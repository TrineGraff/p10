\section{Benchmark modellen}
\begin{frame}{Benchmark modellen}{Den autoregressive model}
\begin{itemize}
\item ordenen bestemmes ud fra BIC, hvor \(p = 1,\ldots, 12\)
\end{itemize}
\begin{table}[h]
\center
\begin{tabular}{ll}
\toprule
$\widehat{\phi}_1$ &$ -0.0162 $ \\
$\widehat{\phi}_2$ & $0.1992^{***}$  \\
$\widehat{\phi}_3$ &$0.1873^{***}$  \\
$\widehat{\phi}_4$ &$0.1686^{***} $ \\ \midrule
BIC & -3.5651 \\
 R$^2_{\text{adj}}$ & 12.31\% \\
LogLik &  211.8617\\ \bottomrule
 \end{tabular}
\caption{Estimationsresultater for en \(\text{AR} \del{4}\), BIC, justeret R$^2$ og log-likehood. Det opløftede symbol betegner signifikans ved henholdsvis $^{***}$0.1\%, $^{**}$1\%, $^{*}$5\% og $^{\dagger}$10\%.} \label{tab:est_ar}
\end{table}
\begin{itemize}
\item afviser normalitet samt at de første 10 autokorrelationer er nul
\item MAE på 0.1312 og MSE på 0.0272
\end{itemize}
\end{frame}

%\begin{frame}{Benchmark modellen}{Den autoregressive model}
%\begin{table}
%\center
%\begin{tabular}{lcc} \toprule
%Skewness & 0.2666 \\
%Kurtosis & 1.4773 \\
%JB-test & \(2.535 \cdot 10^{-13}\) \\ 
%LB$_{10}$-test & 0 \\ 
%\\
%MAE & 0.1312 \\
%MSE & 0.0272 \\ \bottomrule
%\end{tabular}
%\caption{Skewness, excess kurtosis og \(p\)-værdier for Jarque-Bera og Ljung-Box testen for de standardiserede residualer af en \(\text{AR} \del{4}\). Vi lader LB$_{10}$ betegne Ljung-Box testen med lag = 10. 
%\(p\)-værdier \(< 2.2 \cdot 10^{-16}\) sættes til 0.} \label{tab:test_ar}
%\end{table}
%\end{frame}

\begin{frame}{Benchmark modellen}{Faktor modellen}
\begin{itemize}
\item Antallet af faktorer bestemmes ud fra følgende informationskriterier, hvor \(k = 1, \ldots, 20\):
\begin{itemize}
\item $\text{IC}_1 \del{k} = \ln V \del{k, \widehat{\mathbf{F}}} + k \frac{p + T}{p T} \ln \del{\frac{p T}{p + T}}$,
\item $\text{IC}_2 \del{k} = \ln V \del{k, \widehat{\mathbf{F}}} + k \frac{p + T}{p T} \ln \del{ \min \cbr{p, T}}$,
\item $\text{IC}_3 \del{k} = \ln V \del{k, \widehat{\mathbf{F}}} + k \frac{\ln \del{\min \cbr{p, T}}}{\min \cbr{p, T}}$,
\end{itemize}
hvor \(V \del{k, \widehat{\mathbf{F}}} = \del{p T}^{-1} \sum_{j=1}^p \sum_{t=1}^{T} \del{x_{jt} -\boldsymbol{\lambda}_j \widehat{\mathbf{F}}_t}^2\).
\item Lad \(\widehat{\textbf{Z}} = \del{\widehat{{\textbf{F}}}^T \ \boldsymbol{\omega}^T}^T\) være en \(\del{k+m} \times T\) matrix, hvor \(\widehat{{\textbf{F}}}\) er en \(T \times k\) matrix af estimerede faktorer og \(\boldsymbol{\omega}\) er en \(T \times m\) matrix af laggede værdier af arbejdsløshedsraten.
Lad \(m = 4\), da fjernes de første 4 rækker i \(\widehat{\textbf{Z}}\).
\item Parametrene $\widehat{\boldsymbol{\beta}} = \del{ \widehat{\boldsymbol{\beta}}^T_{\textbf{F}} \ \widehat{\boldsymbol{\beta}}^T_{\boldsymbol{\omega}}}^T$ estimeres med OLS.
\end{itemize}
\end{frame}

\begin{frame}{Benchmark modellen}{Faktor modellen}
\begin{table}[h]
\center
\begin{tabular}{lccccc}
\toprule
\multicolumn{5}{c}{Faktor model (IC$_1$)} \\ \midrule
& Værdi &  IC$_1$ &  R$^2_{\text{adj}}$ & LogLik  \\
$k$ & 6 &  $-0.3519$ &  15.79\% &  224.3621  \\ \bottomrule \toprule
\multicolumn{5}{c}{Faktor model (IC$_2$)} \\ \midrule
 & Værdi &  IC$_2$ &  R$^2_{\text{adj}}$ & LogLik \\
 $k$ &11 & $-0.5314$ &  16.85\% &  230.3414 \\\bottomrule \toprule
\multicolumn{5}{c}{Faktor model (IC$_3$)} \\ \midrule
& Værdi &  IC$_3$ &  R$^2_{\text{adj}}$ & LogLik\\
$k$ & 20 & $-0.6931$ & 17.87\% & 238.3753 \\  \bottomrule
 \end{tabular}
 \caption{Antal faktorer, værdien af informationskriteriet, justeret \(R^2\) samt log-likehood for faktormodellerne valgt ud fra IC$_1$, IC$_2$ og IC$_3$, som betegnes faktor model (IC\(_1\)), faktor model (IC\(_2\)) og faktor model (IC\(_3\)).} \label{tab:est_faktor}
\end{table}
\begin{itemize}
\item vælger faktor model (IC\(_2\) som benchmark)
\end{itemize}
\end{frame}

\begin{frame}{Benchmark modellen}{Faktor modellen}
\begin{itemize}
\item Faktor model (IC$_1$): afviser normalitet, men kan ikke afvise at de første 10 autokorrelationer er nul
\item Faktor model (IC$_2$): kan ikke afvise normalitet samt at de første 10 autokorrelationer er nul
\end{itemize}

\begin{table}
\center
\begin{tabular}{lccccccc} \toprule
& Faktor model (IC$_1$) & & Faktor model (IC$_2$)  \\ \midrule
%Skewness & 0.0444 & & $-0.0418$     \\
%Kurtosis & 0.5768 & & 0.4612 \\
%JB-test & 0.0172 & & 0.0712 \\ 
%LB$_{10}$-test & 0.729  && 0.4637  \\ 
%\\
MAE & 0.1190 & & 0.1111 \\ 
MSE &  0.0221  & & 0.0187 \\ \bottomrule
\end{tabular}
\caption{MAE og MSE for faktor modellerne valgt ud fra IC$_1$ og IC$_2$. }
\end{table}
\end{frame}

%%% Local Variables:
%%% mode: latex
%%% TeX-master: "../beamer"
%%% End:

\section{Data}
\begin{frame}{Data}
\begin{itemize}
\item Datasæt fra FRED
\begin{itemize}
\item 128 variable
\item 1. januar 1959 - 1. november 2017 (707 observationer)
\item Opdelt i 8 grupper:
\begin{columns}
\begin{column}{0.38\textwidth}
    \begin{enumerate}
	\item Output og indkomst
	\item Arbejdsmarked
	\item Bolig
	\item Forbrug, ordrer og varebeholdninger
\end{enumerate}
\end{column}
\begin{column}{0.38\textwidth} 
    \begin{enumerate}
    \setcounter{enumi}{4}
	\item Penge og kredit
	\item Renter og valutakurser
	\item Priser
	\item Aktiemarked
\end{enumerate}
\end{column}
\end{columns}
\end{itemize}

\item Transformerede datasæt
\begin{itemize}
\item 123 variable
\item 1. januar 1960 - 1. juli 2017 svarende til 691 variable
\begin{itemize}
\item Træningsmængde: 1. januar 1960 - 1. december 2005 (552 observationer)
\item Testmængde: 1. januar 2006 - 1. juli 2017 (139 observationer)
\end{itemize}
\item centre responsvariablen og standardiser prædiktorerne
\end{itemize}
\end{itemize}
\end{frame}

\begin{frame}{Data}{Arbejdsløshedsraten}
%\item Arbejdsløshedsraten betragtes som responsvariabel
%\begin{itemize}
%\item repræsenterer den procentvise ledighed af arbejdsstyrken
%\end{itemize}
%
\begin{figure}
 \includegraphics[width=1\linewidth, height=0.7\textheight]{slides/unemployment.pdf}
 \caption{Den øverste figur viser arbejdsløshedsraten og den nederste figur illustrerer 1. differensen af arbejdsløshedsraten fra 1. januar 1960 til 1. juli 2017.}
 \end{figure}
%
\end{frame}

%%% Local Variables:
%%% mode: latex
%%% TeX-master: "../beamer"
%%% End:

\begin{table}
\center
\begin{tabular}{lccccccccc} \toprule
& \multicolumn{2}{c}{IC$_1$} & & \multicolumn{2}{c}{IC$_2$} & &\multicolumn{2}{c}{IC$_3$} \\ \midrule
Skewness & \multicolumn{2}{c}{0.0444} & & \multicolumn{2}{c}{$-0.0418$}  & & \multicolumn{2}{c}{$-0.0724$}   \\
Kurtosis & \multicolumn{2}{c}{0.5768} & & \multicolumn{2}{c}{0.4612}  & & \multicolumn{2}{c}{0.2951}\\
JB-test & \multicolumn{2}{c}{0.0172} & & \multicolumn{2}{c}{0.0712}  & & \multicolumn{2}{c}{0.2678} \\ \cmidrule{2-3}\cmidrule{5-6} \cmidrule{8-9} 
& $e_t$ & $e_t^2$ && $e_t$ & $e_t^2$  && $e_t$ & $e_t^2$  \\
LB$_{10}$ & -  &  - && -  &  -&& - & - \\ \bottomrule
\end{tabular}
\caption{Skewness, excess kurtosis, p -værdier for Jarque Bera og Ljung Box test for de standardiserede residualer fra faktor modellerne valgt ud fra IC$_1$, IC$_2$ og IC$_3$. Vi lader LB$_{10}$ betegne Ljung-Box test med lag = 10. } \label{tab:test_faktor}
\end{table}
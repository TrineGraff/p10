\chapter{R koder} \label{app:r_koder}
I dette appendiks inkluderes \Rlang-koderne. 

\section{Autoregressiv model} \label{sec:auto}
Funktionen \texttt{opt.orden} anvendes til at bestemme ordenen $p$ af en autoregressive model.  
\rfile[firstline=10,lastline=28]{../R/unrate/autoregressive_model/insample.R}
%
Vi anvender funktionen \texttt{beta} til at estimere parametrene udfra OLS. 
\rfile[firstline=35,lastline=74]{../R/unrate/autoregressive_model/insample.R}

\texttt{ForecastAR} bruges til prædiktion
\rfile[firstline=5,lastline=20]{../R/unrate/autoregressive_model/outsample.R}

\section{Faktor modellen} \label{sec:faktor}
Funktionen \texttt{estFactors} bruges til at estimerer faktorerne.
\rfile[firstline=24,lastline=58]{../R/unrate/faktor_model/insample.R}

\texttt{estFactors} anvendes til identificere faktorerne. 
%
\rfile[firstline=7,lastline=22]{../R/unrate/faktor_model/insample.R}
%
Vi anvender \texttt{ForecastFaktort} til prædiktion
\rfile[firstline=20,lastline=44]{../R/unrate/faktor_model/outsample.R}

\section{Lasso og dens generaliseringer} \label{sec:lasso}
Koderne i dette afsnit er kun for lasso modellen, men i det det er samme princip for de resterende modeller viser vi ikke disse. 

Vi bruger, som nævnt \texttt{glmnet} og \texttt{cv.glmnet} til at estimerer $\widehat{\lambda}$ ud fra krydsvalidering. 
\rfile[firstline=9,lastline=10]{../R/unrate/shrinkage_metoder/coordinate/lasso/krydsvalidering/insample.R}

Vi anvender funktionen \texttt{BIC} til at estimerer $\widehat{\lambda}$ ud fra  BIC
\rfile[firstline=2,lastline=32]{../R/unrate/shrinkage_metoder/coordinate/bic.R}

Til prædiktion anvender vi funktionen \texttt{Forecast} 
\rfile[firstline=1,lastline=19]{../R/unrate/shrinkage_metoder/coordinate/forecasts.R}


\subsubsection{Inferens} \label{subsubsec:inferens}
Funktionen \texttt{fixedLassoInf} udregner \(p\)-værdier og konfidensintervaller for lasso estimatet for en fast værdi af tuning parameteren \(\lambda\).
Funktionen anvender ``standard'' lasso problemet
\begin{align*}
\frac{1}{2} \Vert \y - \X \tbeta \Vert_2^2 + \lambda \Vert \tbeta \Vert_1.
\end{align*}
Hvorimod \texttt{glmnet} multiplicerer først led med faktoren \(\frac{1}{n}\).
Dette betyder, at efter vi har kørt \texttt{glmnet}, da skal vi sætte \texttt{beta = coef(obj, s=lambda/n)[-1]}, hvor \texttt{obj} er objektet, som er returneret af \texttt{glmnet} og \texttt{[-1]} fjerner skæringen som \texttt{glmnet} altid giver som første komponent, for at finde det beta der svarer til en lambda værdi.
%
\rfile[firstline=5,lastline=8]{../R/unrate/shrinkage_metoder/coordinate/lasso/krydsvalidering/TG/inferens.R}
%





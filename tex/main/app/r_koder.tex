\chapter{R koder} \label{app:r_koder}
I dette appendiks inkluderes \Rlang-koderne. 

\section{Autoregressiv model} \label{sec:auto}
Nedenstående funktion anvendes til at bestemme ordenen $p$ af en autoregressiv model.  
\rfile[firstline=10,lastline=28]{../R/unrate/autoregressive_model/insample.R}
%
Herefter anvender vi følgende funktion til at estimere parametrene udfra OLS. 
\rfile[firstline=35,lastline=74]{../R/unrate/autoregressive_model/insample.R}

Den næste funktion anvendes til prædiktion med AR(4) 
\rfile[firstline=5,lastline=20]{../R/unrate/autoregressive_model/outsample.R}

\section{Faktor modellen} \label{sec:faktor}
Funktionerne \texttt{getFactors} og \texttt{estFactors} anvendes til at estimere og identificere faktorerne. 
%
\rfile[firstline=6,lastline=21]{../R/unrate/faktor_model/insample.R}
\rfile[firstline=23,lastline=57]{../R/unrate/faktor_model/insample.R}
%
Herefter prædikterer faktor modellen arbejdsløshedsraten med følgende funktion.
\rfile[firstline=20,lastline=44]{../R/unrate/faktor_model/outsample.R}


\section{Lasso og dens generaliseringer} \label{sec:lasso}

%\subsection{Krydsvalidering} \label{sec:krydsvalidering}

Vi anvender følgende funktion til at estimerer $\widehat{\lambda}$ med BIC
\rfile[firstline=2,lastline=32]{../R/unrate/shrinkage_metoder/coordinate/bic.R}

\subsubsection{Inferens} \label{subsubsec:inferens}
Funktionen \texttt{fixedLassoInf} udregner \(p\)-værdier og konfidensintervaller for lasso estimatet for en fast værdi af tuning parameteren \(\lambda\).
Funktionen anvender ``standard'' lasso problemet
\begin{align*}
\frac{1}{2} \Vert \y - \X \tbeta \Vert_2^2 + \lambda \Vert \tbeta \Vert_1.
\end{align*}
Hvorimod \texttt{glmnet} multiplicerer først led med faktoren \(\frac{1}{n}\).
Dette betyder, at efter vi har kørt \texttt{glmnet}, da skal vi sætte \texttt{beta = coef(obj, s=lambda/n)[-1]}, hvor \texttt{obj} er objektet, som er returneret af \texttt{glmnet} og \texttt{[-1]} fjerner skæringen som \texttt{glmnet} altid giver som første komponent, for at finde det beta der svarer til en lambda værdi.
%
\rfile[firstline=53,lastline=62]{../R/unrate/Inferens/selectiveInference.R}
%
Koden er her vist for $\widehat{\lambda}$ valgt ud fra krydsvalidering. Det er samme princip for $\widehat{\lambda}$  valgt ud fra BIC,  bare at vi anvender nedenstående kode til at estimerer \texttt{best\_lambda}.




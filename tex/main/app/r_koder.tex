\chapter{R koder} \label{app:r_koder}
I dette appendiks inkluderes \Rlang -koderne. 

\section{Autoregressiv model} \label{sec:auto}
Funktionen \texttt{opt.orden} anvendes til at bestemme ordenen $p$ af en autoregressiv model.  
\rfile[firstline=10,lastline=28]{../R/unrate/autoregressive_model/insample.R}
%
Vi anvender funktionen \texttt{beta} til at estimere parametrene udfra OLS. 
\rfile[firstline=35,lastline=74]{../R/unrate/autoregressive_model/insample.R}

Funktionen \texttt{ForecastAR} anvendes til prædiktion.
\rfile[firstline=5,lastline=20]{../R/unrate/autoregressive_model/outsample.R}

\section{Faktor modellen} \label{sec:faktor}
Funktionen \texttt{estFactors} bruges til at estimerer faktorerne.
\rfile[firstline=24,lastline=58]{../R/unrate/faktor_model/insample.R}
%
Funktionen \texttt{estFactors} anvendes til identificere faktorerne. 
%
\rfile[firstline=7,lastline=22]{../R/unrate/faktor_model/insample.R}
%
Funktionen \texttt{ForecastFaktor} anvendes til prædiktion.
\rfile[firstline=20,lastline=44]{../R/unrate/faktor_model/outsample.R}

\section{Coordinate descent} \label{sec:apprkodercoordinate}
\Rlang-koderne i dette afsnit er kun anvendt på lasso modellen, men princippet er det samme for de resterende modeller.  
Vi anvender funktionen \texttt{glmnet} fra \Rlang-pakken  af samme navn til at løse følgende optimeringsproblem 
\begin{align*}
\argmin_{\tbeta \in \R^p} \cbr{ \frac{1}{2n} \Vert \y - \X \tbeta \Vert_2^2 + \lambda \sbr{\frac{1}{2}(1- \alpha) \Vert \tbeta \Vert_2^2 + \alpha \Vert \tbeta \Vert_1}}.
\end{align*}
%
Vi anvender funktionen \texttt{cv.glmnet} til at udføre et 10-fold krydsvalidering, hvorudfra vi vælger den optimale model. 
\rfile[firstline=9,lastline=10]{../R/Rkoder/unrate/shrinkage_metoder/coordinate/lasso/krydsvalidering/insample.R}
%
Funktionen \texttt{BIC} udvælger den optimale model udfra BIC. 
\rfile[firstline=2,lastline=32]{../R/Rkoder/unrate/shrinkage_metoder/coordinate/bic.R}
%
Til prædiktion anvender vi funktionen \texttt{Forecast}.
\rfile[firstline=1,lastline=17]{../R/Rkoder/unrate/shrinkage_metoder/coordinate/forecasts.R}


\subsubsection{TG test} 
Vi anvender TG testen på lasso modellen for en fast værdi af tuning parameteren. 
Funktionen \texttt{fixedLassoInf} fra \Rlang-pakken \texttt{selectiveInference}  løser lasso problemet
\begin{align*}
\argmin_{\tbeta \in \R^p} \cbr{\frac{1}{2} \Vert \y - \X \tbeta \Vert_2^2 + \lambda \Vert \tbeta \Vert_1}.
\end{align*}
Hvorimod \texttt{glmnet} multiplicerer første led med faktoren \(\frac{1}{n}\).
Dette betyder, at efter vi har kørt \texttt{glmnet}, da skal vi sætte \texttt{beta = coef(obj, s=lambda/n)[-1]}, hvor \texttt{obj} er objektet, som er returneret af \texttt{glmnet} og \texttt{[-1]} fjerner skæringen, som \texttt{glmnet} altid giver som første komponent, for at finde det beta der svarer til en lambda værdi.
%
\rfile[firstline=3,lastline=7]{../R/Rkoder/unrate/shrinkage_metoder/coordinate/lasso/krydsvalidering/TG/insample.R}
%

\section{LARS} \label{sec:apprkoderlars}
\Rlang-koderne i dette afsnit er kun for LARS modellen, men princippet er det samme for lasso LARS.
Vi bruger \texttt{lars} fra  \Rlang-pakken af samme navn til at udfører least angle regression, hvor funktionen \texttt{cv.lars} udfører 10-fold krydsvalidering, hvorudfra vi vælger den optimale model. 
\rfile[firstline=9,lastline=12]{../R/Rkoder/unrate/shrinkage_metoder/lars/lars/krydsvalidering/insample.R}
Funktionen \texttt{larsBIC} vælger den optimale model udfra BIC. 
\rfile[firstline=1,lastline=25]{../R/Rkoder/unrate/shrinkage_metoder/lars/bic_lars.R}
Til prædiktion anvendes funktionen \texttt{Forecast} igen.

\subsubsection{TG test} \label{subsubsec:inferens}
Vi anvender TG testen på LARS modellen, hvor funktionerne \texttt{larInf} og \texttt{lar} fra \Rlang-pakken \texttt{selectiveInference} anvendes. 
Funktionen \texttt{larInf} skal bruge et \texttt{lar}-argument, som har samme output som \texttt{lars}, men har nogle ekstra udregninger, som er nødvendige for polyeder udvælgelsen. 
\rfile[firstline=6,lastline=8]{../R/Rkoder/unrate/shrinkage_metoder/lars/lars/krydsvalidering/TG/insample.R}




\chapter{Generelle statistiske begreber}
I dette appendiks defineres generelle statistiske begreber, som anvendes igennem rapporten.

Løst sagt, siges en estimator at være konsistent, hvis følgen $\boldsymbol{\theta}_n(\mathbf{Y})$ af estimatorer for parameteren $\boldsymbol{\theta}$ konvergerer i sandsynlighed mod den sande værdi $\boldsymbol{\theta}$. Ellers siges estimatoren at være inkonsistent.
Konsistens er altså en asymptopisk egenskab. 
\begin{defn}[Svag konsistens] \label{def:konsistent}
Lad $\{ \boldsymbol{\hat{\theta}}_n \}$ være en følge af estimatorer. Da siges $ \{ \boldsymbol {\hat{\theta}}_n \}$ at være konsistens for $\boldsymbol{\theta}$ hvis
\begin{align*}
\boldsymbol{\hat{\theta}} \overset{p}{\rightarrow} \boldsymbol{\theta}.
%\underset{n \rightarrow \infty}{\lim} P \left( \vert \boldsymbol{\hat{\theta}}_n - \boldsymbol{\theta} \vert \leq \epsilon \right) =1
\end{align*}
\end{defn}

\begin{defn}[Rod-n-konsistent estimator] \label{def:rodn}
En estimator $  \boldsymbol {\hat{\theta}}_n $ for $\boldsymbol{\theta}$ er rod-n-konsistent hvis
\begin{align*}
\boldsymbol {\hat{\theta}}_n  - \boldsymbol{\theta} = O \left( \frac{1}{\sqrt{n}} \right).
\end{align*}
\end{defn}

\begin{thm}[Slutsky's Theorem] \label{thm:sluktsky}
Hvis $X_n \overset{d}{\rightarrow} X$ og $Y_n \overset{p}{\rightarrow} c$, hvor $X$ er en stokastisk variabel og $c$ er en konstant, da gælder, at
\begin{align*}
& X_n + Y_n \overset{d}{\rightarrow} X+c \\
& X_n Y_n \overset{d}{\rightarrow} cX \\
& \frac{X_n}{Y_n} \overset{d}{\rightarrow} \frac{X}{c}, \quad \text{forudsat at } P(c=0)=0.
\end{align*}
\end{thm}
Sætning \ref{thm:sluktsky} tillader at finde grænsefordelingen af $X_n$ og sandsynlighedsgrænsen af $Y_n$ separat.


\begin{defn}[General position] \label{defn:general_position}
Kolonnerne \(\mathbf{X}_1, \ldots, \mathbf{X}_p\) siges at være i generel position hvis the affine span af enhver \(k+1\) vektorer \(s_1 \X_1, \ldots, s_{k+1} \X_{k+1}\) ikke indeholder ethvert element af mængden \(\cbr{\pm \X_i : \ i \neq i_1, \ldots, i_{k+1}}\) for ethvert fortegn \(s_1, \ldots, s_{k+1} \in \cbr{-1,1}\), for \(k < \min \cbr{n,p}\). 
\end{defn}
Ovenstående betingelse er svag og gælder blandt andet når indgangene af \(\X\) kommer af en kontinuert sandsynlighedsfordeling.
At kolonnerne af model matrix er i generel position sikrer at LARS og lasso stierne er entydig, som vises i \citep{lasso_unique}.


\begin{defn}[Konveks] \label{defn:konveks}
En funktion \(f: \ \R^p \rightarrow \R\) er konveks, hvis der for \(\beta\), \(\beta'\) i defintionsmængden af \(f\) og enhver skalar \(s \in \del{0,1}\) gælder
\begin{align*}
f \del{\beta \del{s}} = f \del{s \beta + \del{1-s} \beta'} \leq s f\del{\beta} + \del{1-s} f\del{\beta'}.
\end{align*}
\end{defn}
Geometrisk medfører uligheden at akkorden mellem \(f \del{\beta}\) og  \(f \del{\beta'}\) ligger over grafen af \(f\), som illustreres på figur ---.
Uligheden sikrer, at en konveks funktion ikke kan have et lokalt minimum som ikke også er et globalt minimum, som illustreres på figur --.

%\section{Diebold Mariano test}
%Diebold Mariano testen kan anvendes til at vurdere to konkurrerende prædiktioner i forhold til hinanden.
%
%
%Diebold-Mariano testen er baseret på differensen af tabsfunktionerne
%\begin{align*}
%d_t = L \del{\epsilon_{t+h \given t}^1} - L \del{\epsilon_{t+h \given t}^2}
%\end{align*}
%%
%\begin{defn}[Diebold-Mariano test (DM)]
%Betragt $\mathcal{H}_0: \ \E{d_t}=0$.
%Diebold-Mariano testen er defineret som følgende
%\begin{align*}
%\text{DM}=\frac{1}{1}
%\end{align*}
%\end{defn}

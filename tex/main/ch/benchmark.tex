\chapter{Arbejdsløshed}
I dette kapitel analyserer vi responsvariablen arbejdsløshed. 
Arbejdsløshed er en vigtig økonomisk variable. 
Arbejdsløshed er målt i procent. 
-----
Vi har standardiseret vores forklarende variabler, således at variablerne er centreret omkring 0 og har en varians 1, således vi undgår skæringen i vores regression.
Derudover har vi centreret vores respons variable. 
\section{Benchmark}
Vi har valgt en autoregressiv model, som vores benchmarkmodel. 
Den er defineret i  definition \ref{def:ar}.  
%Vi har valgt at anvende to modeller, som vores benchmarkmodel. 
%Den første er en autoregressiv model, som er defineret  i definition \ref{def:ar}.  
Vi har valgt at inkluderer den, da den er simpel og kun anvender tidligere lags af vores responsvariable. 
Derudover har vi med tidsrække data at gøre. 
Vi vil også bruge modellen til belæg for hvor mange lags der skal tilføjes til vores forklarende variabler. 
%Derudover har vi valgt en faktor model, da det er den første selektions metode der blev introduceret. 
%Vi deler derfor kapitlet op i to dele. 

AR($p$) modellen estimeres med mindste kvadraters metode (OLS), hvor vi lader $p_{\text{max}} = 12$. 
Den maksimale orden på 12, betyder at modellen kun betinger med højst et års tidligere data. 
Dette begrænsning har vi valgt at pålægge modellen fordi vi mener, at en model af højere orden ikke er realistisk for vores data. 
Vi vælger vores optimale orden ved BIC, som er defineret i definition $\ref{def:bic}$.
Vi bestemmer orden på vores træningssæt, og vi får at den mindste BIC er når $p=11$.
Derfor bliver en AR(11) vores benchmarkmodel. 
Vi foretager et rolling-window forecast på vores testsæt.
Forholdet mellem forecastet og de sande værdier er illustreret i figur \ref{fig:ar_arbejds}. 
Det ses, at det ikke er et særligt godt forecast, da den ikke fanger udsvingerne. 
Derudover viser tabel \ref{tab:ar_tab} værdierne af MSE og RMSE, som også er relative høje. 

\imgfigh{ar_fc.pdf}{0.7}{Viser one-step-ahead forecast, hvor den røde linje er en AR(11) model. Hvor den grå linje er de observerede arbejdsløsheds observationer}{ar_arbejds}

\begin{table}
\center
\begin{tabular}{cc}
\toprule
 MAE & MSE \\ \midrule
 0.1312 & 0.0272 \\ \bottomrule
\end{tabular}
\caption{MAE og MSE for \(\text{AR} \del{4}\).} \label{tab:tabs_ar}
\end{table}

 
Vi vil hernæst se om de standardiserede residualer er normalfordelte og ukorrelerede.
Af figur \ref{fig:resid_ar} tydes, at de standardiserede residualer ikke følger en normalfordeling og derudover at der stadig er noget autokorrelation tilbage. 
Tabel \ref{tab:ar_jb_lb} viser skewness, excess kurtosis samt p-værdierne for JB og LB test. 
Vi ser at vi har positive værdier for både excess kurtosis og skewness, samt at JB-testen afviser nul hypotesen for normalitet. 
LB-testen ved lag $10$ afviser også null hypotesen af afhængighed af både de standardiserede residualer og de kvadrede residualer. 

\imgfigh{resid_ar.pdf}{1}{Analyse af de standardiserede residualer for AR(11). }{resid_ar}
\begin{table}
\center
\begin{tabular}{lccc}
\toprule
 Skewness & \multicolumn{2}{c}{0.5399} \\
 Kurtosis & \multicolumn{2}{c}{1.8105}\\
 JB-test & \multicolumn{2}{c}{0} \\
 \cmidrule{2-3}
 & $\omega_t$ &$ \omega_t^2$ \\
 $\text{LB}_{10}$ & 0 & 0 \\ \bottomrule
\end{tabular}
\caption{Skewness, excess kurtosis, p-værdier for Jarque Bera og Ljung-Box test for de standardiserede residualer fra AR(1). $\text{LB}_{10}$ er Ljung box med $lag = 10$  } \label{tab:ar_jb_lb}
\end{table}




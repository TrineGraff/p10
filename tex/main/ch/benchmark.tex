\chapter{Benchmark model}
I dette afsnit vil vi finde en benchmark model ud fra en autoregressive model (se definition \ref{def:ar}) og faktor modellen. 
%Den autoregressive model er primært inkluderet i analysen, da vi anvender ordenen af den autoregressive model, som belæg for hvor mange lags, vi senere vil tilføje til de forklarende variable, da arbejdsløsheden kan være afhængig af de tidligere værdier.
Den autoregressive model er primært inkluderet i analysen, da vi anvender orden af den autoregressive model, som belæg for hvor mange variabler af tidligere værdier, vi senere vil tilføje til de forklarende variable, da arbejdsløsheden kan være afhængige af tidligere værdier. 
Modellen bliver dog stadig inkluderet som en del af analysen, og betragtes som en af benchmark modellerne, da den er simpel og kun anvender tidligere værdier af responsvariablen.

\Rlang-koderne, som er anvendt i dette kapitel, er givet i afsnit \ref{sec:auto} og \ref{sec:faktor}.
\section{Autoregressiv model}
Den autoregressive model af orden \(p\) estimeres med OLS, hvor ordenen bestemmes udfra BIC.
Vi lader $p = 1, \ldots, p_{\max}$, hvor \(p_\text{max}=12\), da vi mener, at arbejdsløshedsraten ikke vil påvirkes mere end et år tilbage.
BIC vælger \(p=4\) og tabel \ref{tab:est_ar} giver estimeringsresultaterne for en \(\text{AR} \del{4}\). \footnote{For en ARMA(p,q) hvor $p_{\max} = 12$ og $q_{\max} = 12$, hvor parametrene estimeres med MLE, vælger BIC også en AR(4).}
%
\begin{table}[h]
\center
\begin{tabular}{ll}
\toprule
$\widehat{\phi}_1$ &$ -0.0162 $ \\
$\widehat{\phi}_2$ & $0.1992^{***}$  \\
$\widehat{\phi}_3$ &$0.1873^{***}$  \\
$\widehat{\phi}_4$ &$0.1686^{***} $ \\ \midrule
BIC & -3.5651 \\
Log-likelihood &  211.8544\\ 
Adj. $R^2$ & 12.30 \% \\ \bottomrule
 \end{tabular}
\caption{Estimationsresultater for en \(\text{AR} \del{4}\) på træningsmængden. Det opløftede symbol betegner signifikansniveauet ved $^{***}$0.1\%, $^{**}$1\%, $^{*}$5\% og $^{\dagger}$10\%.} \label{tab:est_ar}
\end{table}

Alle koefficienter med undtagelse af $\widehat\phi_1$ er signifikante ved et 0.1\% niveau. 
Vi ser også, at justeret $R^2$ kun er 12.31\%, hvilket indikerer, at en autoregressiv model af orden 4 ikke er en optimal model for arbejdsløshedsraten. 

\begin{table}
\center
\begin{tabular}{lll} \toprule
Skewness & \multicolumn{2}{c}{0.2666} \\
Kurtosis & \multicolumn{2}{c}{1.4773} \\
JB-test & \multicolumn{2}{c}{0} \\ \cmidrule{2-3}	
& $\omega_t$ & $\omega_t^2$ \\
LB$_{10}$ & 0 & 0  \\ \bottomrule
\end{tabular}
\caption{Skewness, excess kurtosis, \(p\)-værdier for Jarque Beta og Ljung Box testen for de standardiserede residualer af en \(\text{AR} \del{4}\). Vi lader LB$_{10}$ betegne Ljung-Box testen med lag = 10. } \label{tab:test_ar}
\end{table}

Figur \ref{fig:resid_ar} viser en analyse af de standardiserede residualer for \(\text{AR} \del{4}\). 
Histogrammet og QQ-plottet indikerer, at residualerne har tungere haler end normalfordelingen. 
Derudover observeres få signifikante autokorrelationer.
Dette bekræftes i tabel \ref{tab:test_ar}, som viser skewness, excess kurtosis og \(p\)-værdier for Jarque-Bera og Ljung-Box testen (se definition \ref{def:jbtest} og \ref{def:lbtest}) for de standardiserede residualer. 
Vi ser, at skewness og kurtosis er positive, JB testen afviser nulhypotesen om normalitet og LB testen i lag 10 afviser nulhypotesen om uafhængighed.

Arbejdsløshedsraten prædikteres som nævnt one-step-ahead, hvor estimeringsvinduet udvides med én observeret observation per prædiktion.
Figur \ref{fig:fc_ar} viser arbejdsløshedsraten og den prædikterede arbejdsløshedsrate. 
Heraf ses, at modellen ikke fanger udsvingene, hvilket igen tyder på, at det ikke er en særlig god model for arbejdsløshedsraten. 
Vi finder, at MAE er lig 0.1312 og MSE er lig 0.0272.
%I tabel \ref{tab:tabs_ar} er MAE og MSE udregnet for modellen. 
%\begin{table}
\center
\begin{tabular}{cc}
\toprule
 MAE & MSE \\ \midrule
 0.1312 & 0.0272 \\ \bottomrule
\end{tabular}
\caption{Den gennenmsnitlige absolut fejl og den gennemsnitlige kvadreret fejl for AR(4)} \label{tab:tabs_ar}
\end{table}


\imgfigh{fc_ar.pdf}{1}{Arbejdsløshedsraten og prædiktionen af arbejdsløshedsraten med en \(\text{AR} \del{4}\).}{fc_ar}
\newpage
\section{Faktor modellen}
Parametrene i for faktor modellen estimeres udfra følgende procedure:
\begin{enumerate}
\item Estimer $k$ faktorer for $k = 1, \dots, k_{\max}$, hvor $k_{\max} = 20$.  
\item Vælg $\widehat{k}$ udfra informationskriterierne givet i \eqref{eq:ic1}-\eqref{eq:ic3}.
\item Lad \(\widehat{\textbf{Z}} = \del{\widehat{{\textbf{F}}}^T\boldsymbol{\omega}^T}^T\) være en \(\del{\widehat{k}+m} \times T\) matrix, hvor \(\widehat{{\textbf{F}}}\) er en \(T \times \widehat{k}\) matrix af estimerede faktorer og \(\boldsymbol{\omega}\) er en \(T \times m\) matrix af laggede værdier af arbejdsløshedsraten.
Vi lader \(m = 4\), da den autoregressive model valgte en orden på 4.
Dette medfører nogle uobserveret værdier, og derfor fjernes de første 4 rækker i \(\widehat{\textbf{Z}}\).
\item Estimer parametrene $\widehat{\boldsymbol{\beta}} = \del{ \widehat{\boldsymbol{\beta}}^T_{\textbf{F}} \widehat{\boldsymbol{\beta}}^T_{\boldsymbol{\omega}}}^T$ med OLS.
\end{enumerate}

Tabel \ref{tab:est_faktor} viser antallet af faktorer, værdien af informationskriteriet for dette antal faktorer samt adjusted \(R^2\) for hver af de tre informationskriterier. 

%\begin{table}[h]
%\center
%\begin{tabular}{lccc}
%\toprule
%& Faktor model (IC$_1$) & Faktor model (IC$_2$) & Faktor model (IC$_3$) \\ \midrule
%$\widehat{k}$ & 6 & 11 & 20 \\ 
%IC$\del{\widehat{k}}$ & $-0.3519$  & $-0.5314$ & $-0.6931$  \\
%R$^2_{\text{adj}}$  & 37.08 \% & 50.85 \% & 60.23 \% \\
%LogLike & - & - & -\\ \bottomrule
% \end{tabular}
%\caption{Antal faktorer, værdien af informationskriteriet for dette antal faktorer samt adjusted \(R^2\) for faktormodellerne valgt udfra IC$_1$, IC$_2$ og IC$_3$, som betegnes faktor model (IC\(_1\)), faktor model (IC\(_2\)) og faktor model (IC\(_3\)).} \label{tab:est_faktor}
%\end{table}

\begin{table}[h]
\center
\begin{tabular}{lccccc}
\toprule
\multicolumn{5}{c}{Faktor model (IC$_1$)} \\ \midrule
& Værdi &  IC$_1$ &  R$^2_{\text{adj}}$ & LogLik  \\
$k$ & 6 &  $-0.3519$ &  15.79\% &  224.3621  \\ \bottomrule \toprule
\multicolumn{5}{c}{Faktor model (IC$_2$)} \\ \midrule
 & Værdi &  IC$_2$ &  R$^2_{\text{adj}}$ & LogLik \\
 $k$ &11 & $-0.5314$ &  16.85\% &  230.3414 \\\bottomrule \toprule
\multicolumn{5}{c}{Faktor model (IC$_3$)} \\ \midrule
& Værdi &  IC$_3$ &  R$^2_{\text{adj}}$ & LogLik\\
$k$ & 20 & $-0.6931$ & 17.87\% & 238.3753 \\  \bottomrule
 \end{tabular}
 \caption{Antal faktorer, værdien af informationskriteriet, justeret \(R^2\) samt log-likehood for faktormodellerne valgt udfra IC$_1$, IC$_2$ og IC$_3$, som betegnes faktor model (IC\(_1\)), faktor model (IC\(_2\)) og faktor model (IC\(_3\)).} \label{tab:est_faktor}
\end{table}
Vi observerer, at IC$_1$ vælger den mindst komplekse model, da den vælger det færreste antal faktorer, mens IC$_3$ vælger det maksimale antal faktorer \footnote{For \(k_\text{max} = 50\) vælger IC\(_3\) stadig det maksimale antal faktorer, som er i modstrid med at \(0<k<k_\text{max}\), derfor betragtes faktor modellen valgt udfra IC\(_3\) ikke.}. 
Adjusted \(R^2\) er størst for modellen valgt af IC$_3$ og mindst for modellen valgt af IC$_1$.

Figur \ref{fig:ic1_res} og \ref{fig:ic2_res} viser en analyse af de standardiserede residualer for faktor modellerne.
QQ-plottet viser, at residualerne har tungere haler end normalfordelingen, og ganske få signifikante autokorrelationer.
\begin{table}
\center
\begin{tabular}{lccccccccc} \toprule
& \multicolumn{2}{c}{IC$_1$} & & \multicolumn{2}{c}{IC$_2$} & &\multicolumn{2}{c}{IC$_3$} \\ \midrule
Skewness & \multicolumn{2}{c}{0.0444} & & \multicolumn{2}{c}{$-0.0418$}  & & \multicolumn{2}{c}{$-0.0724$}   \\
Kurtosis & \multicolumn{2}{c}{0.5768} & & \multicolumn{2}{c}{0.4612}  & & \multicolumn{2}{c}{0.2951}\\
JB-test & \multicolumn{2}{c}{0.0172} & & \multicolumn{2}{c}{0.0712}  & & \multicolumn{2}{c}{0.2678} \\ \cmidrule{2-3}\cmidrule{5-6} \cmidrule{8-9} 
& $e_t$ & $e_t^2$ && $e_t$ & $e_t^2$  && $e_t$ & $e_t^2$  \\
LB$_{10}$ & -  &  - && -  &  -&& - & - \\ \bottomrule
\end{tabular}
\caption{Skewness, excess kurtosis, p -værdier for Jarque Beta og Ljung Box test for de standardiserede residualer fra faktor modellerne valgt ud fra IC$_1$, IC$_2$ og IC$_3$. Vi lader LB$_{10}$ betegne Ljung-Box test med lag = 10. } \label{tab:test_faktor}
\end{table}

Dette bekræftes i tabel \ref{tab:test_faktor}, hvor vi observerer lave værdier for skewness og excess kurtosis. 
For JB-testen afvises nulhypotesen om normalitet mens, LB-testens nulhypotesen om uafhængighed ikke kan afvises. 

Tabel \ref{tab:factor_mse_tab} viser MAE og MSE for faktor modellernes prædiktion.
Heraf ses at faktor modellen valgt udfra IC\(_2\) har mindst MAE og MSE.

\begin{table}
\center
\begin{tabular}{lcccc}
\toprule
& $\text{IC}_1$ & $\text{IC}_2$ & $\text{IC}_3$ \\
\midrule 
MAE & 0.1190 & 0.1111 & 0.1048  \\ 
MSE &  0.0221  & 0.0187  & 0.0165 \\ \bottomrule
 \end{tabular}
\caption{MSE og MAE for de forskellige informations kriterier} \label{tab:factor_mse_tab}
\end{table}

Figur \ref{fig:fc_benchmark2} viser arbejdsløshedsraten og den prædikterede arbejdsløshedsrate med faktor modellen valgt udfra IC\(_2\).
\imgfigh{fc_benchmark2.pdf}{0.7}{Arbejdsløshedsraten og prædiktion af arbejdsløshedsraten med faktor modellen valgt udfra IC\(_2\).}{fc_benchmark2}

Vi vælger faktor modellen valgt udfra IC\(_2\) som benchmark model, da den har størst adjusted \(R^2\) og mindst MAE og MSE i forhold til AR(4) og faktor modellen valgt udfra IC\(_1\).





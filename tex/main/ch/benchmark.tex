%\chapter{Arbejdsløshed}
%I dette kapitel analyserer vi responsvariablen arbejdsløshed. 
%Arbejdsløshed er en vigtig økonomisk variable. 
%Arbejdsløshed er målt i procent. 
%Vi har standardiseret vores forklarende variabler, således at variablerne er centreret omkring 0 og har en varians 1, således vi undgår skæringen i vores regression.
%Derudover har vi centreret voreariable. s respons v
\chapter{Benchmark}
Til valg af vores benchmark model har vi valgt, at se på flere modeller. 
Den første er en autoregressiv model, som er defineret  i definition \ref{def:ar}.
Vi har valgt at inkluderer den, da den er simpel og kun anvender tidligere lags af vores responsvariable. 
Derudover har vi med tidsrække data at gøre. 
Vi vil også senere bruge modellen til belæg for hvor mange lags der skal tilføjes til vores forklarende variabler. 
Derudover har vi valgt en faktor model, da det er den første selektions metode der blev introduceret. 
R-koderne, som er anvendt i dette kapitel, er givet i sektionerne \ref{sec:auto} og \ref{sec:faktor} i appendiks .

\section{Autoregressive model}
Den autoregressive model med orden $p$ estimeres med mindste kvadraters metode (OLS) på vores træningsdata, hvor vi lader $p \in \del{1, p_{\max}}$.
Den maksimale orden på 12, betyder at modellen kun betinger med højst et års tidligere data. 
Dette begrænsning har vi valgt at pålægge modellen fordi vi mener, at en model af højere orden ikke er realistisk for vores data. 
Vi estimerer $\widehat{p}$ ved BIC, og får at den mindste BIC er når $p = 4$. 
Derfor ser vi på en AR(4), hvor estimerings resultaterne ses i tabel \ref{tab:est_ar} 
%
\begin{table}[h]
\center
\begin{tabular}{ll}
\toprule
$\widehat{\phi}_1$ &$ -0.0162 $ \\
$\widehat{\phi}_2$ & $0.1992^{***}$  \\
$\widehat{\phi}_3$ &$0.1873^{***}$  \\
$\widehat{\phi}_4$ &$0.1686^{***} $ \\ \midrule
BIC & -3.5651 \\
 R$^2_{\text{adj}}$ & 12.31\% \\
LogLik &  211.8617\\ \bottomrule
 \end{tabular}
\caption{Estimationsresultater for en \(\text{AR} \del{4}\), BIC, justeret R$^2$ og log-likehood. Det opløftede symbol betegner signifikans ved henholdsvis $^{***}$0.1\%, $^{**}$1\%, $^{*}$5\% og $^{\dagger}$10\%.} \label{tab:est_ar}
\end{table}
%
Alle koefficienterne på nær $\widehat\phi_1$ er signifikante ved et 0.1 \% niveau. Vi ser også at den adjusted $R^2$ kun er 12.30 \%, hvilket indikerer på at det ikke er den helt korrekte model for insample data. 

Figur \ref{fig:resid_ar} viser en analyse af de standardiserede residualer for AR(4). 
Vi observerer, at de standardiserede residualer har tunge haler, få signifikante autokorrelationer samt en lille asymmetri i histogrammet sammenlignet med en normal fordeling.
Tabel \ref{tab:test_ar} viser skewness, excess kurtosis og p-værdier for Jarque-Bera og Ljung-Box test for de standardiserede residualer. Vi ser at skewness og kurtosis er positive og at JB test afviser nul hypotesen omkring normalitet. LB testen i lag 10 afviser også nul hypotesen omkring uafhængighed for både de standardiserede residualer samt de kvadrerede standardiserede residualer. 
%
 \begin{table}
\center
\begin{tabular}{lcc} \toprule
Skewness & 0.2666 \\
Kurtosis & 1.4773 \\
JB-test & 0 \\ 
LB$_{10}$-test & 0 \\ \bottomrule
\end{tabular}
\caption{Skewness, excess kurtosis og \(p\)-værdier for Jarque-Bera og Ljung-Box testen for de standardiserede residualer af en \(\text{AR} \del{4}\). Vi lader LB$_{10}$ betegne Ljung-Box testen med lag = 10. } \label{tab:test_ar}
\end{table}

%
Vi vil hernæst performe en out of sample evaluering, dvs vi vil se på forecast performance.
Forecast er som nævnt opnået ved et rolling scheme med expanding estimerings window. 
Figur \ref{fig:fc_ar} viser de prædikterede værdier og de sande observationer. Vi kan se, at en AR(4) ikke fanger udsvingende, hvilket igen tyder på at det ikke er en særlig god model til vores data. 
Vi ser på forecast performance ved anvendelse af tabsfunktionerne, som er defineret i \eqref{eq:mae} og \eqref{eq:mse}. Værdierne vises i tabel \ref{tab:tabs_ar}.  

\begin{table}
\center
\begin{tabular}{cc}
\toprule
 MAE & MSE \\ \midrule
 0.1312 & 0.0272 \\ \bottomrule
\end{tabular}
\caption{MAE og MSE for \(\text{AR} \del{4}\).} \label{tab:tabs_ar}
\end{table}


\imgfigh{fc_ar.pdf}{0.7}{Viser prædiktionen af AR(4) og de sande observationer af responsvariablen.}{fc_ar}







%Vi bestemmer orden på vores træningssæt, og vi får at den mindste BIC er når $p=11$.
%Derfor bliver en AR(11) vores benchmarkmodel. 
%
%
%
%
%
%Vi foretager et rolling-window forecast på vores testsæt.
%Forholdet mellem forecastet og de sande værdier er illustreret i figur \ref{fig:ar_arbejds}. 
%Det ses, at det ikke er et særligt godt forecast, da den ikke fanger udsvingerne. 
%Derudover viser tabel \ref{tab:ar_tab} værdierne af MSE og RMSE, som også er relative høje. 
%
%\imgfigh{ar_fc.pdf}{0.7}{Viser one-step-ahead forecast, hvor den røde linje er en AR(11) model. Hvor den grå linje er de observerede arbejdsløsheds observationer}{ar_arbejds}
%
%\begin{table}
\center
\begin{tabular}{cc}
\toprule
 MAE & MSE \\ \midrule
 0.1312 & 0.0272 \\ \bottomrule
\end{tabular}
\caption{MAE og MSE for \(\text{AR} \del{4}\).} \label{tab:tabs_ar}
\end{table}

% 
%Vi vil hernæst se om de standardiserede residualer er normalfordelte og ukorrelerede.
%Af figur \ref{fig:resid_ar} tydes, at de standardiserede residualer ikke følger en normalfordeling og derudover at der stadig er noget autokorrelation tilbage. 
%Tabel \ref{tab:ar_jb_lb} viser skewness, excess kurtosis samt p-værdierne for JB og LB test. 
%Vi ser at vi har positive værdier for både excess kurtosis og skewness, samt at JB-testen afviser nul hypotesen for normalitet. 
%LB-testen ved lag $10$ afviser også null hypotesen af afhængighed af både de standardiserede residualer og de kvadrede residualer. 
%
%\imgfigh{resid_ar.pdf}{1}{Analyse af de standardiserede residualer for AR(11). }{resid_ar}
%\begin{table}
\center
\begin{tabular}{lccc}
\toprule
 Skewness & \multicolumn{2}{c}{0.5399} \\
 Kurtosis & \multicolumn{2}{c}{1.8105}\\
 JB-test & \multicolumn{2}{c}{0} \\
 \cmidrule{2-3}
 & $\omega_t$ &$ \omega_t^2$ \\
 $\text{LB}_{10}$ & 0 & 0 \\ \bottomrule
\end{tabular}
\caption{Skewness, excess kurtosis, p-værdier for Jarque Bera og Ljung-Box test for de standardiserede residualer fra AR(1). $\text{LB}_{10}$ er Ljung box med $lag = 10$  } \label{tab:ar_jb_lb}
\end{table}

%
\clearpage
\section{Faktor model}
For faktor modellen estimeres parametrene som følgende:
%For at få et bedre overblik beskriver vi fremgangsmåden, for in sample data, i punkter. 
\begin{enumerate}
\item Estimer $r$ faktorer for $r = 1, \dots, r_{\max}$ på træningsmængden, hvor $r_{\max} = 20$.  
\item Vælg $\widehat{r}$ udfra informationskriterierne givet i \eqref{eq:ic1} -  \eqref{eq:ic3}, hvor $\widehat{r}_1$, $\widehat{r}_2$ og $\widehat{r}_3$ betegner antallet af faktorer bestemt udfra informationskriterierne IC$_1$, IC$_2$ og IC$_3$.
\item Vi har, at $\widehat{\textbf{Z}} = \del{\widehat{{\textbf{F}}}^T\boldsymbol{\omega}^T}^T$, hvor $ \boldsymbol{\omega}$ består af laggede værdier af responsvariablen og $\widehat{\textbf{F}}$ består af faktorer. Vi lader $m = 4$, som er bestemt ud fra vores autoregressive model. For at undgå NA's i $\textbf{Z}$, som kommer fra de laggede værdier, fjerner vi de første 4 rækker i $\textbf{Z}$. 
\item Estimer parameterne $\widehat{\boldsymbol{\beta}} = \del{ \widehat{\boldsymbol{\beta}}^T_{\textbf{F}} \widehat{\boldsymbol{\beta}}^T_{\boldsymbol{\omega}}}^T$ med OLS.
\end{enumerate}

Tabel \ref{tab:est_faktor} viser antallet af faktorer, $\widehat{r}$, samt adjusted R squared for hver af de tre informations kriterier. 

\begin{table}[h]
\center
\begin{tabular}{lccc}
\toprule
& IC$_1$ & IC$_2$ & IC$_3$ \\ \midrule
$\widehat{k}$ & 6 & 11 & 20 \\ 
IC$\del{\widehat{k}}$ & $-0.3519$  & $-0.5314$ & $-0.6931$  \\
Adj. R$^2$ & 37.08 \% & 50.85 \% & 60.23 \% \\ \bottomrule
 \end{tabular}
\caption{Antal faktorer, værdien af informationskriteriet for dette antal faktorer samt adjusted \(R^2\) for IC$_1$, IC$_2$ og IC$_3$.} \label{tab:est_faktor}
\end{table}
Vi observerer, at IC$_1$ er modellen med mindst kompleksitet, da den vælger færreste faktorer, hvor IC$_3$ er den model der vælger flest faktorer. Derudover er IC$_3$ modellen med den højeste adjusted R$^2$ værdi og IC$_1$ er den med den mindste adjusted R$^2$.

Figurerne \ref{fig:ic1_res}, \ref{fig:ic3_res} og \ref{fig:ic3_res} viser en analyse af de standardiserede residualer. Vi ser, at der en lille smule tunge haler i QQ plottet for faktor modellerne valgt ud fra IC$_1$ og IC$_2$. Men for faktor modellen valgt ud fra IC$_3$ ses kun en enkelt outliers. 
Derudover ser vi næsten ingen signifikante autokorrelation for de tre faktor modeller. 

\begin{table}
\center
\begin{tabular}{lccccccc} \toprule
& Faktor model (IC$_1$) & & Faktor model (IC$_2$)  \\ \midrule
Skewness & 0.0444 & & $-0.0418$     \\
Kurtosis & 0.5768 & & 0.4612 \\
JB-test & 0.0172 & & 0.0712 \\ 
LB$_{10}$-test & 0.729  && 0.4637  \\ \bottomrule
\end{tabular}
\caption{Skewness, excess kurtosis, $p$-værdier for Jarque-Bera og Ljung-Box testen for de standardiserede residualer fra faktor modellerne valgt ud fra IC$_1$ og IC$_2$. 
Vi lader LB$_{10}$ betegne Ljung-Box testen med lag = 10. } \label{tab:test_faktor}
\end{table}

I tabel  \ref{tab:test_faktor} ser vi for alle tre modeller, at vi har en meget lille skewness, samt en lille kurtosis, hvor faktor modellen valgt ud fra IC$_3$ har mindst værdi af både skewness og kurtosis. 
Vi ser også, at vores null hypotese omkring normalitet for modellerne med informationskriterierne IC$_2$, IC$_3$ ikke kan afvises, men den bliver dog afvist for IC$_1$. uafhængighed ????? how so 
 

Tabel \ref{tab:factor_mse_tab} viser MSE og MAE, som er målt ud fra vores forecasts. Vi kan se, at modellen valgt ud fra IC$_3$ har både mindst MSE og MAE, men den er dog også den mest komplekse model. 

\begin{table}
\center
\begin{tabular}{lcccc}
\toprule
& $\text{IC}_1$ & $\text{IC}_2$ & $\text{IC}_3$ \\
\midrule 
MAE & 0.1190 & 0.1111 & 0.1048  \\ 
MSE &  0.0221  & 0.0187  & 0.0165 \\ \bottomrule
 \end{tabular}
\caption{MAE og MSE for informationskriterierne.} \label{tab:factor_mse_tab}
\end{table}

Vi kan derudover også se, at vores faktor modeller har bedre forecasts performance i forhold til den autoregressive model. 
For in sample ser vi at adjusted R$^2$ er markant bedre for faktor modellerne end den autoregressive model og derudover tyder, det at standardiserede residualer er iid og normalfordelte for faktor modellerne valgt ud fra IC$_2$ og IC$_3$. 

Taget dette i betragtning vil vores faktor model valgt ud fra IC$_3$ være vores benchmark model i det den performer bedst i insample og out of sample. Figur \ref{fig:fc_benchmark} viser de prædikterede værdier for vores benchmark model og de sande observationer.. 

\imgfigh{fc_benchmark.pdf}{0.7}{Prædiktion af arbejdsløshed med faktor modellen og arbejdsløsheden i testmængden.}{fc_benchmark}



%\item Forecast $\widehat{y}_{t+1}$ fra forecast-ligningen givet i ....
%For at bestemme, hvilken informationskriterie vi vil bruge til at finde $r$ i modellen, ser vi på, hvilken af dem der giver det laveste MSE. 
%
%\begin{table}
\center
\begin{tabular}{lcccc}
\toprule
& $\text{IC}_1$ & $\text{IC}_2$ & $\text{IC}_3$ \\
\midrule 
MAE & 0.1190 & 0.1111 & 0.1048  \\ 
MSE &  0.0221  & 0.0187  & 0.0165 \\ \bottomrule
 \end{tabular}
\caption{MAE og MSE for informationskriterierne.} \label{tab:factor_mse_tab}
\end{table}
%
%I Tabel \ref{tab:factor_mse_tab} er for hvert af kriterierne vise deres MSE af forecastet.
%Antallet af faktorer og antallet af lags er defineret til at være ens, såsom vist. Vi ser at $IC_1$ har færrest faktorer, hvor $IC_2$ har lidt flere og $IC_3$ har flest. 
%Vi ser at $IC_3$ giver den med mindst MSE, og derfor anvender vi den som vores benchmark for Faktormodellen. 
%Figur \ref{fig:fc_factor} voser forecast for faktor modellen, hvor $r$ er bestemt udfra $IC_3$
%
%\imgfigh{fc_factor.pdf}{0.7}{Viser forecast, hvor den røde linje er faktor modellen. Hvor den grå linje er de observerede arbejdsløsheds observationer}{fc_factor}











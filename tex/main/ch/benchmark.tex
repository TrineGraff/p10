\chapter{Benchmark model} \label{ch:benhcmarkmodel}
I dette afsnit vil vi finde en benchmark model ud fra en autoregressive model (se definition \ref{def:ar}) og faktor modellen. 
%Den autoregressive model er primært inkluderet i analysen, da vi anvender ordenen af den autoregressive model, som belæg for hvor mange lags, vi senere vil tilføje til de forklarende variable, da arbejdsløsheden kan være afhængig af de tidligere værdier.
Den autoregressive model er primært inkluderet i analysen, da vi anvender orden af den autoregressive model, som belæg for hvor mange variabler af tidligere værdier, vi senere vil tilføje til de forklarende variable, da arbejdsløsheden kan være afhængige af tidligere værdier. 
Modellen bliver dog stadig inkluderet som en del af analysen, og betragtes som en af benchmark modellerne, da den er simpel og kun anvender tidligere værdier af responsvariablen.

\Rlang-koderne, som er anvendt i dette kapitel, er givet i afsnit \ref{sec:auto} og \ref{sec:faktor}.
\section{Autoregressiv model}
Den autoregressive model af orden \(p\) estimeres med OLS, og ordenen bestemmes udfra BIC.
Vi lader $p = 1, \ldots, p_{\max}$, hvor \(p_\text{max}=12\), da vi mener, at arbejdsløshedsraten ikke vil påvirkes mere end et år tilbage.
BIC vælger \(p=4\) og tabel \ref{tab:est_ar} giver estimeringsresultaterne for en \(\text{AR} \del{4}\). \footnote{En ARMA(p,q) estimeret med MLE, hvor $p_{\max} = 12$ og $q_{\max} = 12$ fortrækker også en AR(4) valgt ud fra BIC.}
%
\begin{table}[h]
\center
\begin{tabular}{ll}
\toprule
$\widehat{\phi}_1$ &$ -0.0162 $ \\
$\widehat{\phi}_2$ & $0.1992^{***}$  \\
$\widehat{\phi}_3$ &$0.1873^{***}$  \\
$\widehat{\phi}_4$ &$0.1686^{***} $ \\ \midrule
BIC & -3.5651 \\
 R$^2_{\text{adj}}$ & 12.31\% \\
LogLik &  211.8617\\ \bottomrule
 \end{tabular}
\caption{Estimationsresultater for en \(\text{AR} \del{4}\), BIC, justeret R$^2$ og log-likehood. Det opløftede symbol betegner signifikans ved henholdsvis $^{***}$0.1\%, $^{**}$1\%, $^{*}$5\% og $^{\dagger}$10\%.} \label{tab:est_ar}
\end{table}

Alle koefficienterne med undtagelse af $\widehat\phi_1$ er signifikante ved et 0.1 \% niveau. 
Vi ser også, at adjusted $R^2$ kun er 12.30\%, hvilket indikerer, at en autoregressiv model af orden 4 ikke er en optimal model for arbejdsløshedsraten. 

Figur \ref{fig:resid_ar} viser en analyse af de standardiserede residualer for \(\text{AR} \del{4}\). 
Histogrammet og QQ-plottet indikerer, at residualerne har tungere haler end normalfordelingen. 
Derudover observeres få signifikante autokorrelationer.
Dette bekræftes i tabel \ref{tab:test_ar}, som viser skewness, excess kurtosis og \(p\)-værdier for Jarque-Bera og Ljung-Box testen for de standardiserede residualer. 
Vi ser, at skewness og kurtosis er positive og at JB testen afviser nulhypotesen om normalitet. 
LB testen i lag 10 afviser også nulhypotesen om uafhængighed for både de standardiserede residualer samt de kvadrerede standardiserede residualer. 
%
\begin{table}
\center
\begin{tabular}{lcc} \toprule
Skewness & 0.2666 \\
Kurtosis & 1.4773 \\
JB-test & 0 \\ 
LB$_{10}$-test & 0 \\ \bottomrule
\end{tabular}
\caption{Skewness, excess kurtosis og \(p\)-værdier for Jarque-Bera og Ljung-Box testen for de standardiserede residualer af en \(\text{AR} \del{4}\). Vi lader LB$_{10}$ betegne Ljung-Box testen med lag = 10. } \label{tab:test_ar}
\end{table}
%
%Modellen prædikteres udfra prædiktionsligningen
%\begin{align*}
%\widehat{y}_{t+1 \given t} = \sum_{j = 1}^4 \widehat{\beta}_j y_{t+1-j}.
%\end{align*}
%\qquad t = 552, \ldots, T-1??

Forecast er som nævnt opnået ved et rolling scheme med expanding estimerings window. 
Figur \ref{fig:fc_ar} viser arbejdsløshedsraten og den prædikterede arbejdsløshedsrate med en AR\(\del{4}\).
Vi kan se, at en AR(4) ikke fanger udsvingene, hvilket igen tyder på, at det ikke er en særlig god model. 
I tabel \ref{tab:tabs_ar} er MAE og MSE udregnet for modellen. 

\begin{table}
\center
\begin{tabular}{cc}
\toprule
 MAE & MSE \\ \midrule
 0.1312 & 0.0272 \\ \bottomrule
\end{tabular}
\caption{MAE og MSE for \(\text{AR} \del{4}\).} \label{tab:tabs_ar}
\end{table}


\imgfigh{fc_ar.pdf}{0.7}{Arbejdsløshedsraten og prædiktion af arbejdsløshedsraten med en \(\text{AR} \del{4}\).}{fc_ar}
\newpage
\section{Faktor modellen}
Parametrene i for faktor modellen estimeres udfra følgende procedure:
\begin{enumerate}
\item Estimer $k$ faktorer for $k = 1, \dots, k_{\max}$, hvor $k_{\max} = 20$.  
\item Vælg $\widehat{k}$ udfra informationskriterierne givet i \eqref{eq:ic1}-\eqref{eq:ic3}.
\item Lad \(\widehat{\textbf{Z}} = \del{\widehat{{\textbf{F}}}^T\boldsymbol{\omega}^T}^T\) være en \(\del{\widehat{k}+m} \times T\) matrix, hvor \(\widehat{{\textbf{F}}}\) er en \(T \times \widehat{k}\) matrix af estimerede faktorer og \(\boldsymbol{\omega}\) er en \(T \times m\) matrix af laggede værdier af arbejdsløshedsraten.
Vi lader \(m = 4\), da den autoregressive model valgte en orden på 4.
Dette medfører nogle uobserveret værdier, og derfor fjernes de første 4 rækker i \(\widehat{\textbf{Z}}\).
\item Estimer parametrene $\widehat{\boldsymbol{\beta}} = \del{ \widehat{\boldsymbol{\beta}}^T_{\textbf{F}} \widehat{\boldsymbol{\beta}}^T_{\boldsymbol{\omega}}}^T$ med OLS.
\end{enumerate}

Tabel \ref{tab:est_faktor} viser antallet af faktorer, værdien af informationskriteriet for dette antal faktorer, justerede \(R^2\) samt log-likelihood for hver af de tre informationskriterier. 

\begin{table}[h]
\center
\begin{tabular}{lccc}
\toprule
& IC$_1$ & IC$_2$ & IC$_3$ \\ \midrule
$\widehat{k}$ & 6 & 11 & 20 \\ 
IC$\del{\widehat{k}}$ & $-0.3519$  & $-0.5314$ & $-0.6931$  \\
Adj. R$^2$ & 37.08 \% & 50.85 \% & 60.23 \% \\ \bottomrule
 \end{tabular}
\caption{Antal faktorer, værdien af informationskriteriet for dette antal faktorer samt adjusted \(R^2\) for IC$_1$, IC$_2$ og IC$_3$.} \label{tab:est_faktor}
\end{table}
Vi observerer, at IC$_1$ vælger den mindst komplekse model, da den vælger det færreste antal faktorer, mens IC$_3$ vælger det maksimale antal faktorer \footnote{For \(k_\text{max} = 50\) vælger IC\(_3\) stadig det maksimale antal faktorer, som er i modstrid med at \(0<k<k_\text{max}\), derfor betragtes faktor modellen valgt udfra IC\(_3\) ikke.}. 
Justerede \(R^2\) og log-likelihood er størst for modellen valgt af IC$_3$ og mindst for modellen valgt af IC$_1$.

Figur \ref{fig:ic1_res} og \ref{fig:ic2_res} viser en analyse af de standardiserede residualer for faktor modellerne.
QQ-plottet viser, at residualerne har tungere haler end normalfordelingen, og ganske få signifikante autokorrelationer.
\begin{table}
\center
\begin{tabular}{lccccccc} \toprule
& Faktor model (IC$_1$) & & Faktor model (IC$_2$)  \\ \midrule
Skewness & 0.0444 & & $-0.0418$     \\
Kurtosis & 0.5768 & & 0.4612 \\
JB-test & 0.0172 & & 0.0712 \\ 
LB$_{10}$-test & 0.729  && 0.4637  \\ \bottomrule
\end{tabular}
\caption{Skewness, excess kurtosis, $p$-værdier for Jarque-Bera og Ljung-Box testen for de standardiserede residualer fra faktor modellerne valgt ud fra IC$_1$ og IC$_2$. 
Vi lader LB$_{10}$ betegne Ljung-Box testen med lag = 10. } \label{tab:test_faktor}
\end{table}

Dette bekræftes i tabel \ref{tab:test_faktor}, hvor vi observerer lave værdier for skewness og excess kurtosis. 
For JB-testen afvises nulhypotesen om normalitet mens, LB-testens nulhypotesen om uafhængighed ikke kan afvises. 

Tabel \ref{tab:factor_mse_tab} viser MAE og MSE for faktor modellernes prædiktion.
Heraf ses at faktor modellen valgt udfra IC\(_2\) har mindst MAE og MSE.

\begin{table}
\center
\begin{tabular}{lcccc}
\toprule
& $\text{IC}_1$ & $\text{IC}_2$ & $\text{IC}_3$ \\
\midrule 
MAE & 0.1190 & 0.1111 & 0.1048  \\ 
MSE &  0.0221  & 0.0187  & 0.0165 \\ \bottomrule
 \end{tabular}
\caption{MAE og MSE for informationskriterierne.} \label{tab:factor_mse_tab}
\end{table}

Figur \ref{fig:fc_benchmark2} viser arbejdsløshedsraten og den prædikterede arbejdsløshedsrate med faktor modellen valgt udfra IC\(_2\).
\imgfigh{fc_benchmark2.pdf}{1}{Arbejdsløshedsraten og prædiktion af arbejdsløshedsraten med faktor modellen valgt udfra IC\(_2\).}{fc_benchmark2}

Vi vælger faktor modellen valgt udfra IC\(_2\) som benchmark model, da den har størst justerede \(R^2\) og mindst MAE og MSE i forhold til AR(4) og faktor modellen valgt udfra IC\(_1\).





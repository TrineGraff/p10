%\chapter{Arbejdsløshed}
%I dette kapitel analyserer vi responsvariablen arbejdsløshed. 
%Arbejdsløshed er en vigtig økonomisk variable. 
%Arbejdsløshed er målt i procent. 
%Vi har standardiseret vores forklarende variabler, således at variablerne er centreret omkring 0 og har en varians 1, således vi undgår skæringen i vores regression.
%Derudover har vi centreret voreariable. s respons v
\chapter{Benchmarks}
Som benchmarks har vi valgt at betragte to modeller. 
Den første er en autoregressiv model, som er defineret i definition \ref{def:ar}.
Vi har valgt at inkludere denne, da den er simpel og kun anvender tidligere lags af responsvariablen. 
%Derudover har vi med tidsrække data at gøre. 
Vi vil også senere bruge modellen som belæg for hvor mange lags, der skal tilføjes som forklarende variable udover de resterende 122 variable. 
Derudover har vi valgt en faktor model som benchmark, da det er den første selektions metode der blev introduceret??
\texttt{R}-koderne, som er anvendt i dette kapitel, er givet i sektion \ref{sec:auto} og \ref{sec:faktor} i appendiks.

\section{Autoregressive model}
Den autoregressive model af orden \(p\) estimeres med OLS, og ordenen bestemmes udfra BIC.
Vi lader $p = 1, \ldots, p_{\max}$, hvor \(p_\text{max}=12\), da vi mener, at arbejdsløsheden ikke vil påvirkes mere end et år tilbage.
BIC vælger \(p=4\) og tabel \ref{tab:est_ar} giver estimeringsresultaterne for en \(\text{AR} \del{4}\).
%Den autoregressive model med orden $p$ estimeres med OLS på træningsmængden, hvor vi lader $p = 1, \ldots, p_{\max}$.
%Den maksimale orden på 12, betyder at modellen kun betinger med højst et års tidligere data. 
%Denne begrænsning har vi valgt at pålægge modellen, fordi vi mener, at en model af højere orden ikke er realistisk for vores data. 
%Vi estimerer $\widehat{p}$ ved BIC, og finder at den mindste BIC er når $p = 4$. 
%Derfor ser vi på en AR(4), hvor estimerings resultaterne ses i tabel \ref{tab:est_ar}.
%
\begin{table}[h]
\center
\begin{tabular}{ll}
\toprule
$\widehat{\phi}_1$ &$ -0.0162 $ \\
$\widehat{\phi}_2$ & $0.1992^{***}$  \\
$\widehat{\phi}_3$ &$0.1873^{***}$  \\
$\widehat{\phi}_4$ &$0.1686^{***} $ \\ \midrule
BIC & -3.5651 \\
Log-likelihood &  211.8544\\ 
Adj. $R^2$ & 12.30 \% \\ \bottomrule
 \end{tabular}
\caption{Estimationsresultater for en \(\text{AR} \del{4}\) på træningsmængden. Det opløftede symbol betegner signifikansniveauet ved $^{***}$0.1\%, $^{**}$1\%, $^{*}$5\% og $^{\dagger}$10\%.} \label{tab:est_ar}
\end{table}

Alle koefficienterne med undtagelse af $\widehat\phi_1$ er signifikante ved et 0.1 \% niveau. 
Vi ser også, at adjusted $R^2$ kun er 12.30\%, hvilket indikerer, at det ikke er den helt korrekte model for dette data. 

Figur \ref{fig:resid_ar} viser en analyse af de standardiserede residualer for \(\text{AR} \del{4}\). 
Histogrammet og QQ-plottet indikerer, at residualerne har tungere haler end normalfordelingen. 
Derudover observeres få signifikante autokorrelationer.
%Vi observerer, at de standardiserede residualer har tunge haler, få signifikante autokorrelationer samt en lille asymmetri i histogrammet sammenlignet med en normal fordeling.
Dette bekræftes i tabel \ref{tab:test_ar}, som viser skewness, excess kurtosis og \(p\)-værdier for Jarque-Bera og Ljung-Box testen for de standardiserede residualer. 
Vi ser, at skewness og kurtosis er positive og at JB testen afviser nulhypotesen om normalitet. 
LB testen i lag 10 afviser også nulhypotesen omkring uafhængighed for både de standardiserede residualer samt de kvadrerede standardiserede residualer. 
%
\begin{table}
\center
\begin{tabular}{lll} \toprule
Skewness & \multicolumn{2}{c}{0.2666} \\
Kurtosis & \multicolumn{2}{c}{1.4773} \\
JB-test & \multicolumn{2}{c}{0} \\ \cmidrule{2-3}	
& $\omega_t$ & $\omega_t^2$ \\
LB$_{10}$ & 0 & 0  \\ \bottomrule
\end{tabular}
\caption{Skewness, excess kurtosis, \(p\)-værdier for Jarque Beta og Ljung Box testen for de standardiserede residualer af en \(\text{AR} \del{4}\). Vi lader LB$_{10}$ betegne Ljung-Box testen med lag = 10. } \label{tab:test_ar}
\end{table}

Vi vil hernæst performe en out of sample evaluering, dvs vi vil se på forecast performance.
Forecast er som nævnt opnået ved et rolling scheme med expanding estimerings window. 
Figur \ref{fig:fc_ar} viser de prædikterede værdier og de sande observationer. Vi kan se, at en AR(4) ikke fanger udsvingende, hvilket igen tyder på at det ikke er en særlig god model til vores data. 
Vi ser på forecast performance ved anvendelse af tabsfunktionerne, som er defineret i \eqref{eq:mae} og \eqref{eq:mse}. Værdierne vises i tabel \ref{tab:tabs_ar}.  

\begin{table}
\center
\begin{tabular}{cc}
\toprule
 MAE & MSE \\ \midrule
 0.1312 & 0.0272 \\ \bottomrule
\end{tabular}
\caption{Den gennenmsnitlige absolut fejl og den gennemsnitlige kvadreret fejl for AR(4)} \label{tab:tabs_ar}
\end{table}


\imgfigh{fc_ar.pdf}{0.7}{Prædiktion af arbejdsløshed med en \(\text{AR} \del{4}\) og arbejdsløsheden i testmængden.}{fc_ar}


%Vi bestemmer orden på vores træningssæt, og vi får at den mindste BIC er når $p=11$.
%Derfor bliver en AR(11) vores benchmarkmodel. 
%
%
%
%
%
%Vi foretager et rolling-window forecast på vores testsæt.
%Forholdet mellem forecastet og de sande værdier er illustreret i figur \ref{fig:ar_arbejds}. 
%Det ses, at det ikke er et særligt godt forecast, da den ikke fanger udsvingerne. 
%Derudover viser tabel \ref{tab:ar_tab} værdierne af MSE og RMSE, som også er relative høje. 
%
%\imgfigh{ar_fc.pdf}{0.7}{Viser one-step-ahead forecast, hvor den røde linje er en AR(11) model. Hvor den grå linje er de observerede arbejdsløsheds observationer}{ar_arbejds}
%
%\begin{table}
\center
\begin{tabular}{cc}
\toprule
 MAE & MSE \\ \midrule
 0.1312 & 0.0272 \\ \bottomrule
\end{tabular}
\caption{Den gennenmsnitlige absolut fejl og den gennemsnitlige kvadreret fejl for AR(4)} \label{tab:tabs_ar}
\end{table}

% 
%Vi vil hernæst se om de standardiserede residualer er normalfordelte og ukorrelerede.
%Af figur \ref{fig:resid_ar} tydes, at de standardiserede residualer ikke følger en normalfordeling og derudover at der stadig er noget autokorrelation tilbage. 
%Tabel \ref{tab:ar_jb_lb} viser skewness, excess kurtosis samt p-værdierne for JB og LB test. 
%Vi ser at vi har positive værdier for både excess kurtosis og skewness, samt at JB-testen afviser nul hypotesen for normalitet. 
%LB-testen ved lag $10$ afviser også null hypotesen af afhængighed af både de standardiserede residualer og de kvadrede residualer. 
%
%\imgfigh{resid_ar.pdf}{1}{Analyse af de standardiserede residualer for AR(11). }{resid_ar}
%\begin{table}
\center
\begin{tabular}{lccc}
\toprule
 Skewness & \multicolumn{2}{c}{0.5399} \\
 Kurtosis & \multicolumn{2}{c}{1.8105}\\
 JB-test & \multicolumn{2}{c}{0} \\
 \cmidrule{2-3}
 & $\omega_t$ &$ \omega_t^2$ \\
 $\text{LB}_{10}$ & 0 & 0 \\ \bottomrule
\end{tabular}
\caption{Skewness, excess kurtosis, p-værdier for Jarque Bera og Ljung-Box test for de standardiserede residualer fra AR(1). $\text{LB}_{10}$ er Ljung box med $lag = 10$  } \label{tab:ar_jb_lb}
\end{table}


\section{Faktor model}
For at få et bedre overblik beskriver vi fremgangsmåden i punkter. 





\chapter{Data} \label{ch:data}
I den empiriske del anvendes et datasæt sammensat af Michael McCracken, som er hentet fra Federal Reserve Bank of St. Louis. \footnote{\url{https://research.stlouisfed.org/econ/mccracken/fred-databases/}}
Datasættet består af månedlige observationer for 128 makroøkonomiske variable i perioden 1. januar 1959 til 1. november 2017 svarende til 707 observationer.
Datasættet er givet i appendiks \ref{app:app_data}.

Da 5 variable har et stort antal ikke-observerede værdier fjernes disse fra datasættet.
Herefter transformeres hver variabel med en given transformation foreslået af Michael McCracken, således at de bliver stationære.
Som følge af disse transformationer opstår nogle ikke-observerede værdier i starten af hver tidsrække.
For at undgå ikke-observerede værdier i datasættet fjerner vi de første 12 samt de sidste 4 observationer.
Det transformerede datasæt består hermed af 123 variable i perioden 1. januar 1960 til 1. juli 2017 svarende til 691 observationer.
Datasættet repræsenterer en bred vifte af makroøkonomiske variable, som Michael McCracken har inddelt i 8 grupper:
%
\begin{description}
\item Gruppe 1: Output og indkomst \legendbox{chartreuse4}
\item Gruppe 2: Arbejdsmarked \legendbox{blue3}
\item Gruppe 3: Bolig \legendbox{purple}
\item Gruppe 4: Forbrug, ordrer og varebeholdninger  \legendbox{red3}
\item Gruppe 5: Penge og kredit \legendbox{deeppink}
\item Gruppe 6: Renter og valutakurser \legendbox{orange}
\item Gruppe 7: Priser \legendbox{cadetblue2}
\item Gruppe 8: Aktiemarket \legendbox{goldenrod4}
\end{description} 
%
Farverne, som illustrerer hver gruppe, vil i analysen indikerer, hvilken gruppe variablerne tilhører.
Vi opdeler datasættet i en træningsmængde, som består af 552 observationer fra 1. januar 1960 til 1. december 2005 og en testmængde, som består af 139 observationer fra 1. januar 2006 til 1. juli 2017. 

En betydningsfuld makroøkonomisk variabel er arbejdsløshedsraten, som vi betragter som responsvariabel. 
%\footnote{kilde: U.S. Bureau of Labor Statistics}  
Arbejdsløshedsraten repræsenterer antallet af ledige i procent af den samlede arbejdsstyrke.
Arbejdsstyrken omfatter personer på 16 år eller derover, som er bosiddende i 1 af de 50 stater i USA eller distriktet i Columbia.
Dermed er arbejdsløshedsraten en indikation på, hvorvidt ressourcerne udnyttes.
En stigende arbejdsløshed kan være forbundet med en recession.
Arbejdsløshedsraten er sæsonjusteret og inkluderet i gruppen arbejdsmarked.

På figur \ref{fig:unemployment} illustreres arbejdsløshedsraten samt 1. differensen af arbejdsløshedsraten, som Mccraken foreslår for at opnå stationaritet.
%
\imgfigh{unemployment.pdf}{0.9}{Den øverste figur viser arbejdsløshedsraten  og den nederste figur illustrerer 1. differensen af arbejdsløshedsraten fra 1. januar 1960 til 1. juli 2017.}{unemployment}

%Det overordnet formål med analysen er, at prædiktere arbejdsløshedsraten one-step-ahead.

%De beskrevne modeller i den teoretiske del.
I de følgende kapitler vil vi betragte de beskrevne modeller i den teoretiske del.
Vi centrerer responsvariablen, og standardiserer prædiktorerne.
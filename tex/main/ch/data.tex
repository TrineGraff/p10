\chapter{Data} \label{ch:data}
I den empiriske del anvendes data fra FRED, som er hentet fra Federal Reserve Bank of St. Louis.
Datasættet består af månedlige observationer for 128 makroøkonomiske variable i perioden 1. januar 1959 til 1. november 2017 svarende til 707 observationer.

Tidsrækkerne for variablerne transformeres således at de bliver stationære.
Transformationerne, der anvendes på tidsrækkerne, kan ses i appendiks \ref{app:app_data}.
Transformationerne er forslået af Michael McCracken fra Federal Reserve Bank of St. Louis.  
Som følge at disse transformationer introduceres nogle NA's i starten af hver tidsrække.
Derudover observeres at 5 variable har et stort antal NA's derfor fjernes disse fra datasættet, disse variabler er markeret i appendiks \ref{app:app_data}.
Yderligere fjernes de første 12 samt de sidste 5 observationer således at datasættet er renset for NAs.
Det transformerede datasæt består hermed af 123 variable i perioden 1. januar 1960 til 1. juli 2017 svarende til 691 observationer.
%
%I vores empiriske analyse anvender vi data fra FRED, som er hentet fra Federal Reserve Bank of St. Louis \url{https://research.stlouisfed.org/econ/mccracken/fred-databases/}.
%Der anvendes nogle transformationer for at gøre tidsrækkerne stationære, som er noteret mere detaljeret i appendiks \ref{app:app_data}. 
%Disse transformationer er forslået af Michael McCracken fra Federal Reserve Bank of St. Louis.  
%Der observeres at 5 af disse variabler har mere end 43 NA's inkluderet og derfor bliver de fjernet fra datasættet. 
%Derudover ser vi at de resterende NA's opstår kun i de første eller/og sidste observationer, vi fjerner derfor 17 observationer i hver variable for at undgå NA's. 
%Så vores datasæt inkluderer altså 123 tidsrækker, som indeholder $691$ månedlige observationer og går fra 1. januar 1960 til 1. juli 2017. 
%Hernæst har vi standardiseret vores data, således at variablerne er centreret omkring 0 og har en varians 1, således vi undgår skæringen i vores regression. 
Datasættet repræsenterer en stor brede af makroøkonomiske variable, som Michael McCracken har inddelt i 8 kategorier: 
\begin{enumerate}
\item \textbf{Output og indkomst:} Indeholder 16 tidsrækker
\item \textbf{Arbejdsmarked:}  Indeholder 31 tidsrækker
\item \textbf{Boliger:} Indeholder 10 tidsrækker
\item \textbf{Forbrug, ordrer og varebeholdninger:} Indeholder 7 tidsrækker
\item \textbf{Penge og kredit:} Indeholder 14 tidsrækker
\item\textbf{ Rente og valutakurs:} Indeholder 21 tidsrækker
\item \textbf{Priser:} Indeholder 20 tidsrækker
\item \textbf{Aktiemarked:} Indeholder 4 tidsrækker
\end{enumerate}
%
%Når vi konstruerer vores model anvender vi 122 forklarende variabler og 1 respons variable og for at validerer vores model deler vi vores fulde datasæt i et træningssæt, som består af 552 observationer fra 1. januar 1960 til 1. december 2005 og et testsæt, som består af 139 observationer fra 1. januar 2006 til 1. Juli 2017. 
%En betydningsfuld makroøkonomisk variable er arbejdsløshed, som blandt andet vil være vores responsvariable.  Arbejdsløshed er inkluderet i gruppen arbejdsmarked. 
Herefter opdeles datasættet i en træningsmængde, som består af 552 observationer fra 1. januar 1960 til 1. december 2005 og en testmængde, som består af 139 observationer fra 1. januar 2006 til 1. juli 2017. 

En betydningsfuld makroøkonomisk variable er arbejdsløshed, som vi betragter som responsvariabel.  
Arbejdsløshed er inkluderet i gruppen arbejdsmarked.
For at få et bedre overblik over resultaterne deler vi analysen op således hver responsvariabler har et kapitel. 
Formålet ved analysen er at finde hvilken model er den bedste til forecaste. For at måle dette bruger vi 



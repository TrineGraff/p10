\chapter{Empirisk analyse}
I dette kapitel anvender vi data fra FRED, som er hentet fra Federal Reserve Bank of St. Louis \url{https://research.stlouisfed.org/econ/mccracken/fred-databases/}.
Der anvendes nogle transformationer for at gøre tidsrækkerne stationære, som er noteret mere detaljeret i appendiks ... 
Disse transformationer er forslået af Michael McCracken fra Federal Reserve Bank of St. Louis.  
Der observeres at 5 af disse variabler har mere end 43 NA's inkluderet og derfor bliver de fjernet fra datasættet. 
Derudover ser vi at de resterende NA's opstår kun i de første eller/og sidste observationer, vi fjerner derfor 17 observationer i hver variable for at undgå NA's. 
Så vores datasæt inkluderer altså 123 tidsrækker, som indeholder $691$ månedlige observationer og går fra 1. januar 1960 til 1. juli 2017. 
Hernæst har vi standardiseret vores data, således at variablerne er centreret omkring 0 og har en varians 1. 

Datasættet repræsenterer en stor brede af makroøkonomiske variabler, som Michael McCracken har inddelt i 8 kategorier. 
\begin{itemize}
\item \textbf{Output og indkomst:} Indeholder 16 tidsrækker
\item \textbf{Arbejdsmarked:}  Indeholder 31 tidsrækker
\item \textbf{Boliger:} Indeholder 10 tidsrækker
\item \textbf{Forbrug, ordrer og varebeholdninger:} Indeholder 7 tidsrækker
\item \textbf{Penge og kredit:} Indeholder 14 tidsrækker
\item\textbf{ Rente og valutakurs:} Indeholder 21 tidsrækker
\item \textbf{Priser:} Indeholder 20 tidsrækker
\item \textbf{Aktiemarked:} Indeholder 4 tidsrækker
\end{itemize}

Når vi konstruerer vores model anvender vi 122 forklarende variabler og 1 respons variable og for at validerer vores model deler vi vores fulde datasæt i et træningssæt, som består af 552 observationer fra 1. januar 1960 til 1. december 2005 og et testsæt, som består af 139 observationer fra 1. januar 2006 til 1. Juli 2017. 
En betydningsfuld makroøkonomisk variable er arbejdsløshed, som blandt andet vil være vores responsvariable.  Arbejdsløshed er inkluderet i gruppen arbejdsmarked. 
\section{Benchmark}
Faktormodellen

\newpage

\section{Shrinkage metoder}
I denne sektion vil vi beskrive vores resultater af de forskellige shrinkage metoder. 
Disse metoder er anvendt af en følge af forskellige $\lambda$ værdier.
Vi anvender en 10-gange krydsvalidering for at estimerer den optimale $\lambda$ og derudfra estimerer den bedste mulige model. 
Krydsvalidering bliver målt i gennemsnitlig kvadraters fejl for modellen, som findes for hver værdi af $\lambda$. 
Vi er interesseret i den model med mindst gennemsnitlige kvadraters fejl, men hvor kompleksiteten også spiller en rolle.

\subsection{Lasso}
Figur \ref{fig:cv_lasso_img} viser forholdet mellem værdien af log af lambda, samt krydsvaliderings fejlene. 
Der ses, at det er først når log lambda er større end -6, der sker nogle signifikante ændringer i krydsvalideringfejlene.

\imgfigh{lasso_cv.pdf}{0.7}{Viser krydsvalidering stien, hvor forholdet mellem den gennemsnitlige kvadraters fejl og $\lambda$ værdierne. De lodrette stiplede streger viser logaritmen af $\lambda_min$  og $\lambda_1se$}{cv_lasso_img}
 
Tabel \ref{tab:cv_lasso} viser værdien af $\lambda_{min}$, som giver den mindste krydsvaliderings fejl samt værdien af $\lambda_{1se}$ som er den største værdi af lambda, hvor fejlene er under én standard afvigelse  af minimum værdien.  Værdien af $p$ betegner kompleksiteten, og error er krydsvalideringsfejlen. 
\begin{table}
\center
\begin{tabular}{cccc}
\toprule
\multicolumn{4}{c}{Lasso} \\ \midrule
 & $\lambda$-værdi & fejl &$ p$ \\ \
 $\lambda_{min}$ & 0.0014 & 0.0019 & 20 \\ 
 $\lambda_{1se}$ & 0.0032 & 0.0020 & 15 \\ \bottomrule
\end{tabular}
\caption{Tabellen viser krydsvalideringsfejl samt antal valgte parameter for de to lambda værdier.  } \label{tab:cv_lasso}
\end{table}
Udfra dette kan vi se, at der ikke er stor forskel på krydsvaliderings fejlene, men en betydelig reducering af antal parameter. 
Derfor lader vi den optimale lambda være med én standard afvigelse i fejlene, dvs $\lambda_{1se}$. 
Tabel \ref{tab: lasso_ud} viser hvilken variabler lasso udvælger. Vi kan se, at lasso vælger 9 variabler ud af de 15, som tilhører gruppen arbejdsmarked. 

\begin{table}
\small
\center
\begin{tabular}{lcc}
\toprule
\multicolumn{2}{c}{Lasso} \\ \midrule
Durable Materials (Output og indkomst) & \\
Ratio of Help Wanted/No. Unemployed (Arbejdsmarked) & \\
Civilian Labor Force (Arbejdsmarked)  & \\
Civilian Employment  (Arbejdsmarked) & \\
Civilian Unemployed - Less Than 5 Weeks (Arbejdsmarked)  & \\
Civilians Unemployed for 5-14 weeks (Arbejdsmarked) & \\
Civilian Unemployed - 15 Weeks \& Over (Arbejdsmarked) & \\
All Employees: Total Nonfarm (Arbejdsmarked) &\\
All Employees: Gods-Producing Industries (Arbejdsmarked) & \\
All Employees: Construction (Arbejdsmarked) & \\
Housing Starts, West (Boliger) & \\
Real Personal Consumption expenditures (Forbrug, ordrer og varebeholdning) &\\
New Orders for Durable Goods (Forbrug, ordrer og varebeholdning)  & \\
5 Years Treasure Rate (Rente og valutakurs) &\\
U.S /U.K Foreign Exchange Rate  (Rente og valutakurs)  & \\ \bottomrule
\end{tabular}
\caption{Overstående er de forklarende variable, som bliver udvalgt af lasso. Det der står i parantes er hvilken gruppe hver variable tilhører} \label{tab: lasso_ud}
\end{table}


\newpage

\subsection{Ridge}
Vi ved at ridge regression ikke mindsker antallet af forklarende variable, men derimod bare mindsker estimaterne. Derfor lader vi den optimale $\lambda$-værdi være den med mindst krydsvalideringsfejl, siden at $\lambda$ med en standard afvigelse i fejlene vil ikke ændre på kompleksiteten. Dette ses også i tabellen \ref{tab:cv_ridge}. 

\begin{table}
\center
\begin{tabular}{cccc}
\toprule
\multicolumn{4}{c}{Ridge} \\ \midrule
 & $\lambda$-værdi & fejl &$ p$ \\ 
 $\lambda_{min}$ & 0.012 & 0.0048 & 122 \\ 
 $\lambda_{1se}$ & 0.015 & 0.0051 & 122 \\ \bottomrule
\end{tabular}
\caption{Tabellen viser krydsvalideringsfejl samt antal valgte parameter for de to lambda værdier.  } \label{tab:cv_ridge}
\end{table}
På plottet ses også, at log $\lambda$ ikke skal være meget større end -4 før vi får en høj fejl.
\imgfigh{ridge_cv.pdf}{0.7}{Viser krydsvalidering stien, hvor forholdet mellem den gennemsnitlige kvadraters fejl og $\lambda$ værdierne. De lodrette stiplede streger viser logaritmen af $\lambda_{min}$  og $\lambda_{1se}$ }{cv_ridge_img}


\newpage

\subsection{Elastic Net}
For elastik net ses i ligning ... , at vi har endnu en ubekendt variable andet end lambda nemlig $\alpha$, hvor $\alpha \in [0,1]$. Derfor tester vi en følge med længde 10 af $\alpha$-værdier og ud fra hvert værdi af $\alpha$ finder vi $\lambda_{min}$ og $\lambda_{1se}$ og deres krydsvaliderings fejl. Disse værdier ses i tabel \ref{tab:cv_el}. 

\begin{table}
\center
\begin{tabular}{lllll}
\toprule
\multicolumn{5}{c}{Elastic net} \\ \midrule
 $\lambda_{\min}$& $\text{kvm}_{\min}$ & $\lambda_{1se}$ & $\text{kvm}_{1se}$ & $\alpha$\  \\
0.0005 & 0.0024 & 0.0011 & 0.0025 & 0.1000 \\
0.0007 & 0.0024 & 0.0017 & 0.0024& 0.1889 \\
0.0010 & 0.0023 & 0.0020 & 0.0024& 0.2778 \\
0.0012 &  0.0022& 0.0022& 0.0023& 0.3667 \\
0.0012& 0.0021& 0.0024 & 0.0022&  0.4556 \\
0.0011 &  0.0021 & 0.0026 & 0.0022 &  0.5444 \\
0.0014 &  0.0021 & 0.0025 &  0.0021& 0.6333 \\
0.0015 & 0.0020 &  0.0026 & 0.0021 & 0.7222 \\
0.0014 &  0.0020 &  0.0023 &  0.0020& 0.8111 \\
0.0013& 0.0019& 0.0027& 0.0020 &  0.9000 \\ \bottomrule
\end{tabular}
\caption{Tabellen viser middelværdien af krydsvalideringsfejlene (kvm) for $\lambda_{\min}$ og $\lambda_{1se}$ for hvert valg af $\alpha$. } \label{tab:cv_el}
\end{table}
For at få et bedre overblik viser tabel  \ref{tab:cv_el1}  den $\lambda_min$ samt den $\lambda_{1se}$ med mindst gennemsnitlige krydsvalideringsfejl, samt antallet af parameter. 
Igen ser vi, at krydsvaliderings fejl for $\lambda_{1se}$ ikke er signifikant forskellige fra $\lambda_{\min}$ men der sker en mindskning i kompleksiteten. Derfor vil vores optimale model være modellen med $\lambda_{1se}$. 
\begin{table}
\center
\begin{tabular}{ccccc}
\toprule
\multicolumn{5}{c}{Elastic Net} \\ \midrule
 & $\lambda$-værdi & kvm & alpha & $p$ \\ 
 $\lambda_{min}$ & 0.0013 & 0.0019 & 0.9 & 23  \\ 
 $\lambda_{1se}$ & 0.015 & 0.0020 & 0.8111 & 19 \\ \bottomrule
\end{tabular}
\caption{Tabellen viser det gennemsnitlige krydsvalideringsfejl (kvm) samt antal valgte parameter for de to lambda værdier.  } \label{tab:cv_el1}
\end{table}
Tabel viser hvilken variabler Elastic udvælger, som de forklarende variabler. 
\begin{table}
\small
\center
\begin{tabular}{lcc}
\toprule
\multicolumn{2}{c}{Elastic Net} \\ \midrule
Duarable Materials (Output og indkomst) & \Duarable Materials (1) \\\
Nondurable Materials (Output og indkomst) &\\
Residential Utilities (Output og indkomst) &\\
Civilian Labor Force (Arbejdsmarked) &\\
Civilian Employment (Arbejdsmarked) &\\
Average Duration of Unemployment (weeks) (Arbejdsmarked) &\\
Civilians Unemployed - Less Than 5 Weeks (Arbejdsmarked) &\\
Civilians Unemployed for 5-14 Weeks (Arbejdsmarked) &\\
Civilians Unemployed- 15 Weeks and Over (Arbejdsmarked)& \\
Intial Claims (Arbejdsmarked) &\\
All Employees: Mining and Logging: Minning (Arbejdsmarked) & \\
All Employees: Construction (Arbejdsmarked) &\\
All Employees:  Wholesale Trade (Arbejdsmarked) &\\
Housing Starts West (Boliger) &\\
Real Personal Consumption Expenditures (Forbrug, ordrer og varebeholdning)& \\
New Orders for Durable Goods (Forbrug, ordrer og varebeholdning)&\\
Nonrevolving consumer credit to Personal Income (Penge og kredit) &\\
5-years Treaury Rate (Rente og Valutakurs) &\\
Japan / U.S Foreign Exchange Rate (Rente og Valutakurs) &\\
U.S / U.K Foreign Exchange Rate (Rente og Valutakurs)& \\
PPI: Intermediate Materials (Priser)& \\
PPI: Metals and metal products:(Priser) &\\
CPI: Mediacal Care (Priser)& \\ \bottomrule
\end{tabular}
\caption{Overstående er de forklarende variable, som bliver udvalgt af elastik net. Det der står i parantes er hvilken gruppe hver variable tilhører} \label{tab: el_ud}
\end{table}

\multicolumn{2}{c}{Elastic Net} \\ \midrule

Nondurable Materials (1) &\\
Residential Utilities (1) &\\
Civilian Labor Force (2) &\\
Civilian Employment (2) &\\
Average Duration of Unemployment (weeks) (2) &\\
Civilians Unemployed - Less Than 5 Weeks (2) &\\
Civilians Unemployed for 5-14 Weeks (2) &\\
Civilians Unemployed- 15 Weeks and Over (2)& \\
Intial Claims (2) &\\
All Employees: Mining and Logging: Minning (2) & \\
All Employees: Construction (2) &\\
All Employees:  Wholesale Trade (2) &\\
Housing Starts West (3) &\\
Real Personal Consumption Expenditures (4) & \\
New Orders for Durable Goods (4)&\\
Nonrevolving consumer credit to Personal Income (5) &\\
5-years Treaury Rate (6) &\\
Japan / U.S Foreign Exchange Rate (6) &\\
U.S / U.K Foreign Exchange Rate (6& \\
PPI: Intermediate Materials (7)& \\
PPI: Metals and metal products:(7) &\\
CPI: Mediacal Care (7)& \\ \bottomrule











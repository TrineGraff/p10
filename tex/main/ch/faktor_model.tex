\clearpage
\subsection{Faktor model}
For at få et bedre overblik beskriver vi fremgangsmåden i punkter. 
\begin{enumerate}
\item Estimer $r$ faktorer for $r = 1, \dots, r_{\max}$ ud fra vores forklarende variabler, hvor $r_{\max} = 20$  
\item Vælg bedste $r$ ud fra ét af informationskriterierne givet i ... 
\item Vi har, at $\textbf{Z} = \del{\textbf{F}, \boldsymbol{\omega}}$. Antal laggede værdier bestemmes ud fra $r$. 
\item Estimer parameterne $\widehat{\beta}_F$ og $\widehat{\beta}_\omega$ med OLS
\item Forecast $\widehat{y}_{t+1}$ fra forecast-ligningen givet i ....
\end{enumerate}
For at bestemme, hvilken informationskriterie vi vil bruge til at finde $r$ i modellen, ser vi på, hvilken af dem der giver det laveste MSE. 

\begin{table}
\center
\begin{tabular}{lcccc}
\toprule
& $\text{IC}_1$ & $\text{IC}_2$ & $\text{IC}_3$ \\
\midrule 
MAE & 0.1190 & 0.1111 & 0.1048  \\ 
MSE &  0.0221  & 0.0187  & 0.0165 \\ \bottomrule
 \end{tabular}
\caption{MAE og MSE for informationskriterierne.} \label{tab:factor_mse_tab}
\end{table}

I Tabel \ref{tab:factor_mse_tab} er for hvert af kriterierne vise deres MSE af forecastet.
Antallet af faktorer og antallet af lags er defineret til at være ens, såsom vist. Vi ser at $IC_1$ har færrest faktorer, hvor $IC_2$ har lidt flere og $IC_3$ har flest. 
Vi ser at $IC_3$ giver den med mindst MSE, og derfor anvender vi den som vores benchmark for Faktormodellen. 
Figur \ref{fig:fc_factor} voser forecast for faktor modellen, hvor $r$ er bestemt udfra $IC_3$

\imgfigh{fc_factor.pdf}{0.7}{Viser forecast, hvor den røde linje er faktor modellen. Hvor den grå linje er de observerede arbejdsløsheds observationer}{fc_factor}







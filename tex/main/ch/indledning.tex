\chapter{Indledning}
At prædiktere en makroøkonomisk variable har interesse af flere årsager.
Aktører på finansielle markeder handler på baggrund af forventninger til fremtidige aktiekurser, renter og valutakurser.
Erhvervsfolk fastlægger produktion og investeringer på baggrund af den forventede udvikling i efterspørgsel.
Politiker tager beslutninger om økonomisk politiske indgreb på baggrund af kort- og langsigtede prognoser for økonomien.
Den enkelte forbruger baserer sine indkøb på indkomst.
En makroøkonomisk variabel afhænger 

En central makroøkonomisk variabel er arbejdsløshedsraten.
Arbejdsløshedsraten betegner den procentvise ledighed af arbejdsstyrken.
Arbejdsstyrken omfatter andelen af befolkningen, hvis arbejdskraft er til rådighed og som enten i beskæftigelse eller ledige.

Givet et datasæt med et stort antal makroøkonomiske variable, skal vi prædiktere arbejdsløshedsraten.

En oplagt metode til at estimere koefficienterne vil være mindste kvadraters metode, som minimerer summen af kvadrerede residualer.
Men i dette tilfælde hvor vi betragter et stort antal variable, da kan prædiktionen forbedres ved at reducere antallet af variable.
OLS estimatoren har ofte lav bias, men høj varians, men hvis antallet af variable reduceres, da kan vi reducere variansen på bekostning af lidt bias.
Dette vil forbedre det bias-varians tradeoff og dermed prædiktionen.?????

Vi betragter lasso estimatoren \citep{lasso}, som er en udvidelse af estimatoren for mindste kvadraters metode, hvor der blot tilføjes en \(\ell_1\)-betingelse.
Denne betingelse vil mindske koefficienterne og endda sætte nogle lig 0.
Hermed udfører lasso estimatoren variabeludvælgelse i lineær regression.

Lasso estimatoren har dog
Vi vil også betragte nogle generaliseringer af lasso problemet.
\citep{zou_hastie} viste at ustabiliteten for variabeludvælgelse af lasso kommer af parameter usikkerheden i at estimere en stor kovariansmatrix.
Ved at erstatte den empiriske estimator af kovariansmatricen med en skrinkage estimator, viste de at de resulterende regressions koefficienter og variabeludvælgelses proces er mere stabil.
Dette er ækvivalent med at pålægge en ekstra \(\ell_2\) norm betingelse på lasso problemet.
Denne metoden er kendt som \textit{elastisk net}, eftersom det er som et net der fanger alle "store fisk" til bedre forecasts.

Da prædiktorerne kan opdeles i blokke, pålægges en sparse betingerlser på disse blokke.
Dette gøres ved en two-stage procedure.
I første stage opdeles prædiktorerne i blokkene. I anden stage, group lasso \citep{group_lasso} is lavet således at prædiktorerne i samme block har en tendens til at vælges sammen.

Til slut betragtes også adaptive lasso.


Vi kan reducere dimensionen udfra faktor modellen, som opsummerer al information af variablerne i nogle få underliggende faktorer.
principale komponenter?

Som det kan ses i den empiriske del, har lasso varianterne tilsvarende out-of-sample forcast og generelt outperformes disse dynamisk faktor model, men elastisk net og group lasso giver mere konsistent variabel udvælgelses resultat over ...





Teorien for lasso problemet og dens generalisering er baseret på tværsnitsdata, men anvendes for tidsrækker.
Teorien for faktor modellen er givet på tidsrække data, men oprindelig også udviklet på tværsnitsdata.


Rapporten er struktureret som følgende:
I kapitel --- præsenteres 

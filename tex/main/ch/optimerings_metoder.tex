\chapter{Optimeringsmetoder} \label{ch:optimeringsmetoder}
\textit{I dette kapitel præsenteres først nogle optimeringsbetingelser, og derefter optimeringsalgoritmerne coordinate descent og Least Angle Regression (LARS), som kan løse lasso problemet.
Kapitel er baseret på kapitel 5 i \citep{hastie}, \citep{glmnet1} og \citep{efron}.} 
%
\section{Konvekse optimeringsbetingelser}
Betragt optimeringsproblemet
\begin{align}
\argmin_{\tbeta \in \R^p} f \del{\tbeta}, \text{ underlagt at } \tbeta \in \mathcal{C}, \label{eq:5.2}
\end{align}
hvor \(f: \R^p \rightarrow \R\) er en konveks funktion og \(\mathcal{C}\) er en konveks mængde (se definition \ref{defn:konveksm} og \ref{defn:konveksfkt}).
Hvis \(f\) er differentiabel, da skal første ordens betingelsen
\begin{align}
\nabla f \del{\tbeta^*}^T \del{ \tbeta - \tbeta^*} \geq 0, \label{eq:5.3}
\end{align}
%en nødvendig og tilstrækkelig betingelse 
være opfyldt, for at en vektor \(\tbeta^* \in \mathcal{C}\) er et globalt optimum for alle \(\tbeta \in \mathcal{C}\). 
Hvis \(\mathcal{C} = \R^p\), da er optimeringsproblemet \eqref{eq:5.2} ikke begrænset, og første ordens betingelsen reduceres til \(\nabla f \del{\tbeta^*} = \mathbf{0}\).
Ofte kan betingelsesmængden \(\mathcal{C}\) beskrives ved nogle konvekse betingelsesfunktioner, således at optimeringsproblemet \eqref{eq:5.2} kan omskrives til
\begin{align}
\argmin_{\tbeta \in \R^p} f \del{\tbeta}, \text{ underlagt at } g_j \del{\tbeta} \leq \mathbf{0} \text{ for } j = 1, \ldots, m, \label{eq:5.5}
\end{align}
hvor \(g_j\) for \(j=1, \ldots, m\) er konvekse funktioner, som betegner betingelserne, der skal være opfyldt.
Lad \(f^*\) betegne den optimale værdi af optimeringsproblemet \eqref{eq:5.5}.
Lagrange funktionen \(L: \R^p \times \R^m_+ \rightarrow \R\) for problem \eqref{eq:5.5} er defineret ved
\begin{align*}
L \del{\tbeta, \lambda} = f \del{\tbeta} + \sum_{j=1}^m \lambda_j g_j \del{\tbeta},
\end{align*}
hvor vægtene \(\lambda \geq 0\) kaldes Lagrange multiplikatorer.
Hvis betingelsen \(g_j \del{\tbeta} \leq \mathbf{0}\) ikke er opfyldt, da vil multiplikatoren \(\lambda_j\) pålægge en straf.
Fra dualitetsbegrebet i teorien for Lagrange funktioner og -multiplikatorer, ved vi, at der eksisterer en optimal vektor \(\boldsymbol{\lambda}^* \geq 0\) af Lagrange multiplikatorer, således at 
\(f^* = \argmin_{\tbeta \in \R^p} L \del{\tbeta^*; \boldsymbol{\lambda}^*}\).
Derfor må ethvert optimum \(\tbeta^*\) af \eqref{eq:5.5}, også være et nulgradient punkt af Lagrange funktionen, og dermed opfylde ligningen 
\begin{align}
0 = \nabla_{\tbeta} L \del{\tbeta^*; \boldsymbol{\lambda}^*} = \nabla f \del{\tbeta^*} + \sum_{j=1}^m \lambda_j^* \nabla g_j \del{\tbeta^*}. \label{eq:5.8}
\end{align}
Hvis der blot er en enkelt betingelsesfunktion \(g\), da reduceres denne betingelse til \(\nabla f \del{\tbeta^*} = - \lambda^* \nabla g \del{\tbeta^*}\).

Karush-Kuhn-Tucker (KKT) betingelserne relaterer den optimale Lagrange multiplikator vektor \(\boldsymbol{\lambda}^* \geq 0\) til den optimale vektor \(\tbeta^* \in \R^p\):
\begin{itemize}
\item \(g_j \del{\tbeta^*} \leq 0\) for alle \(j = 1, \ldots, m\)
\item \(\lambda_j^* g_j \del{\tbeta} = 0\) for alle \(j = 1, \ldots, m\)
\item \(\del{\tbeta^*, \boldsymbol{\lambda}^*}\) opfylder betingelse \eqref{eq:5.8}
\end{itemize}
Disse KKT betingelser er nødvendige og tilstrækkelige for at \(\tbeta^*\) er et globalt optimum, når optimeringsproblemet opfylder en regularitetsbetingelse kaldet \textit{strong duality}.
%
\subsection{Subgradienter}
Som nævnt i underafsnittet \ref{subsec:udregning_lasso} er \(\ell_1\)-normen \(g \del{\tbeta} = \sum_{j=1}^p \vert \beta_j \vert\) konveks, men ikke differentialbel i ethvert punkt, hvor mindst et koordinat \(\beta_j = 0\).
For sådan et problem er første ordens betingelsen \eqref{eq:5.3} og Lagrange betingelsen \eqref{eq:5.8} ikke gældende, da de betragter gradienter af \(f\) og \(g\).
Men der findes en naturlig generalisering af begrebet gradient for ikke-differentiable, konvekse funktioner.

For differentiable, konvekse funktioner giver tangentapproksimationen af første orden en nedre grænse.
%Begrebet subgradient er baseret på en generalisering af dette.
Givet en konveks funktion \(f: \ \R^p \rightarrow \R\), siges \(\tz \in \R^p\) at være en subgradient af \(f\) i \(\tbeta\) hvis
\begin{align*}
f \del{\tbeta'} \geq f \del{\tbeta} + \tz^T \del{ \tbeta' - \tbeta}, 
\end{align*}
for alle \(\tbeta' \in \R^p\).
%Geometrisk er subgradient vektoren \(\tz\) normal til et hyperplan som understøtter ---.
Mængden af alle subgradienter af \(f\) i \(\tbeta\) kaldes \textit{subdifferentialet} og betegnes \(\partial f \del{\tbeta}\).
Når \(f\) er differentialbel i \(\tbeta\), da reduceres subdifferentialet til én vektor, givet ved \(\partial f \del{\tbeta} = \cbr{\nabla f \del{\tbeta}}\).
I punkter hvor \(f\) ikke er differentialbel, da er subdifferentialet en konveks mængde bestående af alle mulige subgradienter.

Figur \ref{fig:subgradients} viser en funktion \(f : \R \rightarrow \R\) og nogle eksempler på subgradienter i punkterne \(\beta_1\) og \(\beta_2\).
I punktet \(\beta_1\) er funktionen differentiabel, og derfor har vi blot en subgradient, givet ved \(f' \del{\beta_1}\). I punktet \(\beta_2\) er funktionen ikke differentiabel, og derfor har vi flere subgradienter, som hver specificerer et tangentplan, som giver en nedre grænse på \(f\).
%
\begin{figure}[H]
\centering
\scalebox{1.2}{\begin{tikzpicture}
\draw [->] (-1.8,0) -- (4,0); % x-aksen
\draw (-1,2.5) to[out=300,in=200] (2.5,1);
\draw (2.5,1) -- (3,1.8);
\draw (3.2,2.5) node [right] {\scalebox{0.7}{$f \del{\beta}$}} -- (3,1.8);

\draw [dashed] (-0.5,0) node [below] {\scalebox{0.7}{$\beta_1$}} -- (-0.5,1.8);
\draw [dashed] (2.5,0) node [below] {\scalebox{0.7}{$\beta_2$}} -- (2.5,1);

\draw [dashed] (-1.1,2.4) node [left] {\scalebox{0.7}{$f \del{\beta_1} + z_a \del{\beta - \beta_1}$}} -- (0.55,0.7);

\draw [dashed] (3.8,1.5) node [right] {\scalebox{0.7}{$f \del{\beta_2} + z_b \del{\beta - \beta_2}$}} -- (1.3,0.5);
\draw [dashed] (3.5,1.7) -- (1.5,0.3);
\draw node [label={[xshift=4.6cm, yshift=1.5cm]\scalebox{0.7}{$f \del{\beta_2} + z_c \del{\beta - \beta_2}$}}] {};
\end{tikzpicture}
}
\caption{En konveks funktion \(f : \ \R \rightarrow \R\) med nogle eksempler på subgradienter i \(\beta_1\) og \(\beta_2\).} \label{fig:subgradients}
\end{figure}
%
Antag mindst en af funktionerne \(\cbr{f, g_j}\) er konvekse, men ikke differentiable, da giver Lagrange betingelsen \eqref{eq:5.8} ikke mening, men under milde betingelser for funktionerne, da kan KKT betingelserne modificeres til følgende
\begin{align}
\mathbf{0} \in \partial f \del{\tbeta^*} + \sum_{j=1}^m \lambda_j^* \partial g_j \del{\tbeta^*}, \label{eq:5.11}
\end{align}
hvor gradienterne i KKT betingelsen \eqref{eq:5.8} erstattes med subdifferentialerne.
Da subdifferentialet er en mængde, betyder \eqref{eq:5.11}, at alle nul vektorer tilhører summen af subdifferentialerne.

%\begin{exmp}[Lasso og subgradienter]
%Betragt lasso problemet
%\begin{align*}
%\argmin_{\tbeta} \cbr{ \Vert \y - \X \tbeta \Vert_2^2}, \ \text{underlagt at } \sum_{j=1}^p \vert \beta_j \vert - R \leq 0,
%\end{align*}
%hvor \(R\) er en positiv konstant.
%Betingelsen \( \sum_{j=1}^p \vert \beta_j \vert - R \leq 0\) er ækvivalent med at kræve at \(\tbeta\) tilhører en \(\ell_1\) kugle med radius \(R\).
%Betingelse \eqref{eq:5.11} er da
%\begin{align*}
%\nabla f \del{\tbeta^*} + \lambda^* \tz^* = 0,
%\end{align*}
%hvor 
%%subgradient vektoren opfylder, at \(z_j^* \in \text{sign} \del{\beta_j^*}\) for \(j = 1, \ldots, p\).
%\(z_j =\text{sign} \del{\beta_j^*}\) hvis \(\beta_j^* \neq 0\) og \(z_j \in \sbr{-1,1}\) hvis \(\beta_j^* = 0\).
%\end{exmp}

\section{Coordinate descent} \label{sec:theory_coordinatedescent}
%Ideen bag coordinat descent er, at optimere en target funktion mht én parameter mens de resterende parametre fastholdes. 
%Vi gennemløber alle parametre iterativ indtil konvergens.
%Coordinate descent er specielt attraktiv for problemer som lasso, som har en simpel lukket løsning for en dimension, men ikke for flere dimensioner.
%
Coordinate descent er en iterativ algoritme, som opdaterer fra \(\tbeta^t\) til \(\tbeta^{t+1}\) ved at vælge én koordinat som opdateres, og da udføres en univariat minimering over denne koordinat.
Hvis koordinat $k$ er valgt i iteration $t$, da er opdateringen givet ved
\begin{align}
\beta_k^{t+1} =\underset{ \beta_k}{\arg \min}  f\del{ \beta_1^t, \beta_2^t, \dots, \beta_{k-1}^t, \beta_k, \beta_{k+1}^t, \dots, \beta_p^t  }, \label{eq:5.36}
\end{align}
hvor $\beta_j^{t+1} = \beta_j^t$ for $j \neq k$. 
Typisk gennemløbes koordinaterne i en forudbestemt rækkefølge.
Dette kan generaliseres til \textit{block coordinate descent}, som anvendes for group lasso, hvor prædiktorerne er opdelt i ikke-overlappende blocks, og da udføres en minimering over en enkelt block for hvert koordinat.

For at algoritmen konvergerer til det globale minimum af en konveks funktion, skal funktionen være kontinuert differentiabel og strengt konveks i hver koordinat. 
Men som nævnt er strafleddet for lasso ikke differentiabel.

For mange optimeringsproblemer kan objektfunktionen dekomponeres
\begin{align}
f(\beta_1, \dots, \beta_p) = g(\beta_1, \dots, \beta_p) + \sum_{j = 1}^p h_j \del{\beta_j}, \label{eq:5.37}
\end{align}
hvor \(g: \R^p \rightarrow \R\) er differentiabel og konveks og \(h_j: \R \rightarrow \R\) er konveks, men ikke nødvendigvis differentiabel.
Bemærk at lasso problemet \eqref{eq:2.5} kan dekomponeres som \eqref{eq:5.37} med \(g \del{\tbeta} =\Vert \y - \X \tbeta \Vert_2^2\) og \(h_j \del{\beta_j} = \lambda \vert \beta_j \vert\).
\citep{Tseng_coordinate} viste, at for enhver konveks funktion \(f\) som kan opdeles som \eqref{eq:5.37}, vil coordinate descent algoritmen \eqref{eq:5.36} konvergere til det globale minimum. 
Nøgleegenskaben bag dette resultat er, at den ikke-differentiable komponent \(h \del{\beta} = \sum_{j=1}^p h_j \del{\beta_j}\), kan opsplittes som summen af funktioner af hver individuel parameter.
Resultatet betyder, at coordinate descent kan bruges til at løse lasso og dens generaliseringer, som beskrives senere i specialet.
Hvis den ikke-differentiable komponent \(h\) ikke kan opsplittes, da kan det ikke garanteres at coordinate descent konvergerer.

%\subsubsection{Lineær regression og lasso}
%Optimalitets betingelserne for lasso problemet \eqref{eq:2.5} er
%\begin{align*}
%-2 \sum_{i=1}^n \del{y_i - \sum_{k=1}^p x_{ik} \beta_k} x_{ij} + \lambda s_j = 0,
%\end{align*}
%hvor \(s_j =\text{sign} \del{\beta_j^*}\) hvis \(\beta_j^* \neq 0\) og \(s_j \in \sbr{-1,1}\) hvis \(\beta_j^* = 0\) for \(j=1, \ldots, p\).
%Coordinate descent algoritmen løser disse ligninger og itererer over \(j=1,2,\ldots,p,1,2, \ldots\).
%Lad os definer det partielle residual \(r_i^{(j)} = y_i - \sum_{k \neq j} x_{ik} \widehat{\beta}_k\), som fjerner nuværende fit fra alle undtaget \(j\)'te prædiktor.
%Da er opdateringen givet ved
%\begin{align*}
%\widehat{\beta}_j = S_\lambda \del{\tilde{\beta}_j},
%\end{align*}
%hvor \(\tilde{\beta}_j\) er koefficienten af en simpel lineær regression af det partial residual på variabel \(j\).
\newpage
\begin{alg} [Coordinate descent for lasso problemet]
\begin{enumerate}
\item Standardisér prædiktorerne \(\x_1, \ldots, \x_p\) og centrér responsvariablen.
Definer en følge af værdier \(\lambda_0 > \lambda_1 > \ldots > \lambda_L\), hvor \(\lambda_0\) vælges, således at \(\widehat{\tbeta}^\text{lasso} \del{\lambda_0} =\mathbf{0}\).
\item For hvert \(\lambda \in \cbr{\lambda_0, \ldots, \lambda_L}\), gentages følgende trin for \(j = 1, \ldots, p\) indtil konvergens:
\begin{itemize}
\item Opskriv lasso problemet \eqref{eq:2.5}
\begin{align*}
\sum_{i=1}^n \del{y_i - \sum_{k \neq j} x_{ik} \widehat{\beta}^\text{lasso}_k - x_{ij} \beta_j}^2 + \lambda \sum_{j = 1}^p \vert \beta_j \vert,
\end{align*}
hvor \(\widehat{\beta}^\text{lasso}_k \del{\lambda}\) er det nuværende estimat for \(\beta_k\) for et given \(\lambda\), hvor \(k \neq j\).
\item Udregn de partielle residualer: \(r_{i}^{(j)} = y_i - \sum_{k \neq j} x_{ik} \widehat{\beta}^\text{lasso}_k \del{\lambda}\) for alle \(i\)
\item Udregn koefficienten af en simpel lineær regression af den partielle residual på \(j\)'te prædiktor: \(\tilde{\beta}_j = \frac{1}{n} \sum_{i=1}^n r_{i}^{(j)} x_{ij}\) 
\item Opdater det nuværende estimat \(\widehat{\beta}^\text{lasso}_j\) ud fra soft-thresholding operatoren
\begin{align}
\widehat{\beta}^\text{lasso}_j \del{\lambda}= S_{\frac{\lambda}{2n}} \del{\tilde{\beta}_j}. \label{eq:update_coordinate}
\end{align}
\end{itemize}
\end{enumerate}
\end{alg}
%
Løsningerne udregnes for en aftagende følge af værdier \(\cbr{\lambda_{\ell}}_{\ell = 0}^L\), hvor \(\lambda_0\) vælges således at \(\widehat{\tbeta} \del{\lambda_0} =\mathbf{0}\). 
Algoritmen udnytter \textit{warm start}, hvilket betyder, at \(\widehat{\tbeta} \del{\lambda_\ell}\) anvendes som begyndelsesværdi for løsningen \(\widehat{\tbeta} \del{\lambda_{\ell + 1}}\), dette fører til en mere stabil algoritme. 
Når \(\widehat{\tbeta} = \mathbf{0}\), har vi, at \(\widehat{\beta}_j\) vil forblive nul hvis \(\frac{1}{n} \left\vert \left\langle \mathbf{x}_j, \mathbf{y} \right\rangle \right\vert < \frac{\lambda}{2n}\). Derfor er \( \lambda_0 = 2 \max_j \left\vert \left\langle \mathbf{x}_j, \mathbf{y} \right\rangle \right\vert\).
Strategien er at vælge en minimum værdi \(\lambda_L = \epsilon \lambda_0\) og konstruere en følge af \(K\) værdier af \(\lambda\), som aftager fra \(\lambda_0\) til \(\lambda_L\) på logskalaen.
Typiske værdier er \(\epsilon = 0.001\) og \(K =100\).

Den beskrevne coordinate descent algoritme er implementeret i \Rlang-pakken \texttt{glmnet}.
Koefficientstierne i figur \ref{fig:diabetes_koef} er fundet ud fra denne algoritme.

\section{LARS}
Først vil vi beskrive least angle regression (LARS) algoritmen, hvorefter vi vil introducere en simpel modifikation, som fører til lasso estimater. \\[2mm]
%
I grove træk fungerer algoritmen som følgende. 
Først sættes alle koefficienter lig nul, og vi finder prædiktoren, som er mest korreleret med responsvariablen \(\y\), denne prædiktor betegnes \(x_{j_1}\).
Der udføres så en simpel lineær regression af \(\y\) på \(x_{j_1}\), hvoraf vi finder en residualvektor.
Vi tager det størst mulige step i retningen af denne prædiktor indtil en anden prædiktor, som betegnes  \(x_{j_2}\), har samme korrelation med den nuværende residualvektor.
Istedet for at fortsætte langs retningen af \(x_{j_1}\) fortsætter LARS i en retning, som er ensvinklet mellem de to prædiktorer, indtil en tredje variabel bliver den mest korreleret variabel.
LARS algoritmen fortsætter da ensvinklet imellem \(x_{j_1}\), \(x_{j_2}\) og \(x_{j_3}\), indtil en fjerde variabel medtages, osv.
LARS algoritmen finder estimaterne \(\widehat{\tmu} = \X \widehat{\tbeta}\) ved at tilføje én prædiktor til modellen i hvert trin, således at præcis \(k\) koefficienter er forskellige fra nul efter \(k\) trin.

Figur \ref{fig:lars} illustrerer algoritmen, hvor $p = 2$ og $\X = \del{\textbf{x}_1 \ \textbf{x}_2}$.
Lad \(\mathbf{c} \del{\widehat{\tmu}}\) betegne de nuværende korrelationer
\begin{align}
\widehat{\mathbf{c}} = \mathbf{c} \del{\widehat{\tmu}} = \X^T \del{\y - \widehat{\tmu}}, \label{eq:lars_1.6}
\end{align}
således at \(\widehat{c}_j\) er proportional med korrelationen mellem prædiktor \(\x_j\) og den nuværende residualvektor.
For \(p=2\) afhænger de nuværende korrelationer kun af projektionen \(\bar{\y}_2\) af \(\y\) på det lineære underrum $\mathcal{L} \del{\X}$ udspændt af \(\x_1\) og \(\x_2\)
\begin{align*}
\textbf{c}\del{\boldsymbol{\widehat{\mu}}} =  \X^T \del{ \bar{\y}_2 - \boldsymbol{\widehat{\mu}}}.
\end{align*}
Algoritmen starter i $\widehat{\boldsymbol{\mu}}_0 = \textbf{0}$.
På figur \ref{fig:lars} ses, at vinklen mellem \(\bar{\y}_2 - \widehat{\boldsymbol{\mu}}_0\) og \(\x_1\) er mindre end vinklen mellem \(\bar{\y}_2 - \widehat{\boldsymbol{\mu}}_0\) og \(\x_2\) og dermed fås \(c_1 \del{\widehat{\boldsymbol{\mu}}_0} > c_2 \del{\widehat{\boldsymbol{\mu}}_0}\).
%\(\bar{\y}_2 - \widehat{\boldsymbol{\mu}}_0\) har en mindre vinkel med \(\x_1\) end \(\x_2\), dvs \(c_1 \del{\widehat{\boldsymbol{\mu}}_0} > c_2 \del{\widehat{\boldsymbol{\mu}}_0}\).
Derfor tilføjer LARS \(\widehat{\boldsymbol{\mu}}_0\) i retningen af \(\x_1\), og vi får
\begin{align*}
\widehat{\tmu}_1 = \widehat{\tmu}_0 + \widehat{\gamma}_1 \x_1,
\end{align*}
hvor stepstørrelsen \(\widehat{\gamma}_1\) vælges, således at korrelationen mellem \(\bar{\y}_2 - \widehat{\tmu}_1\) og \(\x_1\) er lig korrelationen mellem \(\bar{\y}_2 - \widehat{\tmu}_1\) og \(\x_2\).
%  \(\bar{\y}_2 - \widehat{\tmu}_1\) er ligeså korreleret med \(\x_1\) som med \(\x_2\).
Dermed halverer \(\bar{\y}_2 - \widehat{\boldsymbol{\mu}}_1\) vinklen mellem \(\x_1\) og \(\x_2\), således at \(c_1 \del{\widehat{\boldsymbol{\mu}}_1} = c_2 \del{\widehat{\boldsymbol{\mu}}_1}\).
%
\begin{figure}[H]
\centering
\scalebox{0.8}{\begin{tikzpicture}
%\draw [<-] (4,0) node [below] {$\x_1$}-- (-3,0);
\draw [blue] [<-] (1,0) node [below] {$\widehat{\tmu}_1$} -- (-3,0);
\draw [<-] (4,0) node [below] {$\x_1$} -- (1,0);
\filldraw [blue] (1,0) circle (2pt) node [below, black] {$\widehat{\boldsymbol{\mu}}_1$};
\filldraw [green] (-3,0)  circle (2pt) node [below, black]{$\widehat{\boldsymbol{\mu}}_0$};
\draw [dashed] [<-] (5,4) node [above] {$\x_2$} --(1,0);
\draw [<-] (1,4) node [above] {$\x_2$} --(-3,0);

\draw [green] (5,1.66) node [above, black] {$\bar{\y}_2$} -- (-3,0);
\filldraw [green] (5,1.66) circle (2pt);
\draw [blue] [->] (1,0) -- (3, 0.83) node [below, black] {$\mathbf{u}_2$} ;
\draw [green] (3,0.83) -- (5,1.66) ; 

\draw [green] (5,0) node [below] {} -- (4.1,0);
\filldraw [green] (5,0) circle (2pt) ;
\draw (5,0) node [black, below] {$\bar{\y}_1$};
\end{tikzpicture}}
\caption{LARS algoritmen for \(p=2\). \(\bar{\y}_2\) er projektionen af \(\y\) på det lineære underrum \(\mathcal{L} \del{\x_1, \x_2}\).
Algoritmen starter i \(\widehat{\tmu}_0=\mathbf{0}\), hvor residualvektoren \(\bar{\y}_2 - \widehat{\tmu}_0\) har en større korrelation med \(\x_1\) end \(\x_2\). Næste LARS estimat er \(\widehat{\tmu}_1 = \widehat{\tmu}_0 + \widehat{\gamma}_1 \x_1\), hvor \(\widehat{\gamma}_1\) vælges, således at \(\bar{\y}_2 - \widehat{\tmu}_1\) halverer vinklen mellem \(\x_1\) og \(\x_2\). Næste LARS estimat er \(\widehat{\tmu}_2 = \widehat{\tmu}_1 + \widehat{\gamma}_2 \mathbf{u}_2\), hvor \(\mathbf{u}_2\) er en enhedsvektor, som ligger langs denne halveringslinje.
Der gælder, at \(\widehat{\tmu}_2 = \bar{\y}_2\) for \(p=2\), dette er ikke tilfældet for \(p>2\), som det ses på figur \ref{fig:lars2}.
 }\label{fig:lars}
\end{figure}
%
Lad $\mathbf{u}_2$ være enhedsvektoren, som ligger langs denne halveringslinje.
Det næste LARS estimat er dermed
\begin{align*}
\widehat{\boldsymbol{\mu}}_2 = \widehat{\boldsymbol{\mu}}_1+ \widehat{\gamma}_2 \mathbf{u}_2,
\end{align*}
hvor $\widehat{\gamma}_2$ er valgt, således at $\widehat{\boldsymbol{\mu}}_2 = \bar{\textbf{y}}_2$ i dette tilfælde hvor $p = 2$. 
For \(p>2\), da vil stepstørrelsen \(\widehat{\gamma}_2\) være mindre, hvilket fører til en anden ændring af retningen, som illustreres på figur \ref{fig:lars2}.
%
\begin{figure}[H]
\centering
\scalebox{0.8}{\begin{tikzpicture}
\filldraw [green] (-3,0) circle (2pt) node [below, black]{$\widehat{\tmu}_0$};
\draw [<-] (6.8,0) node [below] {$\x_1$} -- (-3,0);
\draw [<-] (1,4) node [above] {$\x_2$} --(-3,0);
\draw [<-] (-5,4) node [above] {$\x_3$} --(-3,0);

\draw [green] (4,0) node [above, black] {$\bar{\y}_1$} -- (-3,0);
\filldraw [green] (4,0) circle (2pt) ;
\draw [blue] (1,0) node [below, black] {$\widehat{\tmu}_1$} -- (-3,0);
\draw [blue] [->] (-3,0) -- (-1.5, 0) node [below, black] {$\mathbf{u}_1$} ;

\draw [green] (6,2.1) node [above, black] {$\bar{\y}_2$} -- (1,0);
\filldraw [green] (6,2.1) circle (2pt) ;
\draw [blue] (4,1.25) node [below, black] {$\widehat{\tmu}_2$} -- (1,0);
\draw [blue] [->] (1,0) -- (2.6, 0.65) node [below, black] {$\mathbf{u}_2$} ;

\draw [green] (5,3.5) node [above, black] {$\bar{\y}_3$} -- (4,1.25);
\filldraw [green] (5,3.5) circle (2pt) ;
\draw [blue] [<-] (4.4,2.2)  -- (4,1.25);
\draw [blue] (4.8,3.1)-- (4.4,2.2);
\draw [blue] [<-] (4.5,3.5) -- (4.8,3.1);
\end{tikzpicture}}
\caption{I hvert trin nærmer LARS estimatet \(\widehat{\tmu}_k\) sig det tilhørende OLS estimat \(\bar{\y}_k\), men vil aldrig nå det.
 }\label{fig:lars2}
\end{figure}
%
Efterfølgende LARS trin tages langs ensvinklede vektorer, som generaliserer vektoren \(\mathbf{u}_2\) i figur \ref{fig:lars}.
Vi antager, at prædiktorerne \(\x_1, \ldots, \x_p\) er lineært uafhængige.
Lad \(\A\) være er en delmængde af indekser \(\cbr{1,\ldots, p}\), og definer matricen
\begin{align}
\X_\A = \del{\dots s_j \x_j \dots}_{j \in \A}, \label{eq:lars_2.4}
\end{align}
hvor $s_j = \pm 1$ og \(\X_\A\) er en matrix, som består af søjlerne i \(\X\), der er inkluderet i \(\mathcal{A}\) gange med \(s_j\).
Lad 
\begin{align}
\mathbf{N}_\A = \X_\A^T \X_\A \quad \text{og} \quad A_\A = \del{\mathbf{1}_\A^T \mathbf{N}_\A^{-1} \mathbf{1}_\A}^{-1/2}, \label{eq:lars_2.5}
\end{align}
hvor \(\mathbf{1}_\A\) er en vektor af 1-taller med en længde lig antallet af elementer i \(\A\).
Da defineres en såkaldt ensvinklet vektor
\begin{align}
\mathbf{u}_\A = \X_\A \omega_\A, \quad \text{hvor } \omega_\A = A_\A \mathbf{N}_\A^{-1} \mathbf{1}_\A, \label{eq:lars_2.6}
\end{align}
som er en enhedsvektor, der gør vinkler mellem søjlerne i \(\X_\A\) lige store, dvs
%der resulterer i lige store vinkler der er mindre end \(90^0\), med søjlerne i \(\X_\A\), dvs
\begin{align}
\X_\A^T \mathbf{u}_\A = A_\A \mathbf{1}_\A \quad \text{og} \quad \Vert \mathbf{u}_\A \Vert_2^2 = 1. \label{eq:lars_2.7}
\end{align}
Antag \(\widehat{\tmu}_\A\) er det nuværende LARS estimat.
Lad \(\widehat{\mathbf{c}} = \X^T \del{\y - \widehat{\boldsymbol{\mu}}_\A}\) være en vektor af nuværende korrelationer \eqref{eq:lars_1.6}.
Den \textit{aktive mængde} \(\A\) er en mængde af indekser, som svarer til prædiktorerne med de største absolutte korrelationer
\begin{align}
\widehat{C} = \max_j \cbr{\abs{\widehat{c}_j}}  \qquad \text{og} \qquad \A= \cbr{j: \ \abs{ \widehat{c}_j} = \widehat{C}}. \label{eq:lars_2.9}
\end{align}
Lad 
\begin{align}
s_j = \text{sign} \del{\widehat{c}_j}, \quad j \in \A. \label{eq:lars_2.10}
\end{align}
%
Herefter kan vi give en fyldestgørende beskrivelse af LARS algoritmen.
%
\begin{alg} [LARS algoritmen]
\begin{enumerate}
\item Standardisér prædiktorerne og centrér responsvariablen. 
Start med \(\widehat{\boldsymbol{\mu}}_0 = \mathbf{0}\), \(\widehat{\mathbf{c}} = \X^T \y\), og \(\A = \emptyset\).
\item Find prædiktoren \(\tx_j\) med den største værdi af \(\abs{\widehat{c}_j}\) og definer den aktive mængde \(\A = \cbr{j}\).
\item 
Gentag følgende indtil alle prædiktorer er indeholdt i den aktive mængde:
\begin{itemize}
\item Udregn \(\widehat{\mathbf{c}}\), \(\widehat{C}\), \(\X_\A\), \(A_\A\) og \(\mathbf{u}_\A\) som i \eqref{eq:lars_2.4}-\ref{eq:lars_2.9} samt
\begin{align*}
\mathbf{a} = \X^T \mathbf{u}_\A.
\end{align*}
\item Opdatér \(\widehat{\boldsymbol{\mu}}_\A\) til
\begin{align}
\widehat{\boldsymbol{\mu}}_{\A_+} = \widehat{\boldsymbol{\mu}}_\A + \widehat{\gamma} \mathbf{u}_\A, \label{eq:lars_2.12}
\end{align}
hvor 
\begin{align}
\widehat{\gamma} = \min_{j \in \A^c} \cbr{ \frac{\widehat{C}- \widehat{c}_j}{A_\A - a_j} , \frac{\widehat{C} + \widehat{c}_j}{A_\A + a_j}}_+, \label{eq:lars_2.13}
\end{align}
og hvor \(\min \cbr{\cdot , \cdot}_+\) indikerer, at minimum kun tages over de positive komponenter indenfor valget af \(j\) i \eqref{eq:lars_2.13}.
\item Sæt \(\A = \A \cup \cbr{\widehat{j}}\), hvor \(\widehat{j}\) er minimeringsindekset i \eqref{eq:lars_2.13}.
\end{itemize}
\end{enumerate}
\end{alg}
%
For LARS algoritmen kræves blot \(p\) trin for at finde den fulde løsning.
De beregningsmæssige omkostninger for LARS algoritmen er af samme orden som løsningen af OLS med \(p\) prædiktorer.

%Formlerne \eqref{eq:lars_2.12} og \eqref{eq:lars_2.13} har følgende fortolkning.
%Definer
%\begin{align}
%\tmu \del{\gamma} = \widehat{\tmu}_\A + \gamma \mathbf{u}_\A, \label{eq:lars_2.14}
%\end{align}
%for \(\gamma > 0\), således at den nuværende korrelation er givet ved
%\begin{align}
%c_j \del{\gamma} = \x_j^T \del{\y - \tmu \del{\gamma}} = \widehat{c}_j - \gamma a_j. \label{eq:lars_2.15}
%\end{align}
%For \(j \in \A\) giver \eqref{eq:lars_2.7} og \eqref{eq:lars_2.9} at
%\begin{align}
%\abs{c_j \del{\gamma}} = \widehat{C} - \gamma A_\A,\label{eq:lars_2.16}
%\end{align}
%som viser, at alle af de maksimale absolutte nuværende korrelationer falder ligeligt ?????
%For \(j \in \A^C\), viser \eqref{eq:lars_2.15} og \eqref{eq:lars_2.16} at \(c_j \del{\gamma}\) er lig den maksimale værdi i \(\gamma = \frac{\widehat{C} - \widehat{c}_j}{A_\A - a_j}\).
%Derfor er \(\widehat{\gamma}\) i \eqref{eq:lars_2.13} den mindst positive værdi af \(\gamma\), således at et nyt indeks \(\widehat{j}\) tilføjes til den aktive mængde.
%\(\widehat{j}\) er minimeringsindekset i \eqref{eq:lars_2.13} og den nye aktive mængde \(\A_+\) er \(\A \cup \cbr{\widehat{j}}\) og den nye maksimum absolut korrelation er \(\widehat{C}_+ = \widehat{C}- \widehat{\gamma} A_\A\).

\subsection{Lasso modifikation} \label{subsec:lasso_modifikation}
I dette afsnit beskrives en simpel modifikation af LARS algoritmen, således at den giver lasso estimater.
Lad \(\widehat{\tbeta}^\text{lasso}\) være løsningen til lasso problemet \eqref{eq:2.5} med \(\widehat{\tmu}^\text{lasso} = \X \widehat{\tbeta}^\text{lasso}\).
Da kan det vises, at fortegnet af enhver ikke-nul koefficient \(\widehat{\beta}_j\) og fortegnet \(s_j\) af den nuværende korrelation \(\widehat{c}_j = \x_j^T \del{\y - \widehat{\tmu}}\) må stemme overens
\begin{align}
\text{sign} \del{\widehat{\beta}_j } = \text{sign} \del{\widehat{c}_j } = s_j, \quad j \in \A. \label{eq:lars_3.1}
\end{align}
%
%\begin{lem}
%For \(\widehat{\beta}^\text{lasso}\) må der gælder, at
%\begin{align*}
%\widehat{c}_j = \widehat{C} \cdot \text{sign} \del{\widehat{\beta}_j},
%\end{align*}
%hvor \(\widehat{c}_j = \x_j^T \del{\y - \widehat{\tmu}}= \x_j^T \del{\y - \X \widehat{\beta}}\).
%Dette medfører, at
%\begin{align}
%\text{sign} \del{\widehat{\beta}_j } = \text{sign} \del{\widehat{c}_j }, \quad j \in \A \label{eq:lars_5.29}
%\end{align}
%\end{lem}
%
Denne restriktion er ikke inkluderet i LARS algoritmen, men kan nemt modificeres hertil:
\textit{Når en ikke-nul koefficient ændrer fortegn eller bliver lig nul, da fjernes variablen fra den aktive mængde og vi beregner igen den nuværende ensvinklede retning \eqref{eq:lars_2.12}}.

For at tage denne modifikation i betragtning, defineres en \(p \times 1\) vektor
\begin{align*}
\widehat{\mathbf{d}} = \begin{cases}
s_j \omega_{\A_j}, &\text{hvis } j \in \A, \\
0, & \text{ellers},
\end{cases}
\end{align*}
hvor \(\omega_{\A_j}\) betegner elementet af vektoren \(\omega_{\A}\), som svarer til indeks \(j\).
For \(j \in \A\) opdateres
\begin{align*}
\widehat{\beta}_j \del{\gamma} = \widehat{\beta}_j^\text{prev} + \gamma \widehat{d}_j,
\end{align*}
hvor \(\widehat{\beta}_j^\text{prev}\) er lasso estimaterne fra det tidligere trin.
Lad 
\begin{align*}
\gamma_j = -\frac{\widehat{\beta}_j}{\widehat{d}_j}, \qquad \text{og} \qquad \tilde{\gamma} = \min_{\gamma_j > 0} \cbr{\gamma_j}.
\end{align*}
%
%Antag vi netop har fuldendt et LARS step, som har givet en ny aktiv mængde \(\A\) som i \eqref{eq:lars_2.9}, og at det tilhørende LARS estimat \(\widehat{\tmu}_\A\) svarer til en lasso løsning \(\widehat{\tmu}^\text{lasso} = \X \widehat{\tbeta}^\text{lasso}\).
%Lad
%\begin{align*}
%\omega_\A = A_\A \mathbf{N}_\A^{-1} \mathbf{1}_\A,
%\end{align*}
%være en vektor med længde lig antallet af elementer i \(\A\) og definer en \(p \times 1\) vektor
%\begin{align*}
%\widehat{\mathbf{d}} = \begin{cases}
%s_j \omega_{\A_j}, &\text{hvis } j \in \A, \\
%0, & \text{ellers}.
%\end{cases}
%\end{align*}
%Hvis vi bevæger os i den positive \(\gamma\) retning langs LARS linjen \eqref{eq:lars_2.14}, ser vi, at
%\begin{align*}
%\tmu \del{\gamma} = \X \tbeta \del{\gamma}, \quad \text{hvor } \beta_j \del{\gamma} = \widehat{\beta}_j + \gamma \widehat{d}_j
%\end{align*}
%for \(j \in \A\).
%Derfor vil \(\beta_j \del{\gamma}\) ændre fortegn i
%\begin{align*}
%\gamma_j = -\frac{\widehat{\beta}_j}{\widehat{d}_j},
%\end{align*}
%den første af sådan en ændring kommer i
%\begin{align*}
%\tilde{\gamma} = \min_{\gamma_j > 0} \cbr{\gamma_j},
%\end{align*}
%for prædiktor \(\x_{\tilde{j}}\).
%Hvis der ikke findes en \(\gamma_j > 0\), da er \(\tilde{\gamma}=\infty\) per definition.
%
%Hvis \(\tilde{\gamma} < \widehat{\gamma}\), da kan \(\beta_{\tilde{j}} \del{\gamma}\) ikke være lasso løsningen for \(\gamma > \tilde{\gamma}\), da restriktionen \eqref{eq:lars_3.1} ikke er opfyldt, eftersom \(\beta_{\tilde{j}} \del{\gamma}\) har ændret fortegn, mens \(c_{\tilde{j}}\) ikke har.
%Der gælder, at \(c_{\tilde{j}}\) ikke kan ændre fortegn indenfor ét LARS trin da \(\abs{c_{\tilde{j}} \del{\gamma}} = \widehat{C} - \gamma A_\A> 0\) af \eqref{eq:lars_2.16}. \\[2mm]
%%
%\textbf{Lasso modifikation} \\
Hvis \(\tilde{\gamma} < \widehat{\gamma}\), stoppes det igangværende LARS trin i \(\gamma = \tilde{\gamma}\) og fjern \(\tilde{j}\) fra udregningen af den næste ensvinklede retning.
Dvs
\begin{align*}
\widehat{\tmu}_{\A_+} = \widehat{\tmu}_\A + \tilde{\gamma} \mathbf{u}_\A \quad \text{og} \quad \A_+ = \A - \cbr{\tilde{j}},
\end{align*}
istedet for \eqref{eq:lars_2.12}.

En mere detaljeret gennemgang af LARS algoritmen med lasso modifikationen kan findes i \citep{efron}.
Da variable kan fjernes og tilføjes til den aktive mængde, er antallet af trin i LARS algoritmen med lasso modifikationen større end \(p\).

\begin{eks}
Figur \ref{fig:diabetes_lars} illustrerer koefficientstierne for LARS algoritmen uden og med lasso modifikationen, som funktion af fraktionen af \(\ell_1\)-normen for diabetes data.
Hvis \(\frac{\abs{\tbeta}}{\max \abs{\tbeta}} = 0\), da er ingen variable tilføjet til den aktive mængde og hvis \(\frac{\abs{\tbeta}}{\max \abs{\tbeta}} = 1\) er alle variable inkluderet.
Af figuren kan vi aflæse rækkefølgen, hvori variablerne medtages i modellen.
For LARS algoritmen uden lasso modifikationen udføres 10 trin, mens LARS algoritmen med lasso modifikationen udfører 12 trin.
De 2 ekstra trin som LARS algoritmen med lasso modifikationen udfører, kommer af, at variablen \texttt{hdl} fjernes og tilføjes igen i henholdsvis trin 11 og 12.
Heraf ses det også, at stien er kontinuert og stykvis lineær.
Den aktive mængde og fortegnene af de aktive variable er konstant mellem trinene.

\imgfigh{diabetes_lars.pdf}{0.9}{Koefficientstierne for LARS algoritmen uden og med lasso modifikationen som funktion af fraktionen af \(\ell_1\)-normen for diabetes data.}{diabetes_lars}
\end{eks}

%Coordinate descent kan være hurtigere end LARS algoritmen særligt for store problemer.
%Dette skyldes, at \eqref{eq:update_coordinate} hurtigt kan opdateres som \(j\) varierer, og ofte er opdateringen at lade \(\beta_j = 0\).
%Coordinate descent algoritmen for lasso giver ikke den fulde lasso løsningssti som LARS algoritmen, men kan bruges til at udregne lasso løsningerne for \(\Lambda\).







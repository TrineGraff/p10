\chapter{Lasso modellen og dens generaliseringer} \label{ch:shrinkage_metoder}
%Da den autoregressive model valgte en orden på 4, tilføjes som sagt fire variable med laggede værdier af arbejdsløshedsraten til de øvrige 122 variable.
Da den autoregressive model valgte en orden på 4 tilføjes som nævnt 4 variable med tidligere værdier af arbejdsløshedsraten til de øvrige 122 variable. 
Dette introducerer nogle uobserverede værdier, og derfor fjernes de første 4 observationer af hver tidsrække.
De tilføjede tidsrækker inkluderes i gruppe 2, hvor også arbejdsløshedsraten er inkluderet (se appendiks \ref{app:data}).
I dette kapitel betragter vi derfor 126 variable og en træningsmængde fra 1. maj 1960 til 1. december 2005 svarende til 548 observationer. 

Lasso problemet og dens generaliseringer kan løses med coordinate descent algoritmen og LARS algoritmen.
Vi vil anvende coordinate descent algoritmen til at løse lasso, ridge regression, elastisk net, group lasso og adaptive lasso med henholdsvis OLS og lasso vægte, mens LARS algoritmen anvendes til at løse lasso. 
Da LARS algoritmen er en variabeludvælgelsesmetode, betragtes den også som en model.
Selvom LARS algoritmen ikke er en af generaliseringerne af lasso, er den stadig inkluderet i dette kapitel, og derfor vil lasso og dens generaliseringer også referere til LARS algoritmen.  

\section{Coordinate descent}
Vi anvender funktionen \texttt{glmnet} fra R-pakken af samme navn til at estimerer lasso, elastik net, ridge og adaptive lasso's koefficienter i vores modeller. 
Funktionen genererer ud fra datasættet en følge på 100 $\lambda$-værdier og tilpasser en model til hver af disse ved maksimum likelihood estimation med algoritmen coordinate descent. 
Ud fra dette anvender vi så 10-fold-krydsvalidering og BIC til at vælge $\widehat{\lambda}$, som giver den bedste model. 
Det skal lige bemærkes, at for elastisk net har vi to turning parameter vi skal estimerer nemlig $\alpha$ og $\lambda$.  Så her har vi valgt 10 værdier af $\alpha$, hvor $\alpha \in [0,1]$. 
For group lasso har vi anvendt \texttt{gglasso} fra R-pakken også med samme navn til at estimerer group lasso. 
Denne funktion generer også en følge på 100 $\lambda$-værdier men anvender i stedet algoritmen black-wise descent.  Funktionen kræver også en gruppering af de forklarende variabler. 
Vi anvender grupperne som er forslået af Michael McCracken, som ses i appendiks \ref{app:app_data}. Derudover har kvadratroden af gruppens størrelse som penality faktor. 
For adaptive lasso med lasso vægte, anvender vi kun de forklarende variable, som lasso har udvalgt til at estimerer turning parameteren $\lambda$. 

\subsection{Krydsvalidering}
I denne sektion anvender vi funktionen \texttt{cv.glmnet} og \texttt{cv.gglasso} i pakkerne \texttt{glmnet} og \texttt{gglasso} . 
Tidligere nævnt har elastik net to turning parameter. 
Vi har derfor anvendt 10-fold krydsvalidering for 10 værdier af $\alpha$. 
Derfra har vi fundet $\lambda_{\min}$ og krydsvaliderings fejlen for hver værdi af $\alpha$.  
Den mindste krydsvaliderings fejl for $\lambda_{\min}$ er når $\alpha =1$, som er den samme model som lasso. 
Vi inkluderer derfor ikke elastik net.   

\imgfigh{cv_plot.pdf}{1}{10-fold krydsvaliderings fejl plottede som en function af $ \log(\lambda)$ for vores metoder. De stiplede linjer indikerer minimum fejl, samt fejlen med en standard afvigelse af minimum}{cv_plot}
Figur \ref{fig:cv_plot} illustrer den gennemsnitlige krydsvaliderings fejl for hver værdi af $\log \lambda$ for hver af vores metoder.

\begin{table}[ht]
\center
\begin{tabular}{lcccc | lccccc}
\toprule
   \multicolumn{5}{c}{Lasso} & \multicolumn{1}{c}{ }&  \multicolumn{5}{c}{Ridge regression}  \\ \midrule
 & \(\log \del{\lambda}\) & MSE & $p$ & Adj. R$^2$ &&& \(\log \del{\lambda}\) & MSE & $p$ & Adj. R$^2$\\
 $\lambda_{\min}$ &$-6.6361$& 0.0019 & 28 & 94.28\% &&  $\boldsymbol{\lambda_{\min}}$ &  $\mathbf{-4.3800}$ &   $\mathbf{0.0045} $&  $\mathbf{126}$ & $ \mathbf{86.56 \% }$ \\ 
 $\boldsymbol{\lambda}_{\textbf{1sd}}$ & $\mathbf{-5.7057}$ & $\mathbf{0.0020} $ & $\mathbf{14}$ &$\textbf{ 93.81} \boldsymbol{\%}$ && $\lambda_{ \text{1sd}}$& $-4.1939$ & 0.0047 & 126 &  85.70\%  \\ \bottomrule \toprule
\multicolumn{5}{c}{Group lasso}  &&  \multicolumn{5}{c}{Adap. lasso m. OLS vægte}  \\ \midrule
& \(\log \del{\lambda}\) & MSE &$ p $ &Adj. R$^2$ &&& \(\log \del{\lambda}\) & MSE & $p$ & Adj. R$^2$ \\
$\lambda_{\min}$& $-8.2644$ & 0.0022  & 126 & 93.23\% && $\boldsymbol{\lambda_{\min}}$  & $\mathbf{-2.0630}$ &$ \mathbf{0.0018}$ & $\mathbf{2}$ & $\textbf{94.27} \%$ \\
  $\boldsymbol{\lambda}_{\textbf{1sd}}$  & $\mathbf{-7.6365}$ &$ \mathbf{0.0023}$ & $\mathbf{119}$ &$ \textbf{92.60} \%$ &&  $\lambda_{1\text{sd}}$ & $-0.3884$ & 0.0019 & 2 &  93.93\%\\  \bottomrule 
  \toprule
  \multicolumn{5}{c}{Adap. lasso m. lasso vægte}  \\ \cmidrule{1-5}
& \(\log \del{\lambda}\) & MSE & $p$ & Adj. R$^2$\\
$\boldsymbol{\lambda_{\min}}$   &  $ \mathbf{-2.3674}$ & $ \mathbf{0.0018} $& $ \mathbf{2}$ &   $\textbf{94.28} \%$ \\
$\lambda_{1\text{sd}}$  & $-0.2276$ & 0.0019 & 2 & 93.92\%\\ \cmidrule{1-5}
 \end{tabular}
\caption{Logaritmen af $\lambda_{min}$ og $\lambda_{1\text{sd}}$, gennemsnitlig krydsvalideringsfejl, som er målt i MSE, antallet af paramete og adjusted R$^2$ for lasso og dens generaliseringer. De valgte tuning parameter for hver metode er markeret med tykt.} \label{tab:cv_tab}
\end{table}


For at få et bedre overblik viser tabel  \ref{tab:cv_tab} værdierne af $\log \lambda_{\min}$ og $\log \lambda_{1\text{sd}}$ , antallet af koefficienter og deres krydsvaliderings fejl.  
Vi ser for lasso,  at der sker en reducering af antallet af parameter når $\lambda_{1\text{sd}}$ anvendes i forhold til $\lambda_{\min}$, men MSE ikke er signifikant forskellig.  
Vi anvender derfor $\widehat{\lambda}_{1\text{sd}}$, som vores turning parameter. 
For ridge regression vil der ikke ske reducering af antallet af parameter, men som nævnt tidligere en reducering af værdierne af koefficienterne. 
Vi lader derfor $\widehat{\lambda}_{\min}$ være vores optimale turning parameter, da den har mindst krydsvaliderings fejl.
%
For group lasso kan vi se, at $\lambda_{1\text{sd}}$ ikke reducerer antallet af parameter meget,da den kun sætter 7 variabler til nul. 
Disse variabler stammer alle fra gruppe 5.
Det indikerer, at group lasso måske ikke er den bedste selektions model til vores data. 
Vi lader $\widehat{\lambda}_{1\text{sd}}$ være den optimale $\lambda$ for group lasso, da den stadig har færrest parameter. 

Adaptive lasso m. OLS vægte og adaptive lasso med lasso vægte vælger de færreste antal forklarende variabler. 
Den udvælger kun 2 for både  $\lambda_{\min}$ og $\lambda_{1\text{sd}}$, det er tale om variablerne \textit{CLF16OV: Civilian Labor Force} og \textit{CE16OV: Civilian Employment} begge fra gruppe to. 
Vi lader  $\widehat{\lambda}_{\min}$ være cores turning parameter for både adaptive lasso med OLS vægte og adaptive lasso med lasso vægte. 
Vi har markeret $\widehat{\lambda}$ for hver model med tykt i tabel  \ref{tab:cv_tab}. 

Figur \ref{fig:coef_kryds_coord} viser værdierne af de estimerede koefficienter for lasso og de to adaptive metoder, samt hvilken gruppe de forklarende variablerne tilhører. 
Vi kan se, at lasso hovedsageligt vælger variabler indenfor samme gruppe, som arbejdsløsheden. 
Derudover ses, det tydeligt at \texttt{CLF16OV} og \texttt{CE16OV} har de højest værdier, hvor resten af variablerne er meget tæt på nul. 
Figurerne \ref{fig:coef_ridge_kryds_coord} og \ref{fig:coef_gglasso_kryds_coord} viser det samme for hhv. ridge regression og group lasso.    

\imgfigh{coef_kryds_coord.pdf}{0.6}{Viser de estimerede koefficienters størresle for lasso, adaptive lasso med OLS vægte og adaptive lasso med lasso vægte. Farverne indikerer hvilken gruppe de forklarende variabler tilhører og y-aksen er variablerne udvalgt fra lasso.}{coef_kryds_coord}

For at undersøge tilfældigheden af de estimerede koefficienter anvender vi bootstrap. 
Figurerne \ref{fig:boxplot_lasso_coord_kryds}, \ref{fig:bootstrap_alasso} og \ref{bootstrap_gglasso} viser resultaterne af 1000 bootstrap relisationer for hhv. lasso, adaptive lasso med OLS vægte, adaptive lasso med lasso vægte og gglasso. 
Vi ser at for alle modeller, at variablerne \texttt{CLF16OV} og \texttt{CE16OV} altid bliver valgt til at være forskellige fra nul. 
For lasso ses der at de variable, som ikke er tilhørende af gruppe to ofte bliver valgt til at være nul. 
Derudover ses at variablerne  \texttt{CLF16OV},  \texttt{CE16OV} , \texttt{lag1}, \texttt{UEMPL15OV}, \texttt{UEMP5TO14} og \texttt{UEMPLT5} fra gruppe to ofte estimeret til at være forskellige fra nul, dvs at lasso ofte vælger disse variable. 
%Post-selection intervallerne vises på figur -- for disse 14 variable.
Af figur \ref{fig:fixedLassoInf} observeres at nulhypotesen afvises for \texttt{CLF16OV}, \texttt{CE16OV} samt \texttt{lag 1}.
\begin{table}[ht] 
\centering 
\begin{tabular}{llllllll}
%\multicolumn{4}{c}{Lasso} \\
\toprule
Prædiktor & Koefficient & Z-score & \(p\)-værdi & lowConfPt & UpConfPt & LowTailArea & UpTailArea \\
\midrule
3 & -0.002 & -1.372 & 0.649 & -0.009 & 0.025 & 0.050 & 0.050 \\
14 & -0.003 & -1.111 &  0.275 &   -0.011 &   0.006 & 0.050 & 0.049 \\
21 & 0.002 & 0.723 & 0.191 & -0.003 & 0.014 & 0.049 & 0.050 \\
22 & 0.243 & 36.619 & 0.000 & 0.232 & 0.259 & 0.048 & 0.049 \\
23 & -0.266 & -37.351 & 0.000 & -0.280 & -0.254 & 0.049 & 0.049 \\
25 & 0.001 & 0.243 & 0.401 & -0.005 & 0.008 & 0.049 & 0.049 \\
26 &  0.000 & -0.120 &  0.425 & -0.007 & 0.004 & 0.049 & 0.049 \\
27 & 0.004 & 1.590 & 0.057 & 0.000 & 0.009 &  0.049 & 0.050 \\
31 & 0.001 & 0.249 & 0.236 & -0.007 & 0.027  & 0.049 & 0.050 \\
34 & -0.002 & -0.880 & 0.578 & -0.009 & 0.016 & 0.050 & 0.000 \\
78 & -0.001 & -0.480 & 0.683 & -0.009 & 0.027 & 0.050 & 0.050 \\
80 & -0.003 & -1.131 &  0.218 & -0.025 & 0.007  & 0.050 & 0.050 \\
94 & 0.003 & 1.301 & 0.877 & -0.075 & 0.003 & 0.050 & 0.050 \\
123 & -0.009 & -4.070 & 0.003 & -0.013 & -0.004 & 0.050 & 0.049 \\
\bottomrule
\end{tabular}  
\caption{-}
\end{table} 


Tabel  \ref{tab:adj_r2_shrinkage_tab} viser værdierne af adjusted R$^2$, vi ser at modellerne med de højeste procent er de to adaptive funktioner, hvilket er også dem med færreste estimerede koefficienter. Ridge regression er den med den laveste adjusted R$^2$ men også den med flest koefficienter. 

Figurerne \ref{fig:resid_lasso_coord_kryds} - \ref{fig:resid_adap_ols_coord_kryds} viser en analyse af de standardiserede residualer. 
Vi ser, samme tendens for alle shrinkage metoderne. Histogrammet og QQ-plottet indikerer tungere haler end en normalfordeling og autokorrelation i første lag.
Dette bekræftes i tabel  \ref{tab:res_shrinkage_tab}, som viser skewness, kurtosis og $p$-værdierne fra JB-testen for de standardiserede residualer for alle shrinkage modeller. 
Vi ser, at alle modellerne har en negativ skewness og en kurtosis forskellige fra nul. 
Derudover afvises JB testens nul hypotese omkring normalitet for alle modellerne på nær group lasso, hvor vi kan ikke afvises normalitet af de standardiserede residualer. 
Vi ser også, at group lasso er den model, som har en kurtosis tætteste på nul, samt en meget lille skewness. 






\subsection{BIC}
I dette afsnit finder vi $\widehat{\lambda}$ med BIC. 
Vi bruger funktionen i appendiks \ref{sub:bic} til at finde $\widehat{\lambda}$. 

\begin{table}
\center
\begin{tabular}{llll} 
\toprule
& \multicolumn{1}{c}{${\widehat\lambda}$} & \multicolumn{1}{c}{BIC} & \multicolumn{1}{c}{p} \\ \midrule
Lasso & 0.0023 & -6.1424 & 14  \\
Ridge regression & 0.0112 & -4.2931 & 122 \\
Elastik net, $\alpha = 0.9$ & 0.0019 & -6.1202 &17 \\
Group lasso & $6.34 \cdot 10^{-5}$ & -5.0744 & 122 \\
Adaptive lasso med OLS vægte & 0.0599 & -6.3174 & 2 \\
Adaptive lasso med lasso vægte &  0.1058& -6.3174 & 2\\ \bottomrule
 \end{tabular}
\caption{Tabellen viser ${\widehat\lambda}$ fundet udfra BIC, samt BIC-værdien og antallet af parameter hver metode udvælger} \label{tab:bic_lambda}
\end{table}
Tabel \ref{tab:bic_lambda} viser $\widehat{\lambda}$ værdi, BIC værdien og antallet af parameter. 
Elastik net vælger $\alpha =1$ til at være den model med mindst BIC, hvilket er den samme model, som lasso. 
Derfor ser vi bort fra elastik net. 
Igen ser vi, at de to adaptive lasso modeller udvælger de færreste variabler, men igen er variablerne Civilian Labor Force og Civilian Employment fra samme gruppe, som responsvariablen. 
Derudover ser vi igen, at group lasso har mange forklarende variabler. 

------ 

Figurerne \ref{fig:resid_lasso_coord_bic} - \ref{fig:resid_adap_ols_coord_bic} viser en an analyse af de standardiserede residualer.
Vi ser igen lidt af den samme tendens for alle metoderne. 
Histogrammet og QQ-plottet indikerer tungere haler end en normalfordeling og autokorrelation i det første lag. 
Dog er ridge og groups lasso QQ-plot kun med få outliers. 
I tabel \ref{tab:res_shrinkage_bic_tab} ser vi også at vi ikke kan afvise LB testen omkring normalitet for group lasso. 


% \begin{table}
\small
\center
\begin{tabular}{lllc}
\toprule
\multicolumn{1}{c}{Lasso} & \multicolumn{1}{c}{Elastisk net} \\ \midrule
Durable Materials (1) &Durable Materials (1)   \\
Ratio of Help Wanted/No. Unemployed (2) & Nondurable Materials (1)  \\
Civilian Labor Force (2) &  Civilian Labor Force (2) \\
Civilian Employment (2)& Civilian Employment (2) \\
Civilians Unemployed - Less Than 5 Weeks (2) & Average Duration of Unemployment (Weeks) (2) \\
Civilians Unemployed for 5-14 Weeks (2) & Civilians Unemployed - Less Than 5 Weeks (2)  \\
Civilians Unemployed - 15 Weeks \& Over (2)& Civilians Unemployed for 5-14 Weeks (2) \\
Initial Claims (2)& Civilians Unemployed - 15 Weeks \& Over (2) \\
All Employees: Construction (2)& Initial Claims (2) \\
Housing Starts, West (3)&All Employees: Construction (2) \\
Real personal Consumption expenditures (4)& Housing Starts, West (3) \\
New Orders for Durable Goods (4) & Real personal Consumption expenditures (4) \\
5-Year Treasure Rate (6) & New Orders for Durable Goods (4) \\
U.S. / U.K. Foreign Exchange Rate (6)&Nonrevolving consumer credit to Personal income (5) \\
& 5-Year Treasure Rate (6)  \\
& U.S. / U.K. Foreign Exchange Rate (6) \\
& PPI: MatLA ns metal products (7) \\
\bottomrule 
\toprule
\multicolumn{1}{c}{Adaptive lasso m. OLS vægte} & \multicolumn{1}{c}{ Adaptive lasso m. lasso vægte}  \\ \midrule
Civilian Labor Force (2) & Civilian Labor Force (2) \\
Civilian Employment (2) & Civilian Employment (2) \\
 \bottomrule 
\end{tabular}
\caption{Tabellen indeholder de forklarende variable, som bliver udvalgt af lasso, elastisk net og adaptive lasso med OLS og lasso vægte. Tallene i parantes indikerer hvilken gruppe de forskellige variable tilhører} \label{tab:bic_ud}
\end{table}





\section{LARS}
I dette afsnit vil vi finde den optimale model for LARS og LARS algoritmen med lasso modifikation, som vi betegner lasso LARS. 
Hertil anvendes funktionen \texttt{lars} fra \Rlang-pakken af samme navn.
Mens \texttt{glmnet} fitter en model for 100 værdier af \(\lambda\), fitter \texttt{lars} en model for hvert trin.
For LARS algoritmen tilføjes en variabel i hvert trin, mens variable kan tilføjes og fjernes i hvert trin for LARS algoritmen med lasso modifikationen.
Da første trin svarer til, at alle koefficienter er lig 0, foretager LARS algoritmen 127 trin, mens LARS algoritmen med lasso modifikationen udfører 192 trin.

For at finde den optimale model anvendes igen 10-fold krydsvalidering og BIC til at estimere tuning parameteren, som for LARS algoritmen er fraktion af \(\ell_1\) norm, der er givet ved \(f = \frac{\abs{\tbeta}}{\max \abs{\tbeta}}\), hvor \(f \in \sbr{0,1}\).

Vi betragter TG testen, som udfører inferens i LARS modellen.
Vi lader $\boldsymbol{\eta} = s_k \del{\textbf{X}^+_{\mathcal{A}_k}}^T \mathbf{e}_k$, således at nulhypotesen svarer til, at teste om koefficienten af den sidst tilføjede variabel er lig 0.
Hertil anvendes funktionen \texttt{larInf} fra \Rlang-pakken \texttt{selectiveInference}, som udregner \(p\)-værdier og konfidensintervaller for LARS estimatet.
Derudover anvendes kovarians testen, som udfører inferens i lasso LARS modellen.
Hertil anvender vi funktionen \texttt{covTest} fra \Rlang-pakken af samme navn. 
\Rlang-koderne for dette afsnit er givet i \ref{subsubsec:inferens}.

\subsection{Krydsvalidering}
Funktionen \texttt{cv.lars} fra \Rlang-pakken \texttt{lars} udfører 10-fold krydsvalidering.
For LARS algoritmen betragtes en følge med 127 værdier af $f$, dvs at vi får en værdi af den gennemsnitlige krydsvalideringsfejl, når en variabel tilføjes.
For LARS algoritmen med lasso modifikation betragtes en følge med 100 værdier af $f$, og dermed får vi ikke en gennemsnitlig krydsvalideringsfejl for hver gang en variable tilføjes eller fjernes.

Figur \ref{fig:lars_kryds} illustrerer den gennemsnitlige krydsvalideringsfejl samt nedre og øvre standardafvigelse for hver værdier af $f$ for LARS og LARS algoritmen med lasso modifikation (lasso LARS).
Hvis $f = 0$ er der ingen variabler tilføjet og hvis $f=1$ er alle variabler tilføjet. 
De to lodrette stiplede linjer indikerer \(f_{\text{min}}\) og \(f_\text{1sd}\), hvor \(f_{\text{min}}\) er værdien af \(f\), som giver den mindste gennemsnitlige krydsvalideringsfejl og \(f_\text{1sd}\) er den mindste værdi af \(f\), således at fejlen er indenfor en standardafvigelse af minimum. 

\imgfigh{lars_kryds.pdf}{1}{10-fold krydsvalideringsfejl som funktion af fraktion af \(\ell_1\)-norm LARS og lasso LARS. 
De stiplede linjer indikerer \(f_\text{min}\) og \(f_\text{1sd}\).}{lars_kryds}

Tabel \ref{tab:lars_lasso_tab} giver fraktion af \(\ell_1\)-normen, gennemsnitlig krydsvalideringsfejl, antallet af parametre, justeret R$^2$ og log-likelihood for LARS og lasso LARS.
For begge metoder afviger krydsvalideringsfejlen først på 5. decimal, og derfor vælger vi modellerne med det færreste antal parametre. 
For lasso LARS (CV) udfører algoritmen 22 trin, hvor variablerne  \textcolor{chartreuse4}{CUMFNS}, \textcolor{blue3}{MANEMP} og \textcolor{orange}{GS1} tilføjes og fjernes igen. 
Variablen \textcolor{orange}{TB6MS} bliver tilføjet, fjernet og tilføjet igen.  

%
\begin{table}
\center
\begin{tabular}{cccc | cccccc}
\toprule
   \multicolumn{4}{c}{LARS} &  \multicolumn{4}{c}{LARS med lasso modifikation}  \\ \midrule
 & Steps & MSE & $p$ & & Fraction af \(\ell_1\)-norm & MSE & $p$ \\
 $s_{\min}$ &28 & 0.0019 &27  &	$f_{\min}$ & 0.2626 & 0.0019 & 21\\ 
 $\mathbf{s}_{\textbf{1sd}}$ & \textbf{20}& \textbf{0.0019} & \textbf{19} & $\mathbf{f}_{\textbf{1sd}}$& \textbf{0.2424} & \textbf{0.0019} & \textbf{13 } \\ \bottomrule 
 \end{tabular}
\caption{Antallet af steps eller fraction af \(\ell_1\)-norm, gennemsnitlige krydsvalideringsfejl og antallet af variable for $s_{\min}$, $s_{\text{1sd}}$, $f_{\min}$ og $f_{\text{1sd}}$.
Rækkerne markeret med tykt er de valgte tuning parametre.} \label{tab:lars_lasso_tab}
\end{table}

%
På figur \ref{fig:coef_lars_kryds} vises de 19 estimerede koefficienter for LARS (CV) og de 13 estimerede koefficienter for lasso LARS (CV). 
De største estimerede koefficienter er givet for variablerne \textcolor{blue3}{CE16OV} og \textcolor{blue3}{CLF16OV}, efterfulgt af \textcolor{blue3}{UEMPLT5} \textcolor{blue3}{UEMP5TO14} og \textcolor{blue3}{UEMP15OV}, mens de resterende estimerede koefficienter er meget tæt på nul.


\imgfigh{coef_lars_kryds.pdf}{1}{Estimerede koefficienter for LARS algoritmen uden og med lasso modifikationen (CV).
Farverne indikerer hvilken gruppe, variablerne tilhører.}{coef_lars_kryds}

Figur \ref{fig:lars_kryds_res} og \ref{fig:lars_lasso_kryds_res} viser en analyse af de standardiserede residualer.
Igen ses at fordelingen af de standardiserede residualer har tungere haler end normalfordelingen og autokorrelation i det første lag. 
Tabel \ref{tab:lars_kryds_res_tab} understøtter dette, hvor vi afviser normalitet og uafhængighed i lag 10. 

Figur \ref{fig:boxplot_lars_kryds} og \ref{fig:boxplot_lars_lasso_kryds} viser bootstrap resultaterne for variablerne udvalgt af LARS (CV) og lasso LARS (CV). 
For begge modeller ser vi, at variablerne med de størst estimerede koefficienter i figur \ref{fig:coef_lars_kryds} også er dem, som vælges oftest under bootstrap. 
For LARS (CV) ses, at variablerne \textcolor{blue3}{UEMPL15OV}, \textcolor{blue3}{UEMP5TO14}, \textcolor{blue3}{UEMPLT5}, \textcolor{blue3}{CE16OV} og \textcolor{blue3}{CLF16OV} vælges for alle 1000 bootstrap realisationer, men variablerne \textcolor{orange}{GS1}, \textcolor{blue3}{CLAIMx}, \textcolor{chartreuse4}{CUMFNS}, \textcolor{chartreuse4}{INDPRO} og \textcolor{red3}{DPCERA3M086SBEA} fravælges over 60\%.
For lasso LARS (CV) ser vi, at variablerne \textcolor{blue3}{USGOOD} og \textcolor{blue3}{PAYMENS} fravælges over 75\% af bootstrap realisationerne, mens variablerne \textcolor{blue3}{CE16OV}, \textcolor{blue3}{CLF16OV} og \textcolor{blue3}{lag1} ofte vælges. 

\subsubsection{Kovarians testen}
Tabel \ref{tab:covTest} viser teststørrelsen samt $p$-værdier af kovarians testen for de 13 variable, der bliver udvalgt af lasso LARS (CV). 
For 7 ud af 10 prædiktorer som tilhører gruppe 2 med undtagelse af \textcolor{blue3}{PAYEMS}, \textcolor{blue3}{lag 1} og \textcolor{blue3}{USCONS} afvises nulhypotesen, hvilket betyder, at disse prædiktorer er signifikante.
For de resterende prædiktorer kan nulhypotesen ikke afvises.

\begin{table}[h] 
\centering 
\scalebox{0.9}{
\begin{tabular}{llll}
\multicolumn{4}{l}{LARS algoritmen med lasso modifikation} \\
\toprule
Prædiktor & Cov test & \(p\)-værdi \\
\midrule
\textcolor{blue3}{HWIURATIO} & 864.5594 & 0.00 \\
\textcolor{blue3}{UEMP15OV} & 161.38 & 0.00 \\ 
\textcolor{blue3}{UEMPLT5} & 162.87 & 0.00 \\ 
\textcolor{blue3}{UEMP5TO14} & 121.91 & 0.00 \\
\textcolor{blue3}{CE16OV} & 14.48 & 0.00  \\ 
\textcolor{blue3}{PAYEMS} & 0.37 &  0.69 \\
\textcolor{blue3}{CLF16OV} & 217.42 & 0.00  \\
\textcolor{chartreuse4}{IPDMAT} & 0.06 & 0.94 \\
\textcolor{orange}{GS5} & 0.39 & 0.68  \\ 
\textcolor{blue3}{lag 1} & 0.89 & 0.41 \\ 
\textcolor{orange}{TB6MS} & 0.04 & 0.96  \\
\textcolor{blue3}{USCONS} & 0.01 & 0.99 \\ 
 \textcolor{red3}{DPCERA3M086SBEA} & 0.17 & 0.84 \\
\bottomrule
\end{tabular}
}
\caption{Kovarians testen for LARS algoritmen med lasso modifikation.
Tallene er afrundet til 2 decimaler.
Vi viser kun \(p\)-værdier for prædiktorer som medtages og bliver i modellen for \(\widehat{f}_{1\text{sd}}=0.2424\), dvs hvis en prædiktor medtages i et step og senere forlader modellen, vises denne prædiktor ikke.} \label{tab:covTest}
\end{table} 


\subsubsection{TG testen}
Resultaterne af TG testen for LARS (CV) er givet i tabel \ref{tab:larInf_kryds}.
Variablen  \textcolor{orange}{GS5} afviser som den eneste nulhypotesen, hvilket betyder, at kun denne prædiktor er signifikant.
Vi ser, at $\boldsymbol{\eta}^T \textbf{y}$ er meget tæt på dets trunkerede interval, hvilket resulterer i, at de fleste grænser af konfidensintervallerne er uendelig. 

\begin{table}[ht] 
\centering 
\scalebox{0.8}{
\begin{tabular}{lccccccc}
%\multicolumn{9}{l}{LARS algoritmen} \\
\toprule
Prædiktor&Koefficient  &\(p\)-værdi & Konfidensinterval & $\boldsymbol{\eta^Ty}$ & Z-score &   $\sbr{\mathcal{V}^-;\mathcal{V}^+}$   \\
\midrule
\textcolor{blue3}{HWIURATIO}  &$-0.0017$& 0.160    &  $\del{-\text{Inf}   ;  \text{ Inf} }$ & 0.002  & 0.694   &$\sbr{0.002;0.002} $    \\
 \textcolor{blue3}{UEMP15OV} &  0.0106& 0.923 &     $ \left( -\text{Inf}  ;  0.032\right] $  &    0.004&   1.606   &$\sbr{0.004; 0.005}$   \\
 \textcolor{blue3}{UEMPLT5} & 0.0122 & 0.064  & $ \left[-0.018  ;     \text{ Inf} \right) $ & 0.001   &0.149   & $\sbr{0.000 ;0.001}$   \\
\textcolor{blue3}{MANEMP}  & 0.0030 &0.273 &   $\left[-0.171 ;      \text{ Inf}\right)$  &   0.002 &  0.486  & $\sbr{0.002;0.003}$\\
 \textcolor{blue3}{UEMP5TO14} &0.0068  &0.077  &   $ \left( -\text{Inf}     ;  0.016\right] $&  0.001 & -0.242&      $\sbr{0.000 ;0.001}$ \\
\textcolor{blue3}{CE16OV}&$-0.2272$ & 0.130   &   $\left( -\text{Inf}     ;  0.532\right]  $&0.267 &$-37.446$&    $\sbr{0.267; 0.267}$     \\ 
\textcolor{blue3}{ PAYEMS }&$-0.0009$ & 0.563   &  $\del{-\text{Inf}   ;  \text{ Inf} }$  &   0.000 &  0.006  & $\sbr{0.000 ;0.000}$  \\
 \textcolor{blue3}{USGOOD}  &$-0.0034$&0.638   &   $\del{-\text{Inf}   ;  \text{ Inf} }$ & 0.003  &$-0.498$&    $\sbr{0.003 ;0.003}$\\
\textcolor{chartreuse4}{CUMFNS}  &0.0021& 0.478    & $\del{-\text{Inf}   ;  \text{ Inf} }$ &  0.002  & 0.404 &  $\sbr{0.002 ;0.002 }$  \\
 \textcolor{blue3}{CLF16OV} &0.2058& 0.179   &  $\del{-\text{Inf}   ;  \text{ Inf} }$ &  0.243  &36.643  &  $\sbr{0.243 ;0.243}$ \\  
\textcolor{chartreuse4}{ IPDMAT}& $-0.0040$ &0.874   & $\left[-0.125  ;     \text{ Inf} \right) $&0.006 &$ -1.626 $& $\sbr{0.006; 0.006}$ \\   
\textcolor{orange}{ TB6MS} &$-0.0025 $& 0.569 &     $\del{-\text{Inf}   ;  \text{ Inf} }$& 0.005  &$-0.715$ &   $\sbr{0.005; 0.006}$   \\ 
\textcolor{chartreuse4}{INDPRO}  &0.0003&0.328   &  $\del{-\text{Inf}   ;  \text{ Inf} }$ &  0.003 &  0.513   & $\sbr{0.003 ;0.003}$  \\
\textcolor{orange}{GS1} &0.0021 &0.473  &    $\del{-\text{Inf}   ;  \text{ Inf} }$ &   0.006&   0.577   &$\sbr{0.006 ;0.006}$ \\  
\textcolor{orange}{GS5} &$-0.0029$&0.037 &     $\left( -\text{Inf}   ;  -0.025\right]   $& 0.005 & $-1.146 $ & $\sbr{0.005 ;0.005 }$\\  
 \textcolor{blue3}{lag1} &$-0.0037$ & 0.910   & $\del{-\text{Inf}   ;  \text{ Inf} }$  & 0.009  &$-3.949$  &$\sbr{0.009; 0.009 }$ \\ 
 \textcolor{red3}{DPCERA3M086SBEA} &$-0.0004$& 0.233  &   $\del{-\text{Inf}   ;  \text{ Inf} }$ & 0.003 & $-1.436$&  $\sbr{0.003; 0.003}$ \\ 
\textcolor{orange}{ EXUSUKx} &0.0002  & 0.964   &   $\left( -\text{Inf}     ;-0.053 \right] $&  0.003   &1.383&  $\sbr{0.003; 0.003 }$   \\   
 \textcolor{blue3}{CLAIMSx} &0.0001& 0.226 &    $\del{-\text{Inf}   ;  \text{ Inf} }$&0.002 &  0.813  & $\sbr{0.002 ;0.002 }$   \\ 
\bottomrule
\end{tabular}  
}
\caption{\(p\)-værdier og konfidensintervaller for variablerne udvalgt af LARS algoritmen. Den estimeres standard afvigelse er \(0.043\), og resultaterne er for \(\widehat{f}_{1 \text{sd}} = 0.2542\) med \(\alpha = 0.1\).} \label{tab:larInf_kryds}
\end{table} 


Figur \ref{fig:resid_lars_tg_kryds} viser en analyse af de standardiserede residualer for LARS$_{TG}$ (CV). 
Heraf ses, at QQ-plottet indikerer tungere haler end normalfordelingen samt autokorrelation, hvilket bekræftes i tabel \ref{tab:lars_kryds_res_tab}. 



\subsection{BIC}
Tabel \ref{tab:bic_lars} giver $\widehat{f}$, antallet af parameter, BIC og adjusted R$^2$ for LARS uden og med lasso modifikation. 
Adjusted R$^2$ er højst for LARS med lasso modifikationen, men er også den der vælger færrest antal parameter. 
Den vælger kun 17, hvor LARS vælger 20 parameter. 
På figur \ref{fig:coef_plot_lars_bic} ser vi de estimerede koefficienter for LARS algoritmen uden og med lasso modifikation, og igen ser  vi at variablerne \textcolor{blue3}{CE16OV} og \textcolor{blue3}{CLF16OV} er dem med de største værdier, hvor de resterende er meget tæt på nul. 

\begin{table}
\center
\scalebox{0.8}{
\begin{tabular}{lccccc| lccccc} 
\toprule
\multicolumn{6}{c}{LARS (BIC)}  & \multicolumn{6}{c}{LARS med lasso modifikation (BIC)} \\ \midrule
& Værdi & BIC & $p$ & R$^2_{\text{adj}}$ & LogLik& & Værdi & BIC & $p$ & R$^2_{\text{adj}}$ & LogLik \\
$f_\text{BIC}$ & 0.2623 & $-6.0925$ & 20 &94.43  \% & 975.2909  &$f_\text{BIC}$ &  0.2604 &$-6.1627$& 17 &  94.46 \% & 974.9938 \\ \bottomrule
 \end{tabular}}
\caption{Værdien af $f_\text{BIC}$, antallet af parametre, BIC, justerede R$^2$  og log-likelihood for R$^2$ for LARS uden og med lasso modifikation.} \label{tab:bic_lars}
\end{table}

\imgfigh{coef_plot_lars_bic.pdf}{1}{Estimerede koefficienter for LARS algoritmen uden og med lasso modifikationen, hvor $\widehat{f}$ er fundet ud fra BIC. Farverne indikerer hvilken gruppe, variablerne tilhører. }{coef_plot_lars_bic}

Figurerne \ref{fig:lars_bic_resid} og  \ref{fig:lars_lasso_bic_resid} viser en analyse af de standardiserede residualer for LARS uden og med lasso modifikation. Vi ser, at histogrammet og QQ-plottet viser tungere haler end en normalfordeling og autokorrelation i første lag. Dette ses også i tabel \ref{tab:lars_kryds_res_tab}, hvor også JB testens nulhypotese om normalitet og LB testen om uafhængighed bliver afvist, når turning parameterene er valgt ud fra BIC. 


\subsubsection{Inferens - LARS uden modifikation}
På figur \ref{fig:boxplot_lars_bic} ses bootstrap resultaterne af variablerne udvalgt af LARS uden modifikation. Vi ser at variablerne  \textcolor{blue3}{UEMPL15OV}, \textcolor{blue3}{UEMPLT5}, \textcolor{blue3}{UEMP5TO14}, \textcolor{blue3}{CE16OV}, \textcolor{blue3}{CLF16OV} og \textcolor{blue3}{HWURATIO} bliver valgt i næsten alle bootstraps realisationerne.
 Variablerne \textcolor{red3}{DPCERA3M086SBEA},  \textcolor{chartreuse4}{INDPRO}, \textcolor{chartreuse4}{CUMFNS}, \textcolor{blue3}{CLAIMSx}, \textcolor{red3}{AMDMNOx}, \textcolor{orange}{GS1}, \textcolor{orange}{EXUSUKx}  bliver over 50 \% af bootstrap realisationerne estimeret til at være nul.  

Tabel \ref{tab:larInf_kryds} viser p-værdier og konfidensintervaller for LARS algoritmen. Heraf ses, at nulhypotesen accepteres for alle variable.

\begin{table}[ht] 
\centering 
\scalebox{0.8}{
\begin{tabular}{lcccccc}
%\multicolumn{9}{l}{LARS algoritmen} \\ 
\toprule
Prædiktor&Koefficient  &\(p\)-værdi & Konfidensinterval &   $\boldsymbol{\eta^Ty}$ & Z-score & $\sbr{\mathcal{V}^-;\mathcal{V}^+}$   \\ \midrule
\textcolor{blue3}{HWIURATIO}  &$-0.0010$&0.161    &   $\del{-\text{Inf}   ;  \text{ Inf} }$ &0.002  & 0.720 &  $\sbr{0.002;0.002}$    \\
 \textcolor{blue3}{UEMP15OV} &0.0094 &  0.920 &     $\left( -\text{Inf}  ;      0.034 \right] $& 0.004  & 1.596 & $\sbr{0.004 ;0.005}$  \\
\textcolor{blue3}{UEMPLT5} &0.0099 & 0.065  & $\left[-0.018    ;    \text{ Inf}  \right)$  & 0.001 &  0.148   &$\sbr{0.000;0.001}$    \\
\textcolor{blue3}{MANEMP} &0.0029& 0.766 &       $\left(  -\text{Inf}     ;  0.120\right] $ &0.003 &  0.561   &$\sbr{0.003; 0.003}$  \\
 \textcolor{blue3}{ UEMP5TO14} &0.0052&  0.093   &  $\left( - \text{ Inf}    ;  0.023\right] $& 0.001 & -0.261&   $\sbr{0.000; 0.001}$ \\
\textcolor{blue3}{ CE16OV}  &$-0.2352$ & 0.130  &    $\left( -\text{Inf}     ;   0.574\right] $& 0.267& $-37.412$   &$\sbr{0.266; 0.267}$ \\
 \textcolor{blue3}{PAYEMS} &$-0.0007$& 0.428  &    $\del{-\text{Inf}   ;  \text{ Inf} }$ &  0.000  & 0.012 &  $\sbr{0.000 ;  0.000}$   \\
 \textcolor{blue3}{USGOOD} &$-0.0034$& 0.721   &   $\del{-\text{Inf}   ;  \text{ Inf} } $& 0.004 & $-0.584$&    $\sbr{0.004 ;  0.004   }$   \\
\textcolor{chartreuse4}{CUMFNS}&0.0021 & 0.455     &  $\del{-\text{Inf}   ;  \text{ Inf} }$&  0.002   &0.390 & $\sbr{0.002 ;  0.002 }$   \\
 \textcolor{blue3}{ CLF16OV} & 0.2135 &  0.179    &  $\del{-\text{Inf}   ;  \text{ Inf} }$&   0.243 & 36.646  &$\sbr{0.243 ;  0.243}$    \\
\textcolor{chartreuse4}{ IPDMAT} &$-0.0044 $& 0.869  & $\left[ -0.130  ;      \text{ Inf}  \right)$& 0.006& $ -1.618$ &   $\sbr{0.006;   0.006}$     \\
 \textcolor{orange}{TB6MS}&$-0.0032$& 0.615     &  $\del{-\text{Inf}   ;  \text{ Inf} } $ & 0.006 & $-0.790 $  & $\sbr{0.006 ;  0.006}$   \\
 \textcolor{chartreuse4}{INDPRO} &0.0009& 0.494  &    $\del{-\text{Inf}   ;  \text{ Inf} }$ &   0.003   &0.591 &  $\sbr{0.003;   0.003}$    \\
 \textcolor{orange}{GS1}  &0.0031& 0.571 &      $\del{-\text{Inf}   ;  \text{ Inf} }$ &  0.007&   0.675&   $\sbr{0.007 ;  0.007}$     \\
 \textcolor{orange}{GS5}&$-0.0034$& 0.302&      $\del{-\text{Inf}   ;  \text{ Inf} }$&   0.006 &$ -1.240$ &   $\sbr{0.006;  0.006}$   \\
 \textcolor{blue3}{lag1} &$-0.0047$&  0.912   &  $\del{-\text{Inf}   ;  \text{ Inf} }$& 0.009& $ -3.914  $& $\sbr{0.009;   0.009 }$   \\
  \textcolor{red3}{DPCERA3M086SBEA}  &$-0.0008$& 0.225 &      $\del{-\text{Inf}   ;  \text{ Inf} }$ & 0.002&  $-1.331  $& $\sbr{0.002;   0.002 }$   \\
 \textcolor{orange}{EXUSUKx} &0.0007 & 0.964  &    $\left( -\text{Inf}   ;  -0.051\right] $& 0.003 &  1.357  &  $\sbr{0.003 ;  0.003}$   \\
 \textcolor{blue3}{CLAIMSx} &0.0003& 0.208   &   $\del{-\text{Inf}   ;  \text{ Inf} }$&  0.001  & 0.629  & $\sbr{ 0.001 ;  0.001 }$  \\
 \textcolor{red3}{AMDMNOx} &$-0.0004$ & 0.855     &  $\del{-\text{Inf}   ;  \text{ Inf} }$&  0.002&  $-0.904 $ &$\sbr{0.002 ;  0.002}$  \\ \bottomrule
\end{tabular}  
}
\caption{\(p\)-værdier og konfidensintervaller for variablerne udvalgt af LARS algoritmen. Den estimeres standard afvigelse er \(0.043\), og resultaterne er for \(\widehat{f}_{\text{BIC}} = 0.2623 \) med \(\alpha = 0.1\).} \label{tab:larInf_bic}
\end{table} 

\subsubsection{Inferens - LARS med lasso modifikation}
På figur \ref{fig:boxplot_lars_bic} ses bootstrap resultaterne af variabler udvalgt af LARS med lasso modifikation. 
Her ser vi at variablerne \textcolor{blue3}{UEMPL15OV}, \textcolor{blue3}{UEMPLT5}, \textcolor{blue3}{UEMP5TO14}, \textcolor{blue3}{CE16OV}, \textcolor{blue3}{CLF16OV} og \textcolor{blue3}{lag1} bliver valgt i næsten alle bootstrap realisationerne. Derudover ser vi at \textcolor{orange}{TB6MS} bliver estimeret til at være nul over 30 \% af bootstrap realisationerne. 

Tabel \ref{tab:covTest_bic} viser resultatet af  kovariansen testen. Vi ser at variablerne \textcolor{blue3}{UEMPL15OV}, \textcolor{blue3}{UEMPLT5}, \textcolor{blue3}{UEMP5TO14}, \textcolor{blue3}{CE16OV} og \textcolor{blue3}{CLF16OV} afviser null hypotesen. Disse variabler, er også de første variabler LARS algoritmen tilføjer.  



\begin{table}[ht] 
\centering 
\begin{tabular}{lccc}
%\multicolumn{3}{l}{LARS algoritmen med lasso modifikation} \\
\toprule
Prædiktor & Cov test & \(p\)-værdi \\
\midrule
\textcolor{blue3}{UEMP15OV}    &       161.3770  &0\\
\textcolor{blue3}{UEMPLT5}   &      163.0670 & 0\\
\textcolor{blue3}{UEMP5TO14}    &    122.3840&  0\\
\textcolor{blue3}{CE16OV}         &   14.7416 & 0\\
\textcolor{blue3}{CLF16OV}        &   221.9181 & 0\\
\textcolor{chartreuse4}{IPDMAT}         &    0.0668&  0.9354\\
\textcolor{orange}{GS5}&   0.3856 & 0.6803\\
\textcolor{blue3}{lag1}       &      0.8897 & 0.4115\\
\textcolor{orange}{TB6MS}  &    0.0419 & 0.9590\\
\textcolor{blue3}{USCONS} &    0.0132&  0.9869\\
\textcolor{red3}{ DPCERA3M086SBEA}          &  0.0254 & 0.9750\\
\textcolor{orange}{ EXUSUKx} &     0.2309 & 0.7939\\
\textcolor{blue3}{CLAIMSx} &      0.0082 &  0.9919\\
\textcolor{red3}{ AMDMNOx}  &     0.0464 & 0.9546\\
\textcolor{blue3}{lag4 }     &    0.2281&  0.7962\\
\textcolor{cadetblue2}{CPIMEDSL}  &   0.0719&  0.9307\\
\textcolor{blue3}{USTRADE}   &     0.0029 &  0.9971\\ \bottomrule
\end{tabular}
\caption{Kovarians testen for lasso LARS (BIC).
Vi viser kun \(p\)-værdier for prædiktorer som medtages og bliver i modellen, dvs hvis en prædiktor medtages i et trin og senere forlader modellen, vises denne prædiktor ikke.
\(p\)-værdier \(< 2.2 \cdot 10^{-16}\) sættes lig 0.} \label{tab:covTest_bic}
\end{table} 



\section{Oversigt over in-sample resultater} 
For at få et bedre oversigt over vores 14 modeller i in-sample betragtes tabel \ref{tab:topvariable}. Den viser de 9 meste valgte variabler samt beskrivelse. Vi ser at variablerne \textcolor{blue3}{CLF16OV} og \textcolor{blue3}{CE16OV} bliver valgt af alle 14 modeller, imens de resterende variabler ikke bliver valgt af adaptive lasso med OLS vægte og adaptive lasso med lasso vægte, hvor variablerne er bestemt ud fra  krydsvalidering og BIC. 

\begin{table}[ht] 
\centering 
\begin{tabular}{lll}
\toprule 
Antal & Variable & Beskrivelse \\ \midrule
14 &\textcolor{blue3}{CLF16OV} & Civil arbejdsstyrke \\
14 &\textcolor{blue3}{CE16OV} & Civilbeskæftigelse \\
11 & \textcolor{blue3}{lag 1} & Den tidligere periodes værdi af arbejdsløshedsrate \\
10 & \textcolor{chartreuse4}{IPDMAT} & Holdbart materiale  \\
10 & \textcolor{blue3}{UEMPLT5} & Civile arbejdsløse - mindre end 5 uger \\
10 & \textcolor{blue3}{UEMP5TO14}& Civile arbejdsløse i 5 - 14 uger \\
10 & \textcolor{blue3}{UEMP15OV} &  Civile arbejdsløse i 15 - 26 uger  \\
10 & \textcolor{orange}{TB6MS} & 6-måneders statsskat  \\
10 & \textcolor{orange}{GS5} & 5-årig statsobligationsrente \\
\bottomrule
\end{tabular}  
\caption{Antallet af gange variablerne er blevet inkluderet af lasso og dens generalisering samt beskrivelse af variablerne.} \label{tab:topvariable}
\end{table} 

I forhold til den justerede R$^2$ ser vi, at ridge regression løst med coordinate descent, hvor $\widehat{\lambda}$ er estimeret ud fra henholdsvis krydsvalidering og BIC har samme værdi, som også er den laveste værdi for alle 14 modeller. 
Vi ser også, at alle lasso modellerne har samme justerede R$^2$, selvom at de er løst ud fra to forskellige algoritmer, hvor variablerne er valgt ud fra krydsvalidering og BIC. 

De adaptive lasso modeller i coordinate descent har den laveste log-likelihood, men er også dem med færreste antal parameter. Derudover har vi at log-likehood er størst for ridge regression, men igen vælger den også alle 126 forklarende variable. 

Derudover er alle 14 modeller klart bedre i in-sample resultater i forhold til benchmarkmodellen. Det kunne godt tyde på, at vores benchmark model ikke vil præsterer særligt godt out-of-sample. 

\imgfigh{cf_interval.pdf}{1}{95\% konfidensinterval for de 9 meste valgte variabler for lasso, hvor variablerne er valgt ud fra krydsvalidering og lasso, hvor variablerne er er valgt ud fra BIC samt OLS estimeret af de 9 variabler. }{cf_interval}


Vi anvender, som nævnt TG testen til lasso løst med coordinate descent og LARS uden lasso modifikation. Vi ser for lasso, at variablerne \textcolor{blue3}{CLF16OV} og \textcolor{blue3}{CE16OV} er signifikante for både når variablerne er bestemt ud fra krydsvalidering og BIC. 




Generelt for TG-testen anvendt på LARS uden lasso modifikation ser vi, at variablens grænser i endepunkterne ofte indeholder \textit{inf} og/eller \textit{-inf}. 
Det skyldes at konfidensintervallet er udregnet numerisk, og bliver ustabil når punktet $\boldsymbol{\eta}^T \y$ er for tæt på dens trunkerede interval, altså $\mathcal{V^+}$ og $\mathcal{V^-}$. 



%This function should match (in terms of its output) that from the lars package, but returns additional information (namely, the polyhedral constraints) needed for the selective inference calculations.








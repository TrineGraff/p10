\section{Statistisk inferens}
I dette afsnit vil vi udføre inferens omkring


\subsection{Bootstrap}
På figur \ref{fig:bootstrap_lasso} ses bootstrap resultater af variable valgt af lasso udfra krydsvalidering.

Flere variable har koefficienter meget tæt på nul og 
variabler \texttt{CE16OV} og \texttt{CLF16OV} .
Dette kan bekræftes af figuren til højre, hvor vi ser, at \texttt{lag1}, \texttt{UEMP150V}, \texttt{UEMP5TO14}, \texttt{UEMPLT5}, \texttt{CE16OV} og \texttt{CLF16OV} lader til at være variablerne som vælges ofte af lasso.
%
\begin{figure}[h]
\includegraphics[scale=0.8, clip, trim=0 2.5cm 0 0]{fig/img/bootstrap_lasso.pdf}
\caption{Til venstre ses et boxplot af 1000 bootstrap realisationer af \(\widehat{\tbeta}^\text{lasso} \del{\widehat{\lambda}_\text{CV}}\), mens plottet til højre figur illustrerer andelen af bootstrap realisationer hvor parameter estimaterne er præcis lig nul.}
\label{fig:bootstrap_lasso}
\end{figure}
%
%\fbox{\includegraphics[scale=0.8, trim=0 2.5cm 0 0]{fig/img/bootstrap_lasso.pdf}}



%
%\imgfigh{bootstrap_alassoOLS.pdf}{1}{-.}{Bootstrap_alasso_OLS}
%
%\imgfigh{bootstrap_alassoRidge.pdf}{1}{-.}{bootstrap_alassoRidge}
%
%\imgfigh{bootstrap_grouplasso.pdf}{1}{-.}{bootstrap_grouplasso}
%
%\imgfigh{bootstrap_EN.pdf}{1}{-.}{bootstrap_EN}




\newpage
\subsection{Kovarians test}
Som nævnt udfører LARS med lasso modifikationen 192 steps, hvori variablerne tilføjes og nogle fjernes igen.

Kovarians testen gentager LARS med lasso modifikation og tester nulhypotesen om 

Tildeler \(p\)-værdier til hver variablen som variablen tilføjes til modellen.


Af tabel \ref{tab:covTest} ses, at for prædiktor \texttt{UEMP15OV} (27), \texttt{UEMPLT5} (25), \texttt{CE16OV} (23) og \texttt{CLF16OV} (22) afvises nulhypotesen.
\begin{small}
\begin{table}[ht] 
\centering 
\begin{tabular}{lllllllll}
%\multicolumn{9}{c}{Lasso} \\
\toprule
Step & Prædiktor & Cov test & \(p\)-værdi && Step & Prædiktor & Cov test & \(p\)-værdi \\
\midrule
  1 & 27 & 161.38 & 0.00 & & 46 & 5 & 0.01 & 0.99 \\ 
  2 & 25 & 162.87 & 0.00 & & 47 & 46 & 0.06 & 0.94 \\ 
  3 & 23 & 14.48 & 0.00  & & 48 & 56 & 0.03 & 0.97 \\ 
  4 & 22 & 217.42 & 0.00 & & 49 & 114 & 0.01 & 0.99 \\  
   5 & 80 & 0.39 & 0.68 & & 50 & 67 & 0.00 & 1.00 \\ 
  6 & 123 & 0.89 & 0.41 & & 51 & 42 & 0.02 & 0.98 \\ 
  7 & 78 & 0.04 & 0.96 & & 52 & 87 & 0.01 & 0.99 \\
  8 & 34 & 0.01 & 0.99 & & 53 & 121 & 0.01 & 0.99 \\ 
  9 & 3 & 0.17 & 0.84 & & 54 & 48 & 0.00 & 1.00 \\
  10 & 94 & 0.24 & 0.79 & & 55 & 91 & 0.00 & 1.00 \\ 
  11 & 30 & 0.01 & 0.99 & & 56 & 107 & 0.00 & 1.00 \\  
  12 & 57 & 0.05 & 0.95 & & 57 & 4 & 0.01 & 0.99 \\ 
  13 & 126 & 0.25 & 0.78 & & 58 & 85 & 0.00 & 1.00 \\ 
  14 & 105 & 0.06 & 0.94  & & 59 & 81 & 0.04 & 0.96 \\ 
  15 & 41 & 0.01 & 0.99 & & 60 & 110 & 0.00 & 1.00 \\ 
  16 & 24 & 0.00 & 1.00 & & 61 & 21 & 0.01 & 0.99 \\  
  17 & 40 & 0.01 & 0.99 & & 62 & 99 & 0.00 & 1.00 \\ 
  18 & 101 & 0.01 & 0.99 & & 63 & 111 & 0.01 & 0.99 \\ 
  19 & 15 & 0.37 & 0.69 & & 64 & 77 & 0.00 & 1.00 \\ 
  20 & 33 & 0.01 & 0.99 & & 65 & 62 & 0.01 & 0.99 \\ 
  21 & 51 & 0.00 & 1.00 & & 66 & 116 & 0.00 & 1.00 \\ 
  22 & 93 & 0.04 & 0.96 & & 67 & 117 & 0.15 & 0.86 \\  
  23 & 69 & 0.11 & 0.90 & & 68 & 19 & 0.04 & 0.96 \\
  24 & 98 & 0.02 & 0.98 & & 69 & 92 & 0.00 & 1.00 \\  
  25 & 58 & 0.03 & 0.97 & & 70 & 45 & 0.00 & 1.00 \\  
  26 & 17 & 0.00 & 1.00 & & 71 & 102 & 0.02 & 0.98 \\ 
  27 & 124 & 0.00 & 1.00 & & 72 & 65 & 0.01 & 0.99 \\ 
  28 & 37 & 0.01 & 0.99 & & 73 & 115 & 0.00 & 1.00 \\ 
  29 & 2 & 0.03 & 0.97 & & 74 & 104 & 0.00 & 1.00 \\ 
  30 & 103 & 0.03 & 0.97 & & 75 & 26 & 0.00 & 1.00 \\ 
  31 & 100 & 0.01 & 0.99 & & 76 & 31 & 0.00 & 1.00 \\ 
  32 & 119 & 0.01 & 0.99 & & 77 & 6 & 0.00 & 1.00 \\ 
  33 & 73 & 0.00 & 1.00 & & 78 & 36 & 0.02 & 0.98 \\ 
  34 & 39 & 0.22 & 0.80 & & 79 & 83 & 0.00 & 1.00 \\  
  35 & 118 & 0.00 & 1.00 & & 80 & 59 & 0.00 & 1.00 \\ 
  36 & 113 & 0.01 & 0.99 & & 81 & 120 & 0.00 & 1.00 \\
  37 & 72 & 0.02 & 0.98 & & 82 & 96 & 0.02 & 0.98 \\ 
  38 & 68 & 0.01 & 0.99 & & 83 & 79 & 0.02 & 0.98 \\
  39 & 10 & 0.00 & 1.00 & & 84 & 63 & 0.00 & 1.00 \\ 
  40 & 29 & 0.00 & 1.00 & & 85 & 49 & 0.03 & 0.97 \\ 
  41 & 84 & 0.01 & 0.99 & & 86 & 43 & 0.00 & 1.00 \\ 
  42 & 60 & 0.00 & 1.00 & & 87 & 108 & 0.00 & 1.00 \\ 
  43 & 54 & 0.01 & 0.99 & & 88 & 97 & 0.00 & 1.00 \\ 
  44 & 125 & 0.01 & 0.99 & & 89 & 53 & 0.00 & 1.00 \\ 
  45 & 7 & 0.00 & 1.00 & & 90 & 74 & 0.00 & 1.00 \\ 
\bottomrule
\end{tabular}
\caption{lasso \(p\)-værdier.
Tallene er afrundet til 2 decimaler.
Vi viser kun \(p\)-værdier for steps for hvilket en prædiktor medtages og bliver i modellen resten af stien, dvs hvis en prædiktor medtages i et step og senere forlader, vises denne prædiktor ikke.} \label{tab:covTest}
\end{table} 
\end{small}



\newpage
\subsection{Teste baseret på polyhedral lemmaet}

Da antallet af parametre er mindre end antallet af observationer i træningsmængden, kan \(\sigma^2\) estimeres udfra SSR fra den fulde model med alle prædiktorer.
Vi finder, at \(\widehat{\sigma} \approx 0.0433\).

Tuning parameteren \(\lambda\) vælges udfra krydsvalidering, som vi fandt \(\lambda \approx 0.0033\), hvorfra lasso valgte 14 variable.




Post-selection intervallerne vises på figur -- for disse 14 variable.


\texttt{fixedLassoInf} udregner \(p\)-værdier og konfidens intervaller for lasso estimatet for en fast værdi af tuning parameteren \(\lambda\).
\texttt{fixedLassoInf} anvender ``standard'' lasso
\begin{align*}
\frac{1}{2} \Vert \y - \X \tbeta \Vert_2^2 + \lambda \Vert \tbeta \Vert_1.
\end{align*}
\texttt{glmnet} multiplicerer først led med faktoren \(\frac{1}{n}\).
Efter vi har kørt \texttt{glmnet} og fundet betaen som svarer til lambda værdien, da skal vi \texttt{beta = coef(obj, s=lambda/n)}, hvor \texttt{obj} er objektet som er returneret af \texttt{glmnet}.


Af figur \ref{fig:fixedLassoInf} observeres at nulhypotesen afvises for \texttt{CLF16OV}, \texttt{CE16OV} samt \texttt{lag 1}.

\begin{table}[ht] 
\centering 
\begin{tabular}{llllllll}
%\multicolumn{4}{c}{Lasso} \\
\toprule
Prædiktor & Koefficient & Z-score & \(p\)-værdi & lowConfPt & UpConfPt & LowTailArea & UpTailArea \\
\midrule
3 & -0.002 & -1.372 & 0.649 & -0.009 & 0.025 & 0.050 & 0.050 \\
14 & -0.003 & -1.111 &  0.275 &   -0.011 &   0.006 & 0.050 & 0.049 \\
21 & 0.002 & 0.723 & 0.191 & -0.003 & 0.014 & 0.049 & 0.050 \\
22 & 0.243 & 36.619 & 0.000 & 0.232 & 0.259 & 0.048 & 0.049 \\
23 & -0.266 & -37.351 & 0.000 & -0.280 & -0.254 & 0.049 & 0.049 \\
25 & 0.001 & 0.243 & 0.401 & -0.005 & 0.008 & 0.049 & 0.049 \\
26 &  0.000 & -0.120 &  0.425 & -0.007 & 0.004 & 0.049 & 0.049 \\
27 & 0.004 & 1.590 & 0.057 & 0.000 & 0.009 &  0.049 & 0.050 \\
31 & 0.001 & 0.249 & 0.236 & -0.007 & 0.027  & 0.049 & 0.050 \\
34 & -0.002 & -0.880 & 0.578 & -0.009 & 0.016 & 0.050 & 0.000 \\
78 & -0.001 & -0.480 & 0.683 & -0.009 & 0.027 & 0.050 & 0.050 \\
80 & -0.003 & -1.131 &  0.218 & -0.025 & 0.007  & 0.050 & 0.050 \\
94 & 0.003 & 1.301 & 0.877 & -0.075 & 0.003 & 0.050 & 0.050 \\
123 & -0.009 & -4.070 & 0.003 & -0.013 & -0.004 & 0.050 & 0.049 \\
\bottomrule
\end{tabular}  
\caption{-}
\end{table} 

\section{Statistisk inferens}
I dette afsnit vil vi udføre inferens omkring


\subsection{Bootstrap}
På figur \ref{fig:bootstrap_lassoEN}-\ref{fig:bootstrap_glasso} ses bootstrap resultater af variablene for hver model, som vælges udfra krydsvalidering.

Flere variable har koefficienter meget tæt på nul og 
variabler \texttt{CE16OV} og \texttt{CLF16OV} .
Dette kan bekræftes af figuren til højre, hvor vi ser, at \texttt{lag1}, \texttt{UEMP150V}, \texttt{UEMP5TO14}, \texttt{UEMPLT5}, \texttt{CE16OV} og \texttt{CLF16OV} lader til at være variablerne som vælges ofte af lasso.
%
\imgfigh{bootstrap_lassoEN}{1}{Til venstre ses et boxplot af 1000 bootstrap realisationer af \(\widehat{\tbeta}^\text{lasso} \del{\widehat{\lambda}_\text{CV}}\), mens plottet til højre figur illustrerer andelen af bootstrap realisationer hvor parameter estimaterne er præcis lig nul..}{Bootstrap_lassoEN}
%
\imgfigh{bootstrap_alasso.pdf}{1}{-.}{bootstrap_alasso}
%
\imgfigh{bootstrap_gglasso.pdf}{0.73}{-.}{bootstrap_gglasso}



\newpage
\subsection{Kovarians test}
Som nævnt udfører LARS algoritmen med lasso modifikationen 192 steps, hvori variablerne tilføjes og nogle fjernes igen.
For \(\lambda_\text{min} = 36\) findes 21 prædiktorer, hvorpå kovarians testen udføres.
Tabel \ref{tab:covTest} viser resultatet af dette.
For prædiktorerne valgt i step 1-5 afvises nulhypotesen, hvilket betyder, at ...?
%
\begin{table}[h] 
\centering 
\scalebox{0.9}{
\begin{tabular}{llll}
\multicolumn{4}{l}{LARS algoritmen med lasso modifikation} \\
\toprule
Prædiktor & Cov test & \(p\)-værdi \\
\midrule
\textcolor{blue3}{HWIURATIO} & 864.5594 & 0.00 \\
\textcolor{blue3}{UEMP15OV} & 161.38 & 0.00 \\ 
\textcolor{blue3}{UEMPLT5} & 162.87 & 0.00 \\ 
\textcolor{blue3}{UEMP5TO14} & 121.91 & 0.00 \\
\textcolor{blue3}{CE16OV} & 14.48 & 0.00  \\ 
\textcolor{blue3}{PAYEMS} & 0.37 &  0.69 \\
\textcolor{blue3}{CLF16OV} & 217.42 & 0.00  \\
\textcolor{chartreuse4}{IPDMAT} & 0.06 & 0.94 \\
\textcolor{orange}{GS5} & 0.39 & 0.68  \\ 
\textcolor{blue3}{lag 1} & 0.89 & 0.41 \\ 
\textcolor{orange}{TB6MS} & 0.04 & 0.96  \\
\textcolor{blue3}{USCONS} & 0.01 & 0.99 \\ 
 \textcolor{red3}{DPCERA3M086SBEA} & 0.17 & 0.84 \\
\bottomrule
\end{tabular}
}
\caption{Kovarians testen for LARS algoritmen med lasso modifikation.
Tallene er afrundet til 2 decimaler.
Vi viser kun \(p\)-værdier for prædiktorer som medtages og bliver i modellen for \(\widehat{f}_{1\text{sd}}=0.2424\), dvs hvis en prædiktor medtages i et step og senere forlader modellen, vises denne prædiktor ikke.} \label{tab:covTest}
\end{table} 

%


\newpage
\subsection{Teste baseret på polyhedral lemmaet}
Da antallet af parametre er mindre end antallet af observationer i træningsmængden, kan \(\sigma^2\) estimeres udfra SSR fra den fulde model med alle prædiktorer.
Vi finder, at \(\widehat{\sigma} \approx 0.0433\).

Tuning parameteren \(\lambda\) vælges udfra krydsvalidering, hvor vi fandt, at \(\lambda \approx 0.0033\), hvorfra lasso valgte 14 variable.
Figur \ref{fig:fixedLassoInf} viser inferens af disse variable.


Af figur \ref{fig:fixedLassoInf} observeres at nulhypotesen afvises for \texttt{CLF16OV}, \texttt{CE16OV} samt \texttt{lag 1}.
%
\begin{table}[ht] 
\centering 
\begin{tabular}{llllllll}
%\multicolumn{4}{c}{Lasso} \\
\toprule
Prædiktor & Koefficient & Z-score & \(p\)-værdi & lowConfPt & UpConfPt & LowTailArea & UpTailArea \\
\midrule
3 & -0.002 & -1.372 & 0.649 & -0.009 & 0.025 & 0.050 & 0.050 \\
14 & -0.003 & -1.111 &  0.275 &   -0.011 &   0.006 & 0.050 & 0.049 \\
21 & 0.002 & 0.723 & 0.191 & -0.003 & 0.014 & 0.049 & 0.050 \\
22 & 0.243 & 36.619 & 0.000 & 0.232 & 0.259 & 0.048 & 0.049 \\
23 & -0.266 & -37.351 & 0.000 & -0.280 & -0.254 & 0.049 & 0.049 \\
25 & 0.001 & 0.243 & 0.401 & -0.005 & 0.008 & 0.049 & 0.049 \\
26 &  0.000 & -0.120 &  0.425 & -0.007 & 0.004 & 0.049 & 0.049 \\
27 & 0.004 & 1.590 & 0.057 & 0.000 & 0.009 &  0.049 & 0.050 \\
31 & 0.001 & 0.249 & 0.236 & -0.007 & 0.027  & 0.049 & 0.050 \\
34 & -0.002 & -0.880 & 0.578 & -0.009 & 0.016 & 0.050 & 0.000 \\
78 & -0.001 & -0.480 & 0.683 & -0.009 & 0.027 & 0.050 & 0.050 \\
80 & -0.003 & -1.131 &  0.218 & -0.025 & 0.007  & 0.050 & 0.050 \\
94 & 0.003 & 1.301 & 0.877 & -0.075 & 0.003 & 0.050 & 0.050 \\
123 & -0.009 & -4.070 & 0.003 & -0.013 & -0.004 & 0.050 & 0.049 \\
\bottomrule
\end{tabular}  
\caption{-}
\end{table} 
%


Post-selection intervallerne vises på figur -- for disse 14 variable.


\texttt{fixedLassoInf} udregner \(p\)-værdier og konfidens intervaller for lasso estimatet for en fast værdi af tuning parameteren \(\lambda\).
\texttt{fixedLassoInf} anvender ``standard'' lasso
\begin{align*}
\frac{1}{2} \Vert \y - \X \tbeta \Vert_2^2 + \lambda \Vert \tbeta \Vert_1.
\end{align*}
\texttt{glmnet} multiplicerer først led med faktoren \(\frac{1}{n}\).
Efter vi har kørt \texttt{glmnet} og fundet betaen som svarer til lambda værdien, da skal vi \texttt{beta = coef(obj, s=lambda/n)}, hvor \texttt{obj} er objektet som er returneret af \texttt{glmnet}.



Lars algoritmen udfører 126 steps, hvor én variabel tilføjes i hvert step.
%
\begin{table}[ht] 
\centering 
\scalebox{0.8}{
\begin{tabular}{lccccccc}
%\multicolumn{9}{l}{LARS algoritmen} \\
\toprule
Prædiktor&Koefficient  &\(p\)-værdi & Konfidensinterval & $\boldsymbol{\eta^Ty}$ & Z-score &   $\sbr{\mathcal{V}^-;\mathcal{V}^+}$   \\
\midrule
\textcolor{blue3}{HWIURATIO}  &$-0.0017$& 0.160    &  $\del{-\text{Inf}   ;  \text{ Inf} }$ & 0.002  & 0.694   &$\sbr{0.002;0.002} $    \\
 \textcolor{blue3}{UEMP15OV} &  0.0106& 0.923 &     $ \left( -\text{Inf}  ;  0.032\right] $  &    0.004&   1.606   &$\sbr{0.004; 0.005}$   \\
 \textcolor{blue3}{UEMPLT5} & 0.0122 & 0.064  & $ \left[-0.018  ;     \text{ Inf} \right) $ & 0.001   &0.149   & $\sbr{0.000 ;0.001}$   \\
\textcolor{blue3}{MANEMP}  & 0.0030 &0.273 &   $\left[-0.171 ;      \text{ Inf}\right)$  &   0.002 &  0.486  & $\sbr{0.002;0.003}$\\
 \textcolor{blue3}{UEMP5TO14} &0.0068  &0.077  &   $ \left( -\text{Inf}     ;  0.016\right] $&  0.001 & -0.242&      $\sbr{0.000 ;0.001}$ \\
\textcolor{blue3}{CE16OV}&$-0.2272$ & 0.130   &   $\left( -\text{Inf}     ;  0.532\right]  $&0.267 &$-37.446$&    $\sbr{0.267; 0.267}$     \\ 
\textcolor{blue3}{ PAYEMS }&$-0.0009$ & 0.563   &  $\del{-\text{Inf}   ;  \text{ Inf} }$  &   0.000 &  0.006  & $\sbr{0.000 ;0.000}$  \\
 \textcolor{blue3}{USGOOD}  &$-0.0034$&0.638   &   $\del{-\text{Inf}   ;  \text{ Inf} }$ & 0.003  &$-0.498$&    $\sbr{0.003 ;0.003}$\\
\textcolor{chartreuse4}{CUMFNS}  &0.0021& 0.478    & $\del{-\text{Inf}   ;  \text{ Inf} }$ &  0.002  & 0.404 &  $\sbr{0.002 ;0.002 }$  \\
 \textcolor{blue3}{CLF16OV} &0.2058& 0.179   &  $\del{-\text{Inf}   ;  \text{ Inf} }$ &  0.243  &36.643  &  $\sbr{0.243 ;0.243}$ \\  
\textcolor{chartreuse4}{ IPDMAT}& $-0.0040$ &0.874   & $\left[-0.125  ;     \text{ Inf} \right) $&0.006 &$ -1.626 $& $\sbr{0.006; 0.006}$ \\   
\textcolor{orange}{ TB6MS} &$-0.0025 $& 0.569 &     $\del{-\text{Inf}   ;  \text{ Inf} }$& 0.005  &$-0.715$ &   $\sbr{0.005; 0.006}$   \\ 
\textcolor{chartreuse4}{INDPRO}  &0.0003&0.328   &  $\del{-\text{Inf}   ;  \text{ Inf} }$ &  0.003 &  0.513   & $\sbr{0.003 ;0.003}$  \\
\textcolor{orange}{GS1} &0.0021 &0.473  &    $\del{-\text{Inf}   ;  \text{ Inf} }$ &   0.006&   0.577   &$\sbr{0.006 ;0.006}$ \\  
\textcolor{orange}{GS5} &$-0.0029$&0.037 &     $\left( -\text{Inf}   ;  -0.025\right]   $& 0.005 & $-1.146 $ & $\sbr{0.005 ;0.005 }$\\  
 \textcolor{blue3}{lag1} &$-0.0037$ & 0.910   & $\del{-\text{Inf}   ;  \text{ Inf} }$  & 0.009  &$-3.949$  &$\sbr{0.009; 0.009 }$ \\ 
 \textcolor{red3}{DPCERA3M086SBEA} &$-0.0004$& 0.233  &   $\del{-\text{Inf}   ;  \text{ Inf} }$ & 0.003 & $-1.436$&  $\sbr{0.003; 0.003}$ \\ 
\textcolor{orange}{ EXUSUKx} &0.0002  & 0.964   &   $\left( -\text{Inf}     ;-0.053 \right] $&  0.003   &1.383&  $\sbr{0.003; 0.003 }$   \\   
 \textcolor{blue3}{CLAIMSx} &0.0001& 0.226 &    $\del{-\text{Inf}   ;  \text{ Inf} }$&0.002 &  0.813  & $\sbr{0.002 ;0.002 }$   \\ 
\bottomrule
\end{tabular}  
}
\caption{\(p\)-værdier og konfidensintervaller for variablerne udvalgt af LARS algoritmen. Den estimeres standard afvigelse er \(0.043\), og resultaterne er for \(\widehat{f}_{1 \text{sd}} = 0.2542\) med \(\alpha = 0.1\).} \label{tab:larInf_kryds}
\end{table} 

%
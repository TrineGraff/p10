\section{Statistisk inferens}
I dette afsnit vil vi udføre inferens omkring


\subsection{Bootstrap}
På figur \ref{fig:bootstrap_lassoEN}-\ref{fig:bootstrap_glasso} ses bootstrap resultater af variablene for hver model, som vælges udfra krydsvalidering.

Flere variable har koefficienter meget tæt på nul og 
variabler \texttt{CE16OV} og \texttt{CLF16OV} .
Dette kan bekræftes af figuren til højre, hvor vi ser, at \texttt{lag1}, \texttt{UEMP150V}, \texttt{UEMP5TO14}, \texttt{UEMPLT5}, \texttt{CE16OV} og \texttt{CLF16OV} lader til at være variablerne som vælges ofte af lasso.
%
\imgfigh{bootstrap_lassoEN}{1}{Til venstre ses et boxplot af 1000 bootstrap realisationer af \(\widehat{\tbeta}^\text{lasso} \del{\widehat{\lambda}_\text{CV}}\), mens plottet til højre figur illustrerer andelen af bootstrap realisationer hvor parameter estimaterne er præcis lig nul..}{Bootstrap_lassoEN}
%
\imgfigh{bootstrap_alasso.pdf}{1}{-.}{bootstrap_alasso}
%
\imgfigh{bootstrap_gglasso.pdf}{0.73}{-.}{bootstrap_gglasso}



\newpage
\subsection{Kovarians test}
Som nævnt udfører LARS algoritmen med lasso modifikationen 192 steps, hvori variablerne tilføjes og nogle fjernes igen.
For \(\lambda_\text{min} = 36\) findes 21 prædiktorer, hvorpå kovarians testen udføres.
Tabel \ref{tab:covTest} viser resultatet af dette.
For prædiktorerne valgt i step 1-5 afvises nulhypotesen, hvilket betyder, at ...?
%
\begin{table}[ht] 
\centering 
\begin{tabular}{lccc}
%\multicolumn{3}{l}{LARS algoritmen med lasso modifikation} \\
\toprule
Prædiktor & Cov test & \(p\)-værdi \\
\midrule
 \textcolor{blue3}{HWIURATIO}  &   864.6317 & 0 \\
 \textcolor{blue3}{UEMP15OV}  &   161.3770&  0 \\
 \textcolor{blue3}{UEMPLT5} &  163.0670 & 0 \\
 \textcolor{blue3}{UEMP5TO14}  &   122.3840  &0 \\
 \textcolor{blue3}{CE16OV} & 14.7416  &0 \\
 \textcolor{blue3}{PAYEMS} &  0.3356  &0.7151 \\
  \textcolor{blue3}{USGOOD}  &   5.0872 & 0.0066 \\
 \textcolor{blue3}{CLF16OV}    &   221.9181 & 0 \\
\textcolor{chartreuse4}{IPDMAT}       &    0.0668&  0.9354 \\
\textcolor{orange}{GS5}   &       0.3856 &    0.6803 \\
 \textcolor{blue3}{ lag1 }  &      0.8897 &    0.4115 \\
\textcolor{orange}{ TB6MS }&       0.0419   &  0.9590 \\
 \textcolor{blue3}{USCONS }&   0.0132   &  0.9869 \\ 
\bottomrule
\end{tabular}

\caption{Kovarians testen for LARS algoritmen med lasso modifikation (CV).
Vi viser kun \(p\)-værdier for prædiktorer som medtages og bliver i modellen for \(f_{1\text{sd}}=0.2424\), dvs hvis en prædiktor medtages i et step og senere forlader modellen, vises denne prædiktor ikke.} \label{tab:covTest}
\end{table} 

%


\newpage
\subsection{Teste baseret på polyhedral lemmaet}
Da antallet af parametre er mindre end antallet af observationer i træningsmængden, kan \(\sigma^2\) estimeres udfra SSR fra den fulde model med alle prædiktorer.
Vi finder, at \(\widehat{\sigma} \approx 0.0433\).

Tuning parameteren \(\lambda\) vælges udfra krydsvalidering, hvor vi fandt, at \(\lambda \approx 0.0033\), hvorfra lasso valgte 14 variable.
Figur \ref{fig:fixedLassoInf} viser inferens af disse variable.


Af figur \ref{fig:fixedLassoInf} observeres at nulhypotesen afvises for \texttt{CLF16OV}, \texttt{CE16OV} samt \texttt{lag 1}.
%
\begin{table}[h] 
\centering 
\scalebox{0.8}{
\begin{tabular}{llllllll}
\toprule
Prædiktor & Koefficient & Z-score & \(p\)-værdi & lowConfPt & UpConfPt & LowTailArea & UpTailArea\\
\midrule
\textcolor{red3}{DPCERA3M086SBEA} & -0.002 & -1.365 &  0.666  &  -0.009  &  0.026   &    0.050   &   0.050 \\
\textcolor{chartreuse4}{IPDMAT} & -0.003  &-1.113  & 0.268  &  -0.012 &   0.006    &   0.050    &  0.049 \\
\textcolor{blue3}{HWIURATIO}  & 0.002 &  0.718 &  0.197 &   -0.003 &   0.014  &     0.049  &    0.050 \\
\textcolor{blue3}{CLF16OV} & 0.243 & 36.668  & 0.000   &  0.232  &  0.259     &  0.049  &    0.049 \\
\textcolor{blue3}{CE16OV} &  -0.266 & -37.390 &  0.000  &  -0.280 &  -0.254   &    0.049  &    0.049\\
\textcolor{blue3}{UEMPLT5} & 0.001  & 0.241  & 0.402  &  -0.005  &  0.008     &  0.049   &   0.049\\
\textcolor{blue3}{UEMP5TO14} & 0.000 & -0.118 &  0.429  &  -0.007  &  0.004   &    0.049  &    0.049\\
\textcolor{blue3}{UEMP15OV}& 0.004  & 1.593  & 0.056 &    0.000  &  0.009  &     0.049    &  0.050 \\
\textcolor{blue3}{PAYEMS} & 0.001 &  0.273  & 0.223  &  -0.007  &  0.029     &  0.050   &   0.050 \\
\textcolor{blue3}{USCONS} & -0.002 & -0.883 &  0.569 &   -0.009  &  0.016  &     0.050  &    0.000 \\
\textcolor{orange}{TB6MS} & -0.001 & -0.480 &  0.682 &   -0.009  &  0.026  &     0.050   &   0.050 \\
\textcolor{orange}{GS5} & -0.003 & -1.131  & 0.219  &  -0.025  &  0.007     &  0.050   &   0.049\\
\textcolor{orange}{EXUSUKx} & 0.003 &  1.306 &  0.872 &   -0.072  &  0.003  &     0.050  &    0.050 \\
\textcolor{blue3}{lag 1} & -0.009 & -4.067  & 0.003 &   -0.013 &  -0.004  &     0.049    &  0.049 \\
\bottomrule
\end{tabular}  
}
\caption{\(p\)-værdier og konfidensintervaller for variablerne udvalgt af lasso. Den estimeres standard afvigelse er \(0.043\), og resultaterne er for \(\lambda \approx 1.823\) med \(\alpha = 0.1\).} \label{fig:fixedLassoInf}
\end{table} 
%


Post-selection intervallerne vises på figur -- for disse 14 variable.


\texttt{fixedLassoInf} udregner \(p\)-værdier og konfidens intervaller for lasso estimatet for en fast værdi af tuning parameteren \(\lambda\).
\texttt{fixedLassoInf} anvender ``standard'' lasso
\begin{align*}
\frac{1}{2} \Vert \y - \X \tbeta \Vert_2^2 + \lambda \Vert \tbeta \Vert_1.
\end{align*}
\texttt{glmnet} multiplicerer først led med faktoren \(\frac{1}{n}\).
Efter vi har kørt \texttt{glmnet} og fundet betaen som svarer til lambda værdien, da skal vi \texttt{beta = coef(obj, s=lambda/n)}, hvor \texttt{obj} er objektet som er returneret af \texttt{glmnet}.



Lars algoritmen udfører 126 steps, hvor én variabel tilføjes i hvert step.
%
\begin{table}[ht] 
\centering 
\scalebox{0.8}{
\begin{tabular}{lccccccc}
%\multicolumn{9}{l}{LARS algoritmen} \\
\toprule
Prædiktor& Koefficient & Z-score &\(p\)-værdi & Konfidensinterval &   $\sbr{\mathcal{V}^-;\mathcal{V}^+}$   \\
\midrule
\textcolor{blue3}{HWIURATIO}& 0.002  & 0.694   & 0.160    &  $\del{-\infty   ;  \infty}$   &$\sbr{0.002;0.002} $    \\
 \textcolor{blue3}{UEMP15OV} &    0.004&   1.606 & 0.923 &     $ \left( -\infty  ;  0.032\right] $     &$\sbr{0.004; 0.005}$   \\
 \textcolor{blue3}{UEMPLT5} & 0.001   &0.149   & 0.064  & $ \left[-0.018  ;     \infty\right) $  & $\sbr{0.000 ;0.001}$   \\
\textcolor{blue3}{MANEMP}   &   0.002 &  0.486   &0.273 &   $\left[-0.171 ;      \infty\right)$  & $\sbr{0.002;0.003}$\\
 \textcolor{blue3}{UEMP5TO14}  &-  0.001 & $-0.242 $ &0.077  &   $ \left( -\infty    ;  0.016\right] $&      $\sbr{0.000 ;0.001}$ \\
\textcolor{blue3}{CE16OV} &- 0.267 &$-37.446$ & 0.130   &   $\left( -\infty   ;  0.532\right]  $&    $\sbr{0.267; 0.267}$     \\ 
\textcolor{blue3}{ PAYEMS } &   0.000 &  0.006  & 0.563   &  $\del{-\infty ;  \infty}$   & $\sbr{0.000 ;0.000}$  \\
 \textcolor{blue3}{USGOOD} &- 0.003  &$-0.498$ &0.638   &   $\del{-\infty ;  \infty}$ &    $\sbr{0.003 ;0.003}$\\
\textcolor{chartreuse4}{CUMFNS} &  0.002  & 0.404 & 0.478    & $\del{-\infty   ;  \infty}$  &  $\sbr{0.002 ;0.002 }$  \\
 \textcolor{blue3}{CLF16OV} &  0.243  &36.643   & 0.179   &  $\del{-\infty  ;  \infty}$ &  $\sbr{0.243 ;0.243}$ \\  
\textcolor{chartreuse4}{ IPDMAT}&-0.006 &$ -1.626 $  &0.874   & $\left[-0.125  ;    \infty\right) $& $\sbr{0.006; 0.006}$ \\   
\textcolor{orange}{ TB6MS} & -0.005  &$-0.715$  & 0.569 &     $\del{-\infty  ;  \infty}$&   $\sbr{0.005; 0.006}$   \\ 
\textcolor{chartreuse4}{INDPRO} &  0.003 &  0.513  &0.328   &  $\del{-\infty   ;  \infty }$    & $\sbr{0.003 ;0.003}$  \\
\textcolor{orange}{GS1} &   0.006&   0.577    &0.473  &    $\del{-\infty  ;  \infty}$  &$\sbr{0.006 ;0.006}$ \\  
\textcolor{orange}{GS5} & -0.005 & $-1.146 $ &0.037 &     $\left( -\infty ;  -0.025\right]   $ & $\sbr{0.005 ;0.005 }$\\  
 \textcolor{blue3}{lag1}  & -0.009  &$-3.949$   & 0.910   & $\del{-\infty  ;  \infty }$  &$\sbr{0.009; 0.009 }$ \\ 
 \textcolor{red3}{DPCERA3M086SBEA} &- 0.003 & $-1.436$ & 0.233  &   $\del{-\infty   ;  \infty }$ &  $\sbr{0.003; 0.003}$ \\ 
\textcolor{orange}{ EXUSUKx}  &  0.003   &1.383 & 0.964   &   $\left( -\infty     ;-0.053 \right] $&  $\sbr{0.003; 0.003 }$   \\   
 \textcolor{blue3}{CLAIMSx} &0.002 &  0.813   & 0.226 &    $\del{-\infty  ;  \infty}$& $\sbr{0.002 ;0.002 }$   \\ 
\bottomrule
\end{tabular}  
}
\caption{\(p\)-værdier, konfidensintervaller, $Z$-score samt de trunkerede intervaller for variablerne udvalgt af LARS (CV) . Den estimeres standard afvigelse er \(0.043\), og resultaterne er for \(f_{1 \text{sd}} = 0.2542\) med \(\alpha = 0.1\).} \label{tab:larInf_kryds}
\end{table} 

%
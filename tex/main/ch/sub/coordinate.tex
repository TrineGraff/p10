\subsection{Coordinate descent}
Denne sektion er baseret på pakkerne \textit{glmnet} og \textit{gglasso}, \citep{gglasso}.

Figur \ref{fig:cv_plot} viser den gennemsnitlige krydsvaliderings fejl for hver værdi af $\log \lambda$ for metode.
Det skal lige bemærkes, at for elastisk net har vi to turning parameter $\alpha$ og $\lambda$. 
Så vi har anvendt en 10-fold krydsvalidering for 10 værdier af $\alpha$, hvor $\alpha \in (0,1)$. 
For hvert $\alpha$ har vi fundet $\lambda_{\min}$ og krydsvaliderings fejlen.  
Den mindste krydsvaliderings fejl for $\lambda_{\min}$ er når $\alpha =0.9$. 

\imgfigh{cv_plot.pdf}{1}{10-fold krydsvaliderings fejl plottede som en function af $ \log(\lambda)$ for vores metoder. De stiplede linjer indikerer minimum fejl, samt fejlen med en standard afvigelse af minimum}{cv_plot}

\begin{table}
\center
\begin{tabular}{cccc | cccccc}
\toprule
 &  \multicolumn{3}{c}{Lasso} &  \multicolumn{3}{c}{Ridge}  \\ \midrule
 & værdi & MSE & p & 	værdi & MSE & p \\
 $\lambda_{\min}$ &0.0014& 0.0019 & 20 	& 0.0123 &  0.0049 & 122 \\ 
 $\lambda_{1 \text{sd}}$ & 0.0027 & 0.0020 & 15 & 0.0148 & 0.0051 & 122  \\ \bottomrule \toprule
  &  \multicolumn{3}{c}{Elastic Net}  &  \multicolumn{3}{c}{Group Lasso}  \\ \midrule
 & værdi & MSE & p & værdi & MSE & p \\
 $\lambda_{\min}$ & 0.0014 & 0.0020 & 23 & 0.0003 & 0.0023  & 122\\
  $\lambda_{1\text{sd}}$ & 0.0027 & 0.0021 & 15 & 0.0005 & 0.0024 & 122 \\  \bottomrule \toprule
 &  \multicolumn{3}{c}{Adaptive lasso m. OLS vægte}  &  \multicolumn{3}{c}{Adaptive lasso m. lasso vægte}  \\ \midrule
  & værdi & MSE & p & værdi & MSE & p \\
 $\lambda_{\min}$  & 0.0868 & 0.0018 & 2 & 0.0034 & 0.0018 & 4   \\
 $\lambda_{1\text{sd}}$ & 0.4222 & 0.0018 & 2 & 0.0085 & 0.0019 & 4  \\ \bottomrule
 \end{tabular}
\caption{Tabellen viser $\lambda$ værdierne fundet udfra krydsvalidering, samt krydsvalideringsfejl, som er målt i MSE og antallet af parameter.} \label{tab:cv_tab}
\end{table}


For at få et bedre overblik viser tabel  \ref{tab:cv_tab} værdierne af vores $\lambda$ samt antallet af koefficienter. 
Vi ser for lasso og elastisk net sker der en reducering af antallet af parameter når $\lambda_{1\text{sd}}$ anvendes i forhold til $\lambda_{\min}$ og hvor deres MSE ikke er signifikant forskellig.  
Derfor lader vi den optimale $\lambda$, for lasso og elastisk net være $\lambda_{1\text{sd}}$.
Vi ser dog også, at $\lambda_{1\text{sd}}$ er ens i tabel \ref{tab:cv_tab}. 
De afviger fra hinanden i 5. decimal.  

For ridge regression vil der ikke ske reducering af antallet af parameter, men derimod en reducering af værdierne af koefficienterne og derfor lader vi den optimale $\lambda$, være den med mindst krydvaliderings fejl og anvender $\lambda_{\min}$, som vores optimale $\lambda$. 
%
For group lasso skal de forklarende variabler indelles i  grupper. 
Disse grupper er forslået af Michael McCracken og ses i appendiks \ref{ch:app_data}.
Men group lasso opfører sig anderledes end hvad vi havde forventet. 
Den fejler i reducering af parameter. 
Det indikerer lidt på, at Group lasso ikke er en god model for vores data. (??)
Men vi lader den optimale $\lambda$, være $\lambda_{\min}$, da den har mindst krydsvaliderings fejl. 

Adaptive lasso m. OLS vægte vælger det færreste antal forklarende variabler. 
Den udvælger kun 2 for både  $\lambda_{\min}$ og $\lambda_{1\text{sd}}$. 
Derfor lader vi den optimale $\lambda$ være $\lambda_{\min}$. 

Det skal bemærkes, at når vi anvender adaptive lasso med lasso vægte anvender vi kun de forklarende variable, som lasso har udvalgt. 
Vi giver altså ikke alle 122 forklarende variable, men kun de 15, som den optimale $\lambda$ udvælger for lasso.
For adaptive lasso med lasso vægte lader vi den optimale $\lambda$ være $\lambda_{\min}$.  

Tabel \ref{tab: lasso_ud} viser hvilke koefficienter elastisk net og lasso udvælger. 
Vi får at lasso og elastisk net udvælger de samme variable, dog er værdierne af koefficienterne forskellige. 
Det kan skyldes, at deres optimale $\lambda$ værdi er tæt på den samme. 
Derudover har vi jo, at $\alpha = 1$ for lasso og $\alpha = 0.9$ for elastisk net, så der er altså ikke stor forskel på de to modeller. 
Derudover ses, at de fleste variabler de to metoder udvælger stammer fra samme gruppe, som vores responsvariable.  

Tabel \ref{tab: v_ud} viser hvilke forklarende variable adaptive lasso udvælger. Her udvælger metoderne kun variable fra gruppen Arbejdsmarked. 

Variablerne Civilian Labor Force og Civilian Employment bliver valgt af alle metoderne. 

\begin{table}
\small
\center
\begin{tabular}{lcc}
\toprule
\multicolumn{2}{c}{Lasso} \\ \midrule
Durable Materials (Output og indkomst) & \\
Ratio of Help Wanted/No. Unemployed (Arbejdsmarked) & \\
Civilian Labor Force (Arbejdsmarked)  & \\
Civilian Employment  (Arbejdsmarked) & \\
Civilian Unemployed - Less Than 5 Weeks (Arbejdsmarked)  & \\
Civilians Unemployed for 5-14 weeks (Arbejdsmarked) & \\
Civilian Unemployed - 15 Weeks \& Over (Arbejdsmarked) & \\
All Employees: Total Nonfarm (Arbejdsmarked) &\\
All Employees: Gods-Producing Industries (Arbejdsmarked) & \\
All Employees: Construction (Arbejdsmarked) & \\
Housing Starts, West (Boliger) & \\
Real Personal Consumption expenditures (Forbrug, ordrer og varebeholdning) &\\
New Orders for Durable Goods (Forbrug, ordrer og varebeholdning)  & \\
5 Years Treasure Rate (Rente og valutakurs) &\\
U.S /U.K Foreign Exchange Rate  (Rente og valutakurs)  & \\ \bottomrule
\end{tabular}
\caption{Overstående er de forklarende variable, som bliver udvalgt af lasso. Det der står i parantes er hvilken gruppe hver variable tilhører} \label{tab: lasso_ud}
\end{table}

 \begin{table}
\small
\center
\begin{tabular}{l | l}
\toprule
\multicolumn{1}{c}{Adaptive lasso m. OLS vægte} &  \multicolumn{1}{c}{Adaptive lasso m. lasso vægte}  \\ \midrule
Civilian Labor Force  (2) & Civilian Labor Force (2) \\
Civilian Employment (2) &  Civilian Employment (2) \\
& Civilians Unemployed for 5-15 Weeks (2) \\
& Civilians Unemployed for 15 Weeks \& over (2) \\
 \bottomrule 
\end{tabular}
\caption{Tabellen indeholder de forklarende variable, som bliver udvalgt af adaptive lasso med hhv. OLS vægte og lasso vægte. Tallene i parantes indikerer hvilken gruppe de forskellige variable tilhører} \label{tab: v_ud}
\end{table}


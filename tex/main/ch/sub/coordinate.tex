\section{Coordinate descent}
I dette afsnit vil vi finde den optimale model for lasso og dens generaliseringer udfra coordinate descent.
For lasso, ridge regression, elastisk net og adaptive lasso anvendes funktionen \texttt{glmnet} fra \Rlang-pakken af samme navn.
Funktionen genererer en følge med 100 værdier af $\lambda$ og fitter en model til hver af disse med maksimum likelihood estimation.
For group lasso har vi anvendt funktionen \texttt{gglasso} fra \Rlang-pakken med samme navn. 
Funktionen genererer også en følge med 100 værdier af $\lambda$, men anvender i stedet algoritmen block coordinate descent. 
Her anvendes kvadratroden af gruppens størrelse som straffaktor, som anbefales af \citep{group_lasso}.
Funktionen kræver, at variablerne er opdelt i grupper, hvortil vi betragter grupperne, som er forslået af Michael McCracken, som ses i appendiks \ref{app:app_data}. 

For at finde den optimale model anvendes 10-fold krydsvalidering og BIC til at estimere tuning parameteren $\lambda$.
Bemærk at for elastisk net og adaptive lasso har vi to tuning parametre, som skal estimeres. 
For elastik net betragtes tuning parmetrene $\alpha$ og $\lambda$.
Vi vælger en følge af 10 værdier af \(\alpha\), hvor $\alpha \in \sbr{0,1}$.
For hvert \(\alpha\) finder \(\lambda_\alpha\) udfra krydsvalidering og BIC.
Dette giver en følge af 10 værdier af \(\lambda_\alpha\), hvorfra vi finder den optimale \(\lambda \in \lambda_\alpha\).
Tilsvarende for adaptive lasso, hvor vi betragter tuning parametrene $\gamma$ og $\lambda$.
Her lader vi $\gamma$ være lig 0.5, 1 og 2, som er forslået af \citep{adaptive_lasso}, og følger samme procedure.
For adaptive lasso betragtes OLS og lasso vægte, således at for adaptive lasso med lasso vægte anvender vi kun de forklarende variable, som lasso har valgt. 

I kapitel \ref{ch:statistisk_inferens} introduceres TG testen, som udfører inferens i lasso modellen for en fast værdi af tuning parameteren.
Hertil anvendes funktionen \texttt{fixedLassoInf} fra \Rlang-pakken \texttt{selectiveInference}, som udregner \(p\)-værdier og konfidensintervaller for lasso estimatet.
%Funktionen tager som input tuning parameteren og regressionskoefficienterne returneret af \textttt{glmnet} og estimerer parametrene jævnfør \eqref{eq:tg_beta}, hvilket resulterer i en ny model.
Funktionen kan kun anvendes for lasso modellen og ikke dens generaliseringer. 
\Rlang-koden for dette er givet i \ref{subsubsec:inferens}.
\subsection{Krydsvalidering}
Funktionerne \texttt{cv.glmnet} og \texttt{cv.gglasso} fra pakkerne \texttt{glmnet} og \texttt{gglasso} udfører 10-fold krydsvalidering.

Figur \ref{fig:cv_plot} illustrerer den gennemsnitlige krydsvalideringsfejl samt øvre og nedre standardafvigelse for hver værdier af $\log \del{\lambda}$ for lasso og dens generaliseringer. 
De to lodrette stiplede linjer indikerer \(\lambda_{\text{min}}\) og \(\lambda_\text{1sd}\), hvor \(\lambda_{\text{min}}\) er værdien af \(\lambda\), som giver den mindste gennemsnitlige krydsvalideringsfejl og \(\lambda_\text{1sd}\) er den største værdi af \(\lambda\), således at fejlen er indenfor en standardafvigelse af minimum. 
For elastisk net finder vi, at $\alpha =1$ giver den mindste krydsvalideringsfejl for både \(\lambda_\text{min}\) og  \(\lambda_\text{1sd}\), hvilket svarer til lasso modellen.  Derfor betragter vi ikke elastik net. 

Den mindste krydsvaliderings fejl for $\lambda_\text{1sd}$ er også når $\alpha = 1$. 
\imgfigh{cv_plot.pdf}{1}{10-fold krydsvalideringsfejl som funktion af $\log \del{\lambda}$ for lasso og den generaliseringer. 
De stiplede linjer betegner \(\lambda_\text{min}\) og \(\lambda_\text{1sd}\).}{cv_plot}
%

\begin{table}[ht]
\center
\begin{tabular}{lcccc | lccccc}
\toprule
   \multicolumn{5}{c}{Lasso} & \multicolumn{1}{c}{ }&  \multicolumn{5}{c}{Ridge regression}  \\ \midrule
 & \(\log \del{\lambda}\) & MSE & $p$ & Adj. R$^2$ &&& \(\log \del{\lambda}\) & MSE & $p$ & Adj. R$^2$\\
 $\lambda_{\min}$ &$-6.6361$& 0.0019 & 28 & 94.28\% &&  $\boldsymbol{\lambda_{\min}}$ &  $\mathbf{-4.3800}$ &   $\mathbf{0.0045} $&  $\mathbf{126}$ & $ \mathbf{86.56 \% }$ \\ 
 $\boldsymbol{\lambda}_{\textbf{1sd}}$ & $\mathbf{-5.7057}$ & $\mathbf{0.0020} $ & $\mathbf{14}$ &$\textbf{ 93.81} \boldsymbol{\%}$ && $\lambda_{ \text{1sd}}$& $-4.1939$ & 0.0047 & 126 &  85.70\%  \\ \bottomrule \toprule
\multicolumn{5}{c}{Group lasso}  &&  \multicolumn{5}{c}{Adap. lasso m. OLS vægte}  \\ \midrule
& \(\log \del{\lambda}\) & MSE &$ p $ &Adj. R$^2$ &&& \(\log \del{\lambda}\) & MSE & $p$ & Adj. R$^2$ \\
$\lambda_{\min}$& $-8.2644$ & 0.0022  & 126 & 93.23\% && $\boldsymbol{\lambda_{\min}}$  & $\mathbf{-2.0630}$ &$ \mathbf{0.0018}$ & $\mathbf{2}$ & $\textbf{94.27} \%$ \\
  $\boldsymbol{\lambda}_{\textbf{1sd}}$  & $\mathbf{-7.6365}$ &$ \mathbf{0.0023}$ & $\mathbf{119}$ &$ \textbf{92.60} \%$ &&  $\lambda_{1\text{sd}}$ & $-0.3884$ & 0.0019 & 2 &  93.93\%\\  \bottomrule 
  \toprule
  \multicolumn{5}{c}{Adap. lasso m. lasso vægte}  \\ \cmidrule{1-5}
& \(\log \del{\lambda}\) & MSE & $p$ & Adj. R$^2$\\
$\boldsymbol{\lambda_{\min}}$   &  $ \mathbf{-2.3674}$ & $ \mathbf{0.0018} $& $ \mathbf{2}$ &   $\textbf{94.28} \%$ \\
$\lambda_{1\text{sd}}$  & $-0.2276$ & 0.0019 & 2 & 93.92\%\\ \cmidrule{1-5}
 \end{tabular}
\caption{Logaritmen af $\lambda_{min}$ og $\lambda_{1\text{sd}}$, gennemsnitlig krydsvalideringsfejl, som er målt i MSE, antallet af paramete og adjusted R$^2$ for lasso og dens generaliseringer. De valgte tuning parameter for hver metode er markeret med tykt.} \label{tab:cv_tab}
\end{table}


For at give et bedre overblik giver tabel \ref{tab:cv_tab} værdierne af $\log \del{ \lambda_{\min}}$ og $\log \del{ \lambda_{1\text{sd}}}$, krydsvalideringsfejlen, antallet af parametre, justerede R$^2$ og log-likelihood for lasso og dens generaliseringer.
%Den valgte tuning parameter er markeret med tykt for hver metode.   

For lasso ses en markant reducering i antallet af parametre for $\lambda_{1\text{sd}}$ i forhold til $\lambda_{\min}$, dette øger ikke krydsvalideringsfejlen eller justerede R$^2$ betydeligt, og derfor anvendes $\widehat{\lambda}_{1\text{sd}}$ som tuning parameter for lasso. 
%Ridge regression mindsker blot koefficienter, og derfor vælges alle variable. 
Vi ser også, at ridge regression har den mindste værdi af justerede R$^2$ samt den højeste værdi af log-likelihood, men modellen mindsker også blot koefficienterne, således at alle variablerne bliver valgt. 
For ridge regression vælger vi, derfor $\widehat{\lambda}_{\min}$ som tuning parameter, da den har mindst krydsvalideringsfejl.
For group lasso vælges lidt overraskende alle parametre med \(\lambda_\text{min}\), mens antallet af parametre reduceres med 7 for $\lambda_{1\text{sd}}$. Disse 7 variable tilhører alle gruppe 5.
Vi lader $\widehat{\lambda}_{1\text{sd}}$ være den optimale tuning parameter for group lasso, da den har det færreste antal parametre. 

Adaptive lasso med OLS og lasso vægte vælger blot to variable for både  $\lambda_{\min}$ og $\lambda_{1\text{sd}}$. 
%Variablerne er  \textcolor{blue3}{CLF16OV} og \textcolor{blue3}{CE16OV}.
Vi lader $\widehat{\lambda}_{\min}$ være tuning parameteren for adaptive lasso modellerne. Vi ser også at adaptive lasso med OLS vægte har den højeste justerede R$^2$, men det er også modellen med færreste parametre. 

På figur \ref{fig:coef_kryds_coord} vises de 14 estimerede koefficienter for lasso og de 2 estimerede koefficienter for adaptive lasso.
Heraf ses, at lasso hovedsagligt vælger variable i samme gruppe som arbejdsløshedsraten.
For lasso ses at variablerne valgt af adaptive lasso, \textcolor{blue3}{CLF16OV} og \textcolor{blue3}{CE16OV}, har de største estimerede koefficienter, efterfulgt af \textcolor{blue3}{UEMPLT5}, \textcolor{blue3}{UEMP5TO14}, \textcolor{blue3}{UEMPL15OV} og \textcolor{blue3}{lag 1}, mens de øvrige er meget tæt på nul. 
Figur \ref{fig:coef_ridge_kryds_coord} og \ref{fig:coef_gglasso_kryds_coord} viser de estimerede koefficienter for henholdsvis ridge regression og group lasso.
Igen ser vi, at variablerne \textcolor{blue3}{CLF16OV} og \textcolor{blue3}{CE16OV} klart har de største estimerede koefficienter.    
%
\imgfigh{coef_kryds_coord.pdf}{1}{Estimerede koefficienter for lasso og adaptive lasso med OLS og lasso vægte,  hvor $\widehat{\lambda}$ er fundet ud fra krydsvalidering.
Farverne indikerer hvilken gruppe, variabler tilhører, og y-aksen er variablerne udvalgt af lasso. }{coef_kryds_coord}


Figur \ref{fig:resid_lasso_coord_kryds}-\ref{fig:resid_adap_ols_coord_kryds} viser en analyse af de standardiserede residualer for lasso og dens generaliseringer. 
Vi ser, samme tendens for lasso og dens generaliseringer. Histogrammet og QQ-plottet indikerer tungere haler end normalfordelingen og autokorrelation i første lag.
Dette bekræftes i tabel \ref{tab:res_shrinkage_tab}, som viser skewness, excess kurtosis, $p$-værdier fra JB-testen og LB testen for de standardiserede residualer, hvor $\widehat{\lambda}$ er estimerede udfra krydsvalidering.  
Vi ser, at alle modellerne har en negativ skewness og en kurtosis forskellige fra nul. 
Derudover afvises JB testens nulhypotesen om normalitet for alle modeller med undtagelse af group lasso, dog har den en lille skewness og kurtosis.
For LB testen afvises nulhypotesen om uafhængighed for alle modeller.

\subsubsection{Inferens}
På figur \ref{fig:boxplot_lasso_coord_kryds} ses bootstrap resultater for variablene udvalgt af lasso.
Da adaptive lasso har konsistent variabeludvælgelse, vil variablerne \textcolor{blue3}{CLF16OV} og \textcolor{blue3}{CE16OV} altid vælges, derfor laves ikke bootstrap for disse.

Variablerne \textcolor{orange}{TB6MS}, \textcolor{blue3}{PAYEMS} og \textcolor{red3}{DPCERA3M086SBEA} fravælges over 50\% af bootstrap realisationerne, mens variablerne  \textcolor{blue3}{lag 1}, \textcolor{blue3}{UEMPL15OV}, \textcolor{blue3}{UEMP5TO14}, \textcolor{blue3}{UEMPLT5}, \textcolor{blue3}{CE16OV} og \textcolor{blue3}{CLF16OV} ofte vælges.
Generelt fravælges variablerne, som ikke tilhører gruppe 2.
I forhold til størrelsen af de estimerede koefficienter for lasso er bootstrap resultaterne ikke overraskende. 
%
%\imgfigh{bootstrap_alasso.pdf}{0.7}{Til venstre vises et boxplot af 1000 bootstrap realisationer af $\widehat{\tbeta}^{\text{AL}} \del{{\widehat{\lambda}_\text{min}}}$ med OLS vægte, mens plottet til højre illustrerer andelen af bootstrap realisationer, hvor parameter estimaterne er præcis lig nul.}{bootstrap_alasso}
%
Herefter anvendes TG testen for lasso modellen.
Resultaterne er givet i tabel \ref{tab:fixedLassoInf}.
Heraf ser vi at variablerne \textcolor{blue3}{CLF16OV}, \textcolor{blue3}{CE16OV} og \textcolor{blue3}{lag 1} afviser nulhypotesen, og derfor er signifikante.
%
\begin{table}[ht] 
\centering 
\begin{tabular}{llllllll}
%\multicolumn{4}{c}{Lasso} \\
\toprule
Prædiktor & Koefficient & Z-score & \(p\)-værdi & lowConfPt & UpConfPt & LowTailArea & UpTailArea \\
\midrule
3 & -0.002 & -1.372 & 0.649 & -0.009 & 0.025 & 0.050 & 0.050 \\
14 & -0.003 & -1.111 &  0.275 &   -0.011 &   0.006 & 0.050 & 0.049 \\
21 & 0.002 & 0.723 & 0.191 & -0.003 & 0.014 & 0.049 & 0.050 \\
22 & 0.243 & 36.619 & 0.000 & 0.232 & 0.259 & 0.048 & 0.049 \\
23 & -0.266 & -37.351 & 0.000 & -0.280 & -0.254 & 0.049 & 0.049 \\
25 & 0.001 & 0.243 & 0.401 & -0.005 & 0.008 & 0.049 & 0.049 \\
26 &  0.000 & -0.120 &  0.425 & -0.007 & 0.004 & 0.049 & 0.049 \\
27 & 0.004 & 1.590 & 0.057 & 0.000 & 0.009 &  0.049 & 0.050 \\
31 & 0.001 & 0.249 & 0.236 & -0.007 & 0.027  & 0.049 & 0.050 \\
34 & -0.002 & -0.880 & 0.578 & -0.009 & 0.016 & 0.050 & 0.000 \\
78 & -0.001 & -0.480 & 0.683 & -0.009 & 0.027 & 0.050 & 0.050 \\
80 & -0.003 & -1.131 &  0.218 & -0.025 & 0.007  & 0.050 & 0.050 \\
94 & 0.003 & 1.301 & 0.877 & -0.075 & 0.003 & 0.050 & 0.050 \\
123 & -0.009 & -4.070 & 0.003 & -0.013 & -0.004 & 0.050 & 0.049 \\
\bottomrule
\end{tabular}  
\caption{-}
\end{table} 
%

\newpage
%
%



\newpage
\subsection{BIC}
Hernæst vil vi finde tuning parameteren med BIC. 
Vi bruger funktionen i appendiks \ref{sub:bic} til at finde $\widehat{\lambda}$. 

Tabel \ref{tab:bic_lambda} giver $\log \del{ \widehat\lambda}$, antallet af parameter, BIC og adjusted R$^2$ for hver model. 
Igen vælges \(\alpha = 1\) for elastik net, og derfor ser vi igen bort fra denne model.
Adjusted R$^2$ er igen højst for adaptive lasso og mindst for ridge regression. 
Præcis som ved anvendelse af krydsvalideringer, vælger adaptive lasso færrest antal variable, som igen er \texttt{CLF16OV} og \texttt{CE16OV}.
Derudover vælger group lasso hele 99 variabler. 

\begin{table}
\center
\begin{tabular}{llll} 
\toprule
& \multicolumn{1}{c}{${\widehat\lambda}$} & \multicolumn{1}{c}{BIC} & \multicolumn{1}{c}{p} \\ \midrule
Lasso & 0.0023 & -6.1424 & 14  \\
Ridge regression & 0.0112 & -4.2931 & 122 \\
Elastik net, $\alpha = 0.9$ & 0.0019 & -6.1202 &17 \\
Group lasso & $6.34 \cdot 10^{-5}$ & -5.0744 & 122 \\
Adaptive lasso med OLS vægte & 0.0599 & -6.3174 & 2 \\
Adaptive lasso med lasso vægte &  0.1058& -6.3174 & 2\\ \bottomrule
 \end{tabular}
\caption{Tabellen viser ${\widehat\lambda}$ fundet udfra BIC, samt BIC-værdien og antallet af parameter hver metode udvælger} \label{tab:bic_lambda}
\end{table}

På figur \ref{fig:coef_bic_coord} ser vi også det samme - variablerne \texttt{CLF16OV} og \texttt{CE16OV} har de markant højeste værdier, hvor de resterende er meget tæt på nul. 
Det samme er gældende for ridge og group lasso, vi har derfor valgt ikke at inkluderer disse koefficients plots. 

\imgfigh{coef_bic_coord.pdf}{0.8}{Estimerede koefficienter for lasso og adaptive lasso med OLS og lasso vægte med BIC. 
Farverne indikerer hvilken gruppe, variabler tilhører, og y-aksen er variablerne udvalgt af lasso.}{coef_bic_coord}

Figur \ref{fig:boxplot_lasso_coord_bic} viser boxplot for lasso, og vi ser at variablerne der ikke tilhører gruppe to oftes bliver valgt til at være nul, derudover er variablerne  \texttt{CLF16OV},  \texttt{CE16OV} , \texttt{lag1}, \texttt{UEMPL15OV}, og \texttt{UEMPLT5} fra gruppe to ofte estimeret til at være forskellige fra nul, dvs at lasso ofte vælger disse variable. 
Igen ser vi, for alle modeller at variablerne \texttt{CLF16OV} og \texttt{CE16OV} ofte bliver estimeret til at være forskellige fra nul. 


Af tabel \ref{tab:fixedLassoInf_bic} observeres at nulhypotesen afvises for \texttt{CLF16OV}, \texttt{CE16OV}. 

\begin{table}[h] 
\centering 
\scalebox{0.8}{
\begin{tabular}{llllllll}
\toprule
Prædiktor & Koefficient & Z-score & \(p\)-værdi & lowConfPt & UpConfPt & LowTailArea & UpTailArea\\
\midrule
\textcolor{red3}{DPCERA3M086SBEA}  & -0.002  &-0.960   &0.093  &   -0.071  &  0.003      & 0.000   &  0.050 \\
\textcolor{chartreuse4}{IPDMAT} &-0.002 & -0.680 &  0.159  &  -0.032 &   0.005     &  0.050    &  0.049 \\
\textcolor{blue3}{CLF16OV} & 0.241  &36.686  & 0.000 &    0.235   & 0.350    &   0.050  &    0.050 \\
\textcolor{blue3}{CE160V} &-0.264& -37.339   &0.000  &  -0.455  & -0.260   &    0.050   &   0.050 \\
\textcolor{blue3}{UEMPLT5}  & 0.000 &  0.027 &  0.777   & -0.029    &0.005    &   0.000  &     0.048 \\
\textcolor{blue3}{UEMP5TO14} & -0.001  & -0.266 & 0.599  & -0.007   &  0.014    &   0.050 &      0.050 \\
\textcolor{blue3}{UEMP15OV} &0.004  & 1.299  & 0.249   & -0.005   & 0.008  &     0.050     & 0.049 \\
\textcolor{blue3}{CLAIMSx} & 0.001 &  0.387  & 0.689   & -0.030   & 0.011    &  0.050     & 0.050 \\
\textcolor{blue3}{USCONS}  & -0.001  &  -0.591   &  0.100  &    -0.088  &     0.004  &      0.050  &       0.050 \\
\textcolor{blue3}{USTRADE}  & 0.000  & -0.118  &  0.988     & 0.007     &  Inf     &   0.050  &     0.000\\
\textcolor{red3}{AMDMNOx} &-0.002 &  -0.813 &  0.641  &  -0.008  &  0.020   &    0.049      &0.050 \\
\textcolor{orange}{TB6MS}&-0.001  &-0.415  & 0.677   & -0.008  &  0.023   &    0.049   &   0.000 \\
\textcolor{orange}{GS5} &-0.003 & -1.207  & 0.144    &-0.032  &  0.005      & 0.050    &  0.050 \\
\textcolor{orange}{EXUSUKx} & 0.003  & 1.449   &0.303   & -0.007   & 0.012      & 0.050    &  0.050 \\
\textcolor{cadetblue2}{CPIMEDSL}  &0.002 &  0.855 &  0.865&    -0.054 &   0.003&       0.050    &  0.050 \\
\textcolor{blue3}{lag1} & -0.010&  -4.362 &  0.499  &  -0.011   & 0.033   &    0.050  &    0.050 \\
\textcolor{blue3}{lag4}  & 0.002 &   1.106   & 0.311    & -0.014 &    0.028 &       0.036 &      0.050 \\
\bottomrule
\end{tabular}  
}
\caption{\(p\)-værdier og konfidensintervaller for variablerne udvalgt af lasso med BIC. Den estimeres standard afvigelse er ---, og resultaterne er for ---- med \(\alpha = 0.1\).} \label{tab:fixedLassoInf_bic}
\end{table} 

Figurerne \ref{fig:resid_lasso_coord_bic}-\ref{fig:resid_adap_ols_coord_bic} viser en an analyse af de standardiserede residualer.
Vi ser igen lidt af den samme tendens for alle metoderne. 
Histogrammet og QQ-plottet indikerer tungere haler end normalfordelingen og autokorrelation i det første lag. 
Dog har ridge regression og group lasso kun få outliers som ses på QQ-plottet. 
I tabel \ref{tab:res_shrinkage_bic_tab} ser vi også, at vi ikke kan afvise LB testen omkring normalitet for group lasso. 

%% \begin{table}
\small
\center
\begin{tabular}{lllc}
\toprule
\multicolumn{1}{c}{Lasso} & \multicolumn{1}{c}{Elastisk net} \\ \midrule
Durable Materials (1) &Durable Materials (1)   \\
Ratio of Help Wanted/No. Unemployed (2) & Nondurable Materials (1)  \\
Civilian Labor Force (2) &  Civilian Labor Force (2) \\
Civilian Employment (2)& Civilian Employment (2) \\
Civilians Unemployed - Less Than 5 Weeks (2) & Average Duration of Unemployment (Weeks) (2) \\
Civilians Unemployed for 5-14 Weeks (2) & Civilians Unemployed - Less Than 5 Weeks (2)  \\
Civilians Unemployed - 15 Weeks \& Over (2)& Civilians Unemployed for 5-14 Weeks (2) \\
Initial Claims (2)& Civilians Unemployed - 15 Weeks \& Over (2) \\
All Employees: Construction (2)& Initial Claims (2) \\
Housing Starts, West (3)&All Employees: Construction (2) \\
Real personal Consumption expenditures (4)& Housing Starts, West (3) \\
New Orders for Durable Goods (4) & Real personal Consumption expenditures (4) \\
5-Year Treasure Rate (6) & New Orders for Durable Goods (4) \\
U.S. / U.K. Foreign Exchange Rate (6)&Nonrevolving consumer credit to Personal income (5) \\
& 5-Year Treasure Rate (6)  \\
& U.S. / U.K. Foreign Exchange Rate (6) \\
& PPI: MatLA ns metal products (7) \\
\bottomrule 
\toprule
\multicolumn{1}{c}{Adaptive lasso m. OLS vægte} & \multicolumn{1}{c}{ Adaptive lasso m. lasso vægte}  \\ \midrule
Civilian Labor Force (2) & Civilian Labor Force (2) \\
Civilian Employment (2) & Civilian Employment (2) \\
 \bottomrule 
\end{tabular}
\caption{Tabellen indeholder de forklarende variable, som bliver udvalgt af lasso, elastisk net og adaptive lasso med OLS og lasso vægte. Tallene i parantes indikerer hvilken gruppe de forskellige variable tilhører} \label{tab:bic_ud}
\end{table}




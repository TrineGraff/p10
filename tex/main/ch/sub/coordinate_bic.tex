\subsection{BIC}
Hernæst vil vi estimere tuning parameteren med BIC.
For elastisk net finder vi, at $\alpha = 1$ giver den mindste BIC, derfor betragter vi igen ikke elastisk net.
For adaptive lasso med OLS vægte (BIC) finder vi, at $\gamma = 2$ giver den mindste BIC, og for adaptive lasso med lasso vægte (BIC) får vi, at $\gamma = 0.5$ giver den mindste BIC. 

Tabel \ref{tab:bic_lambda} giver værdierne af $\log \del{ \lambda_\text{BIC}}$, BIC, antallet af parametre, justeret R$^2$ og log-likelihood for lasso og dens generaliseringer. 
Justeret R$^2$ er igen størst for lasso (BIC) og mindst for ridge regression (BIC).

%\begin{table}
%\center
%\begin{tabular}{lccc} 
%\toprule
%& \multicolumn{1}{c}{Lasso} & \multicolumn{1}{c}{Ridge regression}  &  \multicolumn{1}{c}{Group lasso}\\ \midrule
%$\log \del{\widehat{\lambda}_\text{BIC}}$ & $-6.2639$ & $-4.4730$  & $-7.2876$ \\
%$p$ & 17 & 126 & 99 \\
%BIC & $-6.1608 $& $-3.3230$ & $-5.0721$  \\
%Adj. R$^2$ & 94.23 \% & 86.98  \%   & 92.11 \%  \\ \bottomrule \toprule
%& Adap. lasso m. OLS vægte & Adap. lasso m. lasso vægte \\ \midrule
%$\log \del{ \widehat{\lambda}_\text{BIC}}$ &  $-2.6212$& $-1.8390$ \\
%$p$ & 2&2 \\
%BIC &  $-6.3153$&$-6.3142$ \\
%Adj. R$^2$ & 94.28 \% &  94.28 \% \\ \bottomrule
% \end{tabular}
%\caption{Logaritmen af $\widehat{\lambda}_\text{BIC}$, antallet af parametre, BIC og adjusted R$^2$ for lasso og dens generaliseringer.} \label{tab:bic_lambda}
%\end{table}

\begin{table}
\center
\begin{tabular}{lcccc  | lccccc} 
\toprule
 \multicolumn{5}{c}{Lasso} && \multicolumn{5}{c}{Ridge regression}  \\ \midrule
& $\log \del{\widehat{\lambda}_\text{BIC}}$  & BIC & $p$ & R$^2_{\text{adj}}$ && & $\log \del{\widehat{\lambda}_\text{BIC}}$  &  BIC & $p$ &R$^2_{\text{adj}}$  \\
$\widehat{\lambda}_\text{BIC} $&  $-6.2639$ & $-6.1608$ & 17 &  94.23 \% && $\widehat{\lambda}_\text{BIC} $ & $-4.4730$ & $-3.3230$ &  126 & 86.98 \% \\ \bottomrule \toprule 
 \multicolumn{5}{c}{Group lasso} && \multicolumn{5}{c}{Adap. lasso m. OLS vægte}  \\ \midrule
& $\log \del{\widehat{\lambda}_\text{BIC}}$  & BIC & $p$ &R$^2_{\text{adj}}$ && & $\log \del{\widehat{\lambda}_\text{BIC}}$  &  BIC & $p$ &R$^2_{\text{adj}}$   \\
$\widehat{\lambda}_\text{BIC}$ & $-7.2876$ &  $-5.0721$ &99  & 92.11\% &&  $\widehat{\lambda}_\text{BIC}$ & $-2.6212$ &  $-6.3153$  & 2 & $94.28 \%$ \\ \bottomrule \toprule 
 \multicolumn{5}{c}{Adap. lasso m. lasso vægte}  \\
& $\log \del{\widehat{\lambda}_\text{BIC}}$  & BIC & $p$ & R$^2_{\text{adj}}$\\
 $\widehat{\lambda}_\text{BIC}  $&  	$-1.8390$ & $-6.3142$  & 2& 94.28\% \\ \cmidrule{1-5}
 \end{tabular}
\caption{Logaritmen af $\widehat{\lambda}_\text{BIC}$, antallet af parametre, BIC og adjusted R$^2$ for lasso og dens generaliseringer.} \label{tab:bic_lambda}
\end{table}

Figur \ref{fig:coef_bic_coord} vises de 17 estimerede koefficienter for lasso (BIC), 2 estimerede koefficienter for adaptive lasso med OLS vægte (BIC) og 3 estimerede koefficienter for adaptive lasso med lasso vægte (BIC).
Adaptive lasso med OLS vægte (BIC) vælger variablerne \textcolor{blue3}{CLF16OV} og \textcolor{blue3}{CE16OV}, mens adaptive lasso med lasso vægte (BIC) yderligere vælger variablen \textcolor{blue3}{lag 1}. 
Igen har variablerne \textcolor{blue3}{CLF16OV} og \textcolor{blue3}{CE16OV} de største estimerede koefficienter, mens de resterende estimerede koefficienter er meget tæt på nul. 
Vi har valgt ikke at inkludere de estimerede koefficienter for ridge regression (BIC) og group lasso (BIC), da det samme gør sig gældende for disse.

\imgfigh{coef_bic_coord.pdf}{1}{Estimerede koefficienter for lasso (BIC), adaptive lasso med OLS vægte (BIC) og adaptive lasso med lasso vægte (BIC). 
Farverne indikerer hvilken gruppe, variablerne tilhører og y-aksen er variablerne udvalgt af lasso (BIC).}{coef_bic_coord}

Figur \ref{fig:resid_lasso_coord_bic}-\ref{fig:resid_adap_lasso_coord_bic} viser en analyse af de standardiserede residualer for lasso og dens generaliseringer.
For lasso (BIC), adaptive lasso med OLS vægte (BIC) og adaptive lasso med lasso vægte (BIC) observeres udfra histogrammet og QQ-plottet, at fordelingen af de standardiserede residualer har tungere haler end normalfordelingen.
For ridge regression (BIC) og group lasso (BIC) er der blot få outliers fra normalfordelingen i QQ-plottet.
Fælles for alle er, at der er autokorrelation i det første lag. 
I tabel \ref{tab:res_shrinkage_tab} afvises residualerne at være uafhængige og JB testens nulhypotese om normalitet afvises også for alle med undtagelse af group lasso (BIC).

Figur \ref{fig:boxplot_lasso_coord_bic} viser bootstrap resultaterne for variablerne udvalgt af lasso (BIC).
Variablene \textcolor{cadetblue2}{CPIMEDSL}, \textcolor{orange}{TB6MS}, \textcolor{blue3}{USTRADE} og \textcolor{blue3}{CLAIMSx} fravælges over 50\% af bootstrap realisationerne, mens variablerne \textcolor{blue3}{lag 1}, \textcolor{blue3}{UEMPL15OV}, \textcolor{blue3}{CE16OV} og \textcolor{blue3}{CLF16OV} meget ofte vælges.
Resultaterne er ikke overraskende i forhold til størrelsen af de estimerede koefficienter for lasso (BIC).
Som nævnt tidligere har adaptive lasso konsistent variabeludvælgelse, og derfor laves der ikke bootstrap for adaptive lasso med OLS vægte (BIC) og adaptive lasso med lasso vægte (BIC). 

\subsubsection{TG testen}
Resultaterne for TG testen er givet i tabel \ref{tab:fixedLassoInf_bic}.
Heraf ser vi, at variablerne \textcolor{blue3}{CLF16OV} og \textcolor{blue3}{CE16OV} afviser nulhypotesen, og dermed er signifikante.
For \textcolor{blue3}{USTRADE} er den øvre grænse i konfidensintervallet uendelig, hvilket skyldes, at konfidensintervallet er udregnet numerisk og bliver ustabil, når $\boldsymbol\eta^T \mathbf{y}$ er for tæt på det trunkerede interval \(\sbr{\mathcal{V}^+, \mathcal{V}^-}\). 
Variablerne \textcolor{blue3}{CLF16OV} og \textcolor{blue3}{CE16OV} har de største estimerede koefficienter og $Z$-score.


\begin{table}[h] 
\centering 
\scalebox{0.8}{
\begin{tabular}{llllllll}
\toprule
Prædiktor & Koefficient & Z-score & \(p\)-værdi & lowConfPt & UpConfPt & LowTailArea & UpTailArea\\
\midrule
\textcolor{red3}{DPCERA3M086SBEA}  & -0.002  &-0.960   &0.093  &   -0.071  &  0.003      & 0.000   &  0.050 \\
\textcolor{chartreuse4}{IPDMAT} &-0.002 & -0.680 &  0.159  &  -0.032 &   0.005     &  0.050    &  0.049 \\
\textcolor{blue3}{CLF16OV} & 0.241  &36.686  & 0.000 &    0.235   & 0.350    &   0.050  &    0.050 \\
\textcolor{blue3}{CE160V} &-0.264& -37.339   &0.000  &  -0.455  & -0.260   &    0.050   &   0.050 \\
\textcolor{blue3}{UEMPLT5}  & 0.000 &  0.027 &  0.777   & -0.029    &0.005    &   0.000  &     0.048 \\
\textcolor{blue3}{UEMP5TO14} & -0.001  & -0.266 & 0.599  & -0.007   &  0.014    &   0.050 &      0.050 \\
\textcolor{blue3}{UEMP15OV} &0.004  & 1.299  & 0.249   & -0.005   & 0.008  &     0.050     & 0.049 \\
\textcolor{blue3}{CLAIMSx} & 0.001 &  0.387  & 0.689   & -0.030   & 0.011    &  0.050     & 0.050 \\
\textcolor{blue3}{USCONS}  & -0.001  &  -0.591   &  0.100  &    -0.088  &     0.004  &      0.050  &       0.050 \\
\textcolor{blue3}{USTRADE}  & 0.000  & -0.118  &  0.988     & 0.007     &  Inf     &   0.050  &     0.000\\
\textcolor{red3}{AMDMNOx} &-0.002 &  -0.813 &  0.641  &  -0.008  &  0.020   &    0.049      &0.050 \\
\textcolor{orange}{TB6MS}&-0.001  &-0.415  & 0.677   & -0.008  &  0.023   &    0.049   &   0.000 \\
\textcolor{orange}{GS5} &-0.003 & -1.207  & 0.144    &-0.032  &  0.005      & 0.050    &  0.050 \\
\textcolor{orange}{EXUSUKx} & 0.003  & 1.449   &0.303   & -0.007   & 0.012      & 0.050    &  0.050 \\
\textcolor{cadetblue2}{CPIMEDSL}  &0.002 &  0.855 &  0.865&    -0.054 &   0.003&       0.050    &  0.050 \\
\textcolor{blue3}{lag1} & -0.010&  -4.362 &  0.499  &  -0.011   & 0.033   &    0.050  &    0.050 \\
\textcolor{blue3}{lag4}  & 0.002 &   1.106   & 0.311    & -0.014 &    0.028 &       0.036 &      0.050 \\
\bottomrule
\end{tabular}  
}
\caption{\(p\)-værdier og konfidensintervaller for variablerne udvalgt af lasso med BIC. Den estimeres standard afvigelse er 0.043, og resultaterne er for $\lambda \approx 1.0432$  med \(\alpha = 0.1\).} \label{tab:fixedLassoInf_bic}
\end{table} 

Figur \ref{fig:resid_tg_bic} viser en analyse af de standardiserede residualer for lasso$_{TG}$ (BIC). 
Af tabel \ref{tab:res_shrinkage_tab} afvises residualerne at være normalfordelte og uafhængige.

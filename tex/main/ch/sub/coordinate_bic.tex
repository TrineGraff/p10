\subsection{BIC}
Hernæst vil vi finde tuning parameteren med BIC. 
Vi bruger funktionen i appendiks \ref{sub:bic} til at finde $\widehat{\lambda}$. 

Tabel \ref{tab:bic_lambda} giver $\log \del{ \widehat\lambda}$, antallet af parameter, BIC og adjusted R$^2$ for lasso og dens generaliseringer. 
Igen vælges \(\alpha = 1\) for elastik net, og derfor ser vi igen bort fra denne model.
Adjusted R$^2$ er igen højst for adaptive lasso og mindst for ridge regression. 
Præcis som ved anvendelse af krydsvalideringer, vælger adaptive lasso færrest antal variable, som igen er det variablerne \textcolor{blue3}{CLF16OV} og \textcolor{blue3}{CE16OV}.
Derudover vælger group lasso hele 99 variabler. 

%\begin{table}
%\center
%\begin{tabular}{lccc} 
%\toprule
%& \multicolumn{1}{c}{Lasso} & \multicolumn{1}{c}{Ridge regression}  &  \multicolumn{1}{c}{Group lasso}\\ \midrule
%$\log \del{\widehat{\lambda}_\text{BIC}}$ & $-6.2639$ & $-4.4730$  & $-7.2876$ \\
%$p$ & 17 & 126 & 99 \\
%BIC & $-6.1608 $& $-3.3230$ & $-5.0721$  \\
%Adj. R$^2$ & 94.23 \% & 86.98  \%   & 92.11 \%  \\ \bottomrule \toprule
%& Adap. lasso m. OLS vægte & Adap. lasso m. lasso vægte \\ \midrule
%$\log \del{ \widehat{\lambda}_\text{BIC}}$ &  $-2.6212$& $-1.8390$ \\
%$p$ & 2&2 \\
%BIC &  $-6.3153$&$-6.3142$ \\
%Adj. R$^2$ & 94.28 \% &  94.28 \% \\ \bottomrule
% \end{tabular}
%\caption{Logaritmen af $\widehat{\lambda}_\text{BIC}$, antallet af parametre, BIC og adjusted R$^2$ for lasso og dens generaliseringer.} \label{tab:bic_lambda}
%\end{table}

\begin{table}
\center
\begin{tabular}{lcccc  | lccccc} 
\toprule
 \multicolumn{5}{c}{Lasso} && \multicolumn{5}{c}{Ridge regression}  \\ \midrule
& $\log \del{\widehat{\lambda}_\text{BIC}}$  & BIC & $p$ & R$^2_{\text{adj}}$ && & $\log \del{\widehat{\lambda}_\text{BIC}}$  &  BIC & $p$ &R$^2_{\text{adj}}$  \\
$\widehat{\lambda}_\text{BIC} $&  $-6.2639$ & $-6.1608$ & 17 &  94.23 \% && $\widehat{\lambda}_\text{BIC} $ & $-4.4730$ & $-3.3230$ &  126 & 86.98 \% \\ \bottomrule \toprule 
 \multicolumn{5}{c}{Group lasso} && \multicolumn{5}{c}{Adap. lasso m. OLS vægte}  \\ \midrule
& $\log \del{\widehat{\lambda}_\text{BIC}}$  & BIC & $p$ &R$^2_{\text{adj}}$ && & $\log \del{\widehat{\lambda}_\text{BIC}}$  &  BIC & $p$ &R$^2_{\text{adj}}$   \\
$\widehat{\lambda}_\text{BIC}$ & $-7.2876$ &  $-5.0721$ &99  & 92.11\% &&  $\widehat{\lambda}_\text{BIC}$ & $-2.6212$ &  $-6.3153$  & 2 & $94.28 \%$ \\ \bottomrule \toprule 
 \multicolumn{5}{c}{Adap. lasso m. lasso vægte}  \\
& $\log \del{\widehat{\lambda}_\text{BIC}}$  & BIC & $p$ & R$^2_{\text{adj}}$\\
 $\widehat{\lambda}_\text{BIC}  $&  	$-1.8390$ & $-6.3142$  & 2& 94.28\% \\ \cmidrule{1-5}
 \end{tabular}
\caption{Logaritmen af $\widehat{\lambda}_\text{BIC}$, antallet af parametre, BIC og adjusted R$^2$ for lasso og dens generaliseringer.} \label{tab:bic_lambda}
\end{table}

På figur \ref{fig:coef_bic_coord} ser vi at variablerne \textcolor{blue3}{CLF16OV} og \textcolor{blue3}{CE16OV} har de højeste estimerede værdier, hvor de resterende er meget tæt på nul. 
Det samme er gældende for ridge og group lasso, vi har derfor valgt ikke at inkluderer disse koefficients plots. 

\imgfigh{coef_bic_coord.pdf}{1}{Estimerede koefficienter for lasso og adaptive lasso med OLS og lasso vægte, hvor $\widehat{\lambda}$ er fundet ud fra BIC. 
Farverne indikerer hvilken gruppe, variabler tilhører og y-aksen er variablerne udvalgt af lasso.}{coef_bic_coord}

Figurerne \ref{fig:resid_lasso_coord_bic}-\ref{fig:resid_adap_ols_coord_bic} viser en analyse af de standardiserede residualer.
Vi ser igen lidt af den samme tendens for lasso og dens generaliseringer. 
Histogrammet og QQ-plottet indikerer tungere haler end normalfordelingen og autokorrelation i det første lag. 
Dog har ridge regression og group lasso kun få outliers som ses på QQ-plottet. 
I tabel \ref{tab:res_shrinkage_tab} ser vi også, at vi ikke kan afvise LB testen omkring normalitet for group lasso når $\widehat{\lambda}$ er bestemt udfra BIC. 

\newpage
\subsection*{Inferens}
Figur \ref{fig:boxplot_lasso_coord_bic} viser boxplot for lasso, og vi ser at variablerne der ikke tilhører gruppe to oftes bliver valgt til at være nul, derudover er variablerne  \textcolor{blue3}{CLF16OV},  \textcolor{blue3}{CE16OV} , \textcolor{blue3}{lag1}, \textcolor{blue3}{UEMPL15OV}, og \textcolor{blue3}{UEMPLT5} fra gruppe to ofte estimeret til at være forskellige fra nul, dvs at lasso ofte vælger disse variable. 
Det samme er gældende for ridge og groupe lasso. 

Af tabel \ref{tab:fixedLassoInf_bic} observeres at nulhypotesen afvises for \textcolor{blue3}{CLF16OV}, \textcolor{blue3}{CE16OV}. ------

\begin{table}[h] 
\centering 
\scalebox{0.8}{
\begin{tabular}{llllllll}
\toprule
Prædiktor & Koefficient & Z-score & \(p\)-værdi & lowConfPt & UpConfPt & LowTailArea & UpTailArea\\
\midrule
\textcolor{red3}{DPCERA3M086SBEA}  & -0.002  &-0.960   &0.093  &   -0.071  &  0.003      & 0.000   &  0.050 \\
\textcolor{chartreuse4}{IPDMAT} &-0.002 & -0.680 &  0.159  &  -0.032 &   0.005     &  0.050    &  0.049 \\
\textcolor{blue3}{CLF16OV} & 0.241  &36.686  & 0.000 &    0.235   & 0.350    &   0.050  &    0.050 \\
\textcolor{blue3}{CE160V} &-0.264& -37.339   &0.000  &  -0.455  & -0.260   &    0.050   &   0.050 \\
\textcolor{blue3}{UEMPLT5}  & 0.000 &  0.027 &  0.777   & -0.029    &0.005    &   0.000  &     0.048 \\
\textcolor{blue3}{UEMP5TO14} & -0.001  & -0.266 & 0.599  & -0.007   &  0.014    &   0.050 &      0.050 \\
\textcolor{blue3}{UEMP15OV} &0.004  & 1.299  & 0.249   & -0.005   & 0.008  &     0.050     & 0.049 \\
\textcolor{blue3}{CLAIMSx} & 0.001 &  0.387  & 0.689   & -0.030   & 0.011    &  0.050     & 0.050 \\
\textcolor{blue3}{USCONS}  & -0.001  &  -0.591   &  0.100  &    -0.088  &     0.004  &      0.050  &       0.050 \\
\textcolor{blue3}{USTRADE}  & 0.000  & -0.118  &  0.988     & 0.007     &  Inf     &   0.050  &     0.000\\
\textcolor{red3}{AMDMNOx} &-0.002 &  -0.813 &  0.641  &  -0.008  &  0.020   &    0.049      &0.050 \\
\textcolor{orange}{TB6MS}&-0.001  &-0.415  & 0.677   & -0.008  &  0.023   &    0.049   &   0.000 \\
\textcolor{orange}{GS5} &-0.003 & -1.207  & 0.144    &-0.032  &  0.005      & 0.050    &  0.050 \\
\textcolor{orange}{EXUSUKx} & 0.003  & 1.449   &0.303   & -0.007   & 0.012      & 0.050    &  0.050 \\
\textcolor{cadetblue2}{CPIMEDSL}  &0.002 &  0.855 &  0.865&    -0.054 &   0.003&       0.050    &  0.050 \\
\textcolor{blue3}{lag1} & -0.010&  -4.362 &  0.499  &  -0.011   & 0.033   &    0.050  &    0.050 \\
\textcolor{blue3}{lag4}  & 0.002 &   1.106   & 0.311    & -0.014 &    0.028 &       0.036 &      0.050 \\
\bottomrule
\end{tabular}  
}
\caption{\(p\)-værdier og konfidensintervaller for variablerne udvalgt af lasso med BIC. Den estimeres standard afvigelse er 0.043, og resultaterne er for $\lambda \approx 1.0432$  med \(\alpha = 0.1\).} \label{tab:fixedLassoInf_bic}
\end{table} 


%% \begin{table}
\small
\center
\begin{tabular}{lllc}
\toprule
\multicolumn{1}{c}{Lasso} & \multicolumn{1}{c}{Elastisk net} \\ \midrule
Durable Materials (1) &Durable Materials (1)   \\
Ratio of Help Wanted/No. Unemployed (2) & Nondurable Materials (1)  \\
Civilian Labor Force (2) &  Civilian Labor Force (2) \\
Civilian Employment (2)& Civilian Employment (2) \\
Civilians Unemployed - Less Than 5 Weeks (2) & Average Duration of Unemployment (Weeks) (2) \\
Civilians Unemployed for 5-14 Weeks (2) & Civilians Unemployed - Less Than 5 Weeks (2)  \\
Civilians Unemployed - 15 Weeks \& Over (2)& Civilians Unemployed for 5-14 Weeks (2) \\
Initial Claims (2)& Civilians Unemployed - 15 Weeks \& Over (2) \\
All Employees: Construction (2)& Initial Claims (2) \\
Housing Starts, West (3)&All Employees: Construction (2) \\
Real personal Consumption expenditures (4)& Housing Starts, West (3) \\
New Orders for Durable Goods (4) & Real personal Consumption expenditures (4) \\
5-Year Treasure Rate (6) & New Orders for Durable Goods (4) \\
U.S. / U.K. Foreign Exchange Rate (6)&Nonrevolving consumer credit to Personal income (5) \\
& 5-Year Treasure Rate (6)  \\
& U.S. / U.K. Foreign Exchange Rate (6) \\
& PPI: MatLA ns metal products (7) \\
\bottomrule 
\toprule
\multicolumn{1}{c}{Adaptive lasso m. OLS vægte} & \multicolumn{1}{c}{ Adaptive lasso m. lasso vægte}  \\ \midrule
Civilian Labor Force (2) & Civilian Labor Force (2) \\
Civilian Employment (2) & Civilian Employment (2) \\
 \bottomrule 
\end{tabular}
\caption{Tabellen indeholder de forklarende variable, som bliver udvalgt af lasso, elastisk net og adaptive lasso med OLS og lasso vægte. Tallene i parantes indikerer hvilken gruppe de forskellige variable tilhører} \label{tab:bic_ud}
\end{table}
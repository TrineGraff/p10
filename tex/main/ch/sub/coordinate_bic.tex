\subsection{BIC}
Hernæst vil vi finde tuning parameteren med BIC. 
Vi bruger funktionen i appendiks \ref{sub:bic} til at finde $\widehat{\lambda}$. 

Tabel \ref{tab:bic_lambda} giver $\log \del{ \widehat\lambda}$, antallet af parameter, BIC og adjusted R$^2$ for hver model. 
Igen vælges \(\alpha = 1\) for elastik net, og derfor ser vi igen bort fra denne model.
Adjusted R$^2$ er igen højst for adaptive lasso og mindst for ridge regression. 
Præcis som ved anvendelse af krydsvalideringer, vælger adaptive lasso færrest antal variable, som igen er \texttt{CLF16OV} og \texttt{CE16OV}.
Derudover vælger group lasso hele 99 variabler. 

\begin{table}
\center
\begin{tabular}{llll} 
\toprule
& \multicolumn{1}{c}{${\widehat\lambda}$} & \multicolumn{1}{c}{BIC} & \multicolumn{1}{c}{p} \\ \midrule
Lasso & 0.0023 & -6.1424 & 14  \\
Ridge regression & 0.0112 & -4.2931 & 122 \\
Elastik net, $\alpha = 0.9$ & 0.0019 & -6.1202 &17 \\
Group lasso & $6.34 \cdot 10^{-5}$ & -5.0744 & 122 \\
Adaptive lasso med OLS vægte & 0.0599 & -6.3174 & 2 \\
Adaptive lasso med lasso vægte &  0.1058& -6.3174 & 2\\ \bottomrule
 \end{tabular}
\caption{Tabellen viser ${\widehat\lambda}$ fundet udfra BIC, samt BIC-værdien og antallet af parameter hver metode udvælger} \label{tab:bic_lambda}
\end{table}

På figur \ref{fig:coef_bic_coord} ser vi også det samme - variablerne \texttt{CLF16OV} og \texttt{CE16OV} har de markant højeste værdier, hvor de resterende er meget tæt på nul. 
Det samme er gældende for ridge og group lasso, vi har derfor valgt ikke at inkluderer disse koefficients plots. 

\imgfigh{coef_bic_coord.pdf}{0.8}{Estimerede koefficienter for lasso og adaptive lasso med OLS og lasso vægte med BIC. 
Farverne indikerer hvilken gruppe, variabler tilhører, og y-aksen er variablerne udvalgt af lasso.}{coef_bic_coord}

Figur \ref{fig:boxplot_lasso_coord_bic} viser boxplot for lasso, og vi ser at variablerne der ikke tilhører gruppe to oftes bliver valgt til at være nul, derudover er variablerne  \texttt{CLF16OV},  \texttt{CE16OV} , \texttt{lag1}, \texttt{UEMPL15OV}, og \texttt{UEMPLT5} fra gruppe to ofte estimeret til at være forskellige fra nul, dvs at lasso ofte vælger disse variable. 
Igen ser vi, for alle modeller at variablerne \texttt{CLF16OV} og \texttt{CE16OV} ofte bliver estimeret til at være forskellige fra nul. 


Af tabel \ref{tab:fixedLassoInf_bic} observeres at nulhypotesen afvises for \texttt{CLF16OV}, \texttt{CE16OV}. 

\begin{table}[h] 
\centering 
\scalebox{0.8}{
\begin{tabular}{llllllll}
\toprule
Prædiktor & Koefficient & Z-score & \(p\)-værdi & lowConfPt & UpConfPt & LowTailArea & UpTailArea\\
\midrule
\textcolor{red3}{DPCERA3M086SBEA}  & -0.002  &-0.960   &0.093  &   -0.071  &  0.003      & 0.000   &  0.050 \\
\textcolor{chartreuse4}{IPDMAT} &-0.002 & -0.680 &  0.159  &  -0.032 &   0.005     &  0.050    &  0.049 \\
\textcolor{blue3}{CLF16OV} & 0.241  &36.686  & 0.000 &    0.235   & 0.350    &   0.050  &    0.050 \\
\textcolor{blue3}{CE160V} &-0.264& -37.339   &0.000  &  -0.455  & -0.260   &    0.050   &   0.050 \\
\textcolor{blue3}{UEMPLT5}  & 0.000 &  0.027 &  0.777   & -0.029    &0.005    &   0.000  &     0.048 \\
\textcolor{blue3}{UEMP5TO14} & -0.001  & -0.266 & 0.599  & -0.007   &  0.014    &   0.050 &      0.050 \\
\textcolor{blue3}{UEMP15OV} &0.004  & 1.299  & 0.249   & -0.005   & 0.008  &     0.050     & 0.049 \\
\textcolor{blue3}{CLAIMSx} & 0.001 &  0.387  & 0.689   & -0.030   & 0.011    &  0.050     & 0.050 \\
\textcolor{blue3}{USCONS}  & -0.001  &  -0.591   &  0.100  &    -0.088  &     0.004  &      0.050  &       0.050 \\
\textcolor{blue3}{USTRADE}  & 0.000  & -0.118  &  0.988     & 0.007     &  Inf     &   0.050  &     0.000\\
\textcolor{red3}{AMDMNOx} &-0.002 &  -0.813 &  0.641  &  -0.008  &  0.020   &    0.049      &0.050 \\
\textcolor{orange}{TB6MS}&-0.001  &-0.415  & 0.677   & -0.008  &  0.023   &    0.049   &   0.000 \\
\textcolor{orange}{GS5} &-0.003 & -1.207  & 0.144    &-0.032  &  0.005      & 0.050    &  0.050 \\
\textcolor{orange}{EXUSUKx} & 0.003  & 1.449   &0.303   & -0.007   & 0.012      & 0.050    &  0.050 \\
\textcolor{cadetblue2}{CPIMEDSL}  &0.002 &  0.855 &  0.865&    -0.054 &   0.003&       0.050    &  0.050 \\
\textcolor{blue3}{lag1} & -0.010&  -4.362 &  0.499  &  -0.011   & 0.033   &    0.050  &    0.050 \\
\textcolor{blue3}{lag4}  & 0.002 &   1.106   & 0.311    & -0.014 &    0.028 &       0.036 &      0.050 \\
\bottomrule
\end{tabular}  
}
\caption{\(p\)-værdier og konfidensintervaller for variablerne udvalgt af lasso med BIC. Den estimeres standard afvigelse er ---, og resultaterne er for ---- med \(\alpha = 0.1\).} \label{tab:fixedLassoInf_bic}
\end{table} 

Figurerne \ref{fig:resid_lasso_coord_bic}-\ref{fig:resid_adap_ols_coord_bic} viser en an analyse af de standardiserede residualer.
Vi ser igen lidt af den samme tendens for alle metoderne. 
Histogrammet og QQ-plottet indikerer tungere haler end normalfordelingen og autokorrelation i det første lag. 
Dog har ridge regression og group lasso kun få outliers som ses på QQ-plottet. 
I tabel \ref{tab:res_shrinkage_bic_tab} ser vi også, at vi ikke kan afvise LB testen omkring normalitet for group lasso. 

%% \begin{table}
\small
\center
\begin{tabular}{lllc}
\toprule
\multicolumn{1}{c}{Lasso} & \multicolumn{1}{c}{Elastisk net} \\ \midrule
Durable Materials (1) &Durable Materials (1)   \\
Ratio of Help Wanted/No. Unemployed (2) & Nondurable Materials (1)  \\
Civilian Labor Force (2) &  Civilian Labor Force (2) \\
Civilian Employment (2)& Civilian Employment (2) \\
Civilians Unemployed - Less Than 5 Weeks (2) & Average Duration of Unemployment (Weeks) (2) \\
Civilians Unemployed for 5-14 Weeks (2) & Civilians Unemployed - Less Than 5 Weeks (2)  \\
Civilians Unemployed - 15 Weeks \& Over (2)& Civilians Unemployed for 5-14 Weeks (2) \\
Initial Claims (2)& Civilians Unemployed - 15 Weeks \& Over (2) \\
All Employees: Construction (2)& Initial Claims (2) \\
Housing Starts, West (3)&All Employees: Construction (2) \\
Real personal Consumption expenditures (4)& Housing Starts, West (3) \\
New Orders for Durable Goods (4) & Real personal Consumption expenditures (4) \\
5-Year Treasure Rate (6) & New Orders for Durable Goods (4) \\
U.S. / U.K. Foreign Exchange Rate (6)&Nonrevolving consumer credit to Personal income (5) \\
& 5-Year Treasure Rate (6)  \\
& U.S. / U.K. Foreign Exchange Rate (6) \\
& PPI: MatLA ns metal products (7) \\
\bottomrule 
\toprule
\multicolumn{1}{c}{Adaptive lasso m. OLS vægte} & \multicolumn{1}{c}{ Adaptive lasso m. lasso vægte}  \\ \midrule
Civilian Labor Force (2) & Civilian Labor Force (2) \\
Civilian Employment (2) & Civilian Employment (2) \\
 \bottomrule 
\end{tabular}
\caption{Tabellen indeholder de forklarende variable, som bliver udvalgt af lasso, elastisk net og adaptive lasso med OLS og lasso vægte. Tallene i parantes indikerer hvilken gruppe de forskellige variable tilhører} \label{tab:bic_ud}
\end{table}
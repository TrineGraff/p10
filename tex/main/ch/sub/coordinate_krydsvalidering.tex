\subsection{Krydsvalidering}
Funktionerne \texttt{cv.glmnet} og \texttt{cv.gglasso} fra pakkerne \texttt{glmnet} og \texttt{gglasso} udfører 10-fold krydsvalidering.
Figur \ref{fig:cv_plot} illustrerer den gennemsnitlige krydsvalideringsfejl samt øvre og nedre standardafvigelse for hver værdi af $\log \del{\lambda}$ for lasso og dens generaliseringer. 
De to lodrette stiplede linjer indikerer \(\lambda_{\text{min}}\) og \(\lambda_\text{1sd}\), hvor \(\lambda_{\text{min}}\) er værdien af \(\lambda\), som giver den mindste gennemsnitlige krydsvalideringsfejl og \(\lambda_\text{1sd}\) er den største værdi af \(\lambda\), således at fejlen er indenfor en standardafvigelse af minimum. 
For elastisk net finder vi, at $\alpha =1$ giver den mindste krydsvalideringsfejl, som svarer til lasso modellen og derfor betragter vi ikke elastisk net i dette afsnit. 
For adaptive lasso med OLS og lasso vægte finder vi, at $\gamma = 0.5$ giver den mindste krydsvalideringsfejl. 

\imgfigh{cv_plot.pdf}{1}{10-fold krydsvalideringsfejl som funktion af $\log \del{\lambda}$ for lasso og den generaliseringer. 
De stiplede linjer betegner \(\lambda_\text{min}\) og \(\lambda_\text{1sd}\).}{cv_plot}
%
For at give et bedre overblik giver tabel \ref{tab:cv_tab} værdierne af $\log \del{ \lambda_{\min}}$ og $\log \del{ \lambda_{1\text{sd}}}$, den gennemsnitlige krydsvalideringsfejl, antallet af parametre, justeret R$^2$ og log-likelihood for lasso og dens generaliseringer.


\begin{table}
\center
\begin{tabular}{cccc | cccccc}
\toprule
 &  \multicolumn{3}{c}{Lasso} &  \multicolumn{3}{c}{Ridge}  \\ \midrule
 & værdi & MSE & p & 	værdi & MSE & p \\
 $\lambda_{\min}$ &0.0014& 0.0019 & 20 	& 0.0123 &  0.0049 & 122 \\ 
 $\lambda_{1 \text{sd}}$ & 0.0027 & 0.0020 & 15 & 0.0148 & 0.0051 & 122  \\ \bottomrule \toprule
  &  \multicolumn{3}{c}{Elastic Net}  &  \multicolumn{3}{c}{Group Lasso}  \\ \midrule
 & værdi & MSE & p & værdi & MSE & p \\
 $\lambda_{\min}$ & 0.0014 & 0.0020 & 23 & 0.0003 & 0.0023  & 122\\
  $\lambda_{1\text{sd}}$ & 0.0027 & 0.0021 & 15 & 0.0005 & 0.0024 & 122 \\  \bottomrule \toprule
 &  \multicolumn{3}{c}{Adaptive lasso m. OLS vægte}  &  \multicolumn{3}{c}{Adaptive lasso m. lasso vægte}  \\ \midrule
  & værdi & MSE & p & værdi & MSE & p \\
 $\lambda_{\min}$  & 0.0868 & 0.0018 & 2 & 0.0034 & 0.0018 & 4   \\
 $\lambda_{1\text{sd}}$ & 0.4222 & 0.0018 & 2 & 0.0085 & 0.0019 & 4  \\ \bottomrule
 \end{tabular}
\caption{Tabellen viser $\lambda$ værdierne fundet udfra krydsvalidering, samt krydsvalideringsfejl, som er målt i MSE og antallet af parameter.} \label{tab:cv_tab}
\end{table}
 

For lasso ses en reducering i antallet af parametre for $\lambda_{1\text{sd}}$ i forhold til $\lambda_{\min}$, dette øger ikke krydsvalideringsfejlen betydeligt, og derfor anvendes $\lambda_{1\text{sd}}$ som tuning parameter for lasso. 
Ridge regression mindsker blot koefficienterne, og derfor vælges alle parametre. 
For ridge regression vælges \(\lambda_{\min}\) som tuning parameter, da den har mindst krydsvalideringsfejl.
For group lasso vælges lidt overraskende alle parametre med \(\lambda_\text{min}\), mens antallet af parametre reduceres med 7 for $\lambda_{1\text{sd}}$. 
Disse 7 variable tilhører alle gruppe 5. \\
Vi lader $\lambda_{1\text{sd}}$ være den optimale tuning parameter for group lasso, da den har det færreste antal parametre.
For adaptive lasso med OLS og lasso vægte vælges blot to variable for \(\lambda_{1\text{sd}}\), derfor lader vi $\lambda_{1\text{sd}}$ være tuning parameteren for disse.  
Justeret R\(^2\) er størst for lasso (CV) og mindst for ridge regression (CV).

På figur \ref{fig:coef_kryds_coord} vises de 14 estimerede koefficienter for lasso (CV) og de 2 estimerede koefficienter for adaptive lasso med OLS vægte (CV) og adaptive lasso med lasso vægte (CV).
Heraf ses, at lasso (CV) hovedsagligt vælger variable i samme gruppe som arbejdsløshedsraten.
For lasso (CV) ses, at variablerne \textcolor{blue3}{CLF16OV} og \textcolor{blue3}{CE16OV} har de største estimerede koefficienter, efterfulgt af \textcolor{blue3}{UEMPLT5}, \textcolor{blue3}{UEMP5TO14}, \textcolor{blue3}{UEMPL15OV} og \textcolor{blue3}{lag 1}, mens de øvrige koefficienter er meget tæt på nul. 
Figur \ref{fig:coef_ridge_kryds_coord} og \ref{fig:coef_gglasso_kryds_coord} viser de estimerede koefficienter for henholdsvis ridge regression (CV) og group lasso (CV).
Igen ser vi, at variablerne \textcolor{blue3}{CLF16OV} og \textcolor{blue3}{CE16OV} klart har de største estimerede koefficienter.    
%
\imgfigh{coef_kryds_coord.pdf}{1}{Estimerede koefficienter for lasso (CV), adaptive lasso med OLS vægte (CV) og adaptive lasso med lasso vægte (CV).
Farverne indikerer hvilken gruppe, variabler tilhører, og y-aksen er variablerne udvalgt af lasso (CV). }{coef_kryds_coord}

Figur \ref{fig:resid_lasso_coord_kryds}-\ref{fig:resid_adap_ols_coord_kryds} viser en analyse af de standardiserede residualer for lasso og dens generaliseringer. 
Vi ser samme tendens for alle modeller. Histogrammet og QQ-plottet indikerer tungere haler end normalfordelingen, og vi observerer autokorrelation i første lag.
Dette bekræftes i tabel \ref{tab:res_shrinkage_tab}, som viser skewness, excess kurtosis, $p$-værdier for JB testen og LB testen for de standardiserede residualer.
Vi ser, at alle modeller har negativ skewness og excess kurtosis forskellige fra nul. 
Derudover afvises JB testens nulhypotese om normalitet for alle modeller med undtagelse af group lasso (CV).
For LB testen afvises nulhypotesen om at der ingen autokorrelation er.

Figur \ref{fig:boxplot_lasso_coord_kryds} viser bootstrap resultaterne for variablerne udvalgt af lasso (CV).
Variablerne \textcolor{blue3}{lag 1}, \textcolor{blue3}{UEMPL15OV}, \textcolor{blue3}{UEMP5TO14}, \textcolor{blue3}{UEMPLT5}, \textcolor{blue3}{CE16OV} og \textcolor{blue3}{CLF16OV} udvælges ofte, hvilket stemmer overens med, at disse variable har de største estimerede koefficienter.
Omvendt fravælges variablerne \textcolor{orange}{TB6MS}, \textcolor{blue3}{PAYEMS} og \textcolor{red3}{DPCERA3M086SBEA} over 60\% af bootstrap realisationerne. 
Generelt har variablerne, som ikke tilhører gruppe 2, en større sandsynlighed for at blive fravalg.
Da adaptive lasso har konsistent variabeludvælgelse (se definition \ref{defn:orakel}), vil variablerne \textcolor{blue3}{CLF16OV} og \textcolor{blue3}{CE16OV} altid vælges, og derfor laves der ikke bootstrap for adaptive lasso med OLS vægte (CV) og adaptive lasso med lasso vægte (CV) .

\subsubsection{TG testen}
Resultaterne af TG testen er givet i tabel \ref{tab:fixedLassoInf}.
Heraf ser vi, at variablerne \textcolor{blue3}{CLF16OV} og \textcolor{blue3}{CE16OV} har de største estimerede koefficienter og $Z$-scores.
%Vi observerer, at $Z$-scoren er meget stor for variablerne \textcolor{blue3}{CLF16OV} og \textcolor{blue3}{CE16OV}, mens den er relativ lav for de resterende variable. 
Nulhypotesen afvises for variablerne \textcolor{blue3}{CLF16OV}, \textcolor{blue3}{CE16OV} og \textcolor{blue3}{lag 1}. Altså er disse signifikante.

Figur \ref{fig:resid_tg_kryds} viser en analyse af de standardiserede residualer for lasso$_{TG}$ (CV). 
Igen observeres autokorrelation i første lag, mens QQ-plottet indikerer tungere haler end normalfordeling. 
Dette bekræftes i tabel \ref{tab:res_shrinkage_tab}.

\begin{table}[h] 
\centering 
\scalebox{0.8}{
\begin{tabular}{llllllll}
\toprule
Prædiktor & Koefficient & Z-score & \(p\)-værdi & lowConfPt & UpConfPt & LowTailArea & UpTailArea\\
\midrule
\textcolor{red3}{DPCERA3M086SBEA} & -0.002 & -1.365 &  0.666  &  -0.009  &  0.026   &    0.050   &   0.050 \\
\textcolor{chartreuse4}{IPDMAT} & -0.003  &-1.113  & 0.268  &  -0.012 &   0.006    &   0.050    &  0.049 \\
\textcolor{blue3}{HWIURATIO}  & 0.002 &  0.718 &  0.197 &   -0.003 &   0.014  &     0.049  &    0.050 \\
\textcolor{blue3}{CLF16OV} & 0.243 & 36.668  & 0.000   &  0.232  &  0.259     &  0.049  &    0.049 \\
\textcolor{blue3}{CE16OV} &  -0.266 & -37.390 &  0.000  &  -0.280 &  -0.254   &    0.049  &    0.049\\
\textcolor{blue3}{UEMPLT5} & 0.001  & 0.241  & 0.402  &  -0.005  &  0.008     &  0.049   &   0.049\\
\textcolor{blue3}{UEMP5TO14} & 0.000 & -0.118 &  0.429  &  -0.007  &  0.004   &    0.049  &    0.049\\
\textcolor{blue3}{UEMP15OV}& 0.004  & 1.593  & 0.056 &    0.000  &  0.009  &     0.049    &  0.050 \\
\textcolor{blue3}{PAYEMS} & 0.001 &  0.273  & 0.223  &  -0.007  &  0.029     &  0.050   &   0.050 \\
\textcolor{blue3}{USCONS} & -0.002 & -0.883 &  0.569 &   -0.009  &  0.016  &     0.050  &    0.000 \\
\textcolor{orange}{TB6MS} & -0.001 & -0.480 &  0.682 &   -0.009  &  0.026  &     0.050   &   0.050 \\
\textcolor{orange}{GS5} & -0.003 & -1.131  & 0.219  &  -0.025  &  0.007     &  0.050   &   0.049\\
\textcolor{orange}{EXUSUKx} & 0.003 &  1.306 &  0.872 &   -0.072  &  0.003  &     0.050  &    0.050 \\
\textcolor{blue3}{lag 1} & -0.009 & -4.067  & 0.003 &   -0.013 &  -0.004  &     0.049    &  0.049 \\
\bottomrule
\end{tabular}  
}
\caption{\(p\)-værdier og konfidensintervaller for variablerne udvalgt af lasso. Den estimeres standard afvigelse er \(0.043\), og resultaterne er for \(\lambda \approx 1.823\) med \(\alpha = 0.1\).} \label{fig:fixedLassoInf}
\end{table} 
%

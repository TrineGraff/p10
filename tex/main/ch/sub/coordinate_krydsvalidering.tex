\subsection{Krydsvalidering}
Funktionerne \texttt{cv.glmnet} og \texttt{cv.gglasso} fra pakkerne \texttt{glmnet} og \texttt{gglasso} udfører 10-fold krydsvalidering.
Figur \ref{fig:cv_plot} illustrerer den gennemsnitlige krydsvalideringsfejl samt øvre og nedre standardafvigelse for hver værdier af $\log \del{\lambda}$ for lasso og dens generaliseringer. 
De to lodrette stiplede linjer indikerer \(\lambda_{\text{min}}\) og \(\lambda_\text{1sd}\), hvor \(\lambda_{\text{min}}\) er værdien af \(\lambda\), som giver den mindste gennemsnitlige krydsvalideringsfejl og \(\lambda_\text{1sd}\) er den største værdi af \(\lambda\), således at fejlen er indenfor en standardafvigelse af minimum. 
For elastisk net finder vi, at $\alpha =1$ giver den mindste krydsvalideringsfejl, som svarer til lasso modellen, derfor betragter vi ikke elastik net i dette afsnit. 
For adaptive lasso med OLS og lasso vægte finder vi, at $\gamma = 0.5$ giver den mindste krydsvalideringsfejl. 

\imgfigh{cv_plot.pdf}{1}{10-fold krydsvalideringsfejl som funktion af $\log \del{\lambda}$ for lasso og den generaliseringer. 
De stiplede linjer betegner \(\lambda_\text{min}\) og \(\lambda_\text{1sd}\).}{cv_plot}
%
<<<<<<< Updated upstream
For at give et bedre overblik giver tabel \ref{tab:cv_tab} værdierne af $\log \del{ \lambda_{\min}}$ og $\log \del{ \lambda_{1\text{sd}}}$, gennemsnitlig krydsvalideringsfejl, antallet af parametre, justeret R$^2$ og log-likelihood for lasso og dens generaliseringer.
=======
For at give et bedre overblik giver tabel \ref{tab:cv_tab} værdierne af $\log \del{ \lambda_{\min}}$ og $\log \del{ \lambda_{1\text{sd}}}$, krydsvalideringsfejlen, antallet af parametre, justeret R$^2$ og log-likelihood for lasso og dens generaliseringer.
>>>>>>> Stashed changes

\begin{table}[ht]
\center
\begin{tabular}{lcccc | lccccc}
\toprule
   \multicolumn{5}{c}{Lasso} & \multicolumn{1}{c}{ }&  \multicolumn{5}{c}{Ridge regression}  \\ \midrule
 & \(\log \del{\lambda}\) & MSE & $p$ & Adj. R$^2$ &&& \(\log \del{\lambda}\) & MSE & $p$ & Adj. R$^2$\\
 $\lambda_{\min}$ &$-6.6361$& 0.0019 & 28 & 94.28\% &&  $\boldsymbol{\lambda_{\min}}$ &  $\mathbf{-4.3800}$ &   $\mathbf{0.0045} $&  $\mathbf{126}$ & $ \mathbf{86.56 \% }$ \\ 
 $\boldsymbol{\lambda}_{\textbf{1sd}}$ & $\mathbf{-5.7057}$ & $\mathbf{0.0020} $ & $\mathbf{14}$ &$\textbf{ 93.81} \boldsymbol{\%}$ && $\lambda_{ \text{1sd}}$& $-4.1939$ & 0.0047 & 126 &  85.70\%  \\ \bottomrule \toprule
\multicolumn{5}{c}{Group lasso}  &&  \multicolumn{5}{c}{Adap. lasso m. OLS vægte}  \\ \midrule
& \(\log \del{\lambda}\) & MSE &$ p $ &Adj. R$^2$ &&& \(\log \del{\lambda}\) & MSE & $p$ & Adj. R$^2$ \\
$\lambda_{\min}$& $-8.2644$ & 0.0022  & 126 & 93.23\% && $\boldsymbol{\lambda_{\min}}$  & $\mathbf{-2.0630}$ &$ \mathbf{0.0018}$ & $\mathbf{2}$ & $\textbf{94.27} \%$ \\
  $\boldsymbol{\lambda}_{\textbf{1sd}}$  & $\mathbf{-7.6365}$ &$ \mathbf{0.0023}$ & $\mathbf{119}$ &$ \textbf{92.60} \%$ &&  $\lambda_{1\text{sd}}$ & $-0.3884$ & 0.0019 & 2 &  93.93\%\\  \bottomrule 
  \toprule
  \multicolumn{5}{c}{Adap. lasso m. lasso vægte}  \\ \cmidrule{1-5}
& \(\log \del{\lambda}\) & MSE & $p$ & Adj. R$^2$\\
$\boldsymbol{\lambda_{\min}}$   &  $ \mathbf{-2.3674}$ & $ \mathbf{0.0018} $& $ \mathbf{2}$ &   $\textbf{94.28} \%$ \\
$\lambda_{1\text{sd}}$  & $-0.2276$ & 0.0019 & 2 & 93.92\%\\ \cmidrule{1-5}
 \end{tabular}
\caption{Logaritmen af $\lambda_{min}$ og $\lambda_{1\text{sd}}$, gennemsnitlig krydsvalideringsfejl, som er målt i MSE, antallet af paramete og adjusted R$^2$ for lasso og dens generaliseringer. De valgte tuning parameter for hver metode er markeret med tykt.} \label{tab:cv_tab}
\end{table}
 

For lasso ses en reducering i antallet af parametre for $\lambda_{1\text{sd}}$ i forhold til $\lambda_{\min}$, dette øger ikke krydsvalideringsfejlen betydeligt, og derfor anvendes $\lambda_{1\text{sd}}$ som tuning parameter for lasso. 
Ridge regression mindsker blot koefficienterne, og derfor vælges alle parametre. 
For ridge regression vælges \(\lambda_{\min}\) som tuning parameter, da den har mindst krydsvalideringsfejl.
For group lasso vælges lidt overraskende alle parametre med \(\lambda_\text{min}\), mens antallet af parametre reduceres med 7 for $\lambda_{1\text{sd}}$. 
Disse 7 variable tilhører alle gruppe 5.
Vi lader $\lambda_{1\text{sd}}$ være den optimale tuning parameter for group lasso, da den har det færreste antal parametre.
For adaptive lasso med OLS og lasso vægte vælges blot to variable for \(\lambda_{1\text{sd}}\), derfor lader vi $\lambda_{1\text{sd}}$ være tuning parameteren.  
Justeret R\(^2\) er størst lasso (CV) og mindst for ridge regression (CV).

På figur \ref{fig:coef_kryds_coord} vises de 14 estimerede koefficienter for lasso (CV) og de 2 estimerede koefficienter for adaptive lasso med OLS vægte (CV) og adaptive lasso med lasso vægte (CV).
Heraf ses, at lasso (CV) hovedsagligt vælger variable i samme gruppe som arbejdsløshedsraten.
For lasso (CV) ses, at variablerne \textcolor{blue3}{CLF16OV} og \textcolor{blue3}{CE16OV} har de største estimerede koefficienter, efterfulgt af \textcolor{blue3}{UEMPLT5}, \textcolor{blue3}{UEMP5TO14}, \textcolor{blue3}{UEMPL15OV} og \textcolor{blue3}{lag 1}, mens de øvrige koefficienter er meget tæt på nul. 
Figur \ref{fig:coef_ridge_kryds_coord} og \ref{fig:coef_gglasso_kryds_coord} viser de estimerede koefficienter for henholdsvis ridge regression (CV) og group lasso (CV).
Igen ser vi, at variablerne \textcolor{blue3}{CLF16OV} og \textcolor{blue3}{CE16OV} klart har de største estimerede koefficienter.    
%
\imgfigh{coef_kryds_coord.pdf}{1}{Estimerede koefficienter for lasso (CV), adaptive lasso med OLS vægte (CV) og adaptive lasso med lasso vægte (CV).
Farverne indikerer hvilken gruppe, variabler tilhører, og y-aksen er variablerne udvalgt af lasso (CV). }{coef_kryds_coord}

Figur \ref{fig:resid_lasso_coord_kryds}-\ref{fig:resid_adap_ols_coord_kryds} viser en analyse af de standardiserede residualer for lasso og dens generaliseringer. 
Vi ser samme tendens for alle modeller. Histogrammet og QQ-plottet indikerer tungere haler end normalfordelingen og vi observerer autokorrelation i første lag.
Dette bekræftes i tabel \ref{tab:res_shrinkage_tab}, som viser skewness, excess kurtosis, $p$-værdier for JB testen og LB testen for de standardiserede residualer.
Vi ser, at alle modeller har negativ skewness og excess kurtosis forskellige fra nul. 
Derudover afvises JB testens nulhypotesen om normalitet for alle modeller med undtagelse af group lasso (CV), som har en lille skewness og excess kurtosis.
For LB testen afvises nulhypotesen om uafhængighed for alle modeller.

Figur \ref{fig:boxplot_lasso_coord_kryds} viser bootstrap resultaterne for variablerne udvalgt af lasso (CV).
Variablerne \textcolor{orange}{TB6MS}, \textcolor{blue3}{PAYEMS} og \textcolor{red3}{DPCERA3M086SBEA} fravælges over 50\% af bootstrap realisationerne, mens variablerne  \textcolor{blue3}{lag 1}, \textcolor{blue3}{UEMPL15OV}, \textcolor{blue3}{UEMP5TO14}, \textcolor{blue3}{UEMPLT5}, \textcolor{blue3}{CE16OV} og \textcolor{blue3}{CLF16OV} ofte vælges.
Generelt fravælges variablerne, som ikke tilhører gruppe 2.
Resultaterne er ikke overraskende i forhold til størrelsen af de estimerede koefficienter for lasso (CV).
Da adaptive lasso har konsistent variabeludvælgelse (se definition \ref{defn:orakel}), vil variablerne \textcolor{blue3}{CLF16OV} og \textcolor{blue3}{CE16OV} altid vælges, og derfor laves der ikke bootstrap for adaptive lasso med OLS vægte (CV) og adaptive lasso med lasso vægte (CV) .

\subsubsection{TG testen}
Lad $\boldsymbol{\eta} = s_k \del{\textbf{X}^+_{\mathcal{A}_k}}^T \mathbf{e}_k$, da er $\mathcal{H}_0: \beta_k = 0$ jævnfør \eqref{eq:tg_beta}. 
Resultaterne af TG testen er givet i tabel \ref{tab:fixedLassoInf}.
Heraf ser vi, at variablerne \textcolor{blue3}{CLF16OV} og \textcolor{blue3}{CE16OV} har de største estimerede koefficienter og $Z$-score.
%Vi observerer, at $Z$-scoren er meget stor for variablerne \textcolor{blue3}{CLF16OV} og \textcolor{blue3}{CE16OV}, mens den er relativ lav for de resterende variable. 
Nulhypotesen afvises for variablerne \textcolor{blue3}{CLF16OV}, \textcolor{blue3}{CE16OV} og \textcolor{blue3}{lag 1}, hvilket betyder, at disse er signifikante.

Figur \ref{fig:resid_tg_kryds} viser en analyse af de standardiserede residualer for lasso$_{TG}$ (CV). 
Igen observeres autokorrelation i første lag, mens QQ-plottet indikerer tungere haler end normalfordeling. 
Dette bekræftes også i tabel \ref{tab:res_shrinkage_tab}.

\begin{table}[ht] 
\centering 
\begin{tabular}{llllllll}
%\multicolumn{4}{c}{Lasso} \\
\toprule
Prædiktor & Koefficient & Z-score & \(p\)-værdi & lowConfPt & UpConfPt & LowTailArea & UpTailArea \\
\midrule
3 & -0.002 & -1.372 & 0.649 & -0.009 & 0.025 & 0.050 & 0.050 \\
14 & -0.003 & -1.111 &  0.275 &   -0.011 &   0.006 & 0.050 & 0.049 \\
21 & 0.002 & 0.723 & 0.191 & -0.003 & 0.014 & 0.049 & 0.050 \\
22 & 0.243 & 36.619 & 0.000 & 0.232 & 0.259 & 0.048 & 0.049 \\
23 & -0.266 & -37.351 & 0.000 & -0.280 & -0.254 & 0.049 & 0.049 \\
25 & 0.001 & 0.243 & 0.401 & -0.005 & 0.008 & 0.049 & 0.049 \\
26 &  0.000 & -0.120 &  0.425 & -0.007 & 0.004 & 0.049 & 0.049 \\
27 & 0.004 & 1.590 & 0.057 & 0.000 & 0.009 &  0.049 & 0.050 \\
31 & 0.001 & 0.249 & 0.236 & -0.007 & 0.027  & 0.049 & 0.050 \\
34 & -0.002 & -0.880 & 0.578 & -0.009 & 0.016 & 0.050 & 0.000 \\
78 & -0.001 & -0.480 & 0.683 & -0.009 & 0.027 & 0.050 & 0.050 \\
80 & -0.003 & -1.131 &  0.218 & -0.025 & 0.007  & 0.050 & 0.050 \\
94 & 0.003 & 1.301 & 0.877 & -0.075 & 0.003 & 0.050 & 0.050 \\
123 & -0.009 & -4.070 & 0.003 & -0.013 & -0.004 & 0.050 & 0.049 \\
\bottomrule
\end{tabular}  
\caption{-}
\end{table} 
%

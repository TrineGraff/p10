\section{Autoregressiv model}
Den autoregressive model af orden \(p\) estimeres med OLS, og ordenen bestemmes udfra BIC.
Vi lader $p = 1, \ldots, p_{\max}$, hvor \(p_\text{max}=12\), da vi mener, at arbejdsløshedsraten ikke vil påvirkes mere end et år tilbage.
BIC vælger \(p=4\) og tabel \ref{tab:est_ar} giver estimeringsresultaterne for en \(\text{AR} \del{4}\). \footnote{For en ARMA(p,q) hvor $p_{\max} = 12$ og $q_{\max} = 12$, hvor parametrene estimeres med MLE, vælger BIC også en AR(4).}
%
\begin{table}[h]
\center
\begin{tabular}{ll}
\toprule
$\widehat{\phi}_1$ &$ -0.0162 $ \\
$\widehat{\phi}_2$ & $0.1992^{***}$  \\
$\widehat{\phi}_3$ &$0.1873^{***}$  \\
$\widehat{\phi}_4$ &$0.1686^{***} $ \\ \midrule
BIC & -3.5651 \\
 R$^2_{\text{adj}}$ & 12.31\% \\
LogLik &  211.8617\\ \bottomrule
 \end{tabular}
\caption{Estimationsresultater for en \(\text{AR} \del{4}\), BIC, justeret R$^2$ og log-likehood. Det opløftede symbol betegner signifikans ved henholdsvis $^{***}$0.1\%, $^{**}$1\%, $^{*}$5\% og $^{\dagger}$10\%.} \label{tab:est_ar}
\end{table}

Alle koefficienterne med undtagelse af $\widehat\phi_1$ er signifikante ved et 0.1 \% niveau. 
Vi ser også, at adjusted $R^2$ kun er 12.30\% (se definition \ref{def:adjr2}), hvilket indikerer, at en autoregressiv model af orden 4 ikke er en optimal model for arbejdsløshedsraten. 

\begin{table}
\center
\begin{tabular}{lcc} \toprule
Skewness & 0.2666 \\
Kurtosis & 1.4773 \\
JB-test & 0 \\ 
LB$_{10}$-test & 0 \\ \bottomrule
\end{tabular}
\caption{Skewness, excess kurtosis og \(p\)-værdier for Jarque-Bera og Ljung-Box testen for de standardiserede residualer af en \(\text{AR} \del{4}\). Vi lader LB$_{10}$ betegne Ljung-Box testen med lag = 10. } \label{tab:test_ar}
\end{table}

Figur \ref{fig:resid_ar} viser en analyse af de standardiserede residualer for \(\text{AR} \del{4}\). 
Histogrammet og QQ-plottet indikerer, at residualerne har tungere haler end normalfordelingen. 
Derudover observeres få signifikante autokorrelationer.
Dette bekræftes i tabel \ref{tab:test_ar}, som viser skewness, excess kurtosis og \(p\)-værdier for Jarque-Bera og Ljung-Box testen (se definition \ref{def:jbtest} og \ref{def:lbtest}) for de standardiserede residualer. 
Vi ser, at skewness og kurtosis er positive, JB testen afviser nulhypotesen om normalitet og LB testen i lag 10 afviser nulhypotesen om uafhængighed.


Forecast er som nævnt opnået ved et rolling scheme med expanding estimerings window. 
Figur \ref{fig:fc_ar} viser arbejdsløshedsraten og den prædikterede arbejdsløshedsrate med en AR\(\del{4}\).
Vi kan se, at en AR(4) ikke fanger udsvingene, hvilket igen tyder på, at det ikke er en særlig god model. 
For AR(4) fås en MAE på 0.1312 og MSE på 0.0272.
%I tabel \ref{tab:tabs_ar} er MAE og MSE udregnet for modellen. 
%\begin{table}
\center
\begin{tabular}{cc}
\toprule
 MAE & MSE \\ \midrule
 0.1312 & 0.0272 \\ \bottomrule
\end{tabular}
\caption{MAE og MSE for \(\text{AR} \del{4}\).} \label{tab:tabs_ar}
\end{table}


\imgfigh{fc_ar.pdf}{1}{Arbejdsløshedsraten og prædiktion af arbejdsløshedsraten med en \(\text{AR} \del{4}\).}{fc_ar}
\newpage
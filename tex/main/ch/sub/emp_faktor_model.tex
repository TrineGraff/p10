\section{Faktor model}
Parametrene i faktor modellen estimeres udfra følgende procedure:
\begin{enumerate}
\item Estimér $k$ faktorer for $k = 1, \dots, k_{\max}$, hvor $k_{\max} = 20$.  
\item Vælg $k$ udfra informationskriterierne givet i \eqref{eq:ic1}-\eqref{eq:ic3}.
\item Lad \(\widehat{\textbf{Z}} = \del{\widehat{{\textbf{F}}}^T \ \boldsymbol{\omega}^T}^T\) være en \(\del{k+m} \times T\) matrix, hvor \(\widehat{{\textbf{F}}}\) er en \(T \times k\) matrix af estimerede faktorer og \(\boldsymbol{\omega}\) er en \(T \times m\) matrix af laggede værdier af arbejdsløshedsraten.
Vi lader \(m = 4\), da den autoregressive model valgte en orden på 4.
Dette medfører nogle uobserverede værdier, og derfor fjernes de første 4 rækker i \(\widehat{\textbf{Z}}\).
\item Estimér parametrene $\widehat{\boldsymbol{\beta}} = \del{ \widehat{\boldsymbol{\beta}}^T_{\textbf{F}} \ \widehat{\boldsymbol{\beta}}^T_{\boldsymbol{\omega}}}^T$ med OLS.
\end{enumerate}

Tabel \ref{tab:est_faktor} viser antallet af faktorer, værdien af informationskriteriet, justeret \(R^2\) samt log-likelihood for hver af de tre informationskriterier. 

%\begin{table}[h]
%\center
%\begin{tabular}{lccc}
%\toprule
%& Faktor model (IC$_1$) & Faktor model (IC$_2$) & Faktor model (IC$_3$) \\ \midrule
%$\widehat{k}$ & 6 & 11 & 20 \\ 
%IC$\del{\widehat{k}}$ & $-0.3519$  & $-0.5314$ & $-0.6931$  \\
%R$^2_{\text{adj}}$  & 37.08 \% & 50.85 \% & 60.23 \% \\
%LogLike & - & - & -\\ \bottomrule
% \end{tabular}
%\caption{Antal faktorer, værdien af informationskriteriet for dette antal faktorer samt adjusted \(R^2\) for faktormodellerne valgt udfra IC$_1$, IC$_2$ og IC$_3$, som betegnes faktor model (IC\(_1\)), faktor model (IC\(_2\)) og faktor model (IC\(_3\)).} \label{tab:est_faktor}
%\end{table}

\begin{table}[h]
\center
\begin{tabular}{lccccc}
\toprule
\multicolumn{5}{c}{Faktor model (IC$_1$)} \\ \midrule
& Værdi &  IC$_1$ &  R$^2_{\text{adj}}$ & LogLik  \\
$k$ & 6 &  $-0.3519$ &  15.79\% &  224.3621  \\ \bottomrule \toprule
\multicolumn{5}{c}{Faktor model (IC$_2$)} \\ \midrule
 & Værdi &  IC$_2$ &  R$^2_{\text{adj}}$ & LogLik \\
 $k$ &11 & $-0.5314$ &  16.85\% &  230.3414 \\\bottomrule \toprule
\multicolumn{5}{c}{Faktor model (IC$_3$)} \\ \midrule
& Værdi &  IC$_3$ &  R$^2_{\text{adj}}$ & LogLik\\
$k$ & 20 & $-0.6931$ & 17.87\% & 238.3753 \\  \bottomrule
 \end{tabular}
 \caption{Antal faktorer, værdien af informationskriteriet, justeret \(R^2\) samt log-likehood for faktormodellerne valgt udfra IC$_1$, IC$_2$ og IC$_3$, som betegnes faktor model (IC\(_1\)), faktor model (IC\(_2\)) og faktor model (IC\(_3\)).} \label{tab:est_faktor}
\end{table}
Vi observerer, at IC$_1$ vælger den mindst komplekse model, da den vælger det færreste antal faktorer, mens IC$_3$ vælger det maksimale antal faktorer. \footnote{For \(k_\text{max} = 50\) vælger IC\(_3\) stadig det maksimale antal faktorer. Da \(k<k_\text{max}\) betragtes faktor modellen valgt udfra IC\(_3\) ikke.} 

Figur \ref{fig:ic1_res} og \ref{fig:ic2_res} viser en analyse af de standardiserede residualer for faktor modellerne.
Af QQ-plottet ses, at residualerne har tungere haler end normalfordelingen, som bekræftes i tabel \ref{tab:test_faktor}, hvor JB testens nulhypotesen om normalitet afvises.
LB testen i lag 10 kan ikke afvise, at residualerne er uafhængige.

\begin{table}
\center
\begin{tabular}{lccccccccc} \toprule
& \multicolumn{2}{c}{IC$_1$} & & \multicolumn{2}{c}{IC$_2$} & &\multicolumn{2}{c}{IC$_3$} \\ \midrule
Skewness & \multicolumn{2}{c}{0.0444} & & \multicolumn{2}{c}{$-0.0418$}  & & \multicolumn{2}{c}{$-0.0724$}   \\
Kurtosis & \multicolumn{2}{c}{0.5768} & & \multicolumn{2}{c}{0.4612}  & & \multicolumn{2}{c}{0.2951}\\
JB-test & \multicolumn{2}{c}{0.0172} & & \multicolumn{2}{c}{0.0712}  & & \multicolumn{2}{c}{0.2678} \\ \cmidrule{2-3}\cmidrule{5-6} \cmidrule{8-9} 
& $e_t$ & $e_t^2$ && $e_t$ & $e_t^2$  && $e_t$ & $e_t^2$  \\
LB$_{10}$ & -  &  - && -  &  -&& - & - \\ \bottomrule
\end{tabular}
\caption{Skewness, excess kurtosis, p -værdier for Jarque Beta og Ljung Box test for de standardiserede residualer fra faktor modellerne valgt ud fra IC$_1$, IC$_2$ og IC$_3$. Vi lader LB$_{10}$ betegne Ljung-Box test med lag = 10. } \label{tab:test_faktor}
\end{table}

Tabel \ref{tab:factor_mse_tab} viser MAE og MSE for faktor modellernes prædiktion.
Heraf ses at faktor model (IC\(_2\)) har mindst MAE og MSE.

\begin{table}
\center
\begin{tabular}{lcccc}
\toprule
& $\text{IC}_1$ & $\text{IC}_2$ & $\text{IC}_3$ \\
\midrule 
MAE & 0.1190 & 0.1111 & 0.1048  \\ 
MSE &  0.0221  & 0.0187  & 0.0165 \\ \bottomrule
 \end{tabular}
\caption{MSE og MAE for de forskellige informations kriterier} \label{tab:factor_mse_tab}
\end{table}

Figur \ref{fig:fc_benchmark2} viser arbejdsløshedsraten og den prædikterede arbejdsløshedsrate med faktor model (IC\(_2\)).
\imgfigh{fc_benchmark2.pdf}{1}{Arbejdsløshedsraten og prædiktionen af arbejdsløshedsraten med faktor model (IC\(_2\)).}{fc_benchmark2}

Vi vælger faktor model (IC\(_2\)) som benchmark model, da den har mindst MAE og MSE og størst justeret \(R^2\) sammenlignet med AR(4) og faktor model (IC\(_1\)).

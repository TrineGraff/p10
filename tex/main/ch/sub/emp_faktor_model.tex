\section{Faktor model}
Parametrene i faktor modellen estimeres udfra følgende procedure:
\begin{enumerate}
\item Estimér $k$ faktorer for $k = 1, \dots, k_{\max}$, hvor $k_{\max} = 20$.  
\item Vælg $k$ udfra informationskriterierne givet i \eqref{eq:ic1}-\eqref{eq:ic3}.
\item Lad \(\widehat{\textbf{Z}} = \del{\widehat{{\textbf{F}}}^T \ \boldsymbol{\omega}^T}^T\) være en \(\del{k+m} \times T\) matrix, hvor \(\widehat{{\textbf{F}}}\) er en \(T \times k\) matrix af estimerede faktorer og \(\boldsymbol{\omega}\) er en \(T \times m\) matrix af laggede værdier af arbejdsløshedsraten.
Vi lader \(m = 4\), da den autoregressive model valgte en orden på 4.
Dette medfører nogle uobserverede værdier, og derfor fjernes de første 4 rækker i \(\widehat{\textbf{Z}}\).
\item Estimér parametrene $\widehat{\boldsymbol{\beta}} = \del{ \widehat{\boldsymbol{\beta}}^T_{\textbf{F}} \ \widehat{\boldsymbol{\beta}}^T_{\boldsymbol{\omega}}}^T$ med OLS.
\end{enumerate}

Tabel \ref{tab:est_faktor} viser antallet af faktorer, værdien af informationskriteriet, justeret \(R^2\) samt log-likelihood for hver af de tre informationskriterier. 

\begin{table}[h]
\center
\begin{tabular}{lccc}
\toprule
& IC$_1$ & IC$_2$ & IC$_3$ \\ \midrule
$\widehat{k}$ & 6 & 11 & 20 \\ 
IC$\del{\widehat{k}}$ & $-0.3519$  & $-0.5314$ & $-0.6931$  \\
Adj. R$^2$ & 37.08 \% & 50.85 \% & 60.23 \% \\ \bottomrule
 \end{tabular}
\caption{Antal faktorer, værdien af informationskriteriet for dette antal faktorer samt adjusted \(R^2\) for IC$_1$, IC$_2$ og IC$_3$.} \label{tab:est_faktor}
\end{table}
Vi observerer, at IC$_1$ vælger den mindst komplekse model, da den vælger det færreste antal faktorer, mens IC$_3$ vælger det maksimale antal faktorer. \footnote{For \(k_\text{max} = 50\) vælger IC\(_3\) stadig det maksimale antal faktorer. Da \(k<k_\text{max}\) betragtes faktor modellen valgt udfra IC\(_3\) ikke.} 

Figur \ref{fig:ic1_res} og \ref{fig:ic2_res} viser en analyse af de standardiserede residualer for faktor modellerne.
Af QQ-plottet ses, at residualerne har tungere haler end normalfordelingen, som bekræftes i tabel \ref{tab:test_faktor}, hvor JB testens nulhypotesen om normalitet afvises.
LB testen i lag 10 kan ikke afvise, at residualerne er uafhængige.

\begin{table}
\center
\begin{tabular}{lccccccc} \toprule
& Faktor model (IC$_1$) & & Faktor model (IC$_2$)  \\ \midrule
Skewness & 0.0444 & & $-0.0418$     \\
Kurtosis & 0.5768 & & 0.4612 \\
JB-test & 0.0172 & & 0.0712 \\ 
LB$_{10}$-test & 0.729  && 0.4637  \\ \bottomrule
\end{tabular}
\caption{Skewness, excess kurtosis, $p$-værdier for Jarque-Bera og Ljung-Box testen for de standardiserede residualer fra faktor modellerne valgt ud fra IC$_1$ og IC$_2$. 
Vi lader LB$_{10}$ betegne Ljung-Box testen med lag = 10. } \label{tab:test_faktor}
\end{table}

Tabel \ref{tab:factor_mse_tab} viser MAE og MSE for faktor modellernes prædiktion.
Heraf ses at faktor model (IC\(_2\)) har mindst MAE og MSE.

\begin{table}
\center
\begin{tabular}{lcccc}
\toprule
& $\text{IC}_1$ & $\text{IC}_2$ & $\text{IC}_3$ \\
\midrule 
MAE & 0.1190 & 0.1111 & 0.1048  \\ 
MSE &  0.0221  & 0.0187  & 0.0165 \\ \bottomrule
 \end{tabular}
\caption{MAE og MSE for informationskriterierne.} \label{tab:factor_mse_tab}
\end{table}

Figur \ref{fig:fc_benchmark2} viser arbejdsløshedsraten og den prædikterede arbejdsløshedsrate med faktor model (IC\(_2\)).
\imgfigh{fc_benchmark2.pdf}{1}{Arbejdsløshedsraten og prædiktionen af arbejdsløshedsraten med faktor model (IC\(_2\)).}{fc_benchmark2}

Vi vælger faktor model (IC\(_2\)) som benchmark model, da den har mindst MAE og MSE og størst justeret \(R^2\) sammenlignet med AR(4) og faktor model (IC\(_1\)).

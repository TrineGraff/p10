\section{Faktor modellen}
For faktor modellen estimeres parametrene som følgende:
\begin{enumerate}
\item Estimer $k$ faktorer for $k = 1, \dots, k_{\max}$ på træningsmængden, hvor $k_{\max} = 20$.  
\item Vælg $\widehat{k}$ udfra informationskriterierne givet i \eqref{eq:ic1}-\eqref{eq:ic3}, hvor $\widehat{k}_1$, $\widehat{k}_2$ og $\widehat{k}_3$ betegner antallet af faktorer bestemt udfra informationskriterierne IC$_1$, IC$_2$ og IC$_3$.
\item Lad \(\widehat{\textbf{Z}} = \del{\widehat{{\textbf{F}}}^T\boldsymbol{\omega}^T}^T\) være en \(\del{k+m} \times T\) matrix, hvor \(\widehat{{\textbf{F}}}\) er en \(T \times k\) matrix af estimerede faktorer og \(\boldsymbol{\omega}\) er en \(T \times m\) matrix af laggede værdier af arbejdsløshedsraten.
Vi lader \(m = 4\), da den autoregressive model valgte en orden på 4.
Dette medfører nogle uobserveret værdier, og derfor fjernes de første 4 rækker i \(\widehat{\textbf{Z}}\).
\item Estimer parametrene $\widehat{\boldsymbol{\beta}} = \del{ \widehat{\boldsymbol{\beta}}^T_{\textbf{F}} \widehat{\boldsymbol{\beta}}^T_{\boldsymbol{\omega}}}^T$ med OLS.
\end{enumerate}

Tabel \ref{tab:est_faktor} viser antallet af faktorer, værdien af informationskriteriet for dette antal faktorer samt adjusted \(R^2\) for hver af de tre informationskriterier. 

\begin{table}[h]
\center
\begin{tabular}{lccc}
\toprule
& IC$_1$ & IC$_2$ & IC$_3$ \\ \midrule
$\widehat{k}$ & 6 & 11 & 20 \\ 
IC$\del{\widehat{k}}$ & $-0.3519$  & $-0.5314$ & $-0.6931$  \\
Adj. R$^2$ & 37.08 \% & 50.85 \% & 60.23 \% \\ \bottomrule
 \end{tabular}
\caption{Antal faktorer, værdien af informationskriteriet for dette antal faktorer samt adjusted \(R^2\) for IC$_1$, IC$_2$ og IC$_3$.} \label{tab:est_faktor}
\end{table}
Vi observerer, at IC$_1$ vælger den mindst komplekse model, da den vælger det færreste antal faktorer, mens IC$_3$ vælger det højeste antal faktorer. 
Adjusted \(R^2\) er størst for modellen valgt af IC$_3$ og mindst for modellen valgt af IC$_1$.

Figur \ref{fig:ic1_res}, \ref{fig:ic2_res} og \ref{fig:ic3_res} viser en analyse af de standardiserede residualer for modellerne valgt af henholdsvis IC$_1$, IC$_2$ og IC$_3$. 
Vi ser, at der en lille smule tunge haler i QQ-plottet for faktor modellerne valgt udfra IC$_1$ og IC$_2$. 
Men for faktor modellen valgt ud fra IC$_3$ ses kun en enkelt outlier. 
Derudover ser vi næsten ingen signifikante autokorrelation for de tre faktor modeller. 

\begin{table}
\center
\begin{tabular}{lccccccc} \toprule
& Faktor model (IC$_1$) & & Faktor model (IC$_2$)  \\ \midrule
Skewness & 0.0444 & & $-0.0418$     \\
Kurtosis & 0.5768 & & 0.4612 \\
JB-test & 0.0172 & & 0.0712 \\ 
LB$_{10}$-test & 0.729  && 0.4637  \\ \bottomrule
\end{tabular}
\caption{Skewness, excess kurtosis, $p$-værdier for Jarque-Bera og Ljung-Box testen for de standardiserede residualer fra faktor modellerne valgt ud fra IC$_1$ og IC$_2$. 
Vi lader LB$_{10}$ betegne Ljung-Box testen med lag = 10. } \label{tab:test_faktor}
\end{table}

Dette bekræftes i tabel \ref{tab:test_faktor}, hvor vi observerer meget lave værdier af skewness og excess kurtosis. 
Modellen valgt udfra IC$_3$ kommer dog tættest på normalfordelingen.
Vi ser også for JB-testen, at nulhypotesen om normalitet for modellerne med informationskriterierne IC$_2$, IC$_3$ ikke kan afvises, men den bliver dog afvist for IC$_1$.  Derudover har vi at LB-testen afvises for nulhypotesen omkring uafhængighed både for $e_t$ og $e_t^2$. 

Tabel \ref{tab:factor_mse_tab} viser MAE og MSE for hver model, her ses at modellen valgt udfra IC$_3$ har både mindst MAE og MSE, men den er dog også den mest komplekse model. 

\begin{table}
\center
\begin{tabular}{lcccc}
\toprule
& $\text{IC}_1$ & $\text{IC}_2$ & $\text{IC}_3$ \\
\midrule 
MAE & 0.1190 & 0.1111 & 0.1048  \\ 
MSE &  0.0221  & 0.0187  & 0.0165 \\ \bottomrule
 \end{tabular}
\caption{MAE og MSE for informationskriterierne.} \label{tab:factor_mse_tab}
\end{table}

Vi kan derudover også se, at vores faktor modeller har bedre forecasts performance i forhold til den autoregressive model. 
For in-sample ser vi at adjusted R$^2$ er markant bedre for faktor modellerne end den autoregressive model og derudover tyder det på, at de standardiserede residualer er iid og normalfordelte for faktor modellerne valgt ud fra IC$_2$ og IC$_3$. 

Selvom faktor modellen valgt ud fra IC$_3$ performer bedst i in -og out of sample, vælger vi faktor modellen valgt ud fra IC$_2$ til at være benchmark modellen. 
Det skyldes, at IC$_3$ vælger max af antallet af faktorer, og fra afsnit \ref{sec:faktorer} har vi at antallet af faktorer skal være mellem 0 og max af antal faktorer. 

%Taget dette i betragtning vil vores faktor model valgt ud fra IC$_3$ være vores benchmark model i det den performer bedst i in-sample og out-of-sample. 
Figur \ref{fig:fc_benchmark2} viser arbejdsløshedsraten og den prædikterede arbejdsløshedsrate med faktor modellen valgt ud fra IC\(_2\).

%\imgfigh{fc_benchmark.pdf}{0.7}{Arbejdsløshedsraten og prædiktion af arbejdsløshedsraten med faktor modellen valgt udfra IC\(_3\).}{fc_benchmark}

\imgfigh{fc_benchmark2.pdf}{0.7}{Arbejdsløshedsraten og prædiktion af arbejdsløshedsraten med faktor modellen valgt udfra IC\(_2\).}{fc_benchmark2}


%\item Forecast $\widehat{y}_{t+1}$ fra forecast-ligningen givet i ....
%For at bestemme, hvilken informationskriterie vi vil bruge til at finde $r$ i modellen, ser vi på, hvilken af dem der giver det laveste MSE. 
%
%\begin{table}
\center
\begin{tabular}{lcccc}
\toprule
& $\text{IC}_1$ & $\text{IC}_2$ & $\text{IC}_3$ \\
\midrule 
MAE & 0.1190 & 0.1111 & 0.1048  \\ 
MSE &  0.0221  & 0.0187  & 0.0165 \\ \bottomrule
 \end{tabular}
\caption{MAE og MSE for informationskriterierne.} \label{tab:factor_mse_tab}
\end{table}
%
%I Tabel \ref{tab:factor_mse_tab} er for hvert af kriterierne vise deres MSE af forecastet.
%Antallet af faktorer og antallet af lags er defineret til at være ens, såsom vist. Vi ser at $IC_1$ har færrest faktorer, hvor $IC_2$ har lidt flere og $IC_3$ har flest. 
%Vi ser at $IC_3$ giver den med mindst MSE, og derfor anvender vi den som vores benchmark for Faktormodellen. 
%Figur \ref{fig:fc_factor} voser forecast for faktor modellen, hvor $r$ er bestemt udfra $IC_3$
%
%\imgfigh{fc_factor.pdf}{0.7}{Viser forecast, hvor den røde linje er faktor modellen. Hvor den grå linje er de observerede arbejdsløsheds observationer}{fc_factor}







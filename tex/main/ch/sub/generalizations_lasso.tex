\section{Generaliseringer af lasso}
I denne sektion introduceres generaliseringen af lasso.
Alle har de to essentielle egenskaber af standard lasso, nemlig shrinkage og udvælgelse af variable.

Empirisk studier viser at lasso ikke er godt til højt korreleret variabler

\subsection{Elastic net}
Som nævnt er lasso ikke god til at håndtere højt korreleret variable. Dette ses ved at koefficienter stierne er uregelmæssige.


Med ved at kombinere et kvadreret $\ell_2$ strafled med $\ell_1$ strafled fås en metode kaldet elastiske net, som er bedre til korreleret grupper og vælger de korreleret variater (eller ikke) sammen.

Det elastiske net løser det konvekse problem
\begin{align}
\min_{\beta_0, \beta} \cbr{\frac{1}{2n} \sum_{i=1}^n \del{y_i - \beta_0 - x_{i}^T \beta}^2 + \lambda \sbr{\frac{1}{2} (1- \alpha) \Vert \beta \Vert_2^2 + \alpha \Vert \beta \Vert_1}}, \label{eq:4.2}
\end{align}
hvor $\alpha \in [0,1]$ er en parameter som kan varieres. 

Hvis $\alpha=1$, da reduceres strafleddet til $\ell_1$-normen eller strafleddet for lasso og hvis $\alpha=0$ reduceres det til den kvadrerede $\ell_2$-norm, svarende til strafleddet for ridge regression.

For ethvert $\alpha<1$ og $\lambda>0$ da er det elastiske net problem \eqref{eq:4.2} streng konveks, dvs der eksisterer en entydig løsning uanset korrelations strukturen af $X_j$.

Figur -- vises betingelsesområdet for henholdsvis det elastiske net og standard lasso for tre variable.
Heraf ses at det elastiske net deler egenskaber af $\ell_1$ kuglen og $\ell_2$ kuglen: de skarpe hjørner og kanter opfordre til variable udvælgelse og de kurvede konturer opfordre til deling af koefficienterne.
%\input{fig/elastic}

Det elastiske net har en ekstra tuning parameter $\alpha$ som skal bestemmes.
I praksis kan den ses som en højere-level parameter, og kan sættes på subjektiv grunder. Alternativt, kan man inkluderer en sekvens af værdier for $\alpha$ vha krydsvalidering.

Det elastiske net problem \eqref{eq:4.2} er konveks for $(\beta_0, \beta) \in \R \times \R^p$ og vi kan anvende en række algoritmer til at løse det. 
Coordinate descent er særlig effektiv, og opdateringer er blot en simpel udvidelses af dem for standard lasso i --.
Igen centreres kovariaterne, således at skæringen findes til sidst.

Coordinate descent opdateringen for $j$'te koefficient er givet ved
\begin{align*}
\hat{\beta}_j = \frac{S_{\lambda \alpha} \del{\sum_{i=1}^n r_{ij} x_{ij}}}{\sum_{i=1}^n x_{ij}^2 + \lambda (1-\alpha)},
\end{align*} 
hvor $S_\mu(z)=\text{sign}(z)(z-\mu)_+$ er soft-thresholding operatoren og $r_{ij}=y_i - \hat{\beta}_0 - \sum_{k \neq j} x_{ik} \hat{\beta}_k$ er den partial residual.


\subsection{Grouped lasso}
For mange regressions problemer har kovariaterne en naturlig grupperet struktur, og da foretrækkes det at alle koefficienter indenfor en gruppe er ikke-nul (eller nul) samtidig.
Betragt en lineær regressions model som har $J$ grupper af kovariater, hvor vektoren $Z_j \in \R^{p_j}$ for $j=1, \ldots, J$ repræsenterer kovariaterne i gruppe $j$.
Formålet er da at prædiktere responsvariablen $Y \in \R$ baseret på en samling af kovariater $(Z_1,\ldots,Z_J)$.
En lineær model for regressions funktionen $\E{Y \vert Z}$ er givet ved \(\theta_0 + \sum_{j=1}^J Z_j^T \theta_j\), hvor $\theta_j \in \R^{p_j}$ repræsenterer en gruppe af $p_j$ regressions koefficienter. 

Given en samling af $n$ samples \(\{(y_i, z_{i,1}, z_{i,2}, \ldots, z_{i,J})\}_{i=1}^n\) løser group lasso følgende konveks problem
\begin{align}
\min_{\theta_0 \in \R, \ \theta_j \in \R^{p_j}} \cbr{\frac{1}{2} \sum_{i=1}^n \del{y_i - \theta_0 - \sum_{j=1}^J z_{ij}^T \theta_j}^2 + \lambda \sum_{j=1}^J \Vert \theta_j \Vert_2},\label{eq:4.5}
\end{align}
hvor $\Vert \theta_j \Vert_2$ er den euklidiske norm af vektoren $\theta_j$.
Dette er en grupperet generalisering af lasso, som har følgende egenskaber:
\begin{itemize}
\item Afhængig af $\lambda$, vil enten alle indgange i vektoren $\hat{\theta}_j$ være nul eller ikke-nul
\item Når $p_j=1$, da har vi at $\Vert \theta_j \Vert_2 = \vert \theta_j \vert$, således at alle grupper er singletons, dermed reduceres optimerings problemet \eqref{eq:4.5} til standard lasso problemet.
\end{itemize}
På figur -- sammenlignes betingelsesområdet for den grupperet lasso med lasso for tre variable.
Vi ser at den grupperet lasso deler egenskaber med både $\ell_1$ og $\ell_2$ kuglen.

I \eqref{eq:4.5}, straffes alle grupper ligeligt, hvilket betyder at større grupper vil have en tendens til at blive valgt.


\subsubsection{Udregning af group lasso}
Lad os omsrive optimerings problemet \eqref{eq:4.5} på matrix-vektor form
\begin{align*}
\min_{\theta_1, \ldots, \theta_J} \cbr{\frac{1}{2} \Vert \y - \sum_{j=1}^J \mathbf{Z}_{j} \theta_j \Vert_2^2 + \lambda \sum_{j=1}^J \Vert \theta_j \Vert_2}.
\end{align*}
Vi ignorerer skæringen $\theta_0$, da vi centrerer variablerne og responsvariablen.
For dette problem er nul subgradient ligningerne givet ved
\begin{align*}
- \mathbf{Z}_{j}^T \del{\y - \sum_{\ell=1}^J \mathbf{Z}_\ell \hat{\theta}_\ell} + \lambda \hat{s}_j = 0, \quad j=1,\ldots, J,
\end{align*} 
hvor $\hat{s}_j \in \R^{p_j}$ er et element af subdifferentialet af normen $\Vert \cdot \Vert_2$ evalueret i $\hat{\theta}_j$.
Når $\hat{\theta}_j \neq 0$ da har vi, at $\hat{s}_j = \frac{\hat{\theta}_j}{\Vert \hat{\theta}_j \vert_2}$, og når $\hat{\theta}_j=0$ da har vi, at $\hat{s}_j$ er enhver vektor hvor $\Vert \hat{s}_j \Vert_2 \leq 1$.
En metode at løse nul subgradent ligningerne er ved at fastholde alle block vektorer $\{\hat{\theta}_k, k \neq j\}$, og da løse for $\hat{\theta}_j$.
Hermed udføres block coordinate descent på objektfunktionen af group lasso.
Da problemet er konveks, og strafleddet kan separeres efter block, er det garanteret at konvergere til en optimal løsning.
Med $\{\hat{\theta}_k, k \neq j\}$ fastholdt, kan vi skrive
\begin{align*}
- \mathbf{Z}_{j}^T \del{\mathbf{r}_j - \mathbf{Z}_j \hat{\theta}_j} + \lambda \hat{s}_j = 0,
\end{align*}
hvor $\mathbf{r}_j = \y - \sum_{k \neq j} \mathbf{Z}_k \hat{\theta}_k $ er den j'te partial residual.
Fra betingelserne opfyldt af subgradienten $\hat{s}_j$, må vi have at $\hat{\theta}_j =0$ hvis $\Vert \mathbf{Z}_j^T \mathbf{r}_j \Vert_2 < \lambda$, og ellers må $\hat{\theta}_j$ opfylde
\begin{align}
\hat{\theta}_j = \del{\mathbf{Z}_j^T \mathbf{Z}_j + \frac{\lambda}{\Vert \hat{\theta}_j \Vert_2} \mathbf{I}}^{-1} \mathbf{Z}_j^T \mathbf{r}_j. \label{eq:4.14}
\end{align}
Denne opdatering er ens med løsningen af ridge regression, bortset fra at den underliggende straf parameter afhænger af $\Vert \hat{\theta}_j \Vert_2$.
Desværre har ligning \eqref{eq:4.14} ikke en lukket løsning for $\hat{\theta}_j$ medmindre at $\mathbf{Z}_j$ er ortonormal. I dette special tilfælde har vi, at
\begin{align*}
\hat{\theta}_j = \del{1 - \frac{\lambda}{\Vert \mathbf{Z}_j^T \mathbf{r}_j \Vert_2}}_+  \mathbf{Z}_j^T \mathbf{r}_j.
\end{align*}

\subsection{Ikke-konvekse strafled}

tilpasse modeller som er mere sparse end lasso
Den vægtede lasso løser
\begin{align*}
\min_{\beta \in \R^p} \cbr{\frac{1}{2} \Vert \y - \X \beta \Vert_2^2 + \lambda \sum_{j=1}^p w_j \vert \beta_j \vert},
\end{align*}
hvor $w_j = \frac{1}{\vert \tilde{\beta}_j \vert^\nu}$.
Strafleddet for den vægtede lasso kan ses som en approksimation til $\ell_q$ strafleddene med $q=1-\nu$.

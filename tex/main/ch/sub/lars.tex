\section{Lars}
I denne sektion anvender vi algoritmen LARS til at løse vores problem og LARS algoritmen med en modifikation til at løse lasso. 
Vi anvender funktionerne \texttt{lars} fra R-pakken af samme navn. 
For LARS algoritmen producerer vi et fit, hver gang en ny variabler bliver tilføjet, dvs ved hvert step bliver der tilføjet en ny variable. Vi finder $\widehat{s}$, som betegner antallet af steps, ved krydsvalidering, men igen ser vi ikke kun på  $\widehat{s}_{\min}$, som giver den mindste gennemsnitlige krydsvaliderings fejl, men også den mindste værdi således at fejlen er indenfor en standard afvigelse, som betegnes $\widehat{s}_{\text{1sd}}$. 

Krydsvalideringen for lasso deler L1-normen op i en følge af 100 punkter mellem 0 og 1, hvor vi herefter finder $\widehat{f}$. Vi ser ikke kun på den model der giver den mindste krydsvaliderings fejl, som betegnes  $\widehat{f}_{\min}$, men også  $\widehat{f}_{\text{1sd}}$ . 
Figur \ref{fig:lars_kryds} viser krydsvaliderings plot for begge metoder. 
\imgfigh{lars_kryds.pdf}{1}{Viser 10-fold krydsvaliderings fejl plottede som en funktion af steps for metoden LARS, og en 10-fold krydsvaliderings fejl plottede som en funktion af fraktion af L1 normen}{lars_kryds}
\begin{table}
\center
\begin{tabular}{lcccc | lcccc}
\toprule
 \multicolumn{4}{c}{LARS} & \multicolumn{1}{c}{ }&   \multicolumn{4}{c}{LARS med lasso modifikation}  \\ \midrule
& $f$ & MSE & $p$ && & $f$ & MSE & $p$ \\
$f_{\text{min}}$ & $0.2753$ &  0.0019 & 27 && \(f_{\min}\) &  0.2626 & 0.0019 & 21   \\
$\boldsymbol{f}_{\textbf{1sd}}$ & $\textbf{0.2542} $ & \textbf{0.0019} & \textbf{19} &&$\boldsymbol{f}_{\textbf{1sd}}$ & \textbf{0.2424} & \textbf{0.0019} & \textbf{13 } \\ \bottomrule
 \end{tabular}
\caption{Værdien af $f_{min}$ og $f_{1\text{sd}}$, gennemsnitlig krydsvalideringsfejl, som er målt i MSE, og antallet af parameter. De valgte tuning parameter for hver metode er markeret med tykt.} \label{tab:lars_lasso_tab}
\end{table}


\imgfigh{lars_lasso_coef.pdf}{0.6}{h}{lars_lasso_coef}
\imgfigh{lars_coef.pdf}{0.6}{h}{lars_coef}

%SE for CV is found from the sample SE of the squared errors
%choose fraction based 1-se cv error rule
%# largest value of lambda such that
%# error is within 1 standard error of the minimum:
%Krydsvalideringen for lars anvender antallet af steps i lars proceduren. 
%Vi finder her fra det optimale $\widehat{s}$ 
%Herunder finder vi 
%
%Igen deler vi den op i 
%
%this is the number of steps in lars procedure
%
%
%#  Function produces one fit at each new variable entry.
%# Cross-Validation for LASSO chops up the L1-norm into sequence of 100 points. 
%#  SE for CV is found from the sample SE of the squared errors.  NOT from rerunning CV multiple times.
%# Therefore CV may be inappropriate for small n.  Use Cp from Ssummary() instead.
%

%I den her sektion anvender vi lars algoritmen med to modifikationer, som er beskrevet i sektionerne ... .,til at estimerer lasso og elastik net
%Vi anvender funktionerne \texttt{lars} og \texttt{elnet} fra R-pakkerne af hhv. samme navn.
%
%\subsubsection{Krydsvalidering}
%I denne sektion anvender funktionerne  \texttt{cv.lars} og \texttt{cv.enet} fra R pakkerne \texttt{lars} og \texttt{elnet}. 
%I ligning 
%
%Vi kan se på figur \ref{fig:lars_lasso} at jo laverer vores L1 norm er, jo større krydsvalideringsfejl får vi.  
%
%\imgfigh{lars_lasso.pdf}{0.7}{10-fold krydsvaliderings fejl plottede som en function af fraktion af side L1 norm. De stiplede linjer indikerer minimum fejl, samt fejlen med en standard afvigelse af minimum}{lars_lasso}
%
%I tabel \ref{tab:lars_tab} ser vi ikke samme tendens som ved coordinate descent. 
%Vi ser nemlig ikke en reducering af antal parameter, hvis vi anvende r$\lambda_{1\text{sd}}$.
%Derfor anvender vi $\lambda_{\min}$, da den har mindst krydsvaliderings fejl samt mindre kompleksitet. 
%
%\begin{table}
\center
\begin{tabular}{ccccc}
\toprule
 \multicolumn{4}{c}{Lasso} \\ \midrule
 & værdi & MSE & p \\
 $\lambda_{\min}$ & 0.3030 & 0.0020 & 14   \\ 
 $\lambda_{1 \text{sd}}$ & 0.3333 & 0.0022 & 36  \\ \bottomrule
 \end{tabular}
\caption{Tabellen viser lambda værdierne fundet udfra krydsvalidering, samt krydsvalideringsfejl, som er målt i MSE og antallet af parameter.} \label{tab:lars_tab}
\end{table}

%
% \begin{table}
\small
\center
\begin{tabular}{lc }
\toprule
\multicolumn{1}{c}{Lasso} \\ \midrule
Durable Materials (1) \\
Ratio of Help Wanted/No. Unemployed (2) \\
Civilian Labor Force(2) \\
Civilian Employment( 2) \\
Civilians Unemployed - Less Than 5 Weeks (2) \\
Civilans Unemployed for 5-14 Weeks( 2) \\
Civilans Unemployed - 15 Weeks \& and Over (2) \\
Intial Claims (2) \\
All Employees: Construction (2) \\
Housing Starts, West (3) \\
Real Personal Consumption Expenditures (4) \\
New Orders for Duarable Goods (4) \\
5-years Treasure Rate (6) \\
U.S/ U.K Foreign Exchange Rate (6) \\ \bottomrule
\end{tabular}
\caption{Tabellen indeholder de forklarende variable, som bliver udvalgt af lasso. Tallene i parantes indikerer hvilken gruppe de forskellige variable tilhører} \label{tab:lars_ud}
\end{table}
%
%Tabellen viser hvilken variable lasso udvælger, og igen kan vi se at hovedparten af variablerne er i samme gruppe, som vores responsvariable. 
%Trods, at vi har to løsnings metoder for lasso vælger de forholdsvis de samme variable.
%
%
%
%Som tidligere beskrevet kan vi for et fast $\lambda_2$ løse elastik net problemet med LARS-EN algoritmen. 
%Vi udregner
%anvender vi sekvens af værdier for $\lambda_2$, som 
%
%\subsubsection{BIC}
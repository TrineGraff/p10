\section{LARS}
Herefter anvendes LARS algoritmen uden og med lasso modifikationen, hvortil vi anvender \Rlang-pakken \texttt{lars}.
\Rlang-koden for dette afsnit er givet i ---.

For LARS algoritmen uden lasso modifikationen tilføjes en variabel i hvert step og dermed udfører algoritmen 126 steps.
LARS algoritmen med lasso modifikationen udfører 192 steps, hvor variabler kan tilføjes og fjernes igen.

\subsection{Krydsvalidering}
I dette afsnit finder vi turning parameterene for LARS algoritmen og LARS med lasso modifikation med krydsvalidering. 
Hertil anvendes funktionen \texttt{cv.lars} .  
For LARS algoritmen uden lasso modifikationen betragtes antallet af steps, som betegnes \(s\). 
Igen finder vi $\widehat{s}_{\min}$, som angiver den mindste gennemsnitlige krydsvalideringsfejl og $\widehat{s}_{\text{1sd}}$ som betegner den mindste værdi således at fejlen er indenfor en standard afvigelse. 

For LARS algoritmen med lasso modifikationen betragtes en såkaldt fraction af \(\ell_1\)-norm, der er givet ved \(\frac{\vert \tbeta \vert}{\max \vert \tbeta \vert}\).
Vi betragter en følge af 100 værdier af fraction af \(\ell_1\)-norm mellem 0 og 1, hvorudfra vi finder $\widehat{f}$.
Igen betragtes \(\widehat{f}_{\min}\) og \(\widehat{f}_{\text{1sd}}\).


%Vi finder $\widehat{s}$, som betegner antallet af steps, ved krydsvalidering, men igen ser vi ikke kun på  $\widehat{s}_{\min}$, som giver den mindste gennemsnitlige krydsvaliderings fejl, men også den mindste værdi således at fejlen er indenfor en standard afvigelse, som betegnes $\widehat{s}_{\text{1sd}}$. 
%Krydsvalideringen for lasso deler L1-normen op i en følge af 100 punkter mellem 0 og 1, hvor vi herefter finder $\widehat{f}$. Vi ser ikke kun på den model der giver den mindste krydsvaliderings fejl, som betegnes  $\widehat{f}_{\min}$, men også  $\widehat{f}_{\text{1sd}}$ . 

Figur \ref{fig:lars_kryds} viser krydsvalideringskurven og standardfejl som funktion af steps for LARS algoritmen og som funktion af fraction af \(\ell_1\)-norm for LARS algoritmen med lasso modifikationen. 

\imgfigh{lars_kryds.pdf}{1}{10-fold krydsvalideringsfejl som funktion af steps for LARS algoritmen og som funktion af fraction af \(\ell_1\)-norm for LARS algoritmen med lasso modifikationen. De stiplede linjer indikerer værdier af henholdsvis \(s\) og \(f\) med den mindste krydsvalideringsfejl og den mindste værdi af \(s\) og \(f\) således at fejlen er indenfor en standard afvigelse af minimum.}{lars_kryds}
%
Tabel \ref{tab:lars_lasso_tab} giver antallet af steps, gennemsnitlige krydsvalideringsfejl og antallet af variable for LARS algoritmen for \(s_\text{min}\) og \(s_{1 \text{sd}}\) og fraction af \(\ell_1\)-norm, gennemsnitlige krydsvalideringsfejl og antallet af variable for LARS algoritmen med lasso modifikationen for \(f_\text{min}\) og \(f_{1 \text{sd}}\).
%
Krydsvalideringen fejlen afviger først på 5. decimal, derfor vælger vi modellerne med det mindste antal variable. 
Det skal dog bemærkes, at antallet af steps og antallet af parameter ikke er lig med hinanden, som vil forventes. 
Det skyldes at ved udførelse af krydsvalidering er den første gennemsnitlige krydsvaliderings fejl baseret på hvor ingen variabler er blevet tilført. 
Så vi har derfor 127 gennemsnitlige krydsvaliderings fejl. 
%
\begin{table}
\center
\begin{tabular}{lcccc | lcccc}
\toprule
 \multicolumn{4}{c}{LARS} & \multicolumn{1}{c}{ }&   \multicolumn{4}{c}{LARS med lasso modifikation}  \\ \midrule
& $f$ & MSE & $p$ && & $f$ & MSE & $p$ \\
$f_{\text{min}}$ & $0.2753$ &  0.0019 & 27 && \(f_{\min}\) &  0.2626 & 0.0019 & 21   \\
$\boldsymbol{f}_{\textbf{1sd}}$ & $\textbf{0.2542} $ & \textbf{0.0019} & \textbf{19} &&$\boldsymbol{f}_{\textbf{1sd}}$ & \textbf{0.2424} & \textbf{0.0019} & \textbf{13 } \\ \bottomrule
 \end{tabular}
\caption{Værdien af $f_{min}$ og $f_{1\text{sd}}$, gennemsnitlig krydsvalideringsfejl, som er målt i MSE, og antallet af parameter. De valgte tuning parameter for hver metode er markeret med tykt.} \label{tab:lars_lasso_tab}
\end{table}

%
På figur \ref{fig:coef_lars_kryds} vises koefficienter for de valgte variable for LARS uden og med lasso modifikation.

\imgfigh{coef_lars_kryds.pdf}{1}{Estimerede koefficienter for LARS algoritmen uden og med lasso modifikationen, hvor $\widehat{s} $ og $\widehat{f}$ er fundet ud fra krydsvalidering.
Farverne indikerer hvilken gruppe, variablerne tilhører.}{coef_lars_kryds}

Figurerne \ref{fig:lars_kryds_res} og  \ref{fig:lars_lasso_kryds_res} viser en analyse af de standardiserede resiudaler, hvor vi igen ser tungere haler end en normalfordeling og autokorrelation i det første lag. Tabel \ref{tab:lars_kryds_res_tab}  understøtter dette, hvor vi afviser normalitet og uafhængighed i lag 10 når turning parameterne er valgt ved krydsvalidering. 

\newpage
\subsubsection{Inferens - LARS uden modifikation}
Lars algoritmen udfører 126 steps, hvor én variabel tilføjes i hvert step.
Funktionen \texttt{larInf} fra \Rlang-pakken \texttt{selectiveInference} udregner \(p\)-værdier og konfidensintervaller for LARS algoritmen.
I tabel \ref{tab:larInf} vises resultaterne.
%
\begin{table}[ht] 
\centering 
\scalebox{0.8}{
\begin{tabular}{lccccccc}
%\multicolumn{9}{l}{LARS algoritmen} \\
\toprule
Prædiktor& Koefficient & Z-score &\(p\)-værdi & Konfidensinterval &   $\sbr{\mathcal{V}^-;\mathcal{V}^+}$   \\
\midrule
\textcolor{blue3}{HWIURATIO}& 0.002  & 0.694   & 0.160    &  $\del{-\infty   ;  \infty}$   &$\sbr{0.002;0.002} $    \\
 \textcolor{blue3}{UEMP15OV} &    0.004&   1.606 & 0.923 &     $ \left( -\infty  ;  0.032\right] $     &$\sbr{0.004; 0.005}$   \\
 \textcolor{blue3}{UEMPLT5} & 0.001   &0.149   & 0.064  & $ \left[-0.018  ;     \infty\right) $  & $\sbr{0.000 ;0.001}$   \\
\textcolor{blue3}{MANEMP}   &   0.002 &  0.486   &0.273 &   $\left[-0.171 ;      \infty\right)$  & $\sbr{0.002;0.003}$\\
 \textcolor{blue3}{UEMP5TO14}  &-  0.001 & $-0.242 $ &0.077  &   $ \left( -\infty    ;  0.016\right] $&      $\sbr{0.000 ;0.001}$ \\
\textcolor{blue3}{CE16OV} &- 0.267 &$-37.446$ & 0.130   &   $\left( -\infty   ;  0.532\right]  $&    $\sbr{0.267; 0.267}$     \\ 
\textcolor{blue3}{ PAYEMS } &   0.000 &  0.006  & 0.563   &  $\del{-\infty ;  \infty}$   & $\sbr{0.000 ;0.000}$  \\
 \textcolor{blue3}{USGOOD} &- 0.003  &$-0.498$ &0.638   &   $\del{-\infty ;  \infty}$ &    $\sbr{0.003 ;0.003}$\\
\textcolor{chartreuse4}{CUMFNS} &  0.002  & 0.404 & 0.478    & $\del{-\infty   ;  \infty}$  &  $\sbr{0.002 ;0.002 }$  \\
 \textcolor{blue3}{CLF16OV} &  0.243  &36.643   & 0.179   &  $\del{-\infty  ;  \infty}$ &  $\sbr{0.243 ;0.243}$ \\  
\textcolor{chartreuse4}{ IPDMAT}&-0.006 &$ -1.626 $  &0.874   & $\left[-0.125  ;    \infty\right) $& $\sbr{0.006; 0.006}$ \\   
\textcolor{orange}{ TB6MS} & -0.005  &$-0.715$  & 0.569 &     $\del{-\infty  ;  \infty}$&   $\sbr{0.005; 0.006}$   \\ 
\textcolor{chartreuse4}{INDPRO} &  0.003 &  0.513  &0.328   &  $\del{-\infty   ;  \infty }$    & $\sbr{0.003 ;0.003}$  \\
\textcolor{orange}{GS1} &   0.006&   0.577    &0.473  &    $\del{-\infty  ;  \infty}$  &$\sbr{0.006 ;0.006}$ \\  
\textcolor{orange}{GS5} & -0.005 & $-1.146 $ &0.037 &     $\left( -\infty ;  -0.025\right]   $ & $\sbr{0.005 ;0.005 }$\\  
 \textcolor{blue3}{lag1}  & -0.009  &$-3.949$   & 0.910   & $\del{-\infty  ;  \infty }$  &$\sbr{0.009; 0.009 }$ \\ 
 \textcolor{red3}{DPCERA3M086SBEA} &- 0.003 & $-1.436$ & 0.233  &   $\del{-\infty   ;  \infty }$ &  $\sbr{0.003; 0.003}$ \\ 
\textcolor{orange}{ EXUSUKx}  &  0.003   &1.383 & 0.964   &   $\left( -\infty     ;-0.053 \right] $&  $\sbr{0.003; 0.003 }$   \\   
 \textcolor{blue3}{CLAIMSx} &0.002 &  0.813   & 0.226 &    $\del{-\infty  ;  \infty}$& $\sbr{0.002 ;0.002 }$   \\ 
\bottomrule
\end{tabular}  
}
\caption{\(p\)-værdier, konfidensintervaller, $Z$-score samt de trunkerede intervaller for variablerne udvalgt af LARS (CV) . Den estimeres standard afvigelse er \(0.043\), og resultaterne er for \(f_{1 \text{sd}} = 0.2542\) med \(\alpha = 0.1\).} \label{tab:larInf_kryds}
\end{table} 

%
Heraf ses at nulhypotesen primært afvises for variabler i gruppe 2 (\textcolor{blue3}{HWIURATIO}, \textcolor{blue3}{UEMP15OV}, \textcolor{blue3}{UEMPLT5}, \textcolor{blue3}{UEMP5TO14}, \textcolor{blue3}{CLF16OV}) samt en variabel fra gruppe 6 (\textcolor{orange}{EXUSUKx}).

Ifølge spacing testen afvises kun variabler fra gruppe 2, dog kan vi ikke afvise nulhypotesen for variablerne \textcolor{blue3}{PAYEMS} og \textcolor{blue3}{CLAIMSx}.

Ifølge kovarianstesten er alle variable tilhørende gruppe 2 signifikant med undtagelse af \textcolor{blue3}{PAYEMS}, \textcolor{blue3}{lag 1} og \textcolor{blue3}{CLAIMSx}, hvor nulhypotesen ikke kan afvises. 
Derudover ser vi, at også variablerne \textcolor{chartreuse4}{INDPRO}, \textcolor{orange}{GS1} og \textcolor{orange}{GS5} afviser nulhypotesen.

Men kun variablerne \textcolor{blue3}{HWIURATIO}, \textcolor{blue3}{UEMP15OV}, \textcolor{blue3}{UEMPLT5}, \textcolor{blue3}{UEMP5TO14}, \textcolor{blue3}{CLF16OV} afvises for alle teste, hvorfra vi konkluderer at disse er signifikante.


\imgfigh{boxplot_lars_kryds.pdf}{1}{Til venstre vises et boxplot af 1000 bootstrap realisatioer af $\widehat{\beta^{LARS}} \del{\widehat{s}}$, hvor $\widehat{s}$ er estimeret ud fra krydsvalidering. Til højere illustreret andelen af bootstrap realisationer, hvor parameter estimaterne er præcis nul}{coef_lars_kryds}

\newpage
\subsubsection{Inferens - LARS med lasso modifikation}
Som nævnt udfører LARS algoritmen med lasso modifikationen 192 steps, hvori variablerne tilføjes og nogle fjernes igen.
For \(\widehat{f}_{1\text{sd}}=0.2424\) finder vi 13 prædiktorer, hvorpå kovarians testen udføres.
Funktionen \texttt{covTest} fra \Rlang-pakken af samme navn udregner teststørrelsen samt \(p\)-værdie for kovarianstesten for LARS algoritmen med lasso modifikation.
Tabel \ref{tab:covTest} viser resultatet af dette.
For 6 ud af 9 prædiktorer som tilhører gruppe 2 med undtagelse af \textcolor{blue3}{PAYEMS}, \textcolor{blue3}{lag 1} og \textcolor{blue3}{USCONS} afvises nulhypotesen, hvilket betyder, at??
For de resterende variable kan vi ikke afvise nulhypotesen.
%
\begin{table}[ht] 
\centering 
\begin{tabular}{lccc}
%\multicolumn{3}{l}{LARS algoritmen med lasso modifikation} \\
\toprule
Prædiktor & Cov test & \(p\)-værdi \\
\midrule
 \textcolor{blue3}{HWIURATIO}  &   864.6317 & 0 \\
 \textcolor{blue3}{UEMP15OV}  &   161.3770&  0 \\
 \textcolor{blue3}{UEMPLT5} &  163.0670 & 0 \\
 \textcolor{blue3}{UEMP5TO14}  &   122.3840  &0 \\
 \textcolor{blue3}{CE16OV} & 14.7416  &0 \\
 \textcolor{blue3}{PAYEMS} &  0.3356  &0.7151 \\
  \textcolor{blue3}{USGOOD}  &   5.0872 & 0.0066 \\
 \textcolor{blue3}{CLF16OV}    &   221.9181 & 0 \\
\textcolor{chartreuse4}{IPDMAT}       &    0.0668&  0.9354 \\
\textcolor{orange}{GS5}   &       0.3856 &    0.6803 \\
 \textcolor{blue3}{ lag1 }  &      0.8897 &    0.4115 \\
\textcolor{orange}{ TB6MS }&       0.0419   &  0.9590 \\
 \textcolor{blue3}{USCONS }&   0.0132   &  0.9869 \\ 
\bottomrule
\end{tabular}

\caption{Kovarians testen for LARS algoritmen med lasso modifikation (CV).
Vi viser kun \(p\)-værdier for prædiktorer som medtages og bliver i modellen for \(f_{1\text{sd}}=0.2424\), dvs hvis en prædiktor medtages i et step og senere forlader modellen, vises denne prædiktor ikke.} \label{tab:covTest}
\end{table} 

%
For de valgte variable udfører algoritmen 23 steps, hvor variablerne  \textcolor{chartreuse4}{CUMFNS}, \textcolor{blue3}{MANEMP}, \textcolor{orange}{TB6MS}, \textcolor{orange}{GS1}, \textcolor{blue3}{USGOOD} tilføjes og fjernes igen.


\imgfigh{boxplot_lasso_kryds.pdf}{1}{Til venstre vises et boxplot af 1000 bootstrap realisatioer af $\widehat{\beta^{lasso}} \del{\widehat{f}}$, hvor $\widehat{f}$ er estimeret udfra krydsvalidering. Til højere illustreret andelen af bootstrap realisationer, hvor parameter estimaterne er præcis nul}{coef_lasso_kryds}


\newpage
\subsection{BIC}
I dette afsnit findes den optimale model ud fra BIC.
Tabel \ref{tab:bic_lars} giver fraktionen af \(\ell_1\)-normen, BIC, antallet af parametre, justeret R$^2$ og log-likelihood for LARS og lasso LARS.
Justeret R$^2$ er størst for lasso LARS, som også vælger det færreste antal parametre. 
 For de valgte variable udfører algoritmen 32 trin, hvor variablerne \textcolor{chartreuse4}{CUMFNS}, \textcolor{blue3}{MANEMP}, \textcolor{orange}{GS1}, \textcolor{blue3}{HWIURATIO}, \textcolor{blue3}{PAYMENS} og \textcolor{blue3}{USGOOD} tilføjes og fjernes igen og variablen \textcolor{orange}{TB6MS} bliver tilføjet, fjernet og så tilføjet igen. 

\begin{table}
\center
\scalebox{0.8}{
\begin{tabular}{lccccc| lccccc} 
\toprule
\multicolumn{6}{c}{LARS (BIC)}  & \multicolumn{6}{c}{Lasso LARS (BIC)} \\ \midrule
& Værdi & BIC & $p$ & R$^2_{\text{adj}}$ & LogLik& & Værdi & BIC & $p$ & R$^2_{\text{adj}}$ & LogLik \\
$f_\text{BIC}$ & 0.2623 & $-6.0925$ & 20 &94.43\% & 975.2909  &$f_\text{BIC}$ &  0.2604 &$-6.1627$& 17 &  94.46\% & 974.9938 \\ \bottomrule
 \end{tabular}}
\caption{Værdien af $f_\text{BIC}$, antallet af parametre, BIC, justeret R$^2$  og log-likelihood for LARS og lasso LARS.} \label{tab:bic_lars}
\end{table}


Figur \ref{fig:coef_plot_lars_bic} viser, de 20 estimerede koefficienter for LARS (BIC) og de 17 estimerede koefficienter for lasso LARS (BIC).  
Igen ser vi, at  variablerne \textcolor{blue3}{CE16OV} og \textcolor{blue3}{CLF16OV} har de største estimerede koefficienter, mens de resterende koefficienter er meget tæt på nul. 


\imgfigh{coef_plot_lars_bic.pdf}{1}{Estimerede koefficienter for LARS (BIC) og lasso LARS (BIC). Farverne indikerer hvilken gruppe, variablerne tilhører. }{coef_plot_lars_bic}

Figur \ref{fig:lars_bic_resid} og \ref{fig:lars_lasso_bic_resid} viser en analyse af de standardiserede residualer for LARS (BIC) og lasso LARS (BIC). 
Igen ses, at fordelingen af de standardiserede residualer har tungere haler end normalfordelingen og autokorrelation i det første lag. 
Dette bekræftes i tabel \ref{tab:lars_kryds_res_tab}, hvor vi afviser normalitet og at de 10 første autokorrelationer er nul.

Figur \ref{fig:boxplot_lars_bic} og \ref{fig:boxplot_lars_lasso_bic} viser bootstrap resultaterne for variablerne udvalgt af LARS (BIC) og lasso LARS (BIC). 
For lasso LARS (BIC) vælges variablerne \textcolor{blue3}{UEMPL15OV}, \textcolor{blue3}{UEMP5TO14}, \textcolor{blue3}{UEMPLT5}, \textcolor{blue3}{CE16OV} og \textcolor{blue3}{CLF16OV} i alle bootstrap realisationer, mens LARS (BIC) også altid vælger variablen \textcolor{blue3}{HWURATIO}.
For LARS (BIC) estimeres regressionskoefficienterne for variablerne \textcolor{orange}{GS1}, \textcolor{red3}{AMDMNOx}, \textcolor{blue3}{CLAIMSx}, \textcolor{chartreuse4}{CUMFNS}, \textcolor{chartreuse4}{INDPRO} og \textcolor{red3}{DPCERA3M086SBEA} til at være lig nul over 60\% af bootstrap realisationerne.
For lasso LARS (BIC) lader det til, at prædiktorerne er mere signifikante, idet alle regressionskoefficienter har en sandsynlighed mindre end 35\% for at være lig 0. 

\subsubsection{Kovarians testen}
Tabel \ref{tab:covTest_bic} viser teststørrelsen samt $p$-værdier af kovarians testen for de 17 variabler der er udvalgt af lasso LARS (BIC). 
For variablerne  \textcolor{blue3}{UEMPL15OV}, \textcolor{blue3}{UEMPLT5}, \textcolor{blue3}{UEMP5TO14}, \textcolor{blue3}{CE16OV} og \textcolor{blue3}{CLF16OV} afvises nulhypotesen, hvilket betyder, at disse prædiktorer er signifikante. 
For de resterende prædiktorer kan nulhypotesen ikke afvises. 
\newpage
\begin{table}[ht] 
\centering 
\begin{tabular}{lccc}
%\multicolumn{3}{l}{LARS algoritmen med lasso modifikation} \\
\toprule
Prædiktor & Cov test & \(p\)-værdi \\
\midrule
\textcolor{blue3}{UEMP15OV}    &       161.3770  &0\\
\textcolor{blue3}{UEMPLT5}   &      163.0670 & 0\\
\textcolor{blue3}{UEMP5TO14}    &    122.3840&  0\\
\textcolor{blue3}{CE16OV}         &   14.7416 & 0\\
\textcolor{blue3}{CLF16OV}        &   221.9181 & 0\\
\textcolor{chartreuse4}{IPDMAT}         &    0.0668&  0.9354\\
\textcolor{orange}{GS5}&   0.3856 & 0.6803\\
\textcolor{blue3}{lag1}       &      0.8897 & 0.4115\\
\textcolor{orange}{TB6MS}  &    0.0419 & 0.9590\\
\textcolor{blue3}{USCONS} &    0.0132&  0.9869\\
\textcolor{blue3}{ DPCERA3M086SBEA}          &  0.0254 & 0.9750\\
\textcolor{orange}{ EXUSUKx} &     0.2309 & 0.7939\\
\textcolor{blue3}{CLAIMSx} &      0.0082 &  0.9919\\
\textcolor{red3}{ AMDMNOx}  &     0.0464 & 0.9546\\
\textcolor{blue3}{lag4 }     &    0.2281&  0.7962\\
\textcolor{cadetblue2}{CPIMEDSL}  &   0.0719&  0.9307\\
\textcolor{blue3}{USTRADE}   &     0.0029 &  0.9971\\ \bottomrule
\end{tabular}
\caption{Kovarians testen for LARS algoritmen med lasso modifikation.
Vi viser kun \(p\)-værdier for prædiktorer som medtages og bliver i modellen for \(\widehat{f}_{1\text{sd}}=0.2604\), dvs hvis en prædiktor medtages i et step og senere forlader modellen, vises denne prædiktor ikke.} \label{tab:covTest_bic}
\end{table} 


\subsubsection{TG testen}
Resultaterne af TG testen for LARS (BIC) er givet i tabel \ref{tab:larInf_bic}.
Nulhypotesen kan ikke afvises for nogle prædiktorer, hvilket betyder, at ingen er signifikante.
Igen observeres at flere grænser af konfidensintervallerne er uendelige, hvilket kommer af, at $\boldsymbol{\eta}^T \textbf{y}$ er meget tæt på det trunkerede interval \(\sbr{\mathcal{V}^-, \mathcal{V}^+}\).

\begin{table}[ht] 
\centering 
\scalebox{0.8}{
\begin{tabular}{lcccccc}
%\multicolumn{9}{l}{LARS algoritmen} \\ 
\toprule
Prædiktor&  Koefficient & Z-score  & \(p\)-værdi & Konfidensinterval& $\sbr{\mathcal{V}^-;\mathcal{V}^+}$   \\ \midrule
\textcolor{blue3}{HWIURATIO} &0.002  & 0.720   &0.161    &   $\del{-\text{Inf}   ;  \text{ Inf} }$ &  $\sbr{0.002;0.002}$    \\
 \textcolor{blue3}{UEMP15OV} & 0.004  & 1.596  4 &  0.920 &     $\left( -\text{Inf}  ;      0.034 \right] $& $\sbr{0.004 ;0.005}$  \\
\textcolor{blue3}{UEMPLT5}  & 0.001 &  0.148   & 0.065  & $\left[-0.018    ;    \text{ Inf}  \right)$   &$\sbr{0.000;0.001}$    \\
\textcolor{blue3}{MANEMP} &0.003 &  0.561    & 0.766 &       $\left(  -\text{Inf}     ;  0.120\right] $ &$\sbr{0.003; 0.003}$  \\
 \textcolor{blue3}{ UEMP5TO14} & 0.001 & -0.261 &  0.093   &  $\left( - \text{ Inf}    ;  0.023\right] $&   $\sbr{0.000; 0.001}$ \\
\textcolor{blue3}{ CE16OV}   & 0.267& $-37.412$  & 0.130  &    $\left( -\text{Inf}     ;   0.574\right] $  &$\sbr{0.266; 0.267}$ \\
 \textcolor{blue3}{PAYEMS} &  0.000  & 0.012  & 0.428  &    $\del{-\text{Inf}   ;  \text{ Inf} }$ &  $\sbr{0.000 ;  0.000}$   \\
 \textcolor{blue3}{USGOOD} & 0.004 & $-0.584$ & 0.721   &   $\del{-\text{Inf}   ;  \text{ Inf} } $&    $\sbr{0.004 ;  0.004   }$   \\
\textcolor{chartreuse4}{CUMFNS} &  0.002   &0.390  & 0.455     &  $\del{-\text{Inf}   ;  \text{ Inf} }$& $\sbr{0.002 ;  0.002 }$   \\
 \textcolor{blue3}{ CLF16OV}  &   0.243 & 36.646   &  0.179    &  $\del{-\text{Inf}   ;  \text{ Inf} }$&$\sbr{0.243 ;  0.243}$    \\
\textcolor{chartreuse4}{ IPDMAT}& 0.006& $ -1.618$   & 0.869  & $\left[ -0.130  ;      \text{ Inf}  \right)$&   $\sbr{0.006;   0.006}$     \\
 \textcolor{orange}{TB6MS}& 0.006 & $-0.790 $  & 0.615     &  $\del{-\text{Inf}   ;  \text{ Inf} } $  & $\sbr{0.006 ;  0.006}$   \\
 \textcolor{chartreuse4}{INDPRO}&   0.003   &0.591   & 0.494  &    $\del{-\text{Inf}   ;  \text{ Inf} }$ &  $\sbr{0.003;   0.003}$    \\
 \textcolor{orange}{GS1} &  0.007&   0.675  & 0.571 &      $\del{-\text{Inf}   ;  \text{ Inf} }$ &   $\sbr{0.007 ;  0.007}$     \\
 \textcolor{orange}{GS5} &   0.006 &$ -1.240$  & 0.302&      $\del{-\text{Inf}   ;  \text{ Inf} }$&   $\sbr{0.006;  0.006}$   \\
 \textcolor{blue3}{lag1} & 0.009& $ -3.914  $  &  0.912   &  $\del{-\text{Inf}   ;  \text{ Inf} }$& $\sbr{0.009;   0.009 }$   \\
  \textcolor{red3}{DPCERA3M086SBEA}   & 0.002&  $-1.331  $ & 0.225 &      $\del{-\text{Inf}   ;  \text{ Inf} }$& $\sbr{0.002;   0.002 }$   \\
 \textcolor{orange}{EXUSUKx} & 0.003 &  1.357   & 0.964  &    $\left( -\text{Inf}   ;  -0.051\right] $&  $\sbr{0.003 ;  0.003}$   \\
 \textcolor{blue3}{CLAIMSx} &  0.001  & 0.629  & 0.208   &   $\del{-\text{Inf}   ;  \text{ Inf} }$ & $\sbr{ 0.001 ;  0.001 }$  \\
 \textcolor{red3}{AMDMNOx} &  0.002&  $-0.904 $   & 0.855     &  $\del{-\text{Inf}   ;  \text{ Inf} }$&$\sbr{0.002 ;  0.002}$  \\ \bottomrule
\end{tabular}  
}
\caption{\(p\)-værdier, konfidensintervaller samt  $\boldsymbol{\eta}^T\textbf{y}$, dens trunkerede interval og $Z$-score for hver variabel udvalgt af LARS algoritmen. Den estimeres standard afvigelse er \(0.043\), og resultaterne er for \(\widehat{f}_{\text{BIC}} = 0.2623 \) med \(\alpha = 0.1\).} \label{tab:larInf_bic}
\end{table} 

Figur \ref{fig:resid_lars_tg_bic} viser en analyse af de standardiserede residualer for LARS$_{TG}$ (BIC). 
Ud fra tabel \ref{tab:lars_kryds_res_tab} afvises residualerne at være normalfordelte og afhængige i lag 10.








%#  Function produces one fit at each new variable entry.
%# Cross-Validation for LASSO chops up the L1-norm into sequence of 100 points. 
%#  SE for CV is found from the sample SE of the squared errors.  NOT from rerunning CV multiple times.
%# Therefore CV may be inappropriate for small n.  Use Cp from Ssummary() instead.
%

\section{Lars}
I den her sektion anvender vi lars algoritmen med to modifikationer, som er beskrevet i sektionerne ... .,til at estimerer lasso og elastik net
Vi anvender funktionerne \texttt{lars} og \texttt{elnet} fra R-pakkerne af hhv. samme navn.

\subsection{Krydsvalidering}
I denne sektion anvender funktionerne  \texttt{cv.lars} og \texttt{cv.enet} fra R pakkerne \texttt{lars} og \texttt{elnet}. 
I ligning 

Vi kan se på figur \ref{fig:lars_lasso} at jo laverer vores L1 norm er, jo større krydsvalideringsfejl får vi.  

\imgfigh{lars_lasso.pdf}{0.7}{10-fold krydsvaliderings fejl plottede som en function af fraktion af side L1 norm. De stiplede linjer indikerer minimum fejl, samt fejlen med en standard afvigelse af minimum}{lars_lasso}

I tabel \ref{tab:lars_tab} ser vi ikke samme tendens som ved coordinate descent. 
Vi ser nemlig ikke en reducering af antal parameter, hvis vi anvende r$\lambda_{1\text{sd}}$.
Derfor anvender vi $\lambda_{\min}$, da den har mindst krydsvaliderings fejl samt mindre kompleksitet. 

\begin{table}
\center
\begin{tabular}{ccccc}
\toprule
 \multicolumn{4}{c}{Lasso} \\ \midrule
 & værdi & MSE & p \\
 $\lambda_{\min}$ & 0.3030 & 0.0020 & 14   \\ 
 $\lambda_{1 \text{sd}}$ & 0.3333 & 0.0022 & 36  \\ \bottomrule
 \end{tabular}
\caption{Tabellen viser lambda værdierne fundet udfra krydsvalidering, samt krydsvalideringsfejl, som er målt i MSE og antallet af parameter.} \label{tab:lars_tab}
\end{table}


 \begin{table}
\small
\center
\begin{tabular}{lc }
\toprule
\multicolumn{1}{c}{Lasso} \\ \midrule
Durable Materials (1) \\
Ratio of Help Wanted/No. Unemployed (2) \\
Civilian Labor Force(2) \\
Civilian Employment( 2) \\
Civilians Unemployed - Less Than 5 Weeks (2) \\
Civilans Unemployed for 5-14 Weeks( 2) \\
Civilans Unemployed - 15 Weeks \& and Over (2) \\
Intial Claims (2) \\
All Employees: Construction (2) \\
Housing Starts, West (3) \\
Real Personal Consumption Expenditures (4) \\
New Orders for Duarable Goods (4) \\
5-years Treasure Rate (6) \\
U.S/ U.K Foreign Exchange Rate (6) \\ \bottomrule
\end{tabular}
\caption{Tabellen indeholder de forklarende variable, som bliver udvalgt af lasso. Tallene i parantes indikerer hvilken gruppe de forskellige variable tilhører} \label{tab:lars_ud}
\end{table}

Tabellen viser hvilken variable lasso udvælger, og igen kan vi se at hovedparten af variablerne er i samme gruppe, som vores responsvariable. 
Trods, at vi har to løsnings metoder for lasso vælger de forholdsvis de samme variable.



Som tidligere beskrevet kan vi for et fast $\lambda_2$ løse elastik net problemet med LARS-EN algoritmen. 
Vi udregner
anvender vi sekvens af værdier for $\lambda_2$, som 

\subsection{BIC}
\subsection{BIC}
I dette afsnit findes den optimale model udfra BIC.
Tabel \ref{tab:bic_lars} giver fraktion af \(\ell_1\)-norm, BIC, antallet af parametre, justeret R$^2$ og log-likelihood for LARS og lasso LARS.
Justeret R$^2$ er størst for lasso LARS, som også vælger det færreste antal parametre. 
 For de valgte variable udfører algoritmen 32 trin, hvor variablerne \textcolor{chartreuse4}{CUMFNS}, \textcolor{blue3}{MANEMP}, \textcolor{orange}{GS1}, \textcolor{blue3}{HWIURATIO}, \textcolor{blue3}{PAYMENS} og \textcolor{blue3}{USGOOD} tilføjes og fjernes igen og variablen \textcolor{orange}{TB6MS} bliver tilføjet, fjernet og så tilføjet igen. 

\begin{table}
\center
\scalebox{0.8}{
\begin{tabular}{lccccc| lccccc} 
\toprule
\multicolumn{6}{c}{LARS (BIC)}  & \multicolumn{6}{c}{LARS med lasso modifikation (BIC)} \\ \midrule
& Værdi & BIC & $p$ & R$^2_{\text{adj}}$ & LogLik& & Værdi & BIC & $p$ & R$^2_{\text{adj}}$ & LogLik \\
$f_\text{BIC}$ & 0.2623 & $-6.0925$ & 20 &94.43  \% & 975.2909  &$f_\text{BIC}$ &  0.2604 &$-6.1627$& 17 &  94.46 \% & 974.9938 \\ \bottomrule
 \end{tabular}}
\caption{Værdien af $f_\text{BIC}$, antallet af parametre, BIC, justerede R$^2$  og log-likelihood for R$^2$ for LARS uden og med lasso modifikation.} \label{tab:bic_lars}
\end{table}


Figur \ref{fig:coef_plot_lars_bic} viser, de 20 estimerede koefficienter for LARS (BIC) og de 17 estimerede koefficienter for lasso LARS (BIC).  
Igen ser vi, at  variablerne \textcolor{blue3}{CE16OV} og \textcolor{blue3}{CLF16OV} har de største estimerede koefficienter, mens de resterende koefficienter er meget tæt på nul. 


\imgfigh{coef_plot_lars_bic.pdf}{1}{Estimerede koefficienter for LARS (BIC) og lasso LARS (BIC). Farverne indikerer hvilken gruppe, variablerne tilhører. }{coef_plot_lars_bic}

Figur \ref{fig:lars_bic_resid} og \ref{fig:lars_lasso_bic_resid} viser en analyse af de standardiserede residualer for LARS (BIC) og lasso LARS (BIC). 
Igen ses, at fordelingen af de standardiserede residualer har tungere haler end normalfordelingen og autokorrelation i det første lag. 
Dette bekræftes i tabel \ref{tab:lars_kryds_res_tab}, hvor vi afviser normalitet og uafhængighed i lag 10.

Figur \ref{fig:boxplot_lars_bic} og \ref{fig:boxplot_lars_lasso_bic} viser bootstrap resultaterne for variablerne udvalgt af LARS (BIC) og lasso LARS (BIC). 
For lasso LARS (BIC) vælges variablerne \textcolor{blue3}{UEMPL15OV}, \textcolor{blue3}{UEMP5TO14}, \textcolor{blue3}{UEMPLT5}, \textcolor{blue3}{CE16OV} og \textcolor{blue3}{CLF16OV} i alle bootstrap realisationer, mens LARS (BIC) også altid vælger variablen \textcolor{blue3}{HWURATIO}.
For LARS (BIC) estimeres regressionskoefficienterne for variablerne \textcolor{orange}{GS1}, \textcolor{red3}{AMDMNOx}, \textcolor{blue3}{CLAIMSx}, \textcolor{chartreuse4}{CUMFNS}, \textcolor{chartreuse4}{INDPRO} og \textcolor{red3}{DPCERA3M086SBEA} til at være lig nul over 60\% af bootstrap realisationerne.
For lasso LARS (BIC) lader det til, at prædiktorerne er mere signifikante, idet alle regressionskoefficienter har en sandsynlighed mindre end 35\% for at være lig 0. 

\subsubsection{Kovarians testen}
Tabel \ref{tab:covTest_bic} viser teststørrelsen samt $p$-værdier af kovarians testen for de 17 variabler der udvalgt af lasso LARS (BIC). 
For variablerne  \textcolor{blue3}{UEMPL15OV}, \textcolor{blue3}{UEMPLT5}, \textcolor{blue3}{UEMP5TO14}, \textcolor{blue3}{CE16OV} og \textcolor{blue3}{CLF16OV} afvises nulhypotesen, hvilket betyder, at disse prædiktorer er signifikante. 
For de resterende prædiktorer kan nulhypotesen ikke afvises. 

\begin{table}[ht] 
\centering 
\begin{tabular}{lccc}
%\multicolumn{3}{l}{LARS algoritmen med lasso modifikation} \\
\toprule
Prædiktor & Cov test & \(p\)-værdi \\
\midrule
\textcolor{blue3}{UEMP15OV}    &       161.3770  &0\\
\textcolor{blue3}{UEMPLT5}   &      163.0670 & 0\\
\textcolor{blue3}{UEMP5TO14}    &    122.3840&  0\\
\textcolor{blue3}{CE16OV}         &   14.7416 & 0\\
\textcolor{blue3}{CLF16OV}        &   221.9181 & 0\\
\textcolor{chartreuse4}{IPDMAT}         &    0.0668&  0.9354\\
\textcolor{orange}{GS5}&   0.3856 & 0.6803\\
\textcolor{blue3}{lag1}       &      0.8897 & 0.4115\\
\textcolor{orange}{TB6MS}  &    0.0419 & 0.9590\\
\textcolor{blue3}{USCONS} &    0.0132&  0.9869\\
\textcolor{red3}{ DPCERA3M086SBEA}          &  0.0254 & 0.9750\\
\textcolor{orange}{ EXUSUKx} &     0.2309 & 0.7939\\
\textcolor{blue3}{CLAIMSx} &      0.0082 &  0.9919\\
\textcolor{red3}{ AMDMNOx}  &     0.0464 & 0.9546\\
\textcolor{blue3}{lag4 }     &    0.2281&  0.7962\\
\textcolor{cadetblue2}{CPIMEDSL}  &   0.0719&  0.9307\\
\textcolor{blue3}{USTRADE}   &     0.0029 &  0.9971\\ \bottomrule
\end{tabular}
\caption{Kovarians testen for lasso LARS (BIC).
Vi viser kun \(p\)-værdier for prædiktorer som medtages og bliver i modellen, dvs hvis en prædiktor medtages i et trin og senere forlader modellen, vises denne prædiktor ikke.
\(p\)-værdier \(< 2.2 \cdot 10^{-16}\) sættes lig 0.} \label{tab:covTest_bic}
\end{table} 


\subsubsection{TG testen}
Resultaterne af TG testen for LARS (BIC) er givet i tabel \ref{tab:larInf_bic}.
Nulhypotesen kan ikke afvises for nogle prædiktorer, hvilket betyder, at ingen er signifikante.
Igen observeres at flere grænser af konfidensintervallerne er uendelige, hvilket kommer af, at $\boldsymbol{\eta}^T \textbf{y}$ er meget tæt på det trunkerede interval \(\sbr{\mathcal{V}^-, \mathcal{V}^+}\).

\begin{table}[ht] 
\centering 
\scalebox{0.8}{
\begin{tabular}{lcccccc}
%\multicolumn{9}{l}{LARS algoritmen} \\ 
\toprule
Prædiktor&Koefficient  &\(p\)-værdi & Konfidensinterval &   $\boldsymbol{\eta^Ty}$ & Z-score & $\sbr{\mathcal{V}^-;\mathcal{V}^+}$   \\ \midrule
\textcolor{blue3}{HWIURATIO}  &$-0.0010$&0.161    &   $\del{-\text{Inf}   ;  \text{ Inf} }$ &0.002  & 0.720 &  $\sbr{0.002;0.002}$    \\
 \textcolor{blue3}{UEMP15OV} &0.0094 &  0.920 &     $\left( -\text{Inf}  ;      0.034 \right] $& 0.004  & 1.596 & $\sbr{0.004 ;0.005}$  \\
\textcolor{blue3}{UEMPLT5} &0.0099 & 0.065  & $\left[-0.018    ;    \text{ Inf}  \right)$  & 0.001 &  0.148   &$\sbr{0.000;0.001}$    \\
\textcolor{blue3}{MANEMP} &0.0029& 0.766 &       $\left(  -\text{Inf}     ;  0.120\right] $ &0.003 &  0.561   &$\sbr{0.003; 0.003}$  \\
 \textcolor{blue3}{ UEMP5TO14} &0.0052&  0.093   &  $\left( - \text{ Inf}    ;  0.023\right] $& 0.001 & -0.261&   $\sbr{0.000; 0.001}$ \\
\textcolor{blue3}{ CE16OV}  &$-0.2352$ & 0.130  &    $\left( -\text{Inf}     ;   0.574\right] $& 0.267& $-37.412$   &$\sbr{0.266; 0.267}$ \\
 \textcolor{blue3}{PAYEMS} &$-0.0007$& 0.428  &    $\del{-\text{Inf}   ;  \text{ Inf} }$ &  0.000  & 0.012 &  $\sbr{0.000 ;  0.000}$   \\
 \textcolor{blue3}{USGOOD} &$-0.0034$& 0.721   &   $\del{-\text{Inf}   ;  \text{ Inf} } $& 0.004 & $-0.584$&    $\sbr{0.004 ;  0.004   }$   \\
\textcolor{chartreuse4}{CUMFNS}&0.0021 & 0.455     &  $\del{-\text{Inf}   ;  \text{ Inf} }$&  0.002   &0.390 & $\sbr{0.002 ;  0.002 }$   \\
 \textcolor{blue3}{ CLF16OV} & 0.2135 &  0.179    &  $\del{-\text{Inf}   ;  \text{ Inf} }$&   0.243 & 36.646  &$\sbr{0.243 ;  0.243}$    \\
\textcolor{chartreuse4}{ IPDMAT} &$-0.0044 $& 0.869  & $\left[ -0.130  ;      \text{ Inf}  \right)$& 0.006& $ -1.618$ &   $\sbr{0.006;   0.006}$     \\
 \textcolor{orange}{TB6MS}&$-0.0032$& 0.615     &  $\del{-\text{Inf}   ;  \text{ Inf} } $ & 0.006 & $-0.790 $  & $\sbr{0.006 ;  0.006}$   \\
 \textcolor{chartreuse4}{INDPRO} &0.0009& 0.494  &    $\del{-\text{Inf}   ;  \text{ Inf} }$ &   0.003   &0.591 &  $\sbr{0.003;   0.003}$    \\
 \textcolor{orange}{GS1}  &0.0031& 0.571 &      $\del{-\text{Inf}   ;  \text{ Inf} }$ &  0.007&   0.675&   $\sbr{0.007 ;  0.007}$     \\
 \textcolor{orange}{GS5}&$-0.0034$& 0.302&      $\del{-\text{Inf}   ;  \text{ Inf} }$&   0.006 &$ -1.240$ &   $\sbr{0.006;  0.006}$   \\
 \textcolor{blue3}{lag1} &$-0.0047$&  0.912   &  $\del{-\text{Inf}   ;  \text{ Inf} }$& 0.009& $ -3.914  $& $\sbr{0.009;   0.009 }$   \\
  \textcolor{red3}{DPCERA3M086SBEA}  &$-0.0008$& 0.225 &      $\del{-\text{Inf}   ;  \text{ Inf} }$ & 0.002&  $-1.331  $& $\sbr{0.002;   0.002 }$   \\
 \textcolor{orange}{EXUSUKx} &0.0007 & 0.964  &    $\left( -\text{Inf}   ;  -0.051\right] $& 0.003 &  1.357  &  $\sbr{0.003 ;  0.003}$   \\
 \textcolor{blue3}{CLAIMSx} &0.0003& 0.208   &   $\del{-\text{Inf}   ;  \text{ Inf} }$&  0.001  & 0.629  & $\sbr{ 0.001 ;  0.001 }$  \\
 \textcolor{red3}{AMDMNOx} &$-0.0004$ & 0.855     &  $\del{-\text{Inf}   ;  \text{ Inf} }$&  0.002&  $-0.904 $ &$\sbr{0.002 ;  0.002}$  \\ \bottomrule
\end{tabular}  
}
\caption{\(p\)-værdier og konfidensintervaller for variablerne udvalgt af LARS algoritmen. Den estimeres standard afvigelse er \(0.043\), og resultaterne er for \(\widehat{f}_{\text{BIC}} = 0.2623 \) med \(\alpha = 0.1\).} \label{tab:larInf_bic}
\end{table} 

Figur \ref{fig:resid_lars_tg_bic} viser en analyse af de standardiserede residualer for LARS$_{TG}$ (BIC). 
Af tabel \ref{tab:lars_kryds_res_tab} afvises residualerne at være normalfordelte og afhængige i lag 10.







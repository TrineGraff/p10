\subsection{BIC}
Tabel \ref{tab:bic_lars} giver $\widehat{s}$ og $\widehat{f}$, antallet af parameter, BIC og adjusted R$^2$ for LARS uden og med lasso modifikation. 
Adjusted R$^2$ er højst for LARS med lasso modifikationen, men er også den der vælger færrest antal parameter. 
Den vælger kun 17, hvor LARS vælger 20 parameter. 
På figur \ref{fig:coef_plot_lars_bic} ser vi de estimerede koefficienter for LARS algoritmen uden og med lasso modifikation, og igen ser  vi at variablerne \textcolor{blue3}{CE16OV} og \textcolor{blue3}{CLF16OV} er dem med de største værdier, hvor de resterende er meget tæt på nul. 

\begin{table}
\center
\scalebox{0.8}{
\begin{tabular}{lccccc| lccccc} 
\toprule
\multicolumn{6}{c}{LARS (BIC)}  & \multicolumn{6}{c}{Lasso LARS (BIC)} \\ \midrule
& Værdi & BIC & $p$ & R$^2_{\text{adj}}$ & LogLik& & Værdi & BIC & $p$ & R$^2_{\text{adj}}$ & LogLik \\
$f_\text{BIC}$ & 0.2623 & $-6.0925$ & 20 &94.43\% & 975.2909  &$f_\text{BIC}$ &  0.2604 &$-6.1627$& 17 &  94.46\% & 974.9938 \\ \bottomrule
 \end{tabular}}
\caption{Værdien af $f_\text{BIC}$, antallet af parametre, BIC, justeret R$^2$  og log-likelihood for LARS og lasso LARS.} \label{tab:bic_lars}
\end{table}

\imgfigh{coef_plot_lars_bic.pdf}{1}{Estimerede koefficienter for LARS algoritmen uden og med lasso modifikationen, hvor $\widehat{s}$ og $\widehat{f}$ er fundet ud fra BIC. Farverne indikerer hvilken gruppe, variablerne tilhører. }{coef_plot_lars_bic}

Figurerne \ref{fig:lars_bic_resid} og  \ref{fig:lars_lasso_bic_resid} viser en analyse af de standardiserede residualer for lars uden og med lasso modifikation. Vi ser, at histogrammet og QQ-plottet viser tungere haler end en normalfordeling og autokorrelation i første lag. Dette ses også i tabel \ref{tab:lars_kryds_res_tab}, hvor også JB testens nulhypotese om normalitet og LB testen om uafhængighed bliver afvist, når turning parameterene er valgt ud fra BIC. 


\subsubsection{Inferens - LARS uden modifikation}
Tabel  viser p-værdier og konfidensintervaller for LARS algoritmen. Heraf ses, at nulhypotesen accepteres for alle variable.

\begin{table}[ht] 
\centering 
\scalebox{0.8}{
\begin{tabular}{lcccccc}
%\multicolumn{9}{l}{LARS algoritmen} \\ 
\toprule
Prædiktor&  Koefficient & Z-score  & \(p\)-værdi & Konfidensinterval& $\sbr{\mathcal{V}^-;\mathcal{V}^+}$   \\ \midrule
\textcolor{blue3}{HWIURATIO} &0.002  & 0.720   &0.161    &   $\del{-\text{Inf}   ;  \text{ Inf} }$ &  $\sbr{0.002;0.002}$    \\
 \textcolor{blue3}{UEMP15OV} & 0.004  & 1.596  4 &  0.920 &     $\left( -\text{Inf}  ;      0.034 \right] $& $\sbr{0.004 ;0.005}$  \\
\textcolor{blue3}{UEMPLT5}  & 0.001 &  0.148   & 0.065  & $\left[-0.018    ;    \text{ Inf}  \right)$   &$\sbr{0.000;0.001}$    \\
\textcolor{blue3}{MANEMP} &0.003 &  0.561    & 0.766 &       $\left(  -\text{Inf}     ;  0.120\right] $ &$\sbr{0.003; 0.003}$  \\
 \textcolor{blue3}{ UEMP5TO14} & 0.001 & -0.261 &  0.093   &  $\left( - \text{ Inf}    ;  0.023\right] $&   $\sbr{0.000; 0.001}$ \\
\textcolor{blue3}{ CE16OV}   & 0.267& $-37.412$  & 0.130  &    $\left( -\text{Inf}     ;   0.574\right] $  &$\sbr{0.266; 0.267}$ \\
 \textcolor{blue3}{PAYEMS} &  0.000  & 0.012  & 0.428  &    $\del{-\text{Inf}   ;  \text{ Inf} }$ &  $\sbr{0.000 ;  0.000}$   \\
 \textcolor{blue3}{USGOOD} & 0.004 & $-0.584$ & 0.721   &   $\del{-\text{Inf}   ;  \text{ Inf} } $&    $\sbr{0.004 ;  0.004   }$   \\
\textcolor{chartreuse4}{CUMFNS} &  0.002   &0.390  & 0.455     &  $\del{-\text{Inf}   ;  \text{ Inf} }$& $\sbr{0.002 ;  0.002 }$   \\
 \textcolor{blue3}{ CLF16OV}  &   0.243 & 36.646   &  0.179    &  $\del{-\text{Inf}   ;  \text{ Inf} }$&$\sbr{0.243 ;  0.243}$    \\
\textcolor{chartreuse4}{ IPDMAT}& 0.006& $ -1.618$   & 0.869  & $\left[ -0.130  ;      \text{ Inf}  \right)$&   $\sbr{0.006;   0.006}$     \\
 \textcolor{orange}{TB6MS}& 0.006 & $-0.790 $  & 0.615     &  $\del{-\text{Inf}   ;  \text{ Inf} } $  & $\sbr{0.006 ;  0.006}$   \\
 \textcolor{chartreuse4}{INDPRO}&   0.003   &0.591   & 0.494  &    $\del{-\text{Inf}   ;  \text{ Inf} }$ &  $\sbr{0.003;   0.003}$    \\
 \textcolor{orange}{GS1} &  0.007&   0.675  & 0.571 &      $\del{-\text{Inf}   ;  \text{ Inf} }$ &   $\sbr{0.007 ;  0.007}$     \\
 \textcolor{orange}{GS5} &   0.006 &$ -1.240$  & 0.302&      $\del{-\text{Inf}   ;  \text{ Inf} }$&   $\sbr{0.006;  0.006}$   \\
 \textcolor{blue3}{lag1} & 0.009& $ -3.914  $  &  0.912   &  $\del{-\text{Inf}   ;  \text{ Inf} }$& $\sbr{0.009;   0.009 }$   \\
  \textcolor{red3}{DPCERA3M086SBEA}   & 0.002&  $-1.331  $ & 0.225 &      $\del{-\text{Inf}   ;  \text{ Inf} }$& $\sbr{0.002;   0.002 }$   \\
 \textcolor{orange}{EXUSUKx} & 0.003 &  1.357   & 0.964  &    $\left( -\text{Inf}   ;  -0.051\right] $&  $\sbr{0.003 ;  0.003}$   \\
 \textcolor{blue3}{CLAIMSx} &  0.001  & 0.629  & 0.208   &   $\del{-\text{Inf}   ;  \text{ Inf} }$ & $\sbr{ 0.001 ;  0.001 }$  \\
 \textcolor{red3}{AMDMNOx} &  0.002&  $-0.904 $   & 0.855     &  $\del{-\text{Inf}   ;  \text{ Inf} }$&$\sbr{0.002 ;  0.002}$  \\ \bottomrule
\end{tabular}  
}
\caption{\(p\)-værdier, konfidensintervaller samt  $\boldsymbol{\eta}^T\textbf{y}$, dens trunkerede interval og $Z$-score for hver variabel udvalgt af LARS algoritmen. Den estimeres standard afvigelse er \(0.043\), og resultaterne er for \(\widehat{f}_{\text{BIC}} = 0.2623 \) med \(\alpha = 0.1\).} \label{tab:larInf_bic}
\end{table} 
\imgfigh{boxplot_lars_bic.pdf}{1}{Til venstre vises et boxplot af 1000 bootstrap realisationer af $ \widehat{\beta}^{LARS} \del{\widehat{s}}$, hvor $ \widehat{s}$ er estimeret ud fra BIC. Til højere illustreret andelen af bootstrap realisationer, hvor parameter estimaterne er præcis nul. }{boxplot_lars_bic}



\subsubsection{Inferens - LARS med lasso modifikation}
Tabel \ref{tab:covTest_bic}viser resultatet af  kovariansen testen. Vi ser at variablerne \textcolor{blue3}{UEMPL15OV}, \textcolor{blue3}{UEMPLT5}, \textcolor{blue3}{UEMP5TO14}, \textcolor{blue3}{CE16OV} og \textcolor{blue3}{CLF16OV} afviser null hypotesen. Disse variabler, er også de første variabler LARS algoritmen tilføjer.  

\imgfigh{boxplot_lars_lasso_bic.pdf}{1}{Til venstre vises et boxplot af 1000 bootstrap realisationer af $ \widehat{\beta}^{lasso} \del{\widehat{s}}$, hvor $ \widehat{s}$ er estimeret ud fra BIC. Til højere illustreret andelen af bootstrap realisationer, hvor parameter estimaterne er præcis nul. }{boxplot_lars_bic}



\begin{table}[ht] 
\centering 
\begin{tabular}{lccc}
%\multicolumn{3}{l}{LARS algoritmen med lasso modifikation} \\
\toprule
Prædiktor & Cov test & \(p\)-værdi \\
\midrule
\textcolor{blue3}{UEMP15OV}    &       161.3770  &0\\
\textcolor{blue3}{UEMPLT5}   &      163.0670 & 0\\
\textcolor{blue3}{UEMP5TO14}    &    122.3840&  0\\
\textcolor{blue3}{CE16OV}         &   14.7416 & 0\\
\textcolor{blue3}{CLF16OV}        &   221.9181 & 0\\
\textcolor{chartreuse4}{IPDMAT}         &    0.0668&  0.9354\\
\textcolor{orange}{GS5}&   0.3856 & 0.6803\\
\textcolor{blue3}{lag1}       &      0.8897 & 0.4115\\
\textcolor{orange}{TB6MS}  &    0.0419 & 0.9590\\
\textcolor{blue3}{USCONS} &    0.0132&  0.9869\\
\textcolor{blue3}{ DPCERA3M086SBEA}          &  0.0254 & 0.9750\\
\textcolor{orange}{ EXUSUKx} &     0.2309 & 0.7939\\
\textcolor{blue3}{CLAIMSx} &      0.0082 &  0.9919\\
\textcolor{red3}{ AMDMNOx}  &     0.0464 & 0.9546\\
\textcolor{blue3}{lag4 }     &    0.2281&  0.7962\\
\textcolor{cadetblue2}{CPIMEDSL}  &   0.0719&  0.9307\\
\textcolor{blue3}{USTRADE}   &     0.0029 &  0.9971\\ \bottomrule
\end{tabular}
\caption{Kovarians testen for LARS algoritmen med lasso modifikation.
Vi viser kun \(p\)-værdier for prædiktorer som medtages og bliver i modellen for \(\widehat{f}_{1\text{sd}}=0.2604\), dvs hvis en prædiktor medtages i et step og senere forlader modellen, vises denne prædiktor ikke.} \label{tab:covTest_bic}
\end{table} 

\section{LARS}
Først vil vi beskrive least angle regression (LARS) algoritmen, hvorefter vi vil introducere en simpel modifikation, som fører til lasso estimater. \\[2mm]
%
I grove træk fungerer algoritmen som følgende. 
Først sættes alle koefficienter lig nul, og vi finder prædiktoren, som er mest korreleret med responsvariablen \(\y\), denne prædiktor betegnes \(x_{j_1}\).
Der udføres så en simpel lineær regression af \(\y\) på \(x_{j_1}\), hvoraf vi finder en residualvektor.
Vi tager det størst mulige step i retningen af denne prædiktor indtil en anden prædiktor, som betegnes  \(x_{j_2}\), har samme korrelation med den nuværende residualvektor.
Istedet for at fortsætte langs retningen af \(x_{j_1}\) fortsætter LARS i en retning, som er ensvinklet mellem de to prædiktorer, indtil en tredje variabel bliver den mest korreleret variabel.
LARS algoritmen fortsætter da ensvinklet imellem \(x_{j_1}\), \(x_{j_2}\) og \(x_{j_3}\), indtil en fjerde variabel medtages, osv.
LARS algoritmen finder estimaterne \(\widehat{\tmu} = \X \widehat{\tbeta}\) ved at tilføje én prædiktor til modellen i hvert trin, således at præcis \(k\) koefficienter er forskellige fra nul efter \(k\) trin.

Figur \ref{fig:lars} illustrerer algoritmen, hvor $p = 2$ og $\X = \del{\textbf{x}_1 \ \textbf{x}_2}$.
Lad \(\mathbf{c} \del{\widehat{\tmu}}\) betegne de nuværende korrelationer
\begin{align}
\widehat{\mathbf{c}} = \mathbf{c} \del{\widehat{\tmu}} = \X^T \del{\y - \widehat{\tmu}}, \label{eq:lars_1.6}
\end{align}
således at \(\widehat{c}_j\) er proportional med korrelationen mellem prædiktor \(\x_j\) og den nuværende residualvektor.
For \(p=2\) afhænger de nuværende korrelationer kun af projektionen \(\bar{\y}_2\) af \(\y\) på det lineære underrum $\mathcal{L} \del{\X}$ udspændt af \(\x_1\) og \(\x_2\)
\begin{align*}
\textbf{c}\del{\boldsymbol{\widehat{\mu}}} =  \X^T \del{ \bar{\y}_2 - \boldsymbol{\widehat{\mu}}}.
\end{align*}
Algoritmen starter i $\widehat{\boldsymbol{\mu}}_0 = \textbf{0}$.
På figur \ref{fig:lars} ses, at vinklen mellem \(\bar{\y}_2 - \widehat{\boldsymbol{\mu}}_0\) og \(\x_1\) er mindre end vinklen mellem \(\bar{\y}_2 - \widehat{\boldsymbol{\mu}}_0\) og \(\x_2\) og dermed fås \(c_1 \del{\widehat{\boldsymbol{\mu}}_0} > c_2 \del{\widehat{\boldsymbol{\mu}}_0}\).
%\(\bar{\y}_2 - \widehat{\boldsymbol{\mu}}_0\) har en mindre vinkel med \(\x_1\) end \(\x_2\), dvs \(c_1 \del{\widehat{\boldsymbol{\mu}}_0} > c_2 \del{\widehat{\boldsymbol{\mu}}_0}\).
Derfor tilføjer LARS \(\widehat{\boldsymbol{\mu}}_0\) i retningen af \(\x_1\), og vi får
\begin{align*}
\widehat{\tmu}_1 = \widehat{\tmu}_0 + \widehat{\gamma}_1 \x_1,
\end{align*}
hvor stepstørrelsen \(\widehat{\gamma}_1\) vælges, således at korrelationen mellem \(\bar{\y}_2 - \widehat{\tmu}_1\) og \(\x_1\) er lig korrelationen mellem \(\bar{\y}_2 - \widehat{\tmu}_1\) og \(\x_2\).
%  \(\bar{\y}_2 - \widehat{\tmu}_1\) er ligeså korreleret med \(\x_1\) som med \(\x_2\).
Dermed halverer \(\bar{\y}_2 - \widehat{\boldsymbol{\mu}}_1\) vinklen mellem \(\x_1\) og \(\x_2\), således at \(c_1 \del{\widehat{\boldsymbol{\mu}}_1} = c_2 \del{\widehat{\boldsymbol{\mu}}_1}\).
%
\begin{figure}[H]
\centering
\scalebox{0.8}{\begin{tikzpicture}
%\draw [<-] (4,0) node [below] {$\x_1$}-- (-3,0);
\draw [blue] [<-] (1,0) node [below] {$\widehat{\tmu}_1$} -- (-3,0);
\draw [<-] (4,0) node [below] {$\x_1$} -- (1,0);
\filldraw [blue] (1,0) circle (2pt) node [below, black] {$\widehat{\boldsymbol{\mu}}_1$};
\filldraw [green] (-3,0)  circle (2pt) node [below, black]{$\widehat{\boldsymbol{\mu}}_0$};
\draw [dashed] [<-] (5,4) node [above] {$\x_2$} --(1,0);
\draw [<-] (1,4) node [above] {$\x_2$} --(-3,0);

\draw [green] (5,1.66) node [above, black] {$\bar{\y}_2$} -- (-3,0);
\filldraw [green] (5,1.66) circle (2pt);
\draw [blue] [->] (1,0) -- (3, 0.83) node [below, black] {$\mathbf{u}_2$} ;
\draw [green] (3,0.83) -- (5,1.66) ; 

\draw [green] (5,0) node [below] {} -- (4.1,0);
\filldraw [green] (5,0) circle (2pt) ;
\draw (5,0) node [black, below] {$\bar{\y}_1$};
\end{tikzpicture}}
\caption{LARS algoritmen for \(p=2\). \(\bar{\y}_2\) er projektionen af \(\y\) på det lineære underrum \(\mathcal{L} \del{\x_1, \x_2}\).
Algoritmen starter i \(\widehat{\tmu}_0=\mathbf{0}\), hvor residualvektoren \(\bar{\y}_2 - \widehat{\tmu}_0\) har en større korrelation med \(\x_1\) end \(\x_2\). Næste LARS estimat er \(\widehat{\tmu}_1 = \widehat{\tmu}_0 + \widehat{\gamma}_1 \x_1\), hvor \(\widehat{\gamma}_1\) vælges, således at \(\bar{\y}_2 - \widehat{\tmu}_1\) halverer vinklen mellem \(\x_1\) og \(\x_2\). Næste LARS estimat er \(\widehat{\tmu}_2 = \widehat{\tmu}_1 + \widehat{\gamma}_2 \mathbf{u}_2\), hvor \(\mathbf{u}_2\) er en enhedsvektor, som ligger langs denne halveringslinje.
Der gælder, at \(\widehat{\tmu}_2 = \bar{\y}_2\) for \(p=2\), dette er ikke tilfældet for \(p>2\), som det ses på figur \ref{fig:lars2}.
 }\label{fig:lars}
\end{figure}
%
Lad $\mathbf{u}_2$ være enhedsvektoren, som ligger langs denne halveringslinje.
Det næste LARS estimat er dermed
\begin{align*}
\widehat{\boldsymbol{\mu}}_2 = \widehat{\boldsymbol{\mu}}_1+ \widehat{\gamma}_2 \mathbf{u}_2,
\end{align*}
hvor $\widehat{\gamma}_2$ er valgt, således at $\widehat{\boldsymbol{\mu}}_2 = \bar{\textbf{y}}_2$ i dette tilfælde hvor $p = 2$. 
For \(p>2\), da vil stepstørrelsen \(\widehat{\gamma}_2\) være mindre, hvilket fører til en anden ændring af retningen, som illustreres på figur \ref{fig:lars2}.
%
\begin{figure}[H]
\centering
\scalebox{0.8}{\begin{tikzpicture}
\filldraw [green] (-3,0) circle (2pt) node [below, black]{$\widehat{\tmu}_0$};
\draw [<-] (6.8,0) node [below] {$\x_1$} -- (-3,0);
\draw [<-] (1,4) node [above] {$\x_2$} --(-3,0);
\draw [<-] (-5,4) node [above] {$\x_3$} --(-3,0);

\draw [green] (4,0) node [above, black] {$\bar{\y}_1$} -- (-3,0);
\filldraw [green] (4,0) circle (2pt) ;
\draw [blue] (1,0) node [below, black] {$\widehat{\tmu}_1$} -- (-3,0);
\draw [blue] [->] (-3,0) -- (-1.5, 0) node [below, black] {$\mathbf{u}_1$} ;

\draw [green] (6,2.1) node [above, black] {$\bar{\y}_2$} -- (1,0);
\filldraw [green] (6,2.1) circle (2pt) ;
\draw [blue] (4,1.25) node [below, black] {$\widehat{\tmu}_2$} -- (1,0);
\draw [blue] [->] (1,0) -- (2.6, 0.65) node [below, black] {$\mathbf{u}_2$} ;

\draw [green] (5,3.5) node [above, black] {$\bar{\y}_3$} -- (4,1.25);
\filldraw [green] (5,3.5) circle (2pt) ;
\draw [blue] [<-] (4.4,2.2)  -- (4,1.25);
\draw [blue] (4.8,3.1)-- (4.4,2.2);
\draw [blue] [<-] (4.5,3.5) -- (4.8,3.1);
\end{tikzpicture}}
\caption{I hvert trin nærmer LARS estimatet \(\widehat{\tmu}_k\) sig det tilhørende OLS estimat \(\bar{\y}_k\), men vil aldrig nå det.
 }\label{fig:lars2}
\end{figure}
%
Efterfølgende LARS trin tages langs ensvinklede vektorer, som generaliserer vektoren \(\mathbf{u}_2\) i figur \ref{fig:lars}.
Vi antager, at prædiktorerne \(\x_1, \ldots, \x_p\) er lineært uafhængige.
Lad \(\A\) være er en delmængde af indekser \(\cbr{1,\ldots, p}\), og definer matricen
\begin{align}
\X_\A = \del{\dots s_j \x_j \dots}_{j \in \A}, \label{eq:lars_2.4}
\end{align}
hvor $s_j = \pm 1$ og \(\X_\A\) er en matrix, som består af søjlerne i \(\X\), der er inkluderet i \(\mathcal{A}\) gange med \(s_j\).
Lad 
\begin{align}
\mathbf{N}_\A = \X_\A^T \X_\A \quad \text{og} \quad A_\A = \del{\mathbf{1}_\A^T \mathbf{N}_\A^{-1} \mathbf{1}_\A}^{-1/2}, \label{eq:lars_2.5}
\end{align}
hvor \(\mathbf{1}_\A\) er en vektor af 1-taller med en længde lig antallet af elementer i \(\A\).
Da defineres en såkaldt ensvinklet vektor
\begin{align}
\mathbf{u}_\A = \X_\A \omega_\A, \quad \text{hvor } \omega_\A = A_\A \mathbf{N}_\A^{-1} \mathbf{1}_\A, \label{eq:lars_2.6}
\end{align}
som er en enhedsvektor, der gør vinkler mellem søjlerne i \(\X_\A\) lige store, dvs
%der resulterer i lige store vinkler der er mindre end \(90^0\), med søjlerne i \(\X_\A\), dvs
\begin{align}
\X_\A^T \mathbf{u}_\A = A_\A \mathbf{1}_\A \quad \text{og} \quad \Vert \mathbf{u}_\A \Vert_2^2 = 1. \label{eq:lars_2.7}
\end{align}
Antag \(\widehat{\tmu}_\A\) er det nuværende LARS estimat.
Lad \(\widehat{\mathbf{c}} = \X^T \del{\y - \widehat{\boldsymbol{\mu}}_\A}\) være en vektor af nuværende korrelationer \eqref{eq:lars_1.6}.
Den \textit{aktive mængde} \(\A\) er en mængde af indekser, som svarer til prædiktorerne med de største absolutte korrelationer
\begin{align}
\widehat{C} = \max_j \cbr{\abs{\widehat{c}_j}}  \qquad \text{og} \qquad \A= \cbr{j: \ \abs{ \widehat{c}_j} = \widehat{C}}. \label{eq:lars_2.9}
\end{align}
Lad 
\begin{align}
s_j = \text{sign} \del{\widehat{c}_j}, \quad j \in \A. \label{eq:lars_2.10}
\end{align}
%
Herefter kan vi give en fyldestgørende beskrivelse af LARS algoritmen.
%
\begin{alg} [LARS algoritmen]
\begin{enumerate}
\item Standardisér prædiktorerne og centrér responsvariablen. 
Start med \(\widehat{\boldsymbol{\mu}}_0 = \mathbf{0}\), \(\widehat{\mathbf{c}} = \X^T \y\), og \(\A = \emptyset\).
\item Find prædiktoren \(\tx_j\) med den største værdi af \(\abs{\widehat{c}_j}\) og definer den aktive mængde \(\A = \cbr{j}\).
\item 
Gentag følgende indtil alle prædiktorer er indeholdt i den aktive mængde:
\begin{itemize}
\item Udregn \(\widehat{\mathbf{c}}\), \(\widehat{C}\), \(\X_\A\), \(A_\A\) og \(\mathbf{u}_\A\) som i \eqref{eq:lars_2.4}-\ref{eq:lars_2.9} samt
\begin{align*}
\mathbf{a} = \X^T \mathbf{u}_\A.
\end{align*}
\item Opdatér \(\widehat{\boldsymbol{\mu}}_\A\) til
\begin{align}
\widehat{\boldsymbol{\mu}}_{\A_+} = \widehat{\boldsymbol{\mu}}_\A + \widehat{\gamma} \mathbf{u}_\A, \label{eq:lars_2.12}
\end{align}
hvor 
\begin{align}
\widehat{\gamma} = \min_{j \in \A^c} \cbr{ \frac{\widehat{C}- \widehat{c}_j}{A_\A - a_j} , \frac{\widehat{C} + \widehat{c}_j}{A_\A + a_j}}_+, \label{eq:lars_2.13}
\end{align}
og hvor \(\min \cbr{\cdot , \cdot}_+\) indikerer, at minimum kun tages over de positive komponenter indenfor valget af \(j\) i \eqref{eq:lars_2.13}.
\item Sæt \(\A = \A \cup \cbr{\widehat{j}}\), hvor \(\widehat{j}\) er minimeringsindekset i \eqref{eq:lars_2.13}.
\end{itemize}
\end{enumerate}
\end{alg}
%
For LARS algoritmen kræves blot \(p\) trin for at finde den fulde løsning.
De beregningsmæssige omkostninger for LARS algoritmen er af samme orden som løsningen af OLS med \(p\) prædiktorer.

%Formlerne \eqref{eq:lars_2.12} og \eqref{eq:lars_2.13} har følgende fortolkning.
%Definer
%\begin{align}
%\tmu \del{\gamma} = \widehat{\tmu}_\A + \gamma \mathbf{u}_\A, \label{eq:lars_2.14}
%\end{align}
%for \(\gamma > 0\), således at den nuværende korrelation er givet ved
%\begin{align}
%c_j \del{\gamma} = \x_j^T \del{\y - \tmu \del{\gamma}} = \widehat{c}_j - \gamma a_j. \label{eq:lars_2.15}
%\end{align}
%For \(j \in \A\) giver \eqref{eq:lars_2.7} og \eqref{eq:lars_2.9} at
%\begin{align}
%\abs{c_j \del{\gamma}} = \widehat{C} - \gamma A_\A,\label{eq:lars_2.16}
%\end{align}
%som viser, at alle af de maksimale absolutte nuværende korrelationer falder ligeligt ?????
%For \(j \in \A^C\), viser \eqref{eq:lars_2.15} og \eqref{eq:lars_2.16} at \(c_j \del{\gamma}\) er lig den maksimale værdi i \(\gamma = \frac{\widehat{C} - \widehat{c}_j}{A_\A - a_j}\).
%Derfor er \(\widehat{\gamma}\) i \eqref{eq:lars_2.13} den mindst positive værdi af \(\gamma\), således at et nyt indeks \(\widehat{j}\) tilføjes til den aktive mængde.
%\(\widehat{j}\) er minimeringsindekset i \eqref{eq:lars_2.13} og den nye aktive mængde \(\A_+\) er \(\A \cup \cbr{\widehat{j}}\) og den nye maksimum absolut korrelation er \(\widehat{C}_+ = \widehat{C}- \widehat{\gamma} A_\A\).

\subsection{Lasso modifikation} \label{subsec:lasso_modifikation}
I dette afsnit beskrives en simpel modifikation af LARS algoritmen, således at den giver lasso estimater.
Lad \(\widehat{\tbeta}^\text{lasso}\) være løsningen til lasso problemet \eqref{eq:2.5} med \(\widehat{\tmu}^\text{lasso} = \X \widehat{\tbeta}^\text{lasso}\).
Da kan det vises, at fortegnet af enhver ikke-nul koefficient \(\widehat{\beta}_j\) og fortegnet \(s_j\) af den nuværende korrelation \(\widehat{c}_j = \x_j^T \del{\y - \widehat{\tmu}}\) må stemme overens
\begin{align}
\text{sign} \del{\widehat{\beta}_j } = \text{sign} \del{\widehat{c}_j } = s_j, \quad j \in \A. \label{eq:lars_3.1}
\end{align}
%
%\begin{lem}
%For \(\widehat{\beta}^\text{lasso}\) må der gælder, at
%\begin{align*}
%\widehat{c}_j = \widehat{C} \cdot \text{sign} \del{\widehat{\beta}_j},
%\end{align*}
%hvor \(\widehat{c}_j = \x_j^T \del{\y - \widehat{\tmu}}= \x_j^T \del{\y - \X \widehat{\beta}}\).
%Dette medfører, at
%\begin{align}
%\text{sign} \del{\widehat{\beta}_j } = \text{sign} \del{\widehat{c}_j }, \quad j \in \A \label{eq:lars_5.29}
%\end{align}
%\end{lem}
%
Denne restriktion er ikke inkluderet i LARS algoritmen, men kan nemt modificeres hertil:
\textit{Når en ikke-nul koefficient ændrer fortegn eller bliver lig nul, da fjernes variablen fra den aktive mængde og vi beregner igen den nuværende ensvinklede retning \eqref{eq:lars_2.12}}.

For at tage denne modifikation i betragtning, defineres en \(p \times 1\) vektor
\begin{align*}
\widehat{\mathbf{d}} = \begin{cases}
s_j \omega_{\A_j}, &\text{hvis } j \in \A, \\
0, & \text{ellers},
\end{cases}
\end{align*}
hvor \(\omega_{\A_j}\) betegner elementet af vektoren \(\omega_{\A}\), som svarer til indeks \(j\).
For \(j \in \A\) opdateres
\begin{align*}
\widehat{\beta}_j \del{\gamma} = \widehat{\beta}_j^\text{prev} + \gamma \widehat{d}_j,
\end{align*}
hvor \(\widehat{\beta}_j^\text{prev}\) er lasso estimaterne fra det tidligere trin.
Lad 
\begin{align*}
\gamma_j = -\frac{\widehat{\beta}_j}{\widehat{d}_j}, \qquad \text{og} \qquad \tilde{\gamma} = \min_{\gamma_j > 0} \cbr{\gamma_j}.
\end{align*}
%
%Antag vi netop har fuldendt et LARS step, som har givet en ny aktiv mængde \(\A\) som i \eqref{eq:lars_2.9}, og at det tilhørende LARS estimat \(\widehat{\tmu}_\A\) svarer til en lasso løsning \(\widehat{\tmu}^\text{lasso} = \X \widehat{\tbeta}^\text{lasso}\).
%Lad
%\begin{align*}
%\omega_\A = A_\A \mathbf{N}_\A^{-1} \mathbf{1}_\A,
%\end{align*}
%være en vektor med længde lig antallet af elementer i \(\A\) og definer en \(p \times 1\) vektor
%\begin{align*}
%\widehat{\mathbf{d}} = \begin{cases}
%s_j \omega_{\A_j}, &\text{hvis } j \in \A, \\
%0, & \text{ellers}.
%\end{cases}
%\end{align*}
%Hvis vi bevæger os i den positive \(\gamma\) retning langs LARS linjen \eqref{eq:lars_2.14}, ser vi, at
%\begin{align*}
%\tmu \del{\gamma} = \X \tbeta \del{\gamma}, \quad \text{hvor } \beta_j \del{\gamma} = \widehat{\beta}_j + \gamma \widehat{d}_j
%\end{align*}
%for \(j \in \A\).
%Derfor vil \(\beta_j \del{\gamma}\) ændre fortegn i
%\begin{align*}
%\gamma_j = -\frac{\widehat{\beta}_j}{\widehat{d}_j},
%\end{align*}
%den første af sådan en ændring kommer i
%\begin{align*}
%\tilde{\gamma} = \min_{\gamma_j > 0} \cbr{\gamma_j},
%\end{align*}
%for prædiktor \(\x_{\tilde{j}}\).
%Hvis der ikke findes en \(\gamma_j > 0\), da er \(\tilde{\gamma}=\infty\) per definition.
%
%Hvis \(\tilde{\gamma} < \widehat{\gamma}\), da kan \(\beta_{\tilde{j}} \del{\gamma}\) ikke være lasso løsningen for \(\gamma > \tilde{\gamma}\), da restriktionen \eqref{eq:lars_3.1} ikke er opfyldt, eftersom \(\beta_{\tilde{j}} \del{\gamma}\) har ændret fortegn, mens \(c_{\tilde{j}}\) ikke har.
%Der gælder, at \(c_{\tilde{j}}\) ikke kan ændre fortegn indenfor ét LARS trin da \(\abs{c_{\tilde{j}} \del{\gamma}} = \widehat{C} - \gamma A_\A> 0\) af \eqref{eq:lars_2.16}. \\[2mm]
%%
%\textbf{Lasso modifikation} \\
Hvis \(\tilde{\gamma} < \widehat{\gamma}\), stoppes det igangværende LARS trin i \(\gamma = \tilde{\gamma}\) og fjern \(\tilde{j}\) fra udregningen af den næste ensvinklede retning.
Dvs
\begin{align*}
\widehat{\tmu}_{\A_+} = \widehat{\tmu}_\A + \tilde{\gamma} \mathbf{u}_\A \quad \text{og} \quad \A_+ = \A - \cbr{\tilde{j}},
\end{align*}
istedet for \eqref{eq:lars_2.12}.

En mere detaljeret gennemgang af LARS algoritmen med lasso modifikationen kan findes i \citep{efron}.
Da variable kan fjernes og tilføjes til den aktive mængde, er antallet af trin i LARS algoritmen med lasso modifikationen større end \(p\).

\begin{eks}
Figur \ref{fig:diabetes_lars} illustrerer koefficientstierne for LARS algoritmen uden og med lasso modifikationen, som funktion af fraktionen af \(\ell_1\)-normen for diabetes data.
Hvis \(\frac{\abs{\tbeta}}{\max \abs{\tbeta}} = 0\), da er ingen variable tilføjet til den aktive mængde og hvis \(\frac{\abs{\tbeta}}{\max \abs{\tbeta}} = 1\) er alle variable inkluderet.
Af figuren kan vi aflæse rækkefølgen, hvori variablerne medtages i modellen.
For LARS algoritmen uden lasso modifikationen udføres 10 trin, mens LARS algoritmen med lasso modifikationen udfører 12 trin.
De 2 ekstra trin som LARS algoritmen med lasso modifikationen udfører, kommer af, at variablen \texttt{hdl} fjernes og tilføjes igen i henholdsvis trin 11 og 12.
Heraf ses det også, at stien er kontinuert og stykvis lineær.
Den aktive mængde og fortegnene af de aktive variable er konstant mellem trinene.

\imgfigh{diabetes_lars.pdf}{0.9}{Koefficientstierne for LARS algoritmen uden og med lasso modifikationen som funktion af fraktionen af \(\ell_1\)-normen for diabetes data.}{diabetes_lars}
\end{eks}

%Coordinate descent kan være hurtigere end LARS algoritmen særligt for store problemer.
%Dette skyldes, at \eqref{eq:update_coordinate} hurtigt kan opdateres som \(j\) varierer, og ofte er opdateringen at lade \(\beta_j = 0\).
%Coordinate descent algoritmen for lasso giver ikke den fulde lasso løsningssti som LARS algoritmen, men kan bruges til at udregne lasso løsningerne for \(\Lambda\).





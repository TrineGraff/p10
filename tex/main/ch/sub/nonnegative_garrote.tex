\section{Nonnegative garrote} \label{sec:nonnegativegarrote}
\textit{Nonnegative garrote}, introduceret i \citep{nonnegative_garrote}, er en to-trins procedure, som er tæt relateret til lasso. \footnote{\citep{nonnegative_garrote} var inspirationen til \citep{lasso}} 
\begin{defn}[Nonnegative garrote]
Givet et initial estimat af regressionskoefficienterne \(\tilde{\tbeta} \in \mathbb{R}^p\), kan vi løse optimeringsproblemet
\begin{align}
\argmin_{\mathbf{c} \in \mathbb{R}^p}  \cbr{ \sum_{i=1}^n \del{y_i - \sum_{j=1}^p c_j x_{ij} \tilde{\beta}_j}^2}, \ \text{underlagt at } \mathbf{c} \geq \mathbf{0} \text{ og } \Vert \mathbf{c} \Vert_1 \leq t, \label{eq:2.19}
\end{align}
hvor \(\mathbf{c} \geq \mathbf{0}\) betyder, at elementer af \(\mathbf{c}\) er ikke-negative.
Estimatoren for nonnegative garrote er da givet ved \(\widehat{\beta}_j^\text{NG} = \widehat{c}_j \cdot \tilde{\beta}_j\), for \(j = 1, \ldots, p\).
\end{defn}
Der er et ækvivalent Lagrange problem for denne procedure
\begin{align}
\argmin_{\mathbf{c} \in \mathbb{R}^p}  \cbr{\Vert \y - \X \tbeta \Vert_2^2 + \lambda \Vert \mathbf{c} \Vert_1}, \ \text{underlagt at } \mathbf{c} \geq \mathbf{0}, \label{eq:nonnegative_garrote}
\end{align}
hvor \(\lambda \geq 0\).
I den originale artikel \citep{nonnegative_garrote}, vælges \(\tilde{\tbeta}\) til at være \(\widehat{\tbeta}^\text{OLS}\).
 
Antag \(\X\) er ortogonal, og \(t\) er valgt, således at betingelsen \(\Vert \mathbf{c} \Vert_1 = t\) er opfyldt, da er
\begin{align*}
\widehat{c}_j = \del{1 - \frac{\lambda}{\tilde{\beta}_j^2}}_+, \ j = 1, \ldots, p,
\end{align*}
hvor \(\lambda\) er valgt, således at \(\Vert \widehat{\mathbf{c}} \Vert_1 = t\).
Hvis koefficienten \(\tilde{\beta}_j\) er høj, da vil faktoren \(\widehat{c}_j\) være tæt på 1, dvs ingen straf.
Omvendt hvis koefficienten \(\tilde{\beta}_j\) er lav, da vil estimatet \(\widehat{c}_j\) blive trukket mod 0.
På figur \ref{fig:nonnegative_garrote} ses, at nonnegative garrote straffer lave værdier af \(\beta\) hårdere end lasso, og omvendt for høje værdier.
%
\begin{figure}[H]
\centering
\scalebox{0.8}{\begin{tikzpicture}
\draw[loosely dotted] (-3,-3) -- (3,3);
\draw[dotted] (-3,-2.2) -- (-0.75,0) -- (0.75,0) -- (3,2.2);
\draw[dashed] (-3,-2.6) -- (-1,0) -- (1,0) -- (3,2.6);
\draw [<-] (0,3.5) node [left] {$\widehat{\beta}$}-- (0,-3.5);
\draw[<-] (3.5,0) node [below] {$\beta$} -- (-3.5,0);
\node[draw,align=left] at (-2.5,2.1) { \tikz[baseline]{\draw[dotted] (0,.5ex)--++(.5,0) ;} lasso \\
\tikz[baseline]{\draw[dashed] (0,.5ex)--++(.5,0) ;} nonnegative garrote};
\end{tikzpicture}}
\caption[optional short text]{Straf af lasso (\tikz[baseline]{\draw[dashed] (0,.5ex)--++(.5,0) ;}) og nonnegative garrote (\tikz[baseline]{\draw[dotted] (0,.5ex)--++(.5,0) ;}) for en variabel. 
%Da parametriseringen af \(\lambda\) er forskellig, har vi anvendt \(\lambda = 2\) for lasso og \(\lambda = 7\) for nonnegative garrote, således at de kan sammenlignes.
%Eksakte løsninger for lasso (\tikz[baseline]{\draw[dashed] (0,.5ex)--++(.5,0) ;}) og nonnegative garrote (\tikz[baseline]{\draw[dotted] (0,.5ex)--++(.5,0) ;})
} \label{fig:nonnegative_garrote}
\end{figure}
%

Nonnegative garrote er et specialtilfælde af adaptive lasso med en ekstra betingelse, som vi vil diskutere nærmere i afsnit \ref{sec:asymptotics_nonnegative}.

%\citep{nonnegative_garrote_2007} og \citep{zou_hastie} viste, at nonnegative garrote er \textit{path-konsistent} under mindre strenge betingelser end lasso.
%Dette gælder, hvis initial estimaterne er \(\sqrt{n}\)-konsistent.
%Af sætning \ref{thm:asymp_ols} har vi, at \(\widehat{\tbeta}^\text{OLS}\) er \(\sqrt{n}\)-konsistent, og af sætning \ref{thm:asymp_lasso} er \(\widehat{\tbeta}^\text{OLS}\) er \(\sqrt{n}\)-konsistent.
%% som inkluderer OLS (når \(p < n\)), lasso, ridge regression og elastisk net.
%``Path-konsistent'' betyder, at løsningsstien inkluderer den sande model.
%Dog er konvergensen af parameter estimaterne for nonnegative garrote langsommere end den er for initial estimatet.
%
%
%\citep{nonnegative_garrote_2007} beviste, at hvis initial estimatet er konsistent i estimation, da er nonnegative garotte estimatet konsistent i estimation mens også i variabeludvælgelse.

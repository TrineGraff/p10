For en estimator, som udfører variabeludvælgelse, findes nogle såkaldte \textit{orakelegenskaber}.
En estimator, som opfylder disse egenskaber, er konsistent i variabeludvælgelse og følger asymptotisk en normalfordeling.

Lad $\mathcal{A} =\cbr{j:\beta_j^* \neq 0}$ betegne den aktive mængde, hvor $\beta_j^*$ betegner koefficienten af $\mathbf{x}_j$ i den sande model og antag at $\vert \mathcal{A} \vert=p_0 <p$, således at den sande model blot afhænger af en delmængde af prædiktorerne. 
\begin{defn}[Orakelegenskaber]
\begin{itemize}
\item Variabeludvælgelsen er konsistent, dvs for
\begin{align*}
\mathcal{A}_n=\lbrace j :\widehat{\beta}_j \neq 0 \rbrace \ \text{og} \ \mathcal{A} =\{j:\beta_j^* \neq 0\},
\end{align*}
gælder der, at $\lim_{n \rightarrow \infty }P(\mathcal{A}_n=\mathcal{A})=1$.
\item Estimatoren er asymptotisk normalfordelt, dvs
\begin{align*}
\sqrt{n}(\widehat{\boldsymbol{\beta}}_\mathcal{A}-\boldsymbol{\beta}^*_\mathcal{A}) \overset{d}{\rightarrow} N(\mathbf{0}, \boldsymbol{\Sigma}^*_I),
\end{align*}
hvor $\boldsymbol{\beta}^*_\mathcal{A}=\{ \beta_j^*, j \in \mathcal{A} \}$ og $\boldsymbol{\Sigma}^*_I$ er kovariansmatricen, hvor vi antager, at vi kender den sande model.
\end{itemize}
\end{defn}
En god procedure bør have disse orakelegenskaber.
Dog bør proceduren have nogle ekstra betingelser udover orakelegenskaberne for at være optimal.
Derfor er det vigtigt at understrege, at orakelegenskaber ikke alene resulterer i en optimal procedure.
Lad $\mathcal{A} =\cbr{j:\beta_j^* \neq 0}$ betegne den aktive mængde, hvor $\beta_j^*$ betegner koefficienten til $\mathbf{x}_j$ i den sande model og antag at $\abs{ \mathcal{A}}=p_0 <p$, således at den sande model blot afhænger af en delmængde af prædiktorerne.
Nedenfor introduceres orakelegenskaberne. 
\begin{defn}[Orakelegenskaber] \label{defn:orakel}
\begin{itemize}
\item Variabeludvælgelsen er konsistent, dvs for
\begin{align*}
\mathcal{A}_n=\cbr{ j :\widehat{\beta}_j \neq 0} \quad \text{og} \quad \mathcal{A} =\cbr{j:\beta_j^* \neq 0},
\end{align*}
gælder der, at $\lim_{n \rightarrow \infty } \mathbb{P}\del{\mathcal{A}_n=\mathcal{A}}=1$.
\item Estimatoren er asymptotisk normalfordelt, dvs
\begin{align*}
\sqrt{n}\del{\widehat{\boldsymbol{\beta}}_\mathcal{A}-\boldsymbol{\beta}^*_\mathcal{A}} \overset{d}{\rightarrow} N \del{\mathbf{0}, \boldsymbol{\Sigma}^*_I},
\end{align*}
hvor $\boldsymbol{\beta}^*_\mathcal{A}=\cbr{ \beta_j^*, j \in \mathcal{A}}$ og $\boldsymbol{\Sigma}^*_I$ er den sande kovariansmatrix.
\end{itemize}
\end{defn}
En god procedure bør besidde disse orakelegenskaber, dog bør proceduren underlægges nogle ekstra betingelser udover orakelegenskaberne for at være optimal.
Derfor er det vigtigt at understrege, at orakelegenskaber ikke alene resulterer i en optimal procedure.
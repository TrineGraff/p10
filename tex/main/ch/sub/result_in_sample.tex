\section{Oversigt over in-sample resultater} 
For at få et bedre overblik over de valgte variable for de 18 modeller af lasso og dens generaliseringer betragtes tabel \ref{tab:topvariable}.
Tabellen viser de 9 hyppigst valgte variable samt beskrivelsen af dem. 
Heraf observeres, at variablerne \textcolor{blue3}{CLF16OV} og \textcolor{blue3}{CE16OV} bliver valgt af alle modeller.
Variablerne \textcolor{chartreuse4}{IPDMAT}, \textcolor{blue3}{UEMPLT5}, \textcolor{blue3}{UEMP5TO14}, \textcolor{blue3}{UEMP15OV}, \textcolor{orange}{TB6MS} og \textcolor{orange}{GS5} vælges af alle undtagen adaptive lasso modellerne, mens \textcolor{blue3}{lag 1} også vælges af adaptive lasso med lasso vægte (BIC).

\begin{table}[ht] 
\centering 
\begin{tabular}{lll}
\toprule 
Antal & Variable & Beskrivelse \\ \midrule
16 &\textcolor{blue3}{CLF16OV} & Civilarbejdsstyrke \\
16 &\textcolor{blue3}{CE16OV} & Civilbeskæftigelse \\
13 & \textcolor{blue3}{lag 1} & Den tidligere værdi af arbejdsløshedsraten \\
12 & \textcolor{chartreuse4}{IPDMAT} & Holdbart materiale  \\
12 & \textcolor{blue3}{UEMPLT5} & Civile arbejdsløse - mindre end 5 uger \\
12 & \textcolor{blue3}{UEMP5TO14}& Civile arbejdsløse i 5 - 14 uger \\
12 & \textcolor{blue3}{UEMP15OV} &  Civile arbejdsløse i 15 - 26 uger  \\
12 & \textcolor{orange}{TB6MS} & 6-måneders statsskat  \\
12 & \textcolor{orange}{GS5} & 5-årig statsobligationsrente \\
\bottomrule
\end{tabular}  
\caption{Antallet af gange variablerne vælges af de ialt 16 modeller samt beskrivelse af variablerne.} \label{tab:topvariable}
\end{table} 

Justeret R$^2$ er mindst for ridge regression modellerne og størst for lasso modellerne med undtagelse af lasso LARS (CV).
Krydsvalidering tager ikke højde for en models kompleksitet, men idet vi vælger tuning parameteren \(f_\text{1sd}\) eller \(\lambda_\text{1sd}\) for alle med undtagelse af ridge regression, vælger vi i virkeligheden på baggrund af modellens kompleksitet.
Dette kan skyldes, at vi ikke umiddelbart observerer en forskel i justeret \(R^2\) ved at anvende krydsvalidering fremfor BIC for coordinate descent algoritmen.
Dog ses en lille forbedring for LARS algoritmen, hvis vi betragter BIC fremfor krydsvalidering.

Mht lasso problemet, så udvælger krydsvalidering henholdsvis 13 og 14 variable, mens BIC vælger 17 variable, og dermed er antallet af variable ikke specielt afhængig af, hvilken optimeringsalgoritme der anvendes.
%På trods af forskellige optimeringsalgoritmer vælger krydvalidering henholdsvis 13 og 14 variable, mens BIC vælger 17 variable for lasso problemet.
Optimeringsalgoritmerne giver også omtrent samme justerede \(R^2\) for lasso problemet.

Derudover ses, at de 18 modeller alle har klart bedre in-sample resultater end benchmark modellen. 

På figur \ref{fig:cf_interval} vises det betingede 95\% konfidensinterval for lasso\(_{TG}\) (CV) og lasso\(_{TG}\) (BIC), som er givet i tabel \ref{tab:fixedLassoInf} og \ref{tab:fixedLassoInf_bic}, samt 95\% konfidensinterval for OLS estimatorerne for de udvalgte variable af lasso modellerne. 
For lasso\(_{TG}\) (CV) observeres, at konfidensintervallerne er approksimativt ens for høje estimerede koefficienter, dvs for variablerne \textcolor{blue3}{CLF16OV} og \textcolor{blue3}{CE16OV}, mens  konfidensintervallet for lasso\(_{TG}\) (CV) er større end OLS konfidensintervallet for lave estimerede koefficienter.
Konfidensintervallet for de estimerede koefficienter af lasso\(_{TG}\) (BIC) er generelt større end konfidensintervallet for OLS estimatorerne for disse prædiktorer.

\imgfigh{cf_interval.pdf}{1}{Betinget 95\% konfidensinterval for lasso\(_{TG}\) (CV) og lasso\(_{TG}\) (BIC) (rød) og 95\% konfidensinterval for OLS estimatorerne for de udvalgte prædiktorer (blå).}{cf_interval}

For LARS\(_{TG}\) (CV) og LARS\(_{TG}\) (BIC) observeres, at grænserne i  konfidensintervallerne ofte er uendelige, og derfor betragter vi ikke disse.







\section{Oversigt over in-sample resultater} 
For at forsimple formuleringer anvender vi CV og BIC efter modellerne til at betegne, hvordan modellernes tuning parametre er estimeret.

For at få et bedre oversigt over vores 14 modeller i in-sample betragtes tabel \ref{tab:topvariable}.
Den viser de 9 mest valgte variable samt beskrivelse. Vi ser at variablerne \textcolor{blue3}{CLF16OV} og \textcolor{blue3}{CE16OV} bliver valgt af alle 14 modeller, imens \textcolor{blue3}{lag1} ikke blev valgt af adaptive lasso med OLS vægte (CV), adaptive lasso med lasso vægte (CV) og adaptive lasso med OLS vægte (BIC). 
De resterende variabler bliver ikke valgt af adaptive lasso modellerne. 

\begin{table}[ht] 
\centering 
\begin{tabular}{lll}
\toprule 
Antal & Variable & Beskrivelse \\ \midrule
16 &\textcolor{blue3}{CLF16OV} & Civilarbejdsstyrke \\
16 &\textcolor{blue3}{CE16OV} & Civilbeskæftigelse \\
13 & \textcolor{blue3}{lag 1} & Den tidligere værdi af arbejdsløshedsraten \\
12 & \textcolor{chartreuse4}{IPDMAT} & Holdbart materiale  \\
12 & \textcolor{blue3}{UEMPLT5} & Civile arbejdsløse - mindre end 5 uger \\
12 & \textcolor{blue3}{UEMP5TO14}& Civile arbejdsløse i 5 - 14 uger \\
12 & \textcolor{blue3}{UEMP15OV} &  Civile arbejdsløse i 15 - 26 uger  \\
12 & \textcolor{orange}{TB6MS} & 6-måneders statsskat  \\
12 & \textcolor{orange}{GS5} & 5-årig statsobligationsrente \\
\bottomrule
\end{tabular}  
\caption{Antallet af gange variablerne vælges af de ialt 16 modeller samt beskrivelse af variablerne.} \label{tab:topvariable}
\end{table} 

I forhold til den justerede R$^2$ ser vi, at ridge regression (BIC) og ridge regression (CV) har samme værdi, som også er den laveste for alle 14 modeller. 
Den højeste  justerede R$^2$  har lasso (BIC) og lasso (CV).
Adaptive lasso med lasso vægte (BIC) har den laveste log-likelihood efterfyldt af de 3 andre adaptive lasso modeller, hvor ridge regression (CV) og ridge regression (BIC) har den største log-likelihood. 

%Derudover er alle 14 modeller klart bedre i in-sample resultater i forhold til benchmarkmodellen. Det kunne godt tyde på, at vores benchmark model ikke vil præsterer særligt godt out-of-sample. 

På figur \ref{fig:cf_interval} har vi den betinget 95\%  konfidensinterval udregnet ved anvendelse af TG-testen for lasso (CV) og lasso (BIC), samt det 95\% konfidensinterval for OLS, som er estimeres ud fra de udvalgte variabler af lasso (CV) og lasso (BIC). 
For lasso (CV), ser vi at de to konfidensintervaller næsten er ens når vi har høje koefficienter, imens for de lave koefficienter er konfidensintervallet for lasso størst. 
For lasso (BIC) ser vi at konfidensintervallet er generelt større end konfidens intervallet for OLS.

\imgfigh{cf_interval.pdf}{1}{Betinget 95 \% konfidensinterval for lasso (CV) og lasso (BIC). De blå linjer er 95 \% konfidensinterval for OLS ved anvendelse af de valgte variabler for henholdsvis lasso (CV) og lasso (BIC). }{cf_interval}

Generelt for TG-testen anvendt på LARS uden lasso modifikation ser vi, at variablens grænser i endepunkterne ofte indeholder \textit{inf} og/eller \textit{-inf}, derfor har vi ikke plottet det betinget 95 \% konfidensinterval. 










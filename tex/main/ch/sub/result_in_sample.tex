\section{Oversigt over in-sample resultater} 
For at få et bedre oversigt over vores 14 modeller i in-sample betragtes tabel \ref{tab:topvariable}. Den viser de 12 meste valgte variabler samt beskrivelse. Vi ser at variablerne \textcolor{blue3}{CLF16OV} og \textcolor{blue3}{CE16OV} bliver valgt af alle 14 modeller, imens de resterende variabler ikke bliver valgt af adaptive lasso med OLS vægte og adaptive lasso med lasso vægte, hvor krydsvalidering og BIC er anvendt til valg af $\widehat{\lambda}$. 

\begin{table}[ht] 
\centering 
\begin{tabular}{lll}
\toprule 
Antal & Variable & Beskrivelse \\ \midrule
14 &\textcolor{blue3}{CLF16OV} & Civil arbejdsstyrke \\
14 &\textcolor{blue3}{CE16OV} & Civilbeskæftigelse \\
11 & \textcolor{blue3}{lag 1} & Den tidligere periodes værdi af arbejdsløshedsrate \\
10 & \textcolor{chartreuse4}{IPDMAT} & Holdbart materiale  \\
10 & \textcolor{blue3}{UEMPLT5} & Civile arbejdsløse - mindre end 5 uger \\
10 & \textcolor{blue3}{UEMP5TO14}& Civile arbejdsløse i 5 - 14 uger \\
10 & \textcolor{blue3}{UEMP15OV} &  Civile arbejdsløse i 15 - 26 uger  \\
10 & \textcolor{orange}{TB6MS} & 6-måneders statsskat  \\
10 & \textcolor{orange}{GS5} & 5-årig statsobligationsrente \\
\bottomrule
\end{tabular}  
\caption{Antallet af gange variablerne er blevet inkluderet af lasso og dens generalisering samt beskrivelse af variablerne.} \label{tab:topvariable}
\end{table} 

I forhold til den justerede R$^2$ ser vi, at ridge regression løst med coordinate descent,
hvor $\widehat{\lambda}$ er estimeret ud fra henholdvis krydsvalidering og BIC har samme værdi, som også er den laveste værdi for alle 14 modeller. 
Vi ser også, at alle lasso modellerne har samme justerede R$^2$, selvom at de er løst ud fra to forskellige algoritmer, hvor der også er anvendt krydsvalidering og BIC til valg af $\widehat{\lambda}$. 

De adaptive lasso modeller i coordinate descent har den laveste log-likelihood, men er også dem med færreste antal parameter. Derudover har vi at log-likehood er størst for ridge regression, men igen vælger den også alle 126 forklarende variable. 

Derudover er alle 14 modeller klart bedre i in-sample resultater i forhold til benchmarkmodellen. Det kunne godt tyde på, at vores benchmark model ikke vil præsterer særligt godt out-of-sample. 
\chapter{Out-of-sample}
I dette afsnit vil vi betragte modellernes prædiktion.
Som nævnt prædikteres arbejdsløshedsraten one-step-ahead for et rolling scheme med expanding window.

Arbejdsløshedsraten og de prædikterede værdier sammenlignes med MAE, MSE og forhold i mellem den alternative model og benchmark modellen for både MAE og MSE.
Som nævnt er faktor modellen valgt udfra informationskriterie 2 benchmark model.
Resultaterne af dette er givet i tabel \ref{tab:mae_mse_vurdering}.
%
\begin{table}[ht]
\center
\begin{tabular}{lccccc}
\toprule
 & MAE & \(\text{R}_{\text{MAE}}\) && MSE & \(\text{R}_{\text{MSE}}\) \\ \midrule
AR\(\del{4}\) & 0.1312 & 1.1811 && 0.0272 & 1.454 \\  
Faktor model (IC\(_1\)) & 0.119 & 1.0717 && 0.0221 & 1.1798 \\
Benchmark model & 0.1111 & 1 && 0.0187 & 1 \\
Faktor model (IC\(_3\)) & 0.1048 & 0.9437 && 0.0165 & 0.8809 \\
Lasso (CV) & 0.032 & 0.2877 && 0.0016 & 0.0876 \\
Lasso (BIC) & 0.0308 & 0.277 && 0.0015 & 0.0795 \\
Ridge regression (CV) & 0.0582 & 0.5239 && 0.0052 & 0.28 \\
Ridge regression (BIC) & 0.0573 & 0.5155 && 0.0051 & 0.2706 \\
Group lasso (CV) & 0.0352 & 0.3168 && 0.0019 & 0.1042  \\
Group lasso (BIC) & 0.0382 & 0.3437 && 0.0022 & 0.1202 \\
Adap. lasso m. OLS (CV) & 0.0309 & 0.2783 && 0.0014 & 0.0744 \\
Adap. lasso m. OLS (BIC) & 0.031 & 0.2787 && $\mathbf{0.0014}$ & $\mathbf{0.0743}$ \\
Adap. lasso m. lasso (CV) & 0.031 & 0.2787 && 0.0014 & 0.0743 \\
Adap. lasso m. lasso (BIC) & 0.031 & 0.2788 && 0.0014 & 0.0743 \\
LARS u. lasso modifikation (CV) &  0.0307 & 0.2761 && 0.0015 & 0.0802 \\
LARS u. lasso modifikation (BIC) & $\mathbf{0.0305}$ & $\mathbf{0.2747}$ && 0.0015 & 0.0793 \\
LARS m. lasso modifikation (CV) &  0.0352 & 0.317 && 0.002 & 0.1089 \\
LARS m. lasso modifikation (BIC) & 0.0322 & 0.2901 && 0.0017 & 0.0903 \\ \bottomrule
\end{tabular}
\caption{Den gennemsnitlige absolutte fejl og den gennemsnitlige kvadrerede fejl samt forholdet mellem disse og benchmark modellen.} \label{tab:mae_mse_vurdering}
\end{table}

LARS algoritmen uden lasso modifikationen, hvor variablerne er valgt ud fra BIC, har den mindste MAE, mens adaptive lasso modellerne har den mindste MSE. 
%For MAE ses, at LARS algoritmen uden lasso modifikationen, hvor variablerne er valgt udfra BIC, har den mindste MAE, mens adaptive lasso modellerne foretrukket ifølge MSE.
Først på 7. decimal af MSE for adaptive lasso modellerne ses at adaptive lasso m. OLS vægte valgt udfra BIC faktisk har mindst MSE. 
% og dermed betragtes som den ``bedste'' out-of-sample.
I forhold til benchmark modellen har AR(4) den dårligste prædiktion tæt efterfulgt af faktor modellen, mens der for de resterende modeller ses en tydelig forbedring.
Den indbyrdes forskel i MAE og MSE for lasso modellerne og dens generaliseringer er forholdsvis lille.

For at vurdere modellernes prædiktion i forhold til benchmark modellen fra måned til måned, betragtes et rullende gennemsnitlig fejl ratio.
Et rullende gennemsnitlig absolut fejl ratio for samtlige modeller illustreres på figur \ref{fig:rolling_mae}.
Heraf ses at AR(4) prædikterer dårligere end benchmark modellen over hele testmængden.
Også faktor modellen valgt udfra IC$_1$ lader til at prædiktere dårligere end benchmark modellen med undtagelse af maj, juni og august i 2006.
Derudover ser vi, at ridge regression modellerne prædikterer bedre end benchmark modellen, mens alle lasso baseret modeller klart har bedre prædiktion end de øvrige modeller.
%
\imgfigh{rolling_mae.pdf}{1}{Rullende gennemsnitlige absolutte fejl ratio.}{rolling_mae}
%

%
\imgfigh{rolling_mse.pdf}{1}{Rullende gennemsnitlige kvarerede fejl ratio.}{rolling_mse}
%

\subsection{Diebold Mariano testen}
I tabel \ref{tab:dm_test} ses resultaterne fra Diebold-Mariano testen, hvor hver model testes imod benchmark modellen.
Heraf ses at nulhypotesen ikke kan afvises for faktor modellen valgt udfra IC\(_1\), hvilket vil sige, at modellen ikke er signifikant forskellig fra benchmark modellen.
For de resterende modeller afvises nulhypotesen, hvilket betyder, at disse ikke er signifikant forskellige fra benchmark modellen.
%
\begin{table}[ht]
\center
\begin{tabular}{lcc}
\toprule
% & \(\abs{y_t - \widehat{y}_{i,t}}\) & \(\del{y_t - \widehat{y}_{i,t}}^2\) \\ \midrule
 & Absolutte fejl & Kvadrerede fejl \\ \midrule
AR\(\del{4}\) & 0.0021 & 0.0032 \\  
Faktor model (IC\(_1\)) & 0.1692 & 0.1183 \\
Lasso (CV) & 0 & $0$ \\
Lasso (BIC) & 0 & $0$ \\
Ridge regression (CV) & $0$  & $0$  \\
Ridge regression (BIC) &$0 $& $0$\\
Group lasso (CV) & 0 &$ 0$  \\
Group lasso (BIC) & 0& $0$ \\
Adap. lasso m. OLS vægte (CV) & 0& $0 $\\
Adap. lasso m. OLS vægte (BIC) & 0& $0$\\
Adap. lasso m. lasso vægte (CV) & 0 & $0 $\\
Adap. lasso m. lasso vægte (BIC) & 0& $0$\\
Lasso$_{TG}$ (CV)&  0 & 0 \\
Lasso$_{TG}$ (BIC) & 0 & 0 \\
LARS (CV) & 0 & $0$  \\
LARS (BIC) & 0& $0$ \\
Lasso LARS (CV) & 0& $0$ \\
Lasso LARS (BIC) & 0& $0$ \\ 
LARS$_{TG}$ (CV) & 0 & 0 \\
LARS$_{TG}$ (BIC) &0 & 0\\ \bottomrule
\end{tabular}
\caption{\(p\)-værdier for Diebold-Mariano testen for hver model imod benchmark modellen.
\(p\)-værdier \(< 0.001\) sættes til 0.} \label{tab:dm_test}
\end{table}

%
\newpage
\subsection{MCS} 
Herefter vil vi teste alle modellerne imod hinanden, hvortil vi betragter model confidence procedure.
Proceduren udføres for \(\alpha = 0.1\) og \(\alpha = 0.2\), teststørrelserne \(\text{T}_\text{R}\) og \(\text{T}_\text{max}\) og 5000 bootstrapped samples.
I tabel \ref{tab:mcs_tab} er modellerne i 80\% og 90\% MCS angivet for \(\text{T}_\text{R}\) og \(\text{T}_\text{max}\).
Vi har testet for absolutte fejl og kvadrerede fejl, hvilket gav samme resultat.
For alle teste elimineres faktor modellen valgt udfra IC\(_1\) fra MCS.
For teststørrelsen \(\text{T}_\text{R}\) elimineres yderligere ridge regression med CV og BIC for både \(\alpha = 0.1\) og \(\alpha = 0.2\).
For teststørrelsen \(\text{T}_\text{max}\) elimineres foruden faktor modellen valgt udfra IC\(_1\) også LARS algoritmen med lasso modifikation med CV for \(\alpha = 0.2\), men kun faktor modellen valgt udfra IC\(_1\) for \(\alpha = 0.1\).
%
\begin{sidewaystable}[ht]
\center
\scalebox{0.9}{
\begin{tabular}{lllll}
\toprule
%\multicolumn{4}{c}{\(\abs{y_t - \widehat{y}_{i,t}}\) } & \multicolumn{4}{c}{\(\del{y_t - \widehat{y}_{i,t}}^2\)} \\
\multicolumn{2}{c}{\(\text{T}_\text{R}\)} & & \multicolumn{2}{c}{\(\text{T}_\text{max}\)} \\
\cmidrule{1-2} \cmidrule{4-5} 
\(\alpha = 0.1\) & \(\alpha = 0.2\) & & \(\alpha = 0.1\) & \(\alpha = 0.2\) \\ \midrule
Benchmark model & Benchmark model  && Benchmark model  & Benchmark model \\
AR\((4)\) & AR\((4)\) && AR\((4)\) & AR\((4)\) \\
Lasso (CV) & Lasso (CV) && Lasso (CV) & Lasso (CV) \\
Lasso (BIC) & Lasso (BIC) && Lasso (BIC) & Lasso (BIC) \\
Group lasso (CV) & Group lasso (CV) && Ridge regression (CV) & Ridge regression (CV) \\
Group lasso (BIC) & Group lasso (BIC) && Ridge regression (BIC) & Ridge regression (BIC) \\
Adap. lasso m. OLS vægte (CV) & Adap. lasso m. OLS vægte (CV) && Group lasso (CV) & Group lasso (CV) \\
Adap. lasso m. OLS vægte (BIC) & Adap. lasso m. OLS vægte (BIC) && Group lasso (BIC) & Group lasso (BIC) \\
Adap. lasso m. lasso vægte (CV) & Adap. lasso m. lasso vægte (CV) && Adap. lasso m. OLS vægte (CV) & Adap. lasso m. OLS vægte (CV) \\
Adap. lasso m. lasso vægte (BIC) & Adap. lasso m. lasso vægte (BIC) && Adap. lasso m. OLS vægte (BIC) & Adap. lasso m. OLS vægte (BIC) \\
LARS (CV) & LARS (CV) && Adap. lasso m. lasso vægte (CV) & Adap. lasso m. lasso vægte (CV) \\
LARS (BIC) & LARS (BIC) && Adap. lasso m. lasso vægte (BIC) & Adap. lasso m. lasso vægte (BIC) \\
Lasso LARS (CV) & Lasso LARS (CV) && LARS (CV) & LARS (CV)\\
Lasso LARS (BIC) & Lasso LARS (BIC) && LARS (BIC) & LARS (BIC) \\
& && Lasso LARS (CV) & Lasso LARS (BIC) \\
& && Lasso LARS (BIC) & \\ \bottomrule
\end{tabular}
}
\caption{80\% og 90\% model confidence set for arbejdsløshedsraten for absolutte og kvadrerede fejl.} \label{tab:mcs_tab}
\end{sidewaystable}
%

\section{Oversigt over out-of-sample resultater}
Vi ser, at de modeller hvor variablerne er estimeret ud fra BIC har en lidt bedre MAE, på nær for group lasso. 

 

%Generelt for out-of-sample resultater får vi at lasso og dens generaliseringer er signifikante bedre end benchmark modellen. 
%Vi
%
% da de alle er signifikante forskellige fra benchmark modellen, samt har en markant mindre MSE og MAE end benchmark modellen. 
%Derudover ser vi, at AR(4) har de dårligste resultater både i in-sample og out-of-sample stærkt efterfulgt af faktor modellen valgt ud fra IC$_1$. 
%Vi ser også, at de to ridge regression modeller har den største MSE og MAE, efterfulgt af de to group lasso modeller. De fire modeller er også dem med flest parameter. 
%
% er den der har størst MSE og MAE, hvor $\widehat{\lambda}$ estimeret ud fra både krydsvalidering og BIC. 
%Ridge regression er som nævnt også den med fleste variabler, hvor på 
%
%
%af lasso og dens generaliseringer.
%
%
%
%For lasso og dens generaliseringer, ser vi at ridge regression er den der har størst MSE og MAE, samt at det er den med flest variabler. Vi ser at 
%
%
%
%
%Lasso og dens generaliseringer er alle signifikante bedre end benchmark, da vi har at de er alle signifikante forskellige fra benchmark og at de har markant mindre MSE og MAE end benchmark modellen. 
%
%AR(4) har de dårligste resultater både i in-sample og out-of-sample stærkt efterfulgt af faktor modellen valgt ud fra IC$_1$.  
%
%
%
%
%
%Overordnet set præsterer AR(4) dårligst både in- og out-of-sample.
%Faktor modellen valgt udfra IC\(_1\)
%adaptive lasso inkludere kun 2 prædiktorer og er stadig iblandt de bedste modeller.
%For group lasso som inkludere 99 prædiktorerne 119 (CV) og 99 (BIC), hvilket ikke forbedrer prædiktionen fra de øvrige lasso baseret modeller.
%
%Vi ser ikke den store forskel ved at anvende krydsvalidering fremfor BIC
%LARS med lasso modifikation i forhold til lasso 
%
%LARS uden lasso modifikation lader til at præstrere bedre end LARS med lasso modifikation.
%lidt overraskende.
%least angle regression er ikke direkte under lasso modellerne eller shrinkage.



\section{Out-of-sample}
I dette afsnit vil vi betragte modellernes prædiktion.
Som nævnt foretager vi et rolling scheme med expanding window.

Arbejdsløshedsraten og de prædikterede værdier sammenlignes med MAE, MSE og forhold i mellem MAE og MSE af den alternative model og benchmark modellen.
Vi har valgt faktor modellen udfra informationskriterie 3 som overordnet benchmark model.
Resultaterne af dette er givet i tabel \ref{tab:mae_mse_vurdering}.
%
\begin{table}[ht]
\center
\begin{tabular}{lccccc}
\toprule
 & MAE & \(\text{R}_{\text{MAE}}\) && MSE & \(\text{R}_{\text{MSE}}\) \\ \midrule
AR\(\del{4}\) & 0.1312 & 1.1811 && 0.0272 & 1.454 \\  
Faktor model (IC\(_1\)) & 0.119 & 1.0717 && 0.0221 & 1.1798 \\
Benchmark model & 0.1111 & 1 && 0.0187 & 1 \\
Faktor model (IC\(_3\)) & 0.1048 & 0.9437 && 0.0165 & 0.8809 \\
Lasso (CV) & 0.032 & 0.2877 && 0.0016 & 0.0876 \\
Lasso (BIC) & 0.0308 & 0.277 && 0.0015 & 0.0795 \\
Ridge regression (CV) & 0.0582 & 0.5239 && 0.0052 & 0.28 \\
Ridge regression (BIC) & 0.0573 & 0.5155 && 0.0051 & 0.2706 \\
Group lasso (CV) & 0.0352 & 0.3168 && 0.0019 & 0.1042  \\
Group lasso (BIC) & 0.0382 & 0.3437 && 0.0022 & 0.1202 \\
Adap. lasso m. OLS (CV) & 0.0309 & 0.2783 && 0.0014 & 0.0744 \\
Adap. lasso m. OLS (BIC) & 0.031 & 0.2787 && $\mathbf{0.0014}$ & $\mathbf{0.0743}$ \\
Adap. lasso m. lasso (CV) & 0.031 & 0.2787 && 0.0014 & 0.0743 \\
Adap. lasso m. lasso (BIC) & 0.031 & 0.2788 && 0.0014 & 0.0743 \\
LARS u. lasso modifikation (CV) &  0.0307 & 0.2761 && 0.0015 & 0.0802 \\
LARS u. lasso modifikation (BIC) & $\mathbf{0.0305}$ & $\mathbf{0.2747}$ && 0.0015 & 0.0793 \\
LARS m. lasso modifikation (CV) &  0.0352 & 0.317 && 0.002 & 0.1089 \\
LARS m. lasso modifikation (BIC) & 0.0322 & 0.2901 && 0.0017 & 0.0903 \\ \bottomrule
\end{tabular}
\caption{Den gennemsnitlige absolutte fejl og den gennemsnitlige kvadrerede fejl samt forholdet mellem disse og benchmark modellen.} \label{tab:mae_mse_vurdering}
\end{table}

%
For MAE ses, at LARS algoritmen uden lasso modifikationen, hvor variablerne valgt udfra BIC, vælges som den ``bedste''.
For MSE vælges adaptive lasso modellerne som 

7. decimal

\subsection{Diebold Mariano}
I tabel \ref{tab:dm_test} ses resultaterne fra Diebold-Mariano testen, hvor hver model testes imod benchmark modellen.
Heraf ses at nulhypotesen ikke kan afvises for faktor modellerne valgt udfra information 1 og 3, hvilket vil sige, at disse modeller ikke er signifikant forskellige fra benchmark modellen.
%
\begin{table}[ht]
\center
\begin{tabular}{lcc}
\toprule
% & \(\abs{y_t - \widehat{y}_{i,t}}\) & \(\del{y_t - \widehat{y}_{i,t}}^2\) \\ \midrule
 & Absolutte fejl & Kvadrerede fejl \\ \midrule
AR\(\del{4}\) & 0.0021 & 0.0032 \\  
Faktor model (IC\(_1\)) & 0.1692 & 0.1183 \\
Lasso (CV) & 0 & $0$ \\
Lasso (BIC) & 0 & $0$ \\
Ridge regression (CV) & $0$  & $0$  \\
Ridge regression (BIC) &$0 $& $0$\\
Group lasso (CV) & 0 &$ 0$  \\
Group lasso (BIC) & 0& $0$ \\
Adap. lasso m. OLS vægte (CV) & 0& $0 $\\
Adap. lasso m. OLS vægte (BIC) & 0& $0$\\
Adap. lasso m. lasso vægte (CV) & 0 & $0 $\\
Adap. lasso m. lasso vægte (BIC) & 0& $0$\\
Lasso$_{TG}$ (CV)&  0 & 0 \\
Lasso$_{TG}$ (BIC) & 0 & 0 \\
LARS (CV) & 0 & $0$  \\
LARS (BIC) & 0& $0$ \\
Lasso LARS (CV) & 0& $0$ \\
Lasso LARS (BIC) & 0& $0$ \\ 
LARS$_{TG}$ (CV) & 0 & 0 \\
LARS$_{TG}$ (BIC) &0 & 0\\ \bottomrule
\end{tabular}
\caption{\(p\)-værdier for Diebold-Mariano testen for hver model imod benchmark modellen.
\(p\)-værdier \(< 0.001\) sættes til 0.} \label{tab:dm_test}
\end{table}

%
\subsection{MCS} 
For model confidence proceduren 
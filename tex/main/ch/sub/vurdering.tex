\chapter{Out-of-sample} \label{ch:out-of-sample}
I dette kapitel vil vi betragte modellernes prædiktion.
Som nævnt prædikteres arbejdsløshedsraten one-step-ahead, hvor estimeringsvinduet udvides med én observeret observation per prædiktion.

Arbejdsløshedsraten og de prædikterede værdier sammenlignes med MAE, MSE og et gennemsnitlig tabs ratio mellem hver model og benchmark modellen givet i \eqref{eq:gennemsnitligtabsratio}.
Faktor modellen valgt udfra IC\(_2\) betragtes somsagt som benchmark model.
Resultaterne af dette er givet i tabel \ref{tab:mae_mse_vurdering}.

\begin{table}[ht]
\center
\begin{tabular}{lccccc}
\toprule
 & MAE & \(\text{R}^{\text{MAE}}\) && MSE & \(\text{R}^{\text{MSE}}\) \\ \midrule
 Benchmark model & 0.1111 & 1 && 0.0187 & 1 \\
AR\(\del{4}\) & 0.1312 & 1.1811 && 0.0272 & 1.454 \\  
Faktor model (IC\(_1\)) & 0.119 & 1.0717 && 0.0221 & 1.1798 \\
Lasso (CV) & 0.032 & 0.2877 && 0.0016 & 0.0876 \\
Lasso (BIC) & 0.0308 & 0.277 && 0.0015 & 0.0795 \\
Ridge regression (CV) & 0.0582 & 0.5239 && 0.0052 & 0.28 \\
Ridge regression (BIC) & 0.0573 & 0.5155 && 0.0051 & 0.2706 \\
Group lasso (CV) & 0.0352 & 0.3168 && 0.0019 & 0.1042  \\
Group lasso (BIC) & 0.0382 & 0.3437 && 0.0022 & 0.1202 \\
Adap. lasso m. OLS vægte (CV) & 0.0304 & 0.2733 && 0.0014 & 0.0729 \\
Adap. lasso m. OLS vægte (BIC) & 0.0310 & 0.2787 && 0.0014 & 0.0743 \\
Adap. lasso m. lasso vægte (CV) & $\mathbf{0.0298}$ & $\mathbf{0.2684}$ && $\mathbf{0.0013}$ & $\mathbf{0.0716}$ \\
Lasso$_{TG}$ (CV)& 0.0303 & 0.2724 && 0.0014 & 0.0744 \\ 
Lasso$_{TG}$ (BIC) & 0.031 & 0.279 && 0.0014 & 0.0767 \\
Adap. lasso m. lasso vægte (BIC) & 0.0304 & 0.274  && 0.0014 & 0.0729 \\
LARS (CV) &  0.0307 & 0.2761 && 0.0015 & 0.0802 \\
LARS (BIC) & 0.0305 & 0.2747 && 0.0015 & 0.0793 \\
Lasso LARS (CV) &  0.0352 & 0.317 && 0.002 & 0.1089 \\
Lasso LARS (BIC) & 0.0322 & 0.2901 && 0.0017 & 0.0903 \\
LARS$_{TG}$ (CV) & 0.0300 & 0.2701 && 0.0014 & 0.0745 \\
LARS$_{TG}$ (BIC) & 0.0301 & 0.2708 && 0.0014 & 0.0750 \\ \bottomrule
\end{tabular}
\caption{Den gennemsnitlige absolutte og kvadrerede fejl samt gennemsnitlig tabs ratio mellem hver model og benchmark modellen.} \label{tab:mae_mse_vurdering}
\end{table}


Adaptive lasso med lasso vægte (CV) har mindst MAE og MSE.

, tæt efterfulgt af adaptive lasso med OLS vægte (CV), adaptive lasso med OLS vægte (BIC) og adaptive lasso med lasso vægte (BIC). 
Generelt kan vi se, at den indbyrdes forskel i MAE og MSE for lasso modellerne og dens generaliseringer er forholdsvis lille.
 
Vi ser, at kun AR(4) og faktor model (IC$_1$) er dårligere end benchmarkmodellen. 
Ifølge begge tabsfunktioner præsterer  ......

For at vurdere modellernes prædiktion i forhold til benchmark modellens fra måned til måned betragtes et rullende gennemsnitlig tabs ratio.
Figur \ref{fig:rolling_mae} illustrerer et rullende gennemsnitlig absolut tabs ratio for hver model. 
Heraf ses, at AR(4) prædikterer dårligere end benchmark modellen over hele testmængden.
Også faktor modellen valgt udfra IC$_1$ prædiktere dårligere end benchmark modellen med undtagelse af maj, juni og august i 2006.
Derudover ser vi, at ridge regression modellerne prædikterer bedre end benchmark modellen, mens alle lasso baseret modeller klart har bedre prædiktion end de øvrige modeller.
%
\imgfigh{rolling_mae.pdf}{1}{Rullende gennemsnitlig absolut tabs ratio.}{rolling_mae}
%

Et rullende gennemsnitlig kvadreret tabs ratio viser samme konklusion og er dermed undladt.
%
\imgfigh{rolling_mse.pdf}{1}{Rullende gennemsnitlige kvadrerede fejl ratio.}{rolling_mse}
%

\section{Diebold Mariano test}
I tabel \ref{tab:dm_test} ses resultaterne fra Diebold-Mariano testen, hvor hver model testes imod benchmark modellen.
Heraf ses at nulhypotesen ikke kan afvises for faktor model (IC\(_1\)), hvilket vil sige, at modellen ikke er signifikant forskellig fra benchmark modellen.
For de resterende modeller afvises nulhypotesen, hvilket betyder, at disse ikke er signifikant forskellige fra benchmark modellen.
%
\begin{table}[ht]
\center
\begin{tabular}{lcc}
\toprule
% & \(\abs{y_t - \widehat{y}_{i,t}}\) & \(\del{y_t - \widehat{y}_{i,t}}^2\) \\ \midrule
 & Absolutte fejl & Kvadrerede fejl \\ \midrule
AR\(\del{4}\) & 0.002064 & 0.003207 \\  
Faktor model (IC\(_1\)) & 0.1692 & 0.1183 \\
Faktor model (IC\(_3\)) & 0.2426 & 0.1888 \\
Lasso (CV) & < 2.2e-16 & 2.933e-12 \\
Lasso (BIC) & < 2.2e-16 & 2.728e-12 \\
Ridge regression (CV) & 6.418e-13 & 3.551e-09  \\
Ridge regression (BIC) & 2.85e-13 & 2.507e-09 \\
Group lasso (CV) & < 2.2e-16 & 5.999e-12  \\
Group lasso (BIC) & < 2.2e-16 & 8.845e-12 \\
Adap. lasso m. OLS (CV) & < 2.2e-16 & 2.876e-12 \\
Adap. lasso m. OLS (BIC) & < 2.2e-16 & 2.908e-12 \\
Adap. lasso m. lasso (CV) & < 2.2e-16 & 2.905e-12  \\
Adap. lasso m. lasso (BIC) & < 2.2e-16 & 2.908e-12 \\
LARS u. lasso modifikation (CV) & < 2.2e-16 & 2.64e-12  \\
LARS u. lasso modifikation (BIC) & < 2.2e-16 & 2.615e-12 \\
LARS m. lasso modifikation (CV) & < 2.2e-16 & 4.694e-12  \\
LARS m. lasso modifikation (BIC) & < 2.2e-16 & 3.328e-12 \\ \bottomrule
\end{tabular}
\caption{\(p\)-værdier for Diebold-Mariano testen for hver model imod benchmark modellen.} \label{tab:dm_test}
\end{table}

%
\newpage
\section{MCS} 
Herefter vil vi teste alle modeller imod hinanden, hvortil vi betragter model confidence proceduren.
Proceduren udføres for \(\alpha = 0.1\) og \(\alpha = 0.2\), teststørrelserne \(\text{T}_\text{R}\) og \(\text{T}_\text{max}\) og 5000 bootstrapped samples.
I tabel \ref{tab:mcs_tab} er modellerne i 80\% og 90\% MCS angivet for \(\text{T}_\text{R}\) og \(\text{T}_\text{max}\).
Vi har testet for absolutte fejl og kvadrerede fejl, hvilket gav samme resultat.
For alle teste elimineres faktor model (IC\(_1\)) fra MCS.
For teststørrelsen \(\text{T}_\text{R}\) elimineres yderligere ridge regression (CV) og ridge regression (BIC) for både \(\alpha = 0.1\) og \(\alpha = 0.2\).
For teststørrelsen \(\text{T}_\text{max}\) elimineres foruden faktor model (IC\(_1\)) også lasso LARS (CV) for \(\alpha = 0.2\), men kun faktor model (IC\(_1\)) for \(\alpha = 0.1\).
%
\begin{sidewaystable}[ht]
\center
\scalebox{0.9}{
\begin{tabular}{lllll}
\toprule
%\multicolumn{4}{c}{\(\abs{y_t - \widehat{y}_{i,t}}\) } & \multicolumn{4}{c}{\(\del{y_t - \widehat{y}_{i,t}}^2\)} \\
\multicolumn{2}{c}{\(\text{T}_\text{R}\)} & & \multicolumn{2}{c}{\(\text{T}_\text{max}\)} \\
\cmidrule{1-2} \cmidrule{4-5} 
\(\alpha = 0.1\) & \(\alpha = 0.2\) & & \(\alpha = 0.1\) & \(\alpha = 0.2\) \\ \midrule
Benchmark model & Benchmark model  && Benchmark model  & Benchmark model \\
AR\((4)\) & AR\((4)\) && AR\((4)\) & AR\((4)\) \\
Lasso (CV) & Lasso (CV) && Faktor (IC\(_1\)) & Lasso (CV) \\
Lasso (BIC) & Lasso (BIC) && Lasso (CV) & Lasso (BIC) \\
Group lasso (CV) & Group lasso (CV) && Lasso (BIC) & Ridge regression (CV) \\
Group lasso (BIC) & Group lasso (BIC) && Ridge regression (CV) & Ridge regression (BIC) \\
Adap. lasso m. OLS vægte (CV) & Adap. lasso m. OLS vægte (CV) && Ridge regression (BIC) & Group lasso (CV) \\
Adap. lasso m. OLS vægte (BIC) & Adap. lasso m. OLS vægte (BIC) && Group lasso (CV) & Group lasso (BIC) \\
Adap. lasso m. lasso vægte (CV) & Adap. lasso m. lasso vægte (CV) && Group lasso (BIC) & Adap. lasso m. OLS vægte (CV) \\
Adap. lasso m. lasso vægte (BIC) & Adap. lasso m. lasso vægte (BIC) && Adap. lasso m. OLS vægte (CV) & Adap. lasso m. OLS vægte (BIC) \\
Lasso\(_{TG}\) (BIC) & Lasso\(_{TG}\) (BIC) && Adap. lasso m. OLS vægte (BIC) &  Adap. lasso m. lasso vægte (CV)  \\
LARS (CV) & LARS (CV) && Adap. lasso m. lasso vægte (CV) & Adap. lasso m. lasso vægte (BIC) \\
LARS (BIC) & LARS (BIC) && Adap. lasso m. lasso vægte (BIC) & Lasso\(_{TG}\) (CV) \\
Lasso LARS (CV) & Lasso LARS (CV) &&  Lasso\(_{TG}\) (CV) &  Lasso\(_{TG}\) (BIC) \\
Lasso LARS (BIC) & Lasso LARS (BIC) && Lasso\(_{TG}\) (BIC)  & LARS (CV) \\
& && LARS (CV)& LARS (BIC) \\
& && LARS (BIC)& Lasso LARS (CV) \\
& && Lasso LARS (CV) & Lasso LARS (BIC) \\
& && Lasso LARS (BIC) & LARS\(_{TG}\) (CV) \\ 
& && LARS\(_{TG}\) (CV) & LARS\(_{TG}\) (BIC) \\ 
& && LARS\(_{TG}\) (BIC) &  \\ \bottomrule
\end{tabular}
}
\caption{80\% og 90\% model confidence set for arbejdsløshedsraten for absolutte og kvadrerede fejl.} \label{tab:mcs_tab}
\end{sidewaystable}
%

\section{Oversigt over out-of-sample resultater}
Ifølge MAE og MSE prædikterer adaptive lasso med lasso vægte (CV) bedst.

Kun faktor model (IC\(_1\)) kan ikke afvises til at være signifikant fra benchmark modellen.
For MCS proceduren elimineres faktor model (IC\(_1\)) i alle tilfælde.

Der er dog ikke en stor forskel på de resterende 3 adaptive lasso modeller, hvor de også er de modeller med færreste antal parametre. 

Ridge regression, som inkluderer alle variabler, er også dem der prædikterer dårligst af lasso og dens generaliseringer. 
AR(4) er den model, som får de dårligste resultater, hvilket vi nok også havde forventet, da det er en meget simple model. 

Lasso løst med coordinate descent har den mindste MAE og MSE i forhold til lasso løst med LARS. Det er dog ikke den helt store forskel på deres MAE og MSE. 

For hver model estimerede vi tuning parameteren ud fra to forskellige metoder. Vi ser, at lasso, ridge regression, LARS og LARS med lasso modifikation, hvor tuning parameteren er estimeret ud fra BIC har den mindste MAE og MSE i forhold til at anvende krydsvalidering. Det er dog ikke stor en forskel på fejlene og specielt ikke for lasso og dens generaliseringer. 

%Overordnet set præsterer AR(4) dårligst både in- og out-of-sample.
%Faktor modellen valgt udfra IC\(_1\)
%adaptive lasso inkludere kun 2 prædiktorer og er stadig iblandt de bedste modeller.
%For group lasso som inkludere 99 prædiktorerne 119 (CV) og 99 (BIC), hvilket ikke forbedrer prædiktionen fra de øvrige lasso baseret modeller.
%
%Vi ser ikke den store forskel ved at anvende krydsvalidering fremfor BIC
%LARS med lasso modifikation i forhold til lasso 
%
%LARS uden lasso modifikation lader til at præstrere bedre end LARS med lasso modifikation.
%lidt overraskende.
%least angle regression er ikke direkte under lasso modellerne eller shrinkage.



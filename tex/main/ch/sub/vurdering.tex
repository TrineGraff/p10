\chapter{Out-of-sample} \label{ch:out-of-sample}
I dette kapitel vil vi betragte modellernes prædiktionsevne.
Som nævnt prædikteres arbejdsløshedsraten one-step-ahead, hvor estimeringsvinduet udvides med én observeret observation per prædiktion.

Arbejdsløshedsraten og de prædikterede værdier sammenlignes med MAE, MSE og en gennemsnitlig tabs ratio mellem hver model og benchmark modellen givet i \eqref{eq:gennemsnitligtabsratio}.
Faktor modellen valgt ud fra IC\(_2\) betragtes som tidligere nævnt som benchmark model.
Resultaterne af dette er givet i tabel \ref{tab:mae_mse_vurdering}.

\begin{table}[ht]
\center
\begin{tabular}{lccccc}
\toprule
 & MAE & \(\text{R}_{\text{MAE}}\) && MSE & \(\text{R}_{\text{MSE}}\) \\ \midrule
AR\(\del{4}\) & 0.1312 & 1.1811 && 0.0272 & 1.454 \\  
Faktor model (IC\(_1\)) & 0.119 & 1.0717 && 0.0221 & 1.1798 \\
Benchmark model & 0.1111 & 1 && 0.0187 & 1 \\
Faktor model (IC\(_3\)) & 0.1048 & 0.9437 && 0.0165 & 0.8809 \\
Lasso (CV) & 0.032 & 0.2877 && 0.0016 & 0.0876 \\
Lasso (BIC) & 0.0308 & 0.277 && 0.0015 & 0.0795 \\
Ridge regression (CV) & 0.0582 & 0.5239 && 0.0052 & 0.28 \\
Ridge regression (BIC) & 0.0573 & 0.5155 && 0.0051 & 0.2706 \\
Group lasso (CV) & 0.0352 & 0.3168 && 0.0019 & 0.1042  \\
Group lasso (BIC) & 0.0382 & 0.3437 && 0.0022 & 0.1202 \\
Adap. lasso m. OLS (CV) & 0.0309 & 0.2783 && 0.0014 & 0.0744 \\
Adap. lasso m. OLS (BIC) & 0.031 & 0.2787 && $\mathbf{0.0014}$ & $\mathbf{0.0743}$ \\
Adap. lasso m. lasso (CV) & 0.031 & 0.2787 && 0.0014 & 0.0743 \\
Adap. lasso m. lasso (BIC) & 0.031 & 0.2788 && 0.0014 & 0.0743 \\
LARS u. lasso modifikation (CV) &  0.0307 & 0.2761 && 0.0015 & 0.0802 \\
LARS u. lasso modifikation (BIC) & $\mathbf{0.0305}$ & $\mathbf{0.2747}$ && 0.0015 & 0.0793 \\
LARS m. lasso modifikation (CV) &  0.0352 & 0.317 && 0.002 & 0.1089 \\
LARS m. lasso modifikation (BIC) & 0.0322 & 0.2901 && 0.0017 & 0.0903 \\ \bottomrule
\end{tabular}
\caption{Den gennemsnitlige absolutte fejl og den gennemsnitlige kvadrerede fejl samt forholdet mellem disse og benchmark modellen.} \label{tab:mae_mse_vurdering}
\end{table}


Adaptive lasso med lasso vægte (CV) har mindst MAE og MSE, tæt efterfulgt af de øvrige adaptive lasso modeller, lasso\(_{TG}\) modellerne og LARS\(_{TG}\) modellerne.
Generelt kan vi se, at den indbyrdes forskel i MAE og MSE for lasso modellerne og dens generaliseringer er forholdsvis lille.
Ifølge begge tabsfunktioner prædikterer AR(4) dårligt efterfulgt af faktor model (IC\(_1\)).
Begge performer dårligere end benchmark modellen.

For at vurdere modellernes prædiktion i forhold til benchmark modellens prædiktion fra måned til måned, betragtes en rullende gennemsnitlig tabs ratio.
Figur \ref{fig:rolling_mae} illustrerer en rullende gennemsnitlig absolut tabs ratio for hver model. 
Heraf ses, at AR(4) prædikterer dårligere end benchmark modellen over hele testmængden.
Også faktor modellen (IC\(_1\)) prædikterer dårligere end benchmark modellen med undtagelse af maj, juni og august i 2006.
Derudover ser vi, at ridge regression modellerne prædikterer bedre end benchmark modellen, mens alle lasso baserede modeller klart har bedre prædiktion end de øvrige modeller.
%
\imgfigh{rolling_mae.pdf}{1}{Rullende gennemsnitlig absolut tabs ratio.}{rolling_mae}

En rullende gennemsnitlig kvadreret tabs ratio giver samme konklusion og er derfor undladt.


\section{Diebold Mariano testen}
I tabel \ref{tab:dm_test} ses resultaterne fra Diebold-Mariano testen, hvor hver model testes imod benchmark modellen.
Heraf ses, at nulhypotesen ikke kan afvises for faktor model (IC\(_1\)), hvilket vil sige, at modellen ikke er signifikant forskellig fra benchmark modellen.
For de resterende modeller afvises nulhypotesen, hvilket betyder, at disse ikke er signifikant forskellige fra benchmark modellen.
\newpage
%
\begin{table}[ht]
\center
\begin{tabular}{lcc}
\toprule
% & \(\abs{y_t - \widehat{y}_{i,t}}\) & \(\del{y_t - \widehat{y}_{i,t}}^2\) \\ \midrule
 & Absolutte fejl & Kvadrerede fejl \\ \midrule
AR\(\del{4}\) & 0.0021 & 0.0032 \\  
Faktor model (IC\(_1\)) & 0.1692 & 0.1183 \\
Lasso (CV) & 0 & $0$ \\
Lasso (BIC) & 0 & $0$ \\
Ridge regression (CV) & $0$  & $0$  \\
Ridge regression (BIC) &$0 $& $0$\\
Group lasso (CV) & 0 &$ 0$  \\
Group lasso (BIC) & 0& $0$ \\
Adap. lasso m. OLS vægte (CV) & 0& $0 $\\
Adap. lasso m. OLS vægte (BIC) & 0& $0$\\
Adap. lasso m. lasso vægte (CV) & 0 & $0 $\\
Adap. lasso m. lasso vægte (BIC) & 0& $0$\\
Lasso$_{TG}$ (CV)&  0 & 0 \\
Lasso$_{TG}$ (BIC) & 0 & 0 \\
LARS (CV) & 0 & $0$  \\
LARS (BIC) & 0& $0$ \\
Lasso LARS (CV) & 0& $0$ \\
Lasso LARS (BIC) & 0& $0$ \\ 
LARS$_{TG}$ (CV) & 0 & 0 \\
LARS$_{TG}$ (BIC) &0 & 0\\ \bottomrule
\end{tabular}
\caption{\(p\)-værdier for Diebold-Mariano testen for hver model imod benchmark modellen.
\(p\)-værdier \(< 0.001\) sættes til 0.} \label{tab:dm_test}
\end{table}

%
\section{MCS} 
Herefter vil vi teste alle modeller imod hinanden, hvorfor vi betragter model confidence proceduren.
Proceduren udføres for \(\alpha = 0.1\) og \(\alpha = 0.2\), teststørrelserne \(\text{T}_\text{R}\) og \(\text{T}_\text{max}\) og 5000 bootstrapped samples.
I tabel \ref{tab:mcs_tab} er modellerne i 80\% og 90\% MCS angivet for \(\text{T}_\text{R}\) og \(\text{T}_\text{max}\).
Vi har testet for absolutte fejl og kvadrerede fejl, hvilket gav samme resultat.
For teststørrelsen \(\text{T}_\text{R}\) elimineres faktor model (IC\(_1\)), ridge regression (CV), ridge regression (BIC), lasso\(_{TG}\) (CV), LARS\(_{TG}\) (CV) og LARS\(_{TG}\) (BIC) for både \(\alpha = 0.1\) og \(\alpha = 0.2\).
Elimineringen af lasso\(_{TG}\) (CV), LARS\(_{TG}\) (CV) og LARS\(_{TG}\) (BIC) er overraskende, idet de alle har lav MAE og MSE i tabel \ref{tab:mae_mse_vurdering}. 
For teststørrelsen \(\text{T}_\text{max}\) elimineres kun faktor model (IC\(_1\)) for \(\alpha = 0.2\) og ingen modeller for \(\alpha = 0.1\).
\newpage
%
\begin{sidewaystable}[ht]
\center
\scalebox{0.9}{
\begin{tabular}{lllll}
\toprule
%\multicolumn{4}{c}{\(\abs{y_t - \widehat{y}_{i,t}}\) } & \multicolumn{4}{c}{\(\del{y_t - \widehat{y}_{i,t}}^2\)} \\
\multicolumn{2}{c}{\(\text{T}_\text{R}\)} & & \multicolumn{2}{c}{\(\text{T}_\text{max}\)} \\
\cmidrule{1-2} \cmidrule{4-5} 
\(\alpha = 0.1\) & \(\alpha = 0.2\) & & \(\alpha = 0.1\) & \(\alpha = 0.2\) \\ \midrule
Benchmark model & Benchmark model  && Benchmark model  & Benchmark model \\
AR\((4)\) & AR\((4)\) && AR\((4)\) & AR\((4)\) \\
Lasso (CV) & Lasso (CV) && Lasso (CV) & Lasso (CV) \\
Lasso (BIC) & Lasso (BIC) && Lasso (BIC) & Lasso (BIC) \\
Group lasso (CV) & Group lasso (CV) && Ridge regression (CV) & Ridge regression (CV) \\
Group lasso (BIC) & Group lasso (BIC) && Ridge regression (BIC) & Ridge regression (BIC) \\
Adap. lasso m. OLS vægte (CV) & Adap. lasso m. OLS vægte (CV) && Group lasso (CV) & Group lasso (CV) \\
Adap. lasso m. OLS vægte (BIC) & Adap. lasso m. OLS vægte (BIC) && Group lasso (BIC) & Group lasso (BIC) \\
Adap. lasso m. lasso vægte (CV) & Adap. lasso m. lasso vægte (CV) && Adap. lasso m. OLS vægte (CV) & Adap. lasso m. OLS vægte (CV) \\
Adap. lasso m. lasso vægte (BIC) & Adap. lasso m. lasso vægte (BIC) && Adap. lasso m. OLS vægte (BIC) & Adap. lasso m. OLS vægte (BIC) \\
LARS (CV) & LARS (CV) && Adap. lasso m. lasso vægte (CV) & Adap. lasso m. lasso vægte (CV) \\
LARS (BIC) & LARS (BIC) && Adap. lasso m. lasso vægte (BIC) & Adap. lasso m. lasso vægte (BIC) \\
Lasso LARS (CV) & Lasso LARS (CV) && LARS (CV) & LARS (CV)\\
Lasso LARS (BIC) & Lasso LARS (BIC) && LARS (BIC) & LARS (BIC) \\
& && Lasso LARS (CV) & Lasso LARS (BIC) \\
& && Lasso LARS (BIC) & \\ \bottomrule
\end{tabular}
}
\caption{80\% og 90\% model confidence set for arbejdsløshedsraten for absolutte og kvadrerede fejl.} \label{tab:mcs_tab}
\end{sidewaystable}
%
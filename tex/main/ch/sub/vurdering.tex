\section{Out-of-sample}
I dette afsnit vil vi betragte modellernes prædiktion.
Som nævnt prædikteres arbejdsløshedsraten one-step-ahead for et rolling scheme med expanding window.

Arbejdsløshedsraten og de prædikterede værdier sammenlignes med MAE, MSE og forhold i mellem den alternative model og benchmark modellen for både MAE og MSE.
Som nævnt er faktor modellen valgt udfra informationskriterie 2 benchmark model.
Resultaterne af dette er givet i tabel \ref{tab:mae_mse_vurdering}.
%
\begin{table}[ht]
\center
\begin{tabular}{lccccc}
\toprule
 & MAE & \(\text{R}^{\text{MAE}}\) && MSE & \(\text{R}^{\text{MSE}}\) \\ \midrule
 Benchmark model & 0.1111 & 1 && 0.0187 & 1 \\
AR\(\del{4}\) & 0.1312 & 1.1811 && 0.0272 & 1.454 \\  
Faktor model (IC\(_1\)) & 0.119 & 1.0717 && 0.0221 & 1.1798 \\
Lasso (CV) & 0.032 & 0.2877 && 0.0016 & 0.0876 \\
Lasso (BIC) & 0.0308 & 0.277 && 0.0015 & 0.0795 \\
Ridge regression (CV) & 0.0582 & 0.5239 && 0.0052 & 0.28 \\
Ridge regression (BIC) & 0.0573 & 0.5155 && 0.0051 & 0.2706 \\
Group lasso (CV) & 0.0352 & 0.3168 && 0.0019 & 0.1042  \\
Group lasso (BIC) & 0.0382 & 0.3437 && 0.0022 & 0.1202 \\
Adap. lasso m. OLS vægte (CV) & 0.0304 & 0.2733 && 0.0014 & 0.0729 \\
Adap. lasso m. OLS vægte (BIC) & 0.0310 & 0.2787 && 0.0014 & 0.0743 \\
Adap. lasso m. lasso vægte (CV) & $\mathbf{0.0298}$ & $\mathbf{0.2684}$ && $\mathbf{0.0013}$ & $\mathbf{0.0716}$ \\
Lasso$_{TG}$ (CV)& 0.0303 & 0.2724 && 0.0014 & 0.0744 \\ 
Lasso$_{TG}$ (BIC) & 0.031 & 0.279 && 0.0014 & 0.0767 \\
Adap. lasso m. lasso vægte (BIC) & 0.0304 & 0.274  && 0.0014 & 0.0729 \\
LARS (CV) &  0.0307 & 0.2761 && 0.0015 & 0.0802 \\
LARS (BIC) & 0.0305 & 0.2747 && 0.0015 & 0.0793 \\
Lasso LARS (CV) &  0.0352 & 0.317 && 0.002 & 0.1089 \\
Lasso LARS (BIC) & 0.0322 & 0.2901 && 0.0017 & 0.0903 \\
LARS$_{TG}$ (CV) & 0.0300 & 0.2701 && 0.0014 & 0.0745 \\
LARS$_{TG}$ (BIC) & 0.0301 & 0.2708 && 0.0014 & 0.0750 \\ \bottomrule
\end{tabular}
\caption{Den gennemsnitlige absolutte og kvadrerede fejl samt gennemsnitlig tabs ratio mellem hver model og benchmark modellen.} \label{tab:mae_mse_vurdering}
\end{table}


For MAE ses, at LARS algoritmen uden lasso modifikationen, hvor variablerne er valgt udfra BIC, vælges som den ``bedste''.
Ifølge MSE er adaptive lasso modellerne foretrukket.
Først på 7. decimal ses at adaptive lasso m. ols vægte valgt udfra BIC faktisk har mindst MSE og dermed betragtes som den ``bedste''.
Den autoregressive model af orden 4 performer dårligt out-of-sample ifølge begge tabsfunktioner, modellen er forholdsvis tæt efterfulgt af faktor modellerne.
Men for de resterende modeller ses en tydeligt forbedring i MAE og MSE, og den indbyrdes forskel imellem disse modeller og forholdsvis lille.

For at se hvornår vores modeller outperformer vores benchmark betragter vi modellernes prædiktion i forhold til benchmark fra måned til måned
%Herefter vil vi betragte modellernes prædiktion i forhold til benchmark modellen fra måned til måned.
Vi definere derfor et rullende gennemsnitlig absolut fejl ratio 
\begin{align*}
\text{MAE}^\text{Roll} &= \frac{1}{T} \sum_{t = t_0 +1}^{T} \abs{y_t - \widehat{y}_t} \\
R_t &= \frac{\text{MAE}_\text{Alternative}^\text{Roll}}{\text{MAE}_\text{Benchmark}^\text{Roll}}.
\end{align*}
Hvis \(R_t > 1\), da prædikterer benchmark modellen gennemsnitlig bedre end den alternative model op til tid \(t\), og omvendt hvis \(R_t < 1\).
På figur \ref{fig:rolling_mae} er det rullende gennemsnitlige absolutte fejl ratio plottet.
Faktor modellerne omkring benchmark modellen
derimellem ridge regression
mens de øvrige tabellerne klart prædikterer bedre end benhcmark modellen.
%
\imgfigh{rolling_mae.pdf}{1}{Rolling gennemsnitlige absolutte fejl.}{rolling_mae}
%

\subsection{Diebold Mariano}
I tabel \ref{tab:dm_test} ses resultaterne fra Diebold-Mariano testen, hvor hver model testes imod benchmark modellen.
Heraf ses at nulhypotesen ikke kan afvises for faktor modellerne valgt udfra information 1 og 3, hvilket vil sige, at disse modeller ikke er signifikant forskellige fra benchmark modellen.
For de resterende modeller afvises nulhypotesen, hvilket betyder, at disse ikke er signifikant forskellige fra benchmark modellen.
%
\begin{table}[ht]
\center
\begin{tabular}{lcc}
\toprule
% & \(\abs{y_t - \widehat{y}_{i,t}}\) & \(\del{y_t - \widehat{y}_{i,t}}^2\) \\ \midrule
 & Absolutte fejl & Kvadrerede fejl \\ \midrule
AR\(\del{4}\) & 0.002064 & 0.003207 \\  
Faktor model (IC\(_1\)) & 0.1692 & 0.1183 \\
Faktor model (IC\(_3\)) & 0.2426 & 0.1888 \\
Lasso (CV) & < 2.2e-16 & 2.933e-12 \\
Lasso (BIC) & < 2.2e-16 & 2.728e-12 \\
Ridge regression (CV) & 6.418e-13 & 3.551e-09  \\
Ridge regression (BIC) & 2.85e-13 & 2.507e-09 \\
Group lasso (CV) & < 2.2e-16 & 5.999e-12  \\
Group lasso (BIC) & < 2.2e-16 & 8.845e-12 \\
Adap. lasso m. OLS (CV) & < 2.2e-16 & 2.876e-12 \\
Adap. lasso m. OLS (BIC) & < 2.2e-16 & 2.908e-12 \\
Adap. lasso m. lasso (CV) & < 2.2e-16 & 2.905e-12  \\
Adap. lasso m. lasso (BIC) & < 2.2e-16 & 2.908e-12 \\
LARS u. lasso modifikation (CV) & < 2.2e-16 & 2.64e-12  \\
LARS u. lasso modifikation (BIC) & < 2.2e-16 & 2.615e-12 \\
LARS m. lasso modifikation (CV) & < 2.2e-16 & 4.694e-12  \\
LARS m. lasso modifikation (BIC) & < 2.2e-16 & 3.328e-12 \\ \bottomrule
\end{tabular}
\caption{\(p\)-værdier for Diebold-Mariano testen for hver model imod benchmark modellen.} \label{tab:dm_test}
\end{table}

%
\newpage
\subsection{MCS} 
Model confidence proceduren udføres for \(\alpha = 0.1\) og \(\alpha = 0.2\), teststørrelserne \(\text{T}_\text{R}\) og \(\text{T}_\text{max}\) og 5000 bootstrap resamples som bruges til at konstruerer teststørrelserne.
I tabel \ref{tab:mcs_tab} er modellerne i 80\% og 90\% MCS angivet for \(\text{T}_\text{R}\) og \(\text{T}_\text{max}\).
Vi har testet for absolutte fejl og kvadrerede fejl, hvilket gav samme resultat.
For alle teste elimineres faktor modellen valgt udfra IC\(_1\) fra MCS.
For teststørrelsen \(\text{T}_\text{R}\) elimineres yderligere ridge regression med CV og BIC for både \(\alpha = 0.1\) og \(\alpha = 0.2\).
For teststørrelsen \(\text{T}_\text{max}\) elimineres foruden faktor modellen valgt udfra IC\(_1\) også LARS algoritmen med lasso modifikation (CV) for \(\alpha = 0.2\) og blot faktor modellen valgt udfra IC\(_1\) for \(\alpha = 0.1\).
%
\begin{sidewaystable}[ht]
\center
\scalebox{0.9}{
\begin{tabular}{lllll}
\toprule
%\multicolumn{4}{c}{\(\abs{y_t - \widehat{y}_{i,t}}\) } & \multicolumn{4}{c}{\(\del{y_t - \widehat{y}_{i,t}}^2\)} \\
\multicolumn{2}{c}{\(\text{T}_\text{R}\)} & & \multicolumn{2}{c}{\(\text{T}_\text{max}\)} \\
\cmidrule{1-2} \cmidrule{4-5} 
\(\alpha = 0.1\) & \(\alpha = 0.2\) & & \(\alpha = 0.1\) & \(\alpha = 0.2\) \\ \midrule
Benchmark model & Benchmark model  && Benchmark model  & Benchmark model \\
AR\((4)\) & AR\((4)\) && AR\((4)\) & AR\((4)\) \\
Lasso (CV) & Lasso (CV) && Faktor (IC\(_1\)) & Lasso (CV) \\
Lasso (BIC) & Lasso (BIC) && Lasso (CV) & Lasso (BIC) \\
Group lasso (CV) & Group lasso (CV) && Lasso (BIC) & Ridge regression (CV) \\
Group lasso (BIC) & Group lasso (BIC) && Ridge regression (CV) & Ridge regression (BIC) \\
Adap. lasso m. OLS vægte (CV) & Adap. lasso m. OLS vægte (CV) && Ridge regression (BIC) & Group lasso (CV) \\
Adap. lasso m. OLS vægte (BIC) & Adap. lasso m. OLS vægte (BIC) && Group lasso (CV) & Group lasso (BIC) \\
Adap. lasso m. lasso vægte (CV) & Adap. lasso m. lasso vægte (CV) && Group lasso (BIC) & Adap. lasso m. OLS vægte (CV) \\
Adap. lasso m. lasso vægte (BIC) & Adap. lasso m. lasso vægte (BIC) && Adap. lasso m. OLS vægte (CV) & Adap. lasso m. OLS vægte (BIC) \\
Lasso\(_{TG}\) (BIC) & Lasso\(_{TG}\) (BIC) && Adap. lasso m. OLS vægte (BIC) &  Adap. lasso m. lasso vægte (CV)  \\
LARS (CV) & LARS (CV) && Adap. lasso m. lasso vægte (CV) & Adap. lasso m. lasso vægte (BIC) \\
LARS (BIC) & LARS (BIC) && Adap. lasso m. lasso vægte (BIC) & Lasso\(_{TG}\) (CV) \\
Lasso LARS (CV) & Lasso LARS (CV) &&  Lasso\(_{TG}\) (CV) &  Lasso\(_{TG}\) (BIC) \\
Lasso LARS (BIC) & Lasso LARS (BIC) && Lasso\(_{TG}\) (BIC)  & LARS (CV) \\
& && LARS (CV)& LARS (BIC) \\
& && LARS (BIC)& Lasso LARS (CV) \\
& && Lasso LARS (CV) & Lasso LARS (BIC) \\
& && Lasso LARS (BIC) & LARS\(_{TG}\) (CV) \\ 
& && LARS\(_{TG}\) (CV) & LARS\(_{TG}\) (BIC) \\ 
& && LARS\(_{TG}\) (BIC) &  \\ \bottomrule
\end{tabular}
}
\caption{80\% og 90\% model confidence set for arbejdsløshedsraten for absolutte og kvadrerede fejl.} \label{tab:mcs_tab}
\end{sidewaystable}
%
Overordnet konklusion
sammenlign in-sample og out-of.sample
\chapter{Konklusion}
Problemstillingen, som betragtes i dette speciale, er at udvælge variable og dermed forbedre prædiktionen af makroøkonomiske variable, givet et datasæt med et stort antal forklarende variable.
Prædiktionen kan forbedres ved at mindske regressionskoefficienterne og sætte nogle lig 0.
Vi betragter lasso, ridge regression, elastisk net, group lasso og adaptive lasso.
%En optimal procedure bør besidde orakelegenskaberne, således at variabeludvælgelsen er konsistent.
%Lasso proceduren opfylder ikke disse egenskaber, men en simpel udvidelse af lasso, hvor regressionskoefficienterne pålægges en individuel vægt, kaldet adaptive lasso, opfylder egenskaberne.

Vi anvender coordinate descent algoritmen til at løse lasso, ridge regression, elastisk net og adaptive lasso, og block coordinate algoritmen til at løse group lasso.
Derudover betragtes LARS algoritmen til at løse LARS og lasso.
For at finde den optimale model anvendes en 10-fold krydsvalidering og BIC.

Desuden udføres inferens i lasso modellen.


Algoritmerne returnerer omtrent samme resultat for lasso problemet.
Out-of-sample finder vi, at adaptive lasso modellerne er bedst, efterfulgt af lasso modellerne.
Som benchmark model betragtes en faktor modellen.
Vi finder at en autoregressiv model af orden 4 er signifikant dårligere end benchmark modellen, mens alle lasso baseret modellerne er signifikant bedre end benchmark modellen.


\paragraph{}
Vi kunne også have løst elastisk net, group lasso og adaptive lasso med LARS algoritmen, og sammenlignet resultaterne herfor, med resultaterne returneret af coordinate descent.
Vi har valgt at begrænse den empiriske del ved kun at prædiktere arbejdsløhedsraten one-step-ahead dvs en måned frem.
Men eftersom vi prædikterer makroøkonomiske variable, ville en længere prædiktions periode være mere hensigtsmæssigt.
Alternativt kunne vi have betragtet flere makroøkonomiske variable foruden arbejdsløshedsraten, for at opnå bedre resultater for group lasso og for at elastisk net ikke reduceres til lasso.

Specialet kan udvides ved at betragte vektor autoregressive modeller, således at vi også betragter laggede værdier af de forklarende variable og ikke blot af responsvariablen.
%Derudover kan lasso estimaterne fortolkes som posterior mode estimater når regressions parametrene har uafhængige og identiske Laplace priors.
%
%Derudover kunne vi have betragtes Bayes lasso,
%posterior fordeling og en prior fordeling
Benchmark modellen kan eventuelt forbedres ved at betragte dynamiske faktor modeller, hvor lags af faktorerne også medtages.
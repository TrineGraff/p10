\chapter{Summary}
%mellem min 1 side og maks 2 sider.
%indgår i evalueringen
The master thesis is 


In chapter \ref{ch:dfm} the factor model is introduced.
First the classical factor model is presented, at which we rely some assumptions to ensure uniqueness of the estimators.
However 

In the next chapter the lasso is presented.  

Herefter standard lasso,
Hernæst teori for optimeringsmetoderne som anvendes.
Herefter coordinate descent samt least angle regression.
Generaliseringerne for standard lasso præsenteres herefter, elastisk net, group lasso samt adaptive lasso.
Herefter introduceres statiske inferens for lasso.

In the empirical analysis the presented procedures are used to make one-step-ahead forecasts of the unrate.
The data contains --- monthly variables for -- observation is gathered from publicly available by the Federal Reserve Bank of St. Louis.
As benchmark models we will consider the autoregression model and the factor model.
Thought the empirical analysis, the number of lags of the responsvariable chosen by the autoregressive model, will be included as kovariates during  

The presented models will be evaluated out-of-sample by the mean absolutte error (MAE) and the root mean squared error (RMSE).
We will also consider the model confidence set, which identifies a set of models that are significantly better that the other models.
At last the Diebold Mariano test is considered, to 
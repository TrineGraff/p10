{\selectlanguage{danish}
\pdfbookmark[0]{Danish title page}{label:titlepage_da}
\aautitlepage{%
  \danishprojectinfo{
    Prædiktion af makroøkonomiske variable %title
  }{%
    Shrinkage metoder %theme
  }{%
   Forårssemestret 2018 %project period
  }{%
    5.219 % project group
  }{%
    %list of group members
    Louise Nygaard Christensen\\
    Trine Graff
  }{%
    %list of supervisors
    Esben Høg
  }{%
    2??? % number of printed copies
  }{%
    8. juni 2018 % date of completion
  }%
}{%department and address
  \textbf{Institut for matematiske fag}\\
  Skjernvej 4A\\
  9220 Aalborg\\
  \href{http://es.aau.dk}{http://www.math.aau.dk}
}{% the abstract
In the first half of the report we present some theory of the factor model, the lasso model and some of its generalizations such as the elastic net, group lasso and the adaptive lasso. Also the optimization algorithm coordinate descent and the least angle regression is introduced, to solve the lasso problems and its generalizations.
At last we present some tests for statistical infernce of the lasso problem and generalization.

Those models are then used to estimate and forecast the unemployment rate, which is included in a dataset consisting of 1222? variables.

This paper presents out-of-sample forecasts based on --- for monthly macroeconomic data.
The unemploymentrate is forecasted one-step-ahead

We find that no model is significantly better than each of the benchmarks.}
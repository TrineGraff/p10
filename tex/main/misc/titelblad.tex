{\selectlanguage{danish}
\pdfbookmark[0]{Danish title page}{label:titlepage_da}
\aautitlepage{%
  \danishprojectinfo{
    Prædiktion af en makroøkonomisk variabel %title
  }{%
    Lasso estimatoren og dens generaliseringer %theme
  }{%
   Forårssemestret 2018 %project period
  }{%
    5.219 % project group
  }{%
    %list of group members
    Louise Nygaard Christensen\\
    Trine Graff
  }{%
    %list of supervisors
    Esben Høg
  }{%
    3 % number of printed copies
  }{%
    8. juni 2018 % date of completion
  }%
}{%department and address
  \textbf{Institut for matematiske fag}\\
  Skjernvej 4A\\
  9220 Aalborg\\
  \href{http://es.aau.dk}{http://www.math.aau.dk}
}{% the abstract
In the first part of the report we present some theory of the factor model, the lasso estimator and some of its generalizations such as the elastic net, group lasso and the adaptive lasso. 
We also introduce the optimization algorithms coordinate descent and the least angle regression to solve the lasso problem and its generalizations.
Next we present some results of the asymptotics for the lasso estimator, and introduce the oracle properties, which the adaptive lasso are proved to satisfy. 
At last we present some post-selection inference theory for the lasso estimator.
In the last part of the report we consider a dataset of 122 macroeconomic variables, from which we predict the unemployment rate one-step-ahead with the described models.
As a benchmark model we will consider the factor model.
We find that each model considered outperformes the benchmark model, and especially the adaptive lasso model with OLS and lasso weights are preferred.}